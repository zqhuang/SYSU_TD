\documentclass[CJK]{beamer}
\usepackage{CJKutf8}
\usepackage{beamerthemesplit}
\usetheme{Malmoe}
\useoutertheme[footline=authortitle]{miniframes}
\usepackage{amsmath}
\usepackage{amssymb}
\usepackage{graphicx}
\usepackage{eufrak}
\usepackage{color}
\usepackage{slashed}
\usepackage{simplewick}
\usepackage{tikz}
\graphicspath{{../figures/}}
\def\addfig#1#2{\begin{center}\includegraphics[width=#1 in]{#2}\end{center}}
\def\blacktext#1{{\color{black}#1}}
\def\bluetext#1{{\color{blue}#1}}
\def\redtext#1{{\color{red}#1}}
\def\darkbluetext#1{{\color[rgb]{0,0.2,0.6}#1}}
\def\skybluetext#1{{\color[rgb]{0.2,0.7,1.}#1}}
\def\cyantext#1{{\color[rgb]{0.,0.5,0.5}#1}}
\def\greentext#1{{\color[rgb]{0,0.7,0.1}#1}}
\def\darkgray{\color[rgb]{0.2,0.2,0.2}}
\def\lightgray{\color[rgb]{0.6,0.6,0.6}}
\def\gray{\color[rgb]{0.4,0.4,0.4}}
\def\blue{\color{blue}}
\def\red{\color{red}}
\def\green{\color{green}}
\def\darkblue{\color[rgb]{0,0.2,0.6}}
\def\skyblue{\color[rgb]{0.2,0.7,1.}}
\def\fdeg{{^\circ \mathrm{F}}}
\def\cdeg{^\circ \mathrm{C}}
\def\be{\begin{equation}}
\def\ee{\nonumber\end{equation}}
\def\bea{\begin{eqnarray}}
\def\eea{\nonumber\end{eqnarray}}
\def\ii{{\dot{\imath}}}
\def\bch{\begin{CJK}{UTF8}{gbsn}}
\def\ech{\end{CJK}}
\def\bitem{\begin{itemize}}
\def\eitem{\end{itemize}}
\def\bcenter{\begin{center}}
\def\ecenter{\end{center}}
\def\bex{\begin{minipage}{0.3\textwidth}\includegraphics[width=1in]{jugelizi.png}\end{minipage}\begin{minipage}{0.6\textwidth}}
\def\eex{\end{minipage}}
\def\chtitle#1{\frametitle{\bch#1\ech}}
\def\skipline{{\vskip0.1in}}
\def\skiplines{{\vskip0.2in}}
\def\lagr{{\mathcal{L}}}
\def\hamil{{\mathcal{H}}}
\def\vecv{{\mathbf{v}}}
\def\vecx{{\mathbf{x}}}
\def\vecy{{\mathbf{y}}}
\def\veck{{\mathbf{k}}}
\def\vecp{{\mathbf{p}}}
\def\vecn{{\mathbf{n}}}
\def\vecA{{\mathbf{A}}}
\def\vecP{{\mathbf{P}}}
\def\vecsigma{{\mathbf{\sigma}}}
\def\hatJn{{\hat{J_\vecn}}}
\def\hatJx{{\hat{J_x}}}
\def\hatJy{{\hat{J_y}}}
\def\hatJz{{\hat{J_z}}}
\def\hatj#1{\hat{J_{#1}}}
\def\hatphi{{\hat{\phi}}}
\def\hatq{{\hat{q}}}
\def\hatpi{{\hat{\pi}}}
\def\vel{\upsilon}
\def\Dint{{\mathcal{D}}}
\def\adag{{\hat{a}^\dagger}}
\def\bdag{{\hat{b}^\dagger}}
\def\cdag{{\hat{c}^\dagger}}
\def\ddag{{\hat{d}^\dagger}}
\def\hata{{\hat{a}}}
\def\hatb{{\hat{b}}}
\def\hatc{{\hat{c}}}
\def\hatd{{\hat{d}}}
\def\hatN{{\hat{N}}}
\def\hatH{{\hat{H}}}
\def\hatp{{\hat{p}}}
\def\Fup{{F^{\mu\nu}}}
\def\Fdown{{F_{\mu\nu}}}
\def\newl{\nonumber \\}
\def\SIkm{\,\mathrm{km}}
\def\SIyr{\,\mathrm{yr}}
\def\SIGyr{\,\mathrm{Gyr}}
\def\SIeV{\,\mathrm{eV}}
\def\SIkeV{\,\mathrm{keV}}
\def\SIMeV{\,\mathrm{MeV}}
\def\SIGeV{\,\mathrm{GeV}}
\def\SIcal{\,\mathrm{cal}}
\def\SIkcal{\,\mathrm{kcal}}
\def\SImol{\,\mathrm{mol}}
\def\SIm{\,\mathrm{m}}
\def\SIcm{\,\mathrm{cm}}
\def\SIfm{\,\mathrm{fm}}
\def\SImm{\,\mathrm{mm}}
\def\SInm{\,\mathrm{nm}}
\def\SImum{\,\mathrm{\mu m}}
\def\SIJ{\,\mathrm{J}}
\def\SIkJ{\,\mathrm{kJ}}
\def\SIs{\,\mathrm{s}}
\def\SIkg{\,\mathrm{kg}}
\def\SIg{\,\mathrm{g}}
\def\SIK{\,\mathrm{K}}
\def\SImmHg{\,\mathrm{mmHg}}
\def\SIPa{\,\mathrm{Pa}}
\def\vece{\mathrm{e}}
\def\bmat#1{\left(\begin{array}{#1}}
\def\emat{\end{array}\right)}
\def\bcase#1{\left\{\begin{array}{#1}}
\def\ecase{\end{array}\right.}
\def\calM{{\mathcal{M}}}
\def\calT{{\mathcal{T}}}
\def\calR{{\mathcal{R}}}
\def\barpsi{\bar{\psi}}
\def\baru{\bar{u}}
\def\barv{\bar{\upsilon}}
\def\bmini#1{\begin{minipage}{#1\textwidth}}
\def\emini{\end{minipage}}
\def\qeq{\stackrel{?}{=}}
\def\torder#1{\mathcal{T}\left(#1\right)}
\def\rorder#1{\mathcal{R}\left(#1\right)}
\def\contr#1#2{\contraction{}{#1}{}{#2}#1#2}
\def\trof#1{\mathrm{Tr}\left(#1\right)}
\def\trace{\mathrm{Tr}}
\def\comm#1{\ \ \ \left(\mathrm{used}\ #1\right)}
\def\tcomm#1{\ \ \ (\text{#1})}
\def\slp{\slashed{p}}
\def\slk{\slashed{k}}
\def\wulian{\includegraphics[width=0.18in]{emoji_wulian.jpg}}
\def\bye{\includegraphics[width=0.18in]{emoji_bye.jpg}}
\def\calp{{\mathfrak{p}}}
\def\veccalp{\mathbf{\mathfrak{p}}}
\def\atm{\,\mathrm{atm}}
\def\angstrom{\,\text{\AA}}
\def\Tthree{T_{\tiny \textcircled{3}}}
\def\pthree{p_{\tiny \textcircled{3}}}

\def\courseurl{http://zhiqihuang.top}

\def\tpage#1#2{
\begin{frame}
\bch
\begin{center}
\begin{large}
热学 \\
第#1讲 #2

\end{large}

\skiplines

黄志琦


\end{center}

\skiplines

{\small 
教材:《热学》第二版,赵凯华,罗蔚茵,高等教育出版社


课件下载
}
\courseurl 
\ech
\end{frame}
}

\def\bfr#1{
\begin{frame}
\chtitle{#1} 
\bch
}

\def\efr{
\ech 
\end{frame}
}

\title{Lesson 07 - Equipartition Theorem and Heat Capacity}
  \author{}
  \date{}
\begin{document}
\tpage{7}{能均分定理和热容量}

\section{Reivew}

\begin{frame}
\chtitle{上讲内容回顾}
\bch
{\large
\bitem
\item{球坐标的概率密度函数以及泻流速率的计算:
  $$\vleak = \overline{\frac{\upsilon_z+|\upsilon_z|}{2}}$$
在三维分布各向同性的情况下,$\vleak = \frac{1}{4} \overline{\upsilon}$ ($n$维各向同性则 $\vleak = \frac{G_1G_{n-1}}{2G_n} \overline{\upsilon}$,有兴趣可以自行证明)。
}
\item{麦克斯韦-玻尔兹曼分布: 相空间的概率密度函数
$$f(x,y,z;p_x,p_y,p_z) = c e^{-\frac{\varepsilon}{kT}}$$  }
\eitem}
\ech
\end{frame}

\begin{frame}
  \chtitle{练习一下}
  \bch
      {\large
\bmini{0.55}        
把大气理想化为无限高($0<z<\infty$)的温度为$T$的等温气体。气体分子的重力势能零点取在地面。 计算分子的重力势能的平均值。
\emini
\bmini{0.35}
\addfig{1}{mgz.png}
\emini}
  
  \ech
\end{frame}

\begin{frame}
\chtitle{本讲内容}
\bch
\bitem
\item{广义坐标和广义动量}
\item{自由度}
\item{能均分定理}
\item{量子物理和自由度激发的条件}    
\item{物质的热容量}  
\eitem
\ech
\end{frame}


\section{General position/momentum}

\secpage{广义坐标和广义动量}{坐标/动量的定义不具有绝对性}

\begin{frame}
\chtitle{单摆}
\bch
\bmini{0.4}
\lfig{1.4}{pendulum.jpg}
\emini
\bmini{0.55}
      {\large
        摆幅很小的单摆可以用坐标
        $$x = \ell \theta$$
        以及描述$x$如何变化的动量
        $$p_x = m \frac{dx}{dt}$$
        描述。
        
     我特地在动量下加了个下标$x$以明确这是对应于坐标$x$的动量。
      }
      \emini
      

      
{\large 总能量可以用$x$和$p_x$表示出来:
$$ \varepsilon = \frac{p_x^2}{2m} + \frac{1}{2}\frac{mg}{\ell} x^2 . $$
}
      
\ech
\end{frame}


\begin{frame}
\chtitle{思考题}
\bch
\bmini{0.45}
\lfig{1.5}{pendulum.jpg}
\emini
\bmini{0.5}
      {\large
        为什么一定要把$x$作为“坐标”?
        
        用$\theta=\frac{x}{\ell}$会怎么样?
        
        这时候还有对应的动量$p_\theta$吗?
}
     \emini
\ech
\end{frame}


\begin{frame}
\chtitle{重新定义的坐标和动量}
\bch
{\large
因为相空间的可分辨微观态的体积总是$h$(对我们考虑的一维问题而言)。所以把坐标$x$替换为$\theta = \frac{x}{\ell}$后,必须把动量替换为$p_\theta = \ell p_x$,才能保持可分辨微观态体积不变。

\skipline

总能量

$$\varepsilon = \frac{p_\theta^2}{2I}+\frac{Ig}{2\ell}  \theta^2. $$
其中$I=m\ell^2$是单摆的转动惯量。

}
\ech
\end{frame}

\begin{frame}
\chtitle{广义坐标和广义动量: 描述一样简洁}
\bch
    {\large


  \bmini{0.36}
  坐标$x$

  动量$p_x$

  质量$m$

  能量$\frac{p_x^2}{2m} + \frac{mg}{2\ell} x^2$
  \emini
  \bmini{0.36}
  广义坐标$\theta$ 

  广义动量$p_\theta$ 

  转动惯量$I$

  能量$\frac{p_\theta^2}{2I} + \frac{Ig}{2\ell} \theta^2$
  \emini



}
\ech
\end{frame}

\section{DOF}

\secpage{自由度}{静态地描述物体所需要的参数的个数}

\begin{frame}
\chtitle{自由度的定义}
\bch
    {\large
自由度是静态地描述物体需要的参数个数。
    }

    \addfig{2.5}{geyoutang.jpg}

    {\large
一般来说,一个自由度对应一个广义坐标和一个广义动量。例如单摆有一个转动自由度,对应$(x, p_x)$或者$(\theta, p_\theta)$。    
    }
\ech
\end{frame}


\begin{frame}
\chtitle{思考题}
\bch
    {\large
      三维空间的一个质点有几个自由度?
    }

    \addfig{2}{rengyang.jpg}

\ech
\end{frame}


\begin{frame}
\chtitle{思考题}
\bch

    {\large
      牙签有几个自由度?
      }

\addfig{1.2}{toothstick.jpg}

\ech
\end{frame}

\section{Equipartition Theorem}
\secpage{能均分定理}{概率归一化条件分部积分的结果}



\begin{frame}
\chtitle{概率归一化条件}
\bch
{\large
  假设$\theta$是某一个自由度对应的坐标或者动量,粒子能量$\varepsilon$依赖于$\theta$以及其他坐标和动量(用$X$来表示,注意$X$是多个变量的简写,$dX$则代表这些变量的微分的乘积)。

  当粒子处于温度为$T$平衡态时,相空间的概率密度函数为
$$f(\theta,X) = c e^{-\frac{\varepsilon(\theta,X)}{kT}} $$
其中$c$是待定的归一化常数,满足
$$  \int c e^{-\frac{\varepsilon(\theta,X)}{kT}} d\theta dX = 1. $$
}
\ech
\end{frame}


\begin{frame}
\chtitle{分部积分}
\bch
    {\large
      对$\theta$进行分部积分:
      $$ \int dX\, \left. c e^{-\frac{\varepsilon(\theta,X)}{kT}} \theta\right\vert_{\theta_{\min}}^{\theta_{\max}} + \frac{1}{kT}\int d\theta dX \, c e^{-\frac{\varepsilon(\theta,X)}{kT}} \left(\theta\frac{\partial\varepsilon}{\partial\theta}\right)  = 1. $$
      也就是
      $$ \int dX\, \left.  \theta  f(\theta,X) \right\vert_{\theta_{\min}}^{\theta_{\max}} + \frac{1}{kT}\int d\theta dX \,  f(\theta,X)\left(\theta\frac{\partial\varepsilon}{\partial\theta}\right)  = 1. $$
      {\blue 如果$\theta f(\theta,X)$在上下界($\theta_{\min}$,$\theta_{\max})$抵消},那么立刻得到能均分定理的最一般形式:
   \tbox{ \blue{$$\overline{\theta\frac{\partial\varepsilon}{\partial\theta}}=kT.$$}}}
\ech
\end{frame}

\begin{frame}
\chtitle{特殊情形}
\bch
{\large
利用一般的能均分定理,立即可以得到:

\skipline

{\blue 如果$\theta$是某个自由度的坐标或者动量,$f$是相空间概率密度函数,$\theta f$在$\theta$的上下界相等,则正比于$\theta^n$的能量项贡献$\frac{1}{n}kT$的分子平均能量。}


\skipline

经常说的“每个自由度贡献$\frac{kT}{2}$”则是更特殊的$n=2$的情形(的不严谨表述)。


\skipline

{\bf
注意:这里是指$\varepsilon$除了正比于$\theta^n$的项外,其余部分都和$\theta$无关。}
}
\ech
\end{frame}

\begin{frame}
\chtitle{应用举例一: 动量的二次型}
\bch
{\large
理想气体分子平动动量$p_x\rightarrow \pm \infty$时,相空间分布函数$f$以$e^{-\frac{p_x^2}{2mkT}}$的形式快速趋向于零,满足$p_xf$在上下界相等的要求。

\skipline

所以$\frac{p_x^2}{2m}$贡献$\frac{kT}{2}$的分子平均能量。考虑三个方向的动能项则得到分子平均平动动能为$\frac{3kT}{2}$。

\skipline

上述讨论几乎适用于一切动量的二次型(例如单摆的$\frac{p^2_\theta}{2I}$能量项)。

}
\ech
\end{frame}


\begin{frame}
\chtitle{应用举例二: 一开始的思考题}
\bch
    {\large
      把大气理想化为无限高($0<z<\infty$)的温度为$T$的等温气体,则$mgz$是$z$的一次型。

      在边界$z=0$处,显然$zf = 0$;在边界$z=\infty$处,则$f\propto e^{-\frac{mgz}{kT}} $指数衰减为零。所以$zf$在上下界相等,放心应用能均分定理:

      $$\overline{mgz} = kT.$$

      显然,上述讨论中,“无限厚的大气”和势能零点取在$z=0$都至关重要。实验室的一箱气体的分子平均重力势能不能用能均分定理进行计算。
}
\ech
\end{frame}


\begin{frame}
\chtitle{思考题}
\bch
\addfig{1}{molecules.png}
{\large
 根据能均分定理分别计算单原子理想气体和双原子理想气体的分子平均能量。        
}
\ech
\end{frame}


\begin{frame}
\chtitle{思考题}
\bch
单原子理想气体的自由度只有3个吗?如果把原子当成刚性小球(刚体),那么不是也有6个自由度吗?

\addfig{2}{rengyang2.jpg}


\ech
\end{frame}


\begin{frame}
\chtitle{思考题}
\bch
双原子理想气体有三个平动自由度,两个转动自由度,一个振动自由度?

既然分子是有大小的,为什么不允许它以两个原子的连线为轴进行转动呢?

\addfig{2.5}{rotmolecules.png}


\ech
\end{frame}


\section{Quantum Mechanics and DOF Excitation}

\secpage{量子物理和自由度的激发}{最小转动惯量$\sim \frac{\hbar^2}{kT}$; 最大振动频率$\sim \frac{kT}{\hbar}$;}

\begin{frame}
\chtitle{量子力学的离散能级}
\bch
\bmini{0.45}
经典图像:粒子的能量可以连续变化 

\includegraphics[width=1.5in]{slope.jpg}
\emini
\hspace{0.2in}
\bmini{0.45}
量子力学:束缚态的能级一般是离散的

 \includegraphics[width=1.4in]{steps.jpg}  
\emini
\ech
\end{frame}





\begin{frame}
\chtitle{转动的离散能级}
\bch
{\large
考虑绕固定转轴的逆时针转动。若要转动能量低于$E$,则$\frac{p_\theta^2}{2I} \le E$,或者写成$0\le p_\theta \le \sqrt{2I E}$。又$0\le \theta \le 2\pi$。则相空间内能量低于$E$的态个数为
$$n = \frac{2\pi \sqrt{2IE}}{h} = \frac{\sqrt{2IE}}{\hbar}$$
其中{\blue $h$是普朗克常量,$\hbar = \frac{h}{2\pi}$是约化的普朗克常量}。

若把态按能量从低到高排,则第$n$个态的能量为
$$E_n \approx \frac{n^2\hbar^2}{2I}$$
严格的量子力学计算需要考虑所有方向的旋转轴,结果为
$$E_n = \frac{n(n+1)\hbar^2}{2I}$$}
\ech
\end{frame}


\begin{frame}
\chtitle{最小转动惯量}
\bch
    {\large
      如果要转动能量和热运动能量发生充分交换,则至少有
$$kT \gtrsim \frac{\hbar^2}{I}$$
或者说
$$I \gtrsim \frac{\hbar^2}{kT}$$
结论就是
\tbox{转动惯量远低于``最小转动惯量''$\frac{\hbar^2}{kT}$的转动自由度无法被激发。}
}
\ech
\end{frame}


\begin{frame}
\chtitle{振动的离散能级}
\bch
{\large
\addfig{2}{harmonic_oscillator.png}
考虑能量不高于$E$的谐振子振动
$$  \frac{1}{2} m\omega^2 x^2 + \frac{p^2}{2m} \le E$$
其中$x$为坐标,$p$为动量,$m$为质量,$\omega $为固有频率。

这个不等式描述的是相空间的面积为$\frac{2\pi E}{\omega}$的椭圆内部,所以共包含了
$$n = \frac{2\pi E}{ h \omega} = \frac{E}{\hbar}$$
个可分辨微观态。}

\ech
\end{frame}


\begin{frame}
\chtitle{振动的离散能级(续)}
\bch
{\large
若把态按能量从低到高排,则第$n$个态的能量为
$$ E_n \approx  n\hbar\omega$$
量子力学的严格计算结果为
$$ E_n =  (n+\frac{1}{2})\hbar \omega \ (n = 0,1,2,\ldots)$$
}
\ech
\end{frame}

\begin{frame}
\chtitle{最大本征频率}
\bch
    {\large
      如果要振动能量和热运动能量发生充分交换,则至少有
$$kT \gtrsim \hbar \omega$$
或者说
$$\omega \lesssim \frac{kT}{\hbar}$$
结论就是
\tbox{本征频率远高于``最大本征频率''$\frac{kT}{\hbar}$的振动自由度无法被激发。}
}
\ech
\end{frame}


\begin{frame}
\chtitle{平动动能的激发}
\bch
{\large
对于某个维度上的平动动能,如果在该维度分子的运动范围大小是$L$(例如,容器的边长),分子质量为$m$,则
\tbox{$mL^2$远低于``最小转动惯量''$\frac{\hbar^2}{kT}$的平动自由度无法被激发。}

证明非常容易,留为课后思考题。}
\ech
\end{frame}



\begin{frame}
\chtitle{室温时的最小转动惯量}
\bch
{室温$T\sim 300K$时,最小转动惯量
$$ \frac{\hbar^2}{kT} \sim 3 \times 10^{-48} \SIkg\cdot \SIm^2  $$
如果按等效质量$\sim 10^{-26}\SIkg$, 转动半径$\sim 10^{-10}\SIm$估算双原子分子的转动惯量,则比上述最小转动惯量要大一至二个数量级。所以{\blue 双原子分子的转动自由度在室温下一般是被激发的}。

\skipline

如果以双原子连线为轴进行转动,转动惯量要远低于最小转动惯量,自由度无法被激发。
}
\addfig{1.8}{rotmolecules2.png}
\ech
\end{frame}


\begin{frame}
\chtitle{室温时的最大本征频率}
\bch
{室温$T\sim 300K$时,最大本征频率
$$ \frac{kT}{\hbar} \sim 4 \times 10^{13} \SIs^{-1}  $$
  按等效质量$m\sim 10^{-26}\SIkg$, 振幅$r\sim 10^{-10}m$,$\frac{1}{2}mr^2\omega^2\sim 10\mathrm{eV}$(化学键结合能) 估算双原子分子的本征频率,结果和上述最大本征频率是同一数量级,但要略高一些。所以{\blue 很多双原子气体在室温下振动自由度是冻结的}。但$m$较大时,上述估算要打些折扣,就会有部分振动自由度被激发。

  \skiplines
  
  上面的讨论对$m$小且化学键强的固体也适用。然而很多常见的固体$m$要比$10^{-26}\SIkg$大不少,化学键也稍弱一些,上述估算就要大打折扣,往往本征频率要低于最大本征频率,在室温下振动自由度是被激发的。
}

\ech
\end{frame}

\section{Heat Capacity}

\secpage{气体和固体的定体摩尔热容}{$$\cvmol = \frac{R}{2} \times (number\ of\ qudratic\ terms).$$}

\begin{frame}
\chtitle{单原子气体的定体摩尔热容}
\bch
    {\large
      {\blue 定体摩尔热容($\cvmol$):$1\SImol$物质,保持体积不变,温度升高$1\SIK$需要吸收的热量。}

      \skipline
      
      按能均分定理算出分子平均能量 $\frac{3kT}{2}$,所以$1\SImol$气体保持体积不变温度升高$1K$时,需要吸收
      $$ \frac{3kN_A}{2} = \frac{3R}{2}$$
      的热量。
}
\ech
\end{frame}


\begin{frame}
\chtitle{双原子气体的摩尔定体热容随温度的变化}
\bch
如果从低温开始到高温,双原子气体的转动自由度,振动自由度会被相继激发。摩尔定体热容就从$\frac{3}{2}R$增长到$\frac{5}{2}R$再到$\frac{7}{2}R$.

\addfig{2.5}{gascvmol.png}

请再参考教材85页 图2-14。

当分子质量较大或化学键较弱时也会有部分振动自由度在室温下被激发,见教材表2-3。
\ech
\end{frame}



\begin{frame}
\chtitle{固体的定体摩尔热容}
\bch
    {\large

      很多固体的原子除了三个方向的平动动能之外,还有三个独立的振动势能,按能均分定理,分子平均能量是$3kT$,定体摩尔热容是$3R$。

    \skipline
    
    如果$m$小且化学键强(如硼,金刚石),有些振动自由度就会被冻结,导致$\cvmol$远不到$3R$。参考教材表2-6。

}
\ech
\end{frame}





\end{document}

\documentclass[10pt,CJK]{article}
\usepackage{geometry}
\input{reduced_macros.tex}
\geometry{tmargin=0.3in, bmargin=0.5in, lmargin=0.7in, rmargin=0.7in, nohead, nofoot}
\def\mark#1{{\color{blue} (#1分)}}
\renewcommand{\thepage}{}
\begin{document}
\bch
{\large 热学课堂练习 华山论剑版}

{\vskip 0.05in}

姓名 ....................... {\hskip 0.5in}    学号 .......................{\hskip 0.5in}  分数 ...................

\bitem

\item[(一)]{选择题,每题5分,共30分。

  \bitem


\item[(1)]{{\bf 实际气体}绝热自由膨胀,温度和压强一般如何变化? \bropt
  
  \optlist{温度变低,压强变小 }{温度不变,压强变小}{温度不变,压强变大}

}

  
\item[(2)]{某气体分子的速率分布函数为$F(\upsilon)$,那么速率不超过$\upsilon_0$的所有分子的平均速率是
  \bropt

  \optlist{$\upsilon_0 \int_0^{\upsilon_0}  F(\upsilon)d\upsilon$}{$\int_0^{\upsilon_0}\upsilon F(\upsilon)d\upsilon$}{$\frac{\int_0^{\upsilon_0}\upsilon F(\upsilon)d\upsilon}{\int_0^{\upsilon_0} F(\upsilon)d\upsilon}$}
}
  

  \item[(3)]{
  在大气中进行的化学反应过程中的吸热量等于生成物和反应物的 \bropt

  \optlist{内能差}{焓差}{自由焓差}

  }

\item[(4)]{当氮气温度和压强为下列那组数值时,氮气的节流效应是致冷的? \bropt
  
  \optlist{$T=3000\SIK, p = 1\atm $ }{$T=10\SIK, p = 100\atm$}{$T = 300\SIK, p = 0.5\atm $}

}
    

\item[(5)]{
  处于热平衡的氢气中随机抽取一个分子,其速率不小于方均根速率的10倍的概率和下列哪个数的数量级最接近?\bropt

  \optlist{$1/10$}{$e^{-50}$}{$e^{-150}$}
}

\item[(6)]{某种实际气体的状态方程为$\left(p+\frac{a\nu^2}{V^2T}\right)(V-\nu b) = \nu R T$,其中$p, V, T, \nu$分别为压强,体积,热力学温度和摩尔数,$a, b, R$为常量。则该气体的内能$U$的偏导数$\pfrac UVT$为 \bropt

  \optlist{$0$}{$\frac{a\nu^2}{V^2T}$}{$\frac{2a\nu^2}{V^2T}$}}
  
  \eitem
}

\item[(二)]{填空题,每题5分,共25分。

  \bitem
\item[(1)]{服从麦克斯韦分布的理想气体分子,其速率小于方均根速率的$\frac{1}{10\sqrt{3}}$的概率约为 \uline{0.5}.}
\item[(2)]{一个质量为 $1\SIkg$ 的均匀实心金属球初始温度为 $240\SIK$,比热为 $1000\SIJ/\SIK/\SIkg$。把它投入温度为 $360\SIK$ 的大热库并达到热平衡后,金属球的熵变化了\uline{0.5}.}
\item[(3)]{把标准状态下的$1\SImol$氮气和$1\SImol$氧气保持压强和温度不变地混合,熵变化了\uline{0.5}.}    
\item[(4)]{某恒温容器被一块固定的隔板分为容积相等的左右两部分。隔板中间有一个可以开关的面积很小的孔。初始时刻,小孔关闭,容器左半边装有平衡态的单一成分理想气体,右半边为真空。打开小孔后,经过一段时间后测量,发现容器左半边内气体的压强比初始时刻降低$25\%$。那么再经过同样长一段时间后,容器左半边的压强比初始时刻降低了\uline{0.25}\%. }

\item[(5)]{实验室里有很多相同的高压气体样本,初始体积都是$0.5\SIL$,定压热容为$20\SIJ/\SIK$。取两份样本,分别用准静态绝热膨胀法和绝热节流法进行降温,都使气体压强降低了$1\%$。结果发现准静态绝热膨胀法的降温效果要稍好一点,比绝热节流法多降低了$0.3K$。由此可知,这些气体样本初始时的压强大约是\uline{0.3}$\atm$。 }  

  \eitem  
}
  

  
\item[(三)]{某种气体的状态方程为$\left(p+\frac{a\nu^2}{V^3}\right)V = \nu R T$,其中$\nu$为摩尔数,$p$为压强,$V$为体积,$T$为热力学温度,$a>0$为固定常量。 $1\SImol$该气体经过等温过程体积增大了$10\%$。它的内能变大还是变小?\mark{5} 熵变化了多少? \mark{10}
    \vspace{4.5in}
  }


\item[(四)]{某干燥无风地区的地面大气的压强为$1\atm$,温度为$300\SIK$。在该地区的一处山顶,水的沸点为$90\cdeg$。已知水的汽化热为$4.1\times 10^4\SIJ/\SImol$。试估算山顶处的:(1) 大气压强; \mark{10} (2) 大气温度。\mark{5}
\vspace{3.5in}
}
  
  
  \item[(五)]{某可逆理想气体热机按下述循环工作:
      \bitem
    \item{以多方指数$n=\frac{5}{3}$的准静态多方过程从$423\SIK$升温到$777\SIK$;}
    \item{准静态绝热膨胀,降温到$200\SIK$;}
    \item{以$200\SIK$的温度准静态等温压缩;}
    \item{准静态绝热压缩,回到初始状态。}      
      \eitem
      在循环过程的温度范围内,该理想气体的定体摩尔热容随温度变化规律为
      $$C_V^{\rm mol} = \left(\frac{3}{2} + \frac{T}{T_0}\right)R,$$
      其中$T_0$为某固定常量。选择你喜欢的变量为坐标画出该循环的大致示意图\mark{5}, 并求该热机的效率。 \mark{10}
  }  

\eitem


\ech
\end{document}

\documentclass[CJK]{beamer}
\usepackage{CJKutf8}
\usepackage{beamerthemesplit}
\usetheme{Malmoe}
\useoutertheme[footline=authortitle]{miniframes}
\usepackage{amsmath}
\usepackage{amssymb}
\usepackage{graphicx}
\usepackage{eufrak}
\usepackage{color}
\usepackage{slashed}
\usepackage{simplewick}
\usepackage{tikz}
\graphicspath{{../figures/}}
\def\addfig#1#2{\begin{center}\includegraphics[width=#1 in]{#2}\end{center}}
\def\blacktext#1{{\color{black}#1}}
\def\bluetext#1{{\color{blue}#1}}
\def\redtext#1{{\color{red}#1}}
\def\darkbluetext#1{{\color[rgb]{0,0.2,0.6}#1}}
\def\skybluetext#1{{\color[rgb]{0.2,0.7,1.}#1}}
\def\cyantext#1{{\color[rgb]{0.,0.5,0.5}#1}}
\def\greentext#1{{\color[rgb]{0,0.7,0.1}#1}}
\def\darkgray{\color[rgb]{0.2,0.2,0.2}}
\def\lightgray{\color[rgb]{0.6,0.6,0.6}}
\def\gray{\color[rgb]{0.4,0.4,0.4}}
\def\blue{\color{blue}}
\def\red{\color{red}}
\def\green{\color{green}}
\def\darkblue{\color[rgb]{0,0.2,0.6}}
\def\skyblue{\color[rgb]{0.2,0.7,1.}}
\def\fdeg{{^\circ \mathrm{F}}}
\def\cdeg{^\circ \mathrm{C}}
\def\be{\begin{equation}}
\def\ee{\nonumber\end{equation}}
\def\bea{\begin{eqnarray}}
\def\eea{\nonumber\end{eqnarray}}
\def\ii{{\dot{\imath}}}
\def\bch{\begin{CJK}{UTF8}{gbsn}}
\def\ech{\end{CJK}}
\def\bitem{\begin{itemize}}
\def\eitem{\end{itemize}}
\def\bcenter{\begin{center}}
\def\ecenter{\end{center}}
\def\bex{\begin{minipage}{0.3\textwidth}\includegraphics[width=1in]{jugelizi.png}\end{minipage}\begin{minipage}{0.6\textwidth}}
\def\eex{\end{minipage}}
\def\chtitle#1{\frametitle{\bch#1\ech}}
\def\skipline{{\vskip0.1in}}
\def\skiplines{{\vskip0.2in}}
\def\lagr{{\mathcal{L}}}
\def\hamil{{\mathcal{H}}}
\def\vecv{{\mathbf{v}}}
\def\vecx{{\mathbf{x}}}
\def\vecy{{\mathbf{y}}}
\def\veck{{\mathbf{k}}}
\def\vecp{{\mathbf{p}}}
\def\vecn{{\mathbf{n}}}
\def\vecA{{\mathbf{A}}}
\def\vecP{{\mathbf{P}}}
\def\vecsigma{{\mathbf{\sigma}}}
\def\hatJn{{\hat{J_\vecn}}}
\def\hatJx{{\hat{J_x}}}
\def\hatJy{{\hat{J_y}}}
\def\hatJz{{\hat{J_z}}}
\def\hatj#1{\hat{J_{#1}}}
\def\hatphi{{\hat{\phi}}}
\def\hatq{{\hat{q}}}
\def\hatpi{{\hat{\pi}}}
\def\vel{\upsilon}
\def\Dint{{\mathcal{D}}}
\def\adag{{\hat{a}^\dagger}}
\def\bdag{{\hat{b}^\dagger}}
\def\cdag{{\hat{c}^\dagger}}
\def\ddag{{\hat{d}^\dagger}}
\def\hata{{\hat{a}}}
\def\hatb{{\hat{b}}}
\def\hatc{{\hat{c}}}
\def\hatd{{\hat{d}}}
\def\hatN{{\hat{N}}}
\def\hatH{{\hat{H}}}
\def\hatp{{\hat{p}}}
\def\Fup{{F^{\mu\nu}}}
\def\Fdown{{F_{\mu\nu}}}
\def\newl{\nonumber \\}
\def\SIkm{\,\mathrm{km}}
\def\SIyr{\,\mathrm{yr}}
\def\SIGyr{\,\mathrm{Gyr}}
\def\SIeV{\,\mathrm{eV}}
\def\SIkeV{\,\mathrm{keV}}
\def\SIMeV{\,\mathrm{MeV}}
\def\SIGeV{\,\mathrm{GeV}}
\def\SIcal{\,\mathrm{cal}}
\def\SIkcal{\,\mathrm{kcal}}
\def\SImol{\,\mathrm{mol}}
\def\SIm{\,\mathrm{m}}
\def\SIcm{\,\mathrm{cm}}
\def\SIfm{\,\mathrm{fm}}
\def\SImm{\,\mathrm{mm}}
\def\SInm{\,\mathrm{nm}}
\def\SImum{\,\mathrm{\mu m}}
\def\SIJ{\,\mathrm{J}}
\def\SIkJ{\,\mathrm{kJ}}
\def\SIs{\,\mathrm{s}}
\def\SIkg{\,\mathrm{kg}}
\def\SIg{\,\mathrm{g}}
\def\SIK{\,\mathrm{K}}
\def\SImmHg{\,\mathrm{mmHg}}
\def\SIPa{\,\mathrm{Pa}}
\def\vece{\mathrm{e}}
\def\bmat#1{\left(\begin{array}{#1}}
\def\emat{\end{array}\right)}
\def\bcase#1{\left\{\begin{array}{#1}}
\def\ecase{\end{array}\right.}
\def\calM{{\mathcal{M}}}
\def\calT{{\mathcal{T}}}
\def\calR{{\mathcal{R}}}
\def\barpsi{\bar{\psi}}
\def\baru{\bar{u}}
\def\barv{\bar{\upsilon}}
\def\bmini#1{\begin{minipage}{#1\textwidth}}
\def\emini{\end{minipage}}
\def\qeq{\stackrel{?}{=}}
\def\torder#1{\mathcal{T}\left(#1\right)}
\def\rorder#1{\mathcal{R}\left(#1\right)}
\def\contr#1#2{\contraction{}{#1}{}{#2}#1#2}
\def\trof#1{\mathrm{Tr}\left(#1\right)}
\def\trace{\mathrm{Tr}}
\def\comm#1{\ \ \ \left(\mathrm{used}\ #1\right)}
\def\tcomm#1{\ \ \ (\text{#1})}
\def\slp{\slashed{p}}
\def\slk{\slashed{k}}
\def\wulian{\includegraphics[width=0.18in]{emoji_wulian.jpg}}
\def\bye{\includegraphics[width=0.18in]{emoji_bye.jpg}}
\def\calp{{\mathfrak{p}}}
\def\veccalp{\mathbf{\mathfrak{p}}}
\def\atm{\,\mathrm{atm}}
\def\angstrom{\,\text{\AA}}
\def\Tthree{T_{\tiny \textcircled{3}}}
\def\pthree{p_{\tiny \textcircled{3}}}

\def\courseurl{http://zhiqihuang.top}

\def\tpage#1#2{
\begin{frame}
\bch
\begin{center}
\begin{large}
热学 \\
第#1讲 #2

\end{large}

\skiplines

黄志琦


\end{center}

\skiplines

{\small 
教材:《热学》第二版,赵凯华,罗蔚茵,高等教育出版社


课件下载
}
\courseurl 
\ech
\end{frame}
}

\def\bfr#1{
\begin{frame}
\chtitle{#1} 
\bch
}

\def\efr{
\ech 
\end{frame}
}

\title{Lesson 14 Problem Set 2}
  \author{}
  \date{}
\begin{document}
\tpage{14}{第三四章知识的应用(题霸升级版)}

\section{Review}


\begin{frame}
\chtitle{补充知识:Ruchhardt测$\gamma$法}
\bch
参考教材150页图3-19

这是个力学问题,显然由$p$, $V$这些力学量来描述比较合理。选取绝热方程$pV^\gamma = \const$
$$ \frac{dp}{p} + \gamma \frac{dV}{V} = 0$$
把$dp = dF/S$,$dV = S dx$代入上式,
$$ dF = -\gamma \frac{S^2p}{V} dx $$
即等效回复系数$k =\gamma \frac{S^2p}{V} $ 简谐振动圆频率
$$\omega = \sqrt{\frac{k}{m}} = \sqrt{\frac{\gamma p S^2}{mV}}$$
\ech
\end{frame}

\begin{frame}
\chtitle{补充知识:大气垂直温差的修正}
\bch
{\small
对水蒸气饱和的空气,做功等于内能改变量加上汽化热:
$$ -pdV = C_V dT + \Lambda d(c\nu) $$
其中$\nu$为空气摩尔数,$c$为空气中水蒸气的摩尔百分比,$\Lambda$为摩尔汽化热。又
$$pdV + Vdp = \nu R dT$$
两式相加得到
$$ Vdp = C_p  dT + \nu \Lambda dc $$
再由力学平衡$ dp = -\rho g dz$,即有
$$ \frac{dT}{dz} =-\frac{\rho g V + \nu\Lambda\frac{dc}{dz}}{C_p} = -\frac{\gamma-1}{\gamma}\left(\frac{M^{\rm mol}g}{R}+\frac{\Lambda}{R}\frac{dc}{dz}\right)$$
因为饱和蒸气压随着温度降低而降低(请回忆三相图),在有饱和蒸汽的空气里,气团继续上升会造成水蒸气凝结,$ \frac{dc}{dz} < 0 $,垂直温度梯度就会小于$-10\SIK/\SIkm$。
}
\ech
\end{frame}

\section{PDEs}

\begin{frame}
  \chtitle{  第一篇:令人头疼的偏导数}
  \bch


\addfig{4.2}{piantouteng.png}

  \ech
\end{frame}


\begin{frame}
  \chtitle{例题1}
  \bch
  对$pVT$系统证明下述Maxwell关系:
\bitem
\item{$\pfrac TVS = -\pfrac pSV$}
\item{$\pfrac TpS = \pfrac VSp$}
\item{$\pfrac SVT = \pfrac pTV$}
\item{$\pfrac SpT = -\pfrac VTp$}
\eitem

\ech
\end{frame}

\begin{frame}
  \chtitle{例题1解答}
  \bch
\bitem
\item{由$dU = TdS - pdV$为全微分以及全微分成立条件得到$\pfrac TVS = -\pfrac pSV$。}
\item{由$dH = TdS + Vdp$为全微分以及全微分成立条件得到$\pfrac TpS = \pfrac VSp$。}
\item{由$dF = -SdT - pdV$为全微分以及全微分成立条件得到$\pfrac SVT = \pfrac pTV$。}
\item{由$dG = -SdT + Vdp$为全微分以及全微分成立条件得到$\pfrac SpT = -\pfrac VTp$。}
\eitem
  \ech
\end{frame}

\begin{frame}
  \chtitle{例题2}
  \bch
某气体的物态方程为
$$p(V-\nu b) = \nu R T$$
其中$b$为常量。

证明该气体的定压热容和定体热容之差为
$$ C_p - C_V = \nu R $$

  \ech
\end{frame}

\begin{frame}
  \chtitle{例题2解法1}
  \bch
由$C_p  =\pfrac HTp = T \pfrac STp$, $C_V = \pfrac UTp = T\pfrac STV$
得到
$$\frac{C_p - C_V}{T} = \pfrac STp - \pfrac STV = \pfrac SVT  \pfrac VTp $$
最后一步我们用了四变量的偏微分链式法则。然后根据$dF = - SdT - pdV$为全微分得到$\pfrac SVT = \pfrac pTV$,代入上式
$$ C_p - C_V = T\pfrac pTV \pfrac VTp = T \frac{\nu R}{V-\nu b} \frac{\nu R}{p} = \nu R$$

  \ech
\end{frame}


\begin{frame}
  \chtitle{例题2解法2}
  \bch
由内能和状态方程的关系得到(准静态过程吸热量的第一种表达方式)
{\scriptsize 
$$ T dS  = dU + pdV = C_V dT + \left(T\pfrac pTV -p\right)dV+pdV = C_V dT + T\pfrac pTV dV  $$}
由焓和状态方程的关系得到(准静态过程吸热量的第二种表达方式)
{\scriptsize 
$$ TdS = dH - Vdp = C_p dT + (V-T\pfrac VTp)dp - Vdp = C_p dT - T\pfrac VTp dp$$}
两式相减得到
$$ (C_p- C_V)dT =  T\pfrac VTp dp + T\pfrac pTV dV = (V-\nu b)dp + p dV $$
又根据状态方程有$ \nu R dT = (V-\nu b) dp + p dV$,与上式比较即得证$C_p - C_V = \nu R$。
  \ech
\end{frame}


\begin{frame}
  \chtitle{例题3}
  \bch
某固定体积的$pVT$系统,在一定温度范围内($200\SIK < T< 400 \SIK$)内能和温度有如下关系:
  $$U =  aT^2 $$
  其中$a=1 \SIJ/\SIK^2$。
  初始时该系统温度$250\SIK$,自由能$F = 100 \SIkJ$。
  当系统升温到$350\SIK$,求系统自由能$F$。
  \ech
\end{frame}

\begin{frame}
  \chtitle{例题3解法1}
  \bch
  初始状态时,
  $$ S_{\rm ini} = \frac{U_{\rm ini}-F_{\rm ini}}{T_{\rm ini}} = \frac{1\times 250^2 - 10^5}{250} \SIJ/\SIK = -150\SIJ/\SIK$$
  在体积不变的情况下
  $$ dU = T\pfrac STV dT  =  2aTdT$$
  故
  $$ \pfrac STV = 2a $$
  积分得到
  $$ S_{\rm final}  = S_{\rm ini} + 2a\Delta T =  \left[-150 + 2\times 1\times (350-250) \right] \SIJ/\SIK = 50 \SIJ/\SIK $$
  $$ F_{\rm final} = U_{\rm final}-T_{\rm final}S_{\rm final} = \left[1\times 350^2 - 350\times 50\right] \SIJ = 105\SIkJ$$
  
  \ech
\end{frame}

\begin{frame}
  \chtitle{例题3解法2}
  \bch
  $$\pfrac {(F/T)}TV = \frac{1}{T}\pfrac FTV - \frac{F}{T^2} = -\frac{F+TS}{T^2} = -\frac{U}{T^2} = -a $$
  积分得到
  $$\frac{F}{350\SIK} = \frac{10^5\SIJ}{250\SIK} - 1\SIJ/\SIK^2\times(350\SIK - 250\SIK) = 300 \SIJ/\SIK$$
  即
  $$F = 105 \SIkJ$$

  \ech
\end{frame}


\begin{frame}
  \chtitle{例题4}
  \bch
  $pVT$系统中,证明
  $$ \pfrac TpV \pfrac SVp - \pfrac TVp \pfrac SpV = 1 $$
  \ech
\end{frame}

\begin{frame}
  \chtitle{例题4解答}
  \bch
  等式左边是$(T,S)\rightarrow (p, V)$的Jaccobi行列式,代表$(T,S)$和$(p,V)$图上的面积元之比。在研究准静态循环时,我们根据热力学第一定律已经知道了$(T,S)$和$(p,V)$图上的对应的任何闭合曲线包围的面积相等(这包括面积的符号,即闭合曲线的绕行方向)。面积元只是很小的闭合曲线围绕的面积而已,所以得证。

  \skipline

  补充$T$-$S$图和$p$-$V$图上闭合曲线包围面积相等的证明:
$$ \oint TdS - \oint pdV = \oint dU = 0$$
  
  \ech
\end{frame}



\begin{frame}
  \chtitle{例题5}
  \bch
  $pVT$系统中,设$x, y$为任意两个独立变量,证明:
      {\small $$ \pfrac Txy \pfrac Syx - \pfrac Tyx \pfrac Sxy = \pfrac pxy \pfrac Vyx - \pfrac pyx \pfrac Vxy $$}  
  \ech
\end{frame}

\begin{frame}
  \chtitle{例题5解答}
  \bch
等式左边是$(T,S)\rightarrow (x, y)$的Jaccobi行列式,代表$(T,S)$和$(x,y)$图上的面积元之比。等式右边是$(p,V)\rightarrow (x, y)$的Jaccobi行列式,代表$(p,V)$和$(x,y)$图上的面积元之比。由于$(T,S)\rightarrow (p,V)$是保面积映射,故得证。
  \ech
\end{frame}

\begin{frame}
  \chtitle{习题6}
  \bch
  $pVT$系统中,设$x, y$为任意两个独立变量,证明:
      {\scriptsize $$ \pfrac {\left(\frac{1}{T}\right)}xy \pfrac Uyx - \pfrac {\left(\frac{1}{T}\right)}yx \pfrac Uxy = \pfrac Vxy \pfrac {\left(\frac{p}{T}\right)}yx - \pfrac Vyx \pfrac {\left(\frac{p}{T}\right)}xy $$}  
  \ech
\end{frame}


\begin{frame}
  \chtitle{习题6解答}
  \bch
  同样只要证明$\frac{1}{T}$-$U$图上的闭合曲线包围的面积和$V$-$\frac{p}{T}$图上对应的闭合曲线包围的面积相等。
  注意应用环路积分计算$x$-$y$图上闭合曲线面积时既可以用$\oint ydx$,也可以用$\oint -x dy$。对$V$-$\frac{p}{T}$图我们采取后者。
  
  $$\oint \frac{dU}{T} - \oint \frac{-pdV}{T} = \oint \frac{\dbar Q}{T} = \oint dS = 0$$
  
  \ech
\end{frame}

\begin{frame}
  \chtitle{习题7}
  \bch
  对$pVT$系统证明
  $$\pfrac UpV = -T\pfrac VTS$$
  \ech
\end{frame}


\begin{frame}
  \chtitle{习题7解答}
  \bch
  证明:固定熵时
  $$ \dbar Q = C_V dT + T \pfrac pTV dV = 0 $$
  即
  $$ \pfrac VTS = -\frac{C_V}{T\pfrac pTV} = -\frac{\pfrac UTV}{T\pfrac pTV} = -\frac{1}{T} \pfrac UpV $$
  两边乘以$-T$即得证。
  \ech
\end{frame}


\begin{frame}
  \chtitle{习题8}
  \bch
  对$pVT$系统证明
  $$\pfrac UVp = T\pfrac pTS - p$$
  \ech
\end{frame}


\begin{frame}
  \chtitle{习题8解答}
  \bch
  证明:固定熵时
  $$ \dbar Q = C_p dT - T\pfrac VTp dp = 0 $$
  即
  $$ T\pfrac pTS = \frac{C_p}{\pfrac VTp} = \frac{\pfrac HTp}{\pfrac VTp} =  \pfrac HVp = \pfrac UVp + p $$
  两边减去$p$即得证。
  \ech
\end{frame}

\begin{frame}
  \chtitle{习题9}
  \bch
  对$pVT$系统证明
  $$\pfrac TVU = p\pfrac TUV - T\pfrac pUV $$
  \ech
\end{frame}


\begin{frame}
  \chtitle{习题9解答}
  \bch
  证明:固定内能时
  $$ dU = C_V dT + \left(T\pfrac pTV - p\right)dV = 0$$
  即
  {\scriptsize
  $$ \pfrac TVU = \frac{p-T\pfrac pTV }{C_V} = p\pfrac TUV - T\frac{\pfrac pTV}{\pfrac UTV}  = p\pfrac TUV - T\pfrac pUV $$}
  \ech
\end{frame}

\begin{frame}
  \chtitle{习题10}
  \bch
  对$pVT$系统证明
  $$\pfrac TSH +\frac{T^2}{V} \pfrac VHp = \frac{T}{C_p} $$
  \ech
\end{frame}

\begin{frame}
  \chtitle{习题10解答}
  \bch
  {\scriptsize
  证明:由$C_p = T\pfrac STp$得到
  $$ \frac{T}{C_p} - \pfrac TSH = \pfrac TSp - \pfrac TSH = \pfrac THS \pfrac HSp $$
  最后一步我们使用了四变量的链式法则。由$dH = TdS + Vdp$即知$\pfrac HSp = T$,所以
  \begin{equation}
    \frac{T}{C_p} - \pfrac TSH = T \pfrac THS = \frac{T}{V} \pfrac TpS \label{eq1}
  \end{equation}
  上面最后一步我们利用了$S$不变时,$ dH = Vdp$。
  最后,利用$dH  = TdS +Vdp$为全微分,得到
  \begin{equation}
    \pfrac TpS = \pfrac VSp = T\pfrac VHp \label{eq2}
  \end{equation}
  其中最后一步利用了$p$不变时,$dH = TdS$。

  把\eqref{eq2}带入 \eqref{eq1}即证毕。
  
  }
  \ech
\end{frame}

\section{Thermaldynamic Process}

\begin{frame}
  \chtitle{第二篇:热力学过程}
\addfig{4.2}{process.png}
\end{frame}

\begin{frame}
  \chtitle{几个名词}
  \bch
  \bitem
\item{绝热:$Q  = 0$,  $\Delta S = 0$}
\item{自由:$A=0$}
\item{节流:$\Delta H = 0$}
\item{准静态过程 = 可逆过程; 非准静态过程=不可逆过程}
\item{正循环:对外做功的热机$A'>0$;$p$-$V$图和$T$-$S$图上都是顺时针。}
\item{逆循环:外界对系统做功的制冷机$A>0$;$p$-$V$图和$T$-$S$图上都是逆时针。}
\item{可逆循环:对外界影响可以通过相反的一个循环来完全消除。环境熵变为零。}
\item{不可逆循环:对外界影响不能通过任何方式完全消除。环境熵变大于零。}
\eitem
  \ech  
\end{frame}

\begin{frame}
  \chtitle{理想气体的准静态过程}
  \bch
  \bitem
  \item{理想气体内能和体积无关:$dU = C_V dT$}
  \item{等温过程对外做功$A'=\nu RT \Delta \ln V$}
  \item{
  多方过程($n\ne 1$)对外做功
  $A' = -\frac{\nu R}{n-1}\Delta T$
  
  热容相应地即
  $C_n = C_V - \frac{\nu R}{n-1}$}
  \item{理想气体的熵变$ dS = C_V d \ln T + \nu R d\ln V$}
  \item{常数$\gamma$近似下绝热方程$pV^\gamma = C$, $TV^{\gamma-1} =C$, $Tp^{\frac{1}{\gamma}-1} = C$}
  \item{可逆多方循环热机效率$\eta = 1- \frac{T_2}{T_1}$其中$T_2$, $T_1$分别是任意一条绝热线两端的低温值和高温值。可逆多方循环的制冷剂效率则是$T_2/(T_1-T_2)$}
  \eitem
  \ech  
\end{frame}

\begin{frame}
  \chtitle{例题11}
  \bch
  教材166页思考题3-20
  \ech
\end{frame}


\begin{frame}
  \chtitle{例题11解答}
  \bch
  对任何工作物质成立。用$T$-$S$图计算一目了然。
  \ech
\end{frame}

\begin{frame}
  \chtitle{例题12}
  \bch
  教材166页思考题3-19
  \ech
\end{frame}


\begin{frame}
  \chtitle{例题12解答}
  \bch
  空调机比较省电。空调的制热原理是把室外当成需要制冷的低温热源。如果看成可逆循环,空调释放给室内(高温热源)的热量大于外界对空调做功(即空调耗能)。用电炉的话释放的热量不会大于消耗的电能。当然,这些讨论都是基于空调可以看成高效率的可逆制冷机的前提。
  \ech
\end{frame}

\begin{frame}
  \chtitle{例题13}
  \bch
  教材166页习题3-2
  
  \ech
\end{frame}


\begin{frame}
  \chtitle{例题13解答}
  \bch
  {\scriptsize
  氮气的摩尔数$\nu = \frac{14\SIg}{28\SIg/\SImol} = 0.5\SImol$
  \bitem
\item{等温过程$\Delta U = 0$, $A = -\nu R T\Delta \ln V = 787\SIJ$, $Q = \Delta U - A = - 787\SIJ$
}
\item{绝热过程$Q=0$, $\Delta U = C_V \Delta T = \frac{5}{2}\nu R (2^{\gamma-1}T -T) = 907\SIJ$, $A = \Delta U - Q = 907\SIJ$}
\item{等压过程$A = -p\Delta V = \frac{pV}{2} = \frac{1}{2}\nu R T = 567\SIJ$,$\Delta U = C_V \Delta T = \frac{5}{2}\nu R (\frac{T}{2}-T) = -1419\SIJ$,$Q = \Delta U - A = -1986 \SIJ$}
  \eitem
  }
  \ech
\end{frame}

\section{Homework}

\begin{frame}
  \chtitle{第14周作业(序号接第13周)}
  \bch
  {\small 
  \bitem
\item{教材习题3-14}
\item{对$pVT$系统证明$\pfrac TpH = T \pfrac VHp - V\pfrac THp$}
\item{某固定压强的$pVT$系统,在一定温度范围内($200\SIK < T< 400 \SIK$)焓和温度有如下关系:
  $$H =  aT^2 $$
  其中$a=1 \SIJ/\SIK^2$。
  初始时该系统温度$250\SIK$,自由焓$G = 100 \SIkJ$。
  当系统升温到$350\SIK$,求系统自由焓$G$。
}
  \eitem
  }
  \ech
\end{frame}

\end{document}

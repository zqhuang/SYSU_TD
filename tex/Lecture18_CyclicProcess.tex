\documentclass[CJK]{beamer}
\usepackage{CJKutf8}
\usepackage{beamerthemesplit}
\usetheme{Malmoe}
\useoutertheme[footline=authortitle]{miniframes}
\usepackage{amsmath}
\usepackage{amssymb}
\usepackage{graphicx}
\usepackage{eufrak}
\usepackage{color}
\usepackage{slashed}
\usepackage{simplewick}
\usepackage{tikz}
\graphicspath{{../figures/}}
\def\addfig#1#2{\begin{center}\includegraphics[width=#1 in]{#2}\end{center}}
\def\blacktext#1{{\color{black}#1}}
\def\bluetext#1{{\color{blue}#1}}
\def\redtext#1{{\color{red}#1}}
\def\darkbluetext#1{{\color[rgb]{0,0.2,0.6}#1}}
\def\skybluetext#1{{\color[rgb]{0.2,0.7,1.}#1}}
\def\cyantext#1{{\color[rgb]{0.,0.5,0.5}#1}}
\def\greentext#1{{\color[rgb]{0,0.7,0.1}#1}}
\def\darkgray{\color[rgb]{0.2,0.2,0.2}}
\def\lightgray{\color[rgb]{0.6,0.6,0.6}}
\def\gray{\color[rgb]{0.4,0.4,0.4}}
\def\blue{\color{blue}}
\def\red{\color{red}}
\def\green{\color{green}}
\def\darkblue{\color[rgb]{0,0.2,0.6}}
\def\skyblue{\color[rgb]{0.2,0.7,1.}}
\def\fdeg{{^\circ \mathrm{F}}}
\def\cdeg{^\circ \mathrm{C}}
\def\be{\begin{equation}}
\def\ee{\nonumber\end{equation}}
\def\bea{\begin{eqnarray}}
\def\eea{\nonumber\end{eqnarray}}
\def\ii{{\dot{\imath}}}
\def\bch{\begin{CJK}{UTF8}{gbsn}}
\def\ech{\end{CJK}}
\def\bitem{\begin{itemize}}
\def\eitem{\end{itemize}}
\def\bcenter{\begin{center}}
\def\ecenter{\end{center}}
\def\bex{\begin{minipage}{0.3\textwidth}\includegraphics[width=1in]{jugelizi.png}\end{minipage}\begin{minipage}{0.6\textwidth}}
\def\eex{\end{minipage}}
\def\chtitle#1{\frametitle{\bch#1\ech}}
\def\skipline{{\vskip0.1in}}
\def\skiplines{{\vskip0.2in}}
\def\lagr{{\mathcal{L}}}
\def\hamil{{\mathcal{H}}}
\def\vecv{{\mathbf{v}}}
\def\vecx{{\mathbf{x}}}
\def\vecy{{\mathbf{y}}}
\def\veck{{\mathbf{k}}}
\def\vecp{{\mathbf{p}}}
\def\vecn{{\mathbf{n}}}
\def\vecA{{\mathbf{A}}}
\def\vecP{{\mathbf{P}}}
\def\vecsigma{{\mathbf{\sigma}}}
\def\hatJn{{\hat{J_\vecn}}}
\def\hatJx{{\hat{J_x}}}
\def\hatJy{{\hat{J_y}}}
\def\hatJz{{\hat{J_z}}}
\def\hatj#1{\hat{J_{#1}}}
\def\hatphi{{\hat{\phi}}}
\def\hatq{{\hat{q}}}
\def\hatpi{{\hat{\pi}}}
\def\vel{\upsilon}
\def\Dint{{\mathcal{D}}}
\def\adag{{\hat{a}^\dagger}}
\def\bdag{{\hat{b}^\dagger}}
\def\cdag{{\hat{c}^\dagger}}
\def\ddag{{\hat{d}^\dagger}}
\def\hata{{\hat{a}}}
\def\hatb{{\hat{b}}}
\def\hatc{{\hat{c}}}
\def\hatd{{\hat{d}}}
\def\hatN{{\hat{N}}}
\def\hatH{{\hat{H}}}
\def\hatp{{\hat{p}}}
\def\Fup{{F^{\mu\nu}}}
\def\Fdown{{F_{\mu\nu}}}
\def\newl{\nonumber \\}
\def\SIkm{\,\mathrm{km}}
\def\SIyr{\,\mathrm{yr}}
\def\SIGyr{\,\mathrm{Gyr}}
\def\SIeV{\,\mathrm{eV}}
\def\SIkeV{\,\mathrm{keV}}
\def\SIMeV{\,\mathrm{MeV}}
\def\SIGeV{\,\mathrm{GeV}}
\def\SIcal{\,\mathrm{cal}}
\def\SIkcal{\,\mathrm{kcal}}
\def\SImol{\,\mathrm{mol}}
\def\SIm{\,\mathrm{m}}
\def\SIcm{\,\mathrm{cm}}
\def\SIfm{\,\mathrm{fm}}
\def\SImm{\,\mathrm{mm}}
\def\SInm{\,\mathrm{nm}}
\def\SImum{\,\mathrm{\mu m}}
\def\SIJ{\,\mathrm{J}}
\def\SIkJ{\,\mathrm{kJ}}
\def\SIs{\,\mathrm{s}}
\def\SIkg{\,\mathrm{kg}}
\def\SIg{\,\mathrm{g}}
\def\SIK{\,\mathrm{K}}
\def\SImmHg{\,\mathrm{mmHg}}
\def\SIPa{\,\mathrm{Pa}}
\def\vece{\mathrm{e}}
\def\bmat#1{\left(\begin{array}{#1}}
\def\emat{\end{array}\right)}
\def\bcase#1{\left\{\begin{array}{#1}}
\def\ecase{\end{array}\right.}
\def\calM{{\mathcal{M}}}
\def\calT{{\mathcal{T}}}
\def\calR{{\mathcal{R}}}
\def\barpsi{\bar{\psi}}
\def\baru{\bar{u}}
\def\barv{\bar{\upsilon}}
\def\bmini#1{\begin{minipage}{#1\textwidth}}
\def\emini{\end{minipage}}
\def\qeq{\stackrel{?}{=}}
\def\torder#1{\mathcal{T}\left(#1\right)}
\def\rorder#1{\mathcal{R}\left(#1\right)}
\def\contr#1#2{\contraction{}{#1}{}{#2}#1#2}
\def\trof#1{\mathrm{Tr}\left(#1\right)}
\def\trace{\mathrm{Tr}}
\def\comm#1{\ \ \ \left(\mathrm{used}\ #1\right)}
\def\tcomm#1{\ \ \ (\text{#1})}
\def\slp{\slashed{p}}
\def\slk{\slashed{k}}
\def\wulian{\includegraphics[width=0.18in]{emoji_wulian.jpg}}
\def\bye{\includegraphics[width=0.18in]{emoji_bye.jpg}}
\def\calp{{\mathfrak{p}}}
\def\veccalp{\mathbf{\mathfrak{p}}}
\def\atm{\,\mathrm{atm}}
\def\angstrom{\,\text{\AA}}
\def\Tthree{T_{\tiny \textcircled{3}}}
\def\pthree{p_{\tiny \textcircled{3}}}

\def\courseurl{http://zhiqihuang.top}

\def\tpage#1#2{
\begin{frame}
\bch
\begin{center}
\begin{large}
热学 \\
第#1讲 #2

\end{large}

\skiplines

黄志琦


\end{center}

\skiplines

{\small 
教材:《热学》第二版,赵凯华,罗蔚茵,高等教育出版社


课件下载
}
\courseurl 
\ech
\end{frame}
}

\def\bfr#1{
\begin{frame}
\chtitle{#1} 
\bch
}

\def\efr{
\ech 
\end{frame}
}

\title{Lesson 18 Cyclic Process}
  \author{}
  \date{}
\begin{document}
\tpage{18}{循环过程}


\begin{frame}
\chtitle{本讲内容}
\bchL
\bitem
\item{孤立系统的熵增大原理}
\item{循环过程}
\item{卡诺定理}  
\item{理想气体可逆循环的效率}
\eitem
\echL
\end{frame}

\section{Entropy of IS}

\secpage{孤立系统的熵增大原理}{孤立的非热平衡系统的熵总是自发地增大,直到达到热平衡态(熵最大的状态)。}

\begin{frame}
  \chtitle{思考题}
  \bchL
  \addfig{0.6}{think3.jpg}

  温度为$400\SIK$的热库和温度为$300\SIK$的热库短暂接触,传过去热量$Q = 1200J$。计算这个过程两个热库的总熵变。
  \echL
\end{frame}

\begin{frame}
\chtitle{孤立系统的熵增大原理}
\bchL
假想一个孤立的非热平衡系统$\Sigma$,把它划分成很多子系统$\Sigma_1$, $\Sigma_2$, $\ldots$, $\Sigma_N$。



\bmini{0.52}
\lfig{2}{subsystems.jpg}
\emini
\bmini{0.44}
每个子系统$\Sigma_i$在短时间内可以近似看成处于平衡态,有确定的温度$T_i$,压强$p_i$等.

\skipline

假设子系统边界处的分子相互作用可以忽略,系统总熵等于各个子系统熵之和.
\emini
\echL
\end{frame}

\begin{frame}
\chtitle{孤立系统的熵增大原理(续)}
\bchL
因系统处于非热平衡状态,总有两个互相接触的子系统温度不同,不妨设$T_i>T_j$。

\skipline

假设{\blue 在宏观尺度上,热量会且只会自发地从高温物体传递到低温物体},$\Sigma_i$对$\Sigma_j$传热$Q_{i\rightarrow j}$ ($Q_{i\rightarrow j}>0$)。那么系统的总熵变化为
$$\Delta S = \Delta S_i + \Delta S_j = - \frac{Q_{i\rightarrow j}}{T_i}+ \frac{Q_{i\rightarrow j}}{T_j}  > 0.$$

\skipline
这样的熵增大过程会持续进行,直到系统达到平衡态(所有的$T_i = T_j$)。

\echL
\end{frame}

\begin{frame}
\chtitle{孤立系统的熵增大原理(续)}
\bchL

\bitem
\item{\blue 孤立的非热平衡系统的熵总是自发地增大,直到达到热平衡态(熵最大的状态)。热平衡态的熵不再发生变化。}
\item{\blue 孤立系统的可逆过程熵不变,不可逆过程熵增大。}
\eitem

\addfig{2}{dengshan.png}

\echL
\end{frame}


\section{Cyclic Process}
\secpage{循环过程}{(正循环)热机效率 = 对外净做功/吸收热量; \\(逆循环)制冷机效率 = 吸收热量/消耗的功}

\begin{frame}
\chtitle{蒸汽机}
\bch
\addfig{3}{steam_engine.jpg}
\ech
\end{frame}

\begin{frame}
\chtitle{循环过程}
\bchL
{\blue 循环过程}:一系统由某个状态出发,经过一系列过程,最后{\blue 回到原状态。}

\bmini{0.6}
\lfig{2.2}{cycle.jpg}
\emini
\bmini{0.35}
{\blue 经过循环过程,系统的所有态函数($p, V, T, S, U\ldots$)都不变。}
\emini
\echL
\end{frame}


\begin{frame}
\chtitle{可逆循环与不可逆循环}
\bchL



\bmini{0.48}
\lfig{1.8}{ircycle.jpg}
\emini
\bmini{0.48}
\addfig{1.8}{rcycle.jpg}
\emini

\bitem
\item{若循环过程对环境造成的影响不可逆(环境熵增大), {\blue 称为不可逆循环}。}
\item{若循环过程对环境造成的影响是可逆的(环境熵不变),称为{\blue 可逆循环};}
\eitem
\echL
\end{frame}

\begin{frame}
\chtitle{正循环热机}
\bchL

系统对外界净做功$A'_{\rm net}>0$的循环为正循环。

\addfig{3}{heat_engine.jpg}
\echL
\end{frame}

\begin{frame}
\chtitle{逆循环制冷机}
\bchL

外界对系统净做功$A_{\rm net}>0$的循环为逆循环。逆循环热机也叫制冷机。

\addfig{3}{cold_engine.jpg}
\echL
\end{frame}

\begin{frame}
\chtitle{思考题}
\bchL
下列过程是正循环还是逆循环?
\bitem
\item{蒸汽机的一个循环}
\item{制冷机的一个循环}
\item{净吸热量$Q_{\rm net}$大于零的循环}
\item{净放热量$Q_{\rm net}'$大于零的循环}
\item{$p$-$V$图上顺时针的闭合曲线}
\item{$p$-$V$图上逆时针的闭合曲线}
\item{$T$-$S$图上顺时针的闭合曲线}
\item{$T$-$S$图上逆时针的闭合曲线}
\eitem
\echL
\end{frame}


\begin{frame}
\chtitle{正循环热机的效率}
\bchL

\addfig{2}{heat_engine.jpg}


{\bf 正循环把热量转化为机械功。}设正循环热机从高温热源1吸热$Q_1$,对外净做功$A'_{\rm net}$,并对低温热源2放热$Q_2'$。{\bf 热量转化为机械能的百分比称为正循环热机的效率,记作$\eta$。}
$$\eta \equiv \frac{A'_{\rm net}}{Q_1}=\frac{Q_1-Q_2'}{Q_1} $$


\echL
\end{frame}

\section{Carnot Theorem}
\secpage{卡诺定理}{$$\eta \le 1 - \frac{T_2}{T_1}$$}

\begin{frame}
\chtitle{恒温热源间的正循环的效率}
\bch
\addfig{2}{heat_engine.jpg}
$$\eta = \frac{Q_1-Q_2'}{Q_1} = 1-\frac{T_2|\Delta S_2|}{T_1|\Delta S_1|}$$
 其中$|\Delta S_1|$为高温热源的熵减少量,$|\Delta S_2|$为低温热源的熵增加量。注意热机循环后回到原状态,熵不变。所以总熵的变化为 $|\Delta S_2| - |\Delta S_1|$.

 \bitem
 \item{若整个循环过程{\blue 可逆},则总熵不变:$|\Delta S_1| = |\Delta S_2|$,即{\blue $ \eta = 1 - \frac{T_2}{T_1}$}。}
 \item{如果整个循环过程{\blue 不可逆},则环境总熵增大:$|\Delta S_2| > |\Delta S_1|$,即{\blue $\eta < 1 - \frac{T_2}{T_1} $}。}
   \eitem

\ech
\end{frame}


\begin{frame}
\chtitle{卡诺定理}
\bchL
\addfig{2}{heat_engine.jpg}

在温度为$T_1$的高温热源和温度为$T_2$的低温热源之间工作的热机:{\blue 可逆循环$\eta = 1 -\frac{T_2}{T_1}$,不可逆循环$\eta < 1 - \frac{T_2}{T_1}$. 这称为卡诺(Carnot)定理}。
\echL
\end{frame}

\begin{frame}
\chtitle{逆循环热机的效率}
\bchL

\addfig{2}{cold_engine.jpg}

{\bf 逆循环利用外界提供的机械功从低温热源吸热制冷。}设外界对逆循环热机净做功$A_{\rm net}$,使它从低温热源2吸热$Q_2$,并对高温热源放热$Q_1'$。{\bf 制冷量$Q_2$与外功$A$之比称为逆循环热机的制冷系数,记作$\varepsilon$。}
$$\varepsilon \equiv \frac{Q_2}{A_{\rm net}}=\frac{Q_2}{Q_1'-Q_2} $$

\echL
\end{frame}


\begin{frame}
\chtitle{恒温热源间的逆循环的制冷效率}
\bch
{\small
设外界对热机做功$A$,使热机从温度恒为$T_2$的低温热源2吸热$Q_2$,并对温度恒为$T_1$的高温热源1放热$Q_1'$。则
$$\varepsilon = \frac{1}{Q_1'/Q_2 - 1} = \frac{1}{\frac{T_1|\Delta S_1|}{T_2|\Delta S_2|}-1}$$
其中$|\Delta S_1|$为高温热源的熵增加量,$|\Delta S_2|$为低温热源的熵减少量。

如果整个循环过程{\bf 可逆},则环境的总熵不变:$|\Delta S_1| = |\Delta S_2|$。

$$ \varepsilon = \frac{T_2}{T_1-T_2}$$

如果整个循环过程{\bf 不可逆},则环境总熵增大:$|\Delta S_2| < |\Delta S_1|$,即
$$\varepsilon <  \frac{T_2}{T_1-T_2} $$
}
\ech
\end{frame}


\begin{frame}
\chtitle{可逆热机的正循环效率和逆循环效率的关系}
\bchL
如果把一个{\bf 可逆的}正循环热机反向运转,就是逆循环热机,容易根据定义得出两者的效率$\eta$, $\varepsilon$存在如下关系

\tbox{$$\eta(1+\varepsilon) = 1$$}

注意:如果循环不可逆,则不存在反向运转的操作。
\echL
\end{frame}


\begin{frame}
\chtitle{环保常识: 空调温度别开太低}
\bchL

\addfig{1.1}{aclife2.png}

设夏天室外温度为$30\cdeg$。把空调近似看成可逆热机。试估算制冷温度分别为$28\cdeg$和$20\cdeg$时空调制冷效率之比。

\echL
\end{frame}



\begin{frame}
\chtitle{正循环$p$-$V$图}
\bch
\addfig{3.2}{pVdiagram_cycle1.png}
\ech
\end{frame}

\begin{frame}
\chtitle{逆循环$p$-$V$图}
\bch
\addfig{3.2}{pVdiagram_cycle2.png}
\ech
\end{frame}


\begin{frame}
\chtitle{正循环$T$-$S$图}
\bch
\addfig{3.2}{TSdiagram_cycle2.png}
\ech
\end{frame}

\begin{frame}
\chtitle{逆循环$T$-$S$图}
\bch
\addfig{3.2}{TSdiagram_cycle1.png}
\ech
\end{frame}


\section{Ideal gas cycles}

\secpage{理想气体的可逆循环}{多方-绝热-多方-绝热: $\eta = 1 -$ 绝热线两端温度比}


\begin{frame}
\chtitle{dalao也有长得帅的}
\bchL

\bmini{0.45}
\lfig{1.3}{njg_ndmz.png}
\emini
\bmini{0.5}
\addfig{1.4}{carnot.jpg}
\bcenter
Nicolas Léonard Sadi Carnot
\ecenter
\emini
\echL
\end{frame}


\begin{frame}
\chtitle{卡诺循环(Carnot Cycle)}
\bch

\addfig{2}{carnotcycle.jpg}

\ech
\end{frame}

\begin{frame}
\chtitle{理想气体的可逆卡诺循环$p$-$V$图}
\bchL
\addfig{2.2}{Carnot_cycle.png}

思考题:对理想气体计算可逆卡诺循环的效率(设$C_V$仅是温度的函数,等温过程的$T_1$, $T_2$已知)。
\echL
\end{frame}


\begin{frame}
\chtitle{猜一猜}
\bchL
刚才算出理想气体可逆卡诺循环的效率为$\eta = 1-\frac{T_2}{T_1}$,这个结果对非理想气体成立吗?
\echL
\end{frame}

\begin{frame}
\chtitle{可逆卡诺循环的$T$-$S$图}
\bchL
\addfig{2.2}{Carnot_cycle_TS.png}

在这个图里计算热机效率特别容易(和工作物质无关):
$$\eta = \frac{A'}{Q_1} = \frac{(T_1-T_2)(S_2-S_1)}{T_1(S_2-S_1)} = 1- \frac{T_2}{T_1}$$

\echL
\end{frame}


\begin{frame}
\chtitle{理想气体的可逆多方循环}
\bchL
设理想气体定体热容$C_V$为常量。把卡诺循环的两个等温过程换成多方指数为$n$的多方过程(即整个循环为 多方-绝热-多方-绝热),则可逆热机的效率仍然可以写成
{\blue $$\eta = 1-\frac{T_2}{T_1}$$}
其中$T_1, T_2$ ($T_1>T_2$) 是(循环中任一个绝热过程对应的)绝热线的两端的温度。

\skipline

      {\scriptsize 证明可以参考附录1。

        如果两个多方过程的多方指数不同,计算就复杂了(参考附录2)。}
\echL
\end{frame}


\begin{frame}
\chtitle{附录1:定体热容固定的理想气体的(可逆)多方循环}
\bch
\bmini{0.5}
对定体热容固定的理想气体,把卡诺循环中的两个等温过程换成多方指数为$n$的多方过程,如图所示
\emini
\bmini{0.46}
\addfig{1.6}{polytropic_cycle_n.jpg}
\emini
如果$C_V$是常量,则多方热容$C_n =\frac{n-\gamma}{n-1}C_V$也是常量(见上一讲附录)。准静态过程中
$$ TdS = \dbar Q  = C_n dT $$
或者分离变量
$$\frac{dT}{T} = \frac{1}{C_n} dS$$
\ech
\end{frame}


\begin{frame}
\chtitle{附录1:定体热容固定的理想气体的(可逆)多方循环}
\bch
\bmini{0.5}
这个方程的解为
$$T =  c e^{\frac{S}{C_n} }$$
其中$c$为积分常量。
\emini
\bmini{0.46}
\addfig{1.6}{polytropic_cycle_n.jpg}
\emini

因此,两个多方过程在$T$-$S$图上对应的曲线只是相差一个系数(即积分常量$c$不同)。即曲线$CD$下的面积($Q_2'$)和曲线$AB$下的面积($Q_1$)之比为$Q_2'/Q_1=T_D/T_A = T_C/T_B$。热机效率为
$$ \eta = 1 - \frac{T_D}{T_A} = 1 - \frac{T_C}{T_B}$$
{\scriptsize 此外,利用AD过程(或BC过程)的绝热方程很容易把上式中的温度比转化成体积比或者压强比。}

\ech
\end{frame}

\begin{frame}
\chtitle{附录2:定体热容固定的理想气体的的(可逆)广义多方循环}
\bch
\bmini{0.5}
广义多方循环由两个绝热过程和两个多方指数不同的多方过程组成,设多方膨胀的多方指数为$n_1$,多方压缩的多方指数为$n_2$:
\emini
\bmini{0.46}
\addfig{1.6}{polytropic_cycle.jpg}
\emini

{\scriptsize 
不妨设$S_1=0$,$S_2=S$。多方膨胀过程中$T = T_A e^{S/C_{n_1}}$,多方压缩过程中$T= T_D e^{S/C_{n_2}}$。积分求出
$$Q_2' = T_D \int_0^S e^{S/C_{n_2}} dS = T_DC_{n_2}\left(e^{S/C_{n_2}}-1\right)$$
$$Q_1  = T_A \int_0^S e^{S/C_{n_1}} dS = T_AC_{n_1}\left(e^{S/C_{n_1}}-1\right)$$
$$\eta = 1-\frac{Q_2'}{Q_1} = 1 - \frac{T_D}{T_A} \frac{C_{n_2}}{C_{n_1}} \frac{e^{S/C_{n_2}}-1}{e^{S/C_{n_1}}-1}$$
}
\ech
\end{frame}

\begin{frame}
\chtitle{附录2:定体热容固定的理想气体的的广义多方循环(续)}
\bch
{\scriptsize 
若$n_1=n_2$则回到前面讨论的结果。若$n_1\ne n_2$,则由$T_B = T_A e^{S/C_{n_1}}$以及$T_C = T_D e^{S/C_{n_2}}$得到
$$\frac{T_B T_D}{T_AT_C} = e^{S\left(\frac{1}{C_{n_1}} - \frac{1}{C_{n_2}}\right)}$$
即
$$ e^S = \left(\frac{T_B T_D}{T_AT_C}\right)^{\frac{C_{n_1}C_{n_2}}{C_{n_2}-C_{n_1}}}$$
记绝热压缩温度比$r_c = \frac{T_D}{T_A}$,绝热膨胀温度比$r_e = \frac{T_C}{T_B}$,则
$$ e^S = \left(\frac{r_c}{r_e}\right)^{\frac{C_{n_1}C_{n_2}}{C_{n_2}-C_{n_1}}}$$
记{\blue $\lambda = \frac{C_{n_2}}{C_{n_2}-C_{n_1}}$},代入前面的结果得到
{\blue $$\eta =1- \frac{\lambda}{\lambda-1} \frac{r_c^\lambda r_e - r_e^\lambda r_c}{r_c^\lambda - r_e^\lambda}$$}
}
\ech
\end{frame}

\end{document}

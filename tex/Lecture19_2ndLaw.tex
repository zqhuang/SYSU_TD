\documentclass[CJK, 14pt]{beamer}
\usepackage{CJKutf8}
\usepackage{beamerthemesplit}
\usetheme{Malmoe}
\useoutertheme[footline=authortitle]{miniframes}
\usepackage{amsmath}
\usepackage{amssymb}
\usepackage{graphicx}
\usepackage{eufrak}
\usepackage{color}
\usepackage{slashed}
\usepackage{simplewick}
\usepackage{tikz}
\graphicspath{{../figures/}}
\def\addfig#1#2{\begin{center}\includegraphics[width=#1 in]{#2}\end{center}}
\def\blacktext#1{{\color{black}#1}}
\def\bluetext#1{{\color{blue}#1}}
\def\redtext#1{{\color{red}#1}}
\def\darkbluetext#1{{\color[rgb]{0,0.2,0.6}#1}}
\def\skybluetext#1{{\color[rgb]{0.2,0.7,1.}#1}}
\def\cyantext#1{{\color[rgb]{0.,0.5,0.5}#1}}
\def\greentext#1{{\color[rgb]{0,0.7,0.1}#1}}
\def\darkgray{\color[rgb]{0.2,0.2,0.2}}
\def\lightgray{\color[rgb]{0.6,0.6,0.6}}
\def\gray{\color[rgb]{0.4,0.4,0.4}}
\def\blue{\color{blue}}
\def\red{\color{red}}
\def\green{\color{green}}
\def\darkblue{\color[rgb]{0,0.2,0.6}}
\def\skyblue{\color[rgb]{0.2,0.7,1.}}
\def\fdeg{{^\circ \mathrm{F}}}
\def\cdeg{^\circ \mathrm{C}}
\def\be{\begin{equation}}
\def\ee{\nonumber\end{equation}}
\def\bea{\begin{eqnarray}}
\def\eea{\nonumber\end{eqnarray}}
\def\ii{{\dot{\imath}}}
\def\bch{\begin{CJK}{UTF8}{gbsn}}
\def\ech{\end{CJK}}
\def\bitem{\begin{itemize}}
\def\eitem{\end{itemize}}
\def\bcenter{\begin{center}}
\def\ecenter{\end{center}}
\def\bex{\begin{minipage}{0.3\textwidth}\includegraphics[width=1in]{jugelizi.png}\end{minipage}\begin{minipage}{0.6\textwidth}}
\def\eex{\end{minipage}}
\def\chtitle#1{\frametitle{\bch#1\ech}}
\def\skipline{{\vskip0.1in}}
\def\skiplines{{\vskip0.2in}}
\def\lagr{{\mathcal{L}}}
\def\hamil{{\mathcal{H}}}
\def\vecv{{\mathbf{v}}}
\def\vecx{{\mathbf{x}}}
\def\vecy{{\mathbf{y}}}
\def\veck{{\mathbf{k}}}
\def\vecp{{\mathbf{p}}}
\def\vecn{{\mathbf{n}}}
\def\vecA{{\mathbf{A}}}
\def\vecP{{\mathbf{P}}}
\def\vecsigma{{\mathbf{\sigma}}}
\def\hatJn{{\hat{J_\vecn}}}
\def\hatJx{{\hat{J_x}}}
\def\hatJy{{\hat{J_y}}}
\def\hatJz{{\hat{J_z}}}
\def\hatj#1{\hat{J_{#1}}}
\def\hatphi{{\hat{\phi}}}
\def\hatq{{\hat{q}}}
\def\hatpi{{\hat{\pi}}}
\def\vel{\upsilon}
\def\Dint{{\mathcal{D}}}
\def\adag{{\hat{a}^\dagger}}
\def\bdag{{\hat{b}^\dagger}}
\def\cdag{{\hat{c}^\dagger}}
\def\ddag{{\hat{d}^\dagger}}
\def\hata{{\hat{a}}}
\def\hatb{{\hat{b}}}
\def\hatc{{\hat{c}}}
\def\hatd{{\hat{d}}}
\def\hatN{{\hat{N}}}
\def\hatH{{\hat{H}}}
\def\hatp{{\hat{p}}}
\def\Fup{{F^{\mu\nu}}}
\def\Fdown{{F_{\mu\nu}}}
\def\newl{\nonumber \\}
\def\SIkm{\,\mathrm{km}}
\def\SIyr{\,\mathrm{yr}}
\def\SIGyr{\,\mathrm{Gyr}}
\def\SIeV{\,\mathrm{eV}}
\def\SIkeV{\,\mathrm{keV}}
\def\SIMeV{\,\mathrm{MeV}}
\def\SIGeV{\,\mathrm{GeV}}
\def\SIcal{\,\mathrm{cal}}
\def\SIkcal{\,\mathrm{kcal}}
\def\SImol{\,\mathrm{mol}}
\def\SIm{\,\mathrm{m}}
\def\SIcm{\,\mathrm{cm}}
\def\SIfm{\,\mathrm{fm}}
\def\SImm{\,\mathrm{mm}}
\def\SInm{\,\mathrm{nm}}
\def\SImum{\,\mathrm{\mu m}}
\def\SIJ{\,\mathrm{J}}
\def\SIkJ{\,\mathrm{kJ}}
\def\SIs{\,\mathrm{s}}
\def\SIkg{\,\mathrm{kg}}
\def\SIg{\,\mathrm{g}}
\def\SIK{\,\mathrm{K}}
\def\SImmHg{\,\mathrm{mmHg}}
\def\SIPa{\,\mathrm{Pa}}
\def\vece{\mathrm{e}}
\def\bmat#1{\left(\begin{array}{#1}}
\def\emat{\end{array}\right)}
\def\bcase#1{\left\{\begin{array}{#1}}
\def\ecase{\end{array}\right.}
\def\calM{{\mathcal{M}}}
\def\calT{{\mathcal{T}}}
\def\calR{{\mathcal{R}}}
\def\barpsi{\bar{\psi}}
\def\baru{\bar{u}}
\def\barv{\bar{\upsilon}}
\def\bmini#1{\begin{minipage}{#1\textwidth}}
\def\emini{\end{minipage}}
\def\qeq{\stackrel{?}{=}}
\def\torder#1{\mathcal{T}\left(#1\right)}
\def\rorder#1{\mathcal{R}\left(#1\right)}
\def\contr#1#2{\contraction{}{#1}{}{#2}#1#2}
\def\trof#1{\mathrm{Tr}\left(#1\right)}
\def\trace{\mathrm{Tr}}
\def\comm#1{\ \ \ \left(\mathrm{used}\ #1\right)}
\def\tcomm#1{\ \ \ (\text{#1})}
\def\slp{\slashed{p}}
\def\slk{\slashed{k}}
\def\wulian{\includegraphics[width=0.18in]{emoji_wulian.jpg}}
\def\bye{\includegraphics[width=0.18in]{emoji_bye.jpg}}
\def\calp{{\mathfrak{p}}}
\def\veccalp{\mathbf{\mathfrak{p}}}
\def\atm{\,\mathrm{atm}}
\def\angstrom{\,\text{\AA}}
\def\Tthree{T_{\tiny \textcircled{3}}}
\def\pthree{p_{\tiny \textcircled{3}}}

\def\courseurl{http://zhiqihuang.top}

\def\tpage#1#2{
\begin{frame}
\bch
\begin{center}
\begin{large}
热学 \\
第#1讲 #2

\end{large}

\skiplines

黄志琦


\end{center}

\skiplines

{\small 
教材:《热学》第二版,赵凯华,罗蔚茵,高等教育出版社


课件下载
}
\courseurl 
\ech
\end{frame}
}

\def\bfr#1{
\begin{frame}
\chtitle{#1} 
\bch
}

\def\efr{
\ech 
\end{frame}
}

\title{Lesson 19 - 2$^{\rm nd}$ Law of Thermodynamics}
  \author{}
  \date{}

  \begin{document}
  \bch
\tpage{19}{热力学第二定律}

\section{Review}

\begin{frame}
\frametitle{知识点回顾:节流}
           {\blue 节流过程是等焓降压过程。}
           
  \bitem
\item{当焦耳-汤姆孙系数$\pfrac TpH>0$时,节流效应为正,温度随着压强降低而\uline{1};}
\item{当焦耳-汤姆孙系数$\pfrac TpH<0$时,节流效应为负,温度随着压强降低而\uline{1}。}
  \eitem
  这两种效应都称为{\blue 焦耳-汤姆孙效应}。
\end{frame}  

\begin{frame}
\frametitle{汤姆孙是谁}
学习焦耳-汤姆孙效应时,我有个疑问:

\skipline

焦耳这位dalao我熟,但汤姆孙我怎么没听说过?
\addfig{1.3}{tom.jpg}

\end{frame}

\begin{frame}
\frametitle{原来汤姆孙就是开尔文\bigwulian}

\bitem
\item{James Prescott Joule
  
\addfig{0.5}{Joule.jpg}
}
\item{William Thomson, 1st Baron Kelvin 

\addfig{0.5}{Kelvin.jpg}
}
\eitem
这两位可不止是提出了焦耳-汤姆孙效应那么简单……

\end{frame}



\begin{frame}
  \frametitle{第一类永动机(perpetuum mobile of the 1st kind)}
  \bmini{0.46}

  西方的永动机
  
  \lfig{1.6}{magiccycle.jpg}
  
       {\scriptsize 法国人亨内考在13世纪提出的“魔轮”}

  \emini
  \bmini{0.5}
  中国风永动机



  \addfig{0.5}{zuojiaoyoujiao.jpg}
  
  \lfig{2.5}{leftright.png}

  {\scriptsize 摘自温瑞安《惊艳一枪》的精彩片段}

  
  \emini
  
  
\end{frame}

\begin{frame}
  
  \frametitle{{\bf 焦耳确定了热功当量}之后,人们逐渐总结出了{\bf 能量守恒定律}}
  
  \addfig{1.5}{bucunzai_diyilei.jpg}
  
\end{frame}

\begin{frame}
  \frametitle{第二类永动机(perpetuum mobile of the 2nd kind)}

  假想的可以{\bf 从单一热源中吸收热量转化为机械功,并不产生其他影响的机器称为第二类永动机。}

  \addfig{2.5}{pm2nd.jpg}



 

\end{frame}


\begin{frame}

\frametitle{热力学第二定律的开尔文表述}

  \tbox{\bf 不可能从单一热源吸取热量,使之完全变为有用的功而不产生其他影响。}

  \addfig{1.8}{bucunzai_dierlei.jpg}       


\end{frame}



\begin{frame}
\addfig{1}{xdldt.jpg}

为了纪念毁梦者,我们把能量的国际标准单位叫做焦耳($\SIJ$),把温度的国际标准单位叫做开尔文($\SIK$)。
\end{frame}

\begin{frame}
\frametitle{本讲内容}
\bitem
\item{热力学第二定律的多种表述方式}
\item{克劳修斯不等式和热力学第二定律的“微分”形式}
\eitem
\end{frame}

\section{2nd Law}
\secpage{热力学第二定律}{无条件地把内能拿出来做功?或许在梦里可以吧\bye}

\begin{frame}
\frametitle{热力学第二定律的开尔文表述}

\tbox{\bf 不可能从单一热源吸取热量,使之完全变为有用的功而不产生其他影响。}

  这个表述里涉及到三个对象:作为单一热源的热库A,机械系统B,从A吸热并对B做功的热机C。
  
\bitem
\item{“单一热源”指只有一个热库(温度唯一)。}
\item{“吸热”是指净吸热。}
\item{“不产生其他影响”是指C必须恢复原状。}
  \eitem

{\scriptsize “其他”是指“A内能减少”和“对B做功”之外}。  


  
\end{frame}


\begin{frame}
\frametitle{热库的概念}

热库是指比系统大得多,且一直处于平衡态的热源。热库和系统之间的热量交换对热库来说如九牛一毛,不影响热库的温度。

\addfig{1.8}{heatsource.png}

\end{frame}




\begin{frame}
  \frametitle{思考题}

  \addfig{1}{think3.jpg}
  
  热力学第二定律的开尔文表述:{\bf 不可能从单一热源吸取热量,使之完全变为有用的功而不产生其他影响}。如果去掉“单一”或者去掉“不产生其他影响”还成立吗?如果不成立,请举出反例。
             
  
\end{frame}


\begin{frame}
\frametitle{我们再来谈谈另一位dalao}

\bcenter
克劳修斯(Rudolf Clausius)
\ecenter

\addfig{1.2}{Clausius.jpg}

\bcenter
$\uparrow$

这位dalao提出了熵
\ecenter

\end{frame}

\begin{frame}
\frametitle{热力学第二定律的克劳修斯表述}
\tbox{\bf 不可能把热量从低温物体传到高温物体而不引起其他变化。}

\end{frame}

\begin{frame}
\frametitle{补充点逻辑学常识}

\tbox{如果两个逻辑命题等价:$A=B$, 则两个命题的否命题也等价:$\mathrm{not}\ A = \mathrm{not}\ B$}

\end{frame}

\begin{frame}
\frametitle{两种表述的等价性}
{\small
\bitem
\item{
开尔文表述的否命题是: A)可以从单一热源吸热,使之完全变为有用的功而不产生其他影响。
}
\item{
克劳修斯表述的否命题是: B) 可以把热量从低温物体传到高温物体而不引起其他变化。
}
\eitem
\bitem
\item{A $\rightarrow$ B: 设A成立,则可以从高温物体吸热,转化为驱动可逆制冷机的功,从低温物体吸热传递给高温物体。其净效果为从低温热源吸热传递给高温热源。}
\item{B $\rightarrow$ A: 设B成立,则可以从低温物体传热到高温物体,然后在高温物体和低温物体之间通过一个可逆卡诺循环把热量返还给低温物体,并额外做功。净效果就是从高温物体吸热做功。}
  \eitem
}

\end{frame}

\begin{frame}
\frametitle{孤立系统的熵增大原理和热力学第二定律等价}

克劳修斯表述实际上和孤立系统的熵增大原理是一回事情。因为热量$Q$($Q>0$)从低温物体(设温度为$T_2$)向高温物体(设温度为$T_1$)传递,等价于熵减过程
$$\Delta S = Q/T_1 - Q/T_2<0$$

\end{frame}

\begin{frame}
\frametitle{卡诺定理和热力学第二定律的等价}

卡诺定理是熵增大原理推出来的,故热二律$\rightarrow$卡诺定理。

反之,若热二律不成立,存在一个热机A可以从单一热源吸热做功而不引起其他变化,则该热机的效率为$1>1-T_2/T_1$,与卡诺定理矛盾。所以卡诺定理$\rightarrow$热二律。


\end{frame}

\begin{frame}
\frametitle{总结热力学第二定律的多种表述方式}

\bitem
\item{开尔文表述:不可能从单一热源吸热转化为功而无其他影响 (即:第二类永动机不可能)}
\item{克劳修斯表述:不可能低温物体传热给高温物体而无其他影响}
\item{孤立系统的熵增大原理:孤立系统在非平衡态熵会持续增大,直到到达平衡态后熵取到极大值不再改变。}
\item{卡诺定理:所有工作于温度为$T_1$的高温热源和温度为$T_2$的低温热源之间的可逆热机效率均为$1-T_2/T_1$,不可逆热机的效率则低于这个值。}
\eitem

\end{frame}


\begin{frame}
\frametitle{热力学第二定律练习I}


\addfig{1.}{songfen.jpg}

\bitem
\item{论证摩擦生热过程是不可逆的}
\eitem


\end{frame}

\begin{frame}
\frametitle{热力学第二定律练习II}


\addfig{1.}{songfen2.jpg}

\bitem
\item{有人想利用海洋不同深度处温度不同制造一种机器,把海水的内能转化为机械功,这是否违反热力学第二定律?}
\eitem


\end{frame}


\begin{frame}
\frametitle{热力学第二定律练习III}


\addfig{1.}{songfen.jpg}

\bitem
\item{给气筒里的气体加热,使它在保持内能不变的情况下膨胀推动活塞做功,这把热完全转化为了功,是否违反热力学第二定律?}
\eitem

\end{frame}

\section{Clausius Inequality}

\secpage{克劳修斯不等式}{$$\oint \frac{\dbar Q}{T_e} \le 0 $$}



\begin{frame}
\frametitle{克劳修斯不等式}

设在一个循环中,系统依次和温度为$T_1, T_2, \ldots, T_n$的环境接触,分别吸收热量$\dbar Q_1, \dbar Q_2, \ldots, \dbar Q_n$。环境的总熵不减:
$$\sum_{i=1}^n \Delta S_i = \sum_{i=1}^n \frac{-\dbar Q_i}{T_i} \ge 0.$$
当热源温度连续变化时,
{\blue $$\oint \frac{\dbar Q}{T_e} \le 0, $$}
{\bf 等号对且仅对可逆循环成立}。

{\scriptsize 我们用符号$T_e$ 明确这里的温度为环境(environment)热源的温度。}

\end{frame}


\begin{frame}
\frametitle{热力学第二定律的微分形式}

设系统和温度为$T_e$的环境发生接触。在一个微过程中,系统的吸热量为$\dbar Q$,则环境熵变为
$$ dS_e = \frac{-\dbar Q}{T_e} $$
把系统和环境的总和看作一个孤立系统,设系统熵变为$dS$,则
$$dS +dS_e \ge 0$$
综合上面两式即有
{\blue $$\dbar Q \le T_e dS $$
等号当且仅当过程可逆时成立}。

\end{frame}



\begin{frame}
\frametitle{热二律微分形式的另一种写法}

对$pVT$系统,根据热力学第一定律
  $$dU = \dbar Q - p_edV$$
这里$p_e$为环境的压强,以及热力学第二定律
$$ \dbar Q \le T_e dS $$
即有
{\blue $$dU \le T_edS - p_edV$$
等号当且仅当过程可逆时成立}

{\scriptsize 注意不要和态函数全微分$dU = TdS - pdV$混淆。}
\end{frame}


\begin{frame}
\frametitle{附录:热力学温标的理想热机定义法}

如果一开始不引入热力学温标,而仅仅把卡诺定理表述为在两个恒温(“恒温”的定义可以由热力学第零定律得到,不需要具体温标)热源之间工作的热机具有相同效率。则我们可以用可逆热机来定义热力学温标:

取定某标准点$T_1$后,通过可逆热机效率$\eta$来定义$T_2 \equiv (1-\eta)T_1$。这种定义热力学温标的方法和测温物质无关。

\end{frame}



\ech
\end{document}

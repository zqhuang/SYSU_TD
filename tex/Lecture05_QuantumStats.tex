\documentclass[CJK]{beamer}
\usepackage{CJKutf8}
\usepackage{beamerthemesplit}
\usetheme{Malmoe}
\useoutertheme[footline=authortitle]{miniframes}
\usepackage{amsmath}
\usepackage{amssymb}
\usepackage{graphicx}
\usepackage{eufrak}
\usepackage{color}
\usepackage{slashed}
\usepackage{simplewick}
\usepackage{tikz}
\graphicspath{{../figures/}}
\def\addfig#1#2{\begin{center}\includegraphics[width=#1 in]{#2}\end{center}}
\def\blacktext#1{{\color{black}#1}}
\def\bluetext#1{{\color{blue}#1}}
\def\redtext#1{{\color{red}#1}}
\def\darkbluetext#1{{\color[rgb]{0,0.2,0.6}#1}}
\def\skybluetext#1{{\color[rgb]{0.2,0.7,1.}#1}}
\def\cyantext#1{{\color[rgb]{0.,0.5,0.5}#1}}
\def\greentext#1{{\color[rgb]{0,0.7,0.1}#1}}
\def\darkgray{\color[rgb]{0.2,0.2,0.2}}
\def\lightgray{\color[rgb]{0.6,0.6,0.6}}
\def\gray{\color[rgb]{0.4,0.4,0.4}}
\def\blue{\color{blue}}
\def\red{\color{red}}
\def\green{\color{green}}
\def\darkblue{\color[rgb]{0,0.2,0.6}}
\def\skyblue{\color[rgb]{0.2,0.7,1.}}
\def\fdeg{{^\circ \mathrm{F}}}
\def\cdeg{^\circ \mathrm{C}}
\def\be{\begin{equation}}
\def\ee{\nonumber\end{equation}}
\def\bea{\begin{eqnarray}}
\def\eea{\nonumber\end{eqnarray}}
\def\ii{{\dot{\imath}}}
\def\bch{\begin{CJK}{UTF8}{gbsn}}
\def\ech{\end{CJK}}
\def\bitem{\begin{itemize}}
\def\eitem{\end{itemize}}
\def\bcenter{\begin{center}}
\def\ecenter{\end{center}}
\def\bex{\begin{minipage}{0.3\textwidth}\includegraphics[width=1in]{jugelizi.png}\end{minipage}\begin{minipage}{0.6\textwidth}}
\def\eex{\end{minipage}}
\def\chtitle#1{\frametitle{\bch#1\ech}}
\def\skipline{{\vskip0.1in}}
\def\skiplines{{\vskip0.2in}}
\def\lagr{{\mathcal{L}}}
\def\hamil{{\mathcal{H}}}
\def\vecv{{\mathbf{v}}}
\def\vecx{{\mathbf{x}}}
\def\vecy{{\mathbf{y}}}
\def\veck{{\mathbf{k}}}
\def\vecp{{\mathbf{p}}}
\def\vecn{{\mathbf{n}}}
\def\vecA{{\mathbf{A}}}
\def\vecP{{\mathbf{P}}}
\def\vecsigma{{\mathbf{\sigma}}}
\def\hatJn{{\hat{J_\vecn}}}
\def\hatJx{{\hat{J_x}}}
\def\hatJy{{\hat{J_y}}}
\def\hatJz{{\hat{J_z}}}
\def\hatj#1{\hat{J_{#1}}}
\def\hatphi{{\hat{\phi}}}
\def\hatq{{\hat{q}}}
\def\hatpi{{\hat{\pi}}}
\def\vel{\upsilon}
\def\Dint{{\mathcal{D}}}
\def\adag{{\hat{a}^\dagger}}
\def\bdag{{\hat{b}^\dagger}}
\def\cdag{{\hat{c}^\dagger}}
\def\ddag{{\hat{d}^\dagger}}
\def\hata{{\hat{a}}}
\def\hatb{{\hat{b}}}
\def\hatc{{\hat{c}}}
\def\hatd{{\hat{d}}}
\def\hatN{{\hat{N}}}
\def\hatH{{\hat{H}}}
\def\hatp{{\hat{p}}}
\def\Fup{{F^{\mu\nu}}}
\def\Fdown{{F_{\mu\nu}}}
\def\newl{\nonumber \\}
\def\SIkm{\,\mathrm{km}}
\def\SIyr{\,\mathrm{yr}}
\def\SIGyr{\,\mathrm{Gyr}}
\def\SIeV{\,\mathrm{eV}}
\def\SIkeV{\,\mathrm{keV}}
\def\SIMeV{\,\mathrm{MeV}}
\def\SIGeV{\,\mathrm{GeV}}
\def\SIcal{\,\mathrm{cal}}
\def\SIkcal{\,\mathrm{kcal}}
\def\SImol{\,\mathrm{mol}}
\def\SIm{\,\mathrm{m}}
\def\SIcm{\,\mathrm{cm}}
\def\SIfm{\,\mathrm{fm}}
\def\SImm{\,\mathrm{mm}}
\def\SInm{\,\mathrm{nm}}
\def\SImum{\,\mathrm{\mu m}}
\def\SIJ{\,\mathrm{J}}
\def\SIkJ{\,\mathrm{kJ}}
\def\SIs{\,\mathrm{s}}
\def\SIkg{\,\mathrm{kg}}
\def\SIg{\,\mathrm{g}}
\def\SIK{\,\mathrm{K}}
\def\SImmHg{\,\mathrm{mmHg}}
\def\SIPa{\,\mathrm{Pa}}
\def\vece{\mathrm{e}}
\def\bmat#1{\left(\begin{array}{#1}}
\def\emat{\end{array}\right)}
\def\bcase#1{\left\{\begin{array}{#1}}
\def\ecase{\end{array}\right.}
\def\calM{{\mathcal{M}}}
\def\calT{{\mathcal{T}}}
\def\calR{{\mathcal{R}}}
\def\barpsi{\bar{\psi}}
\def\baru{\bar{u}}
\def\barv{\bar{\upsilon}}
\def\bmini#1{\begin{minipage}{#1\textwidth}}
\def\emini{\end{minipage}}
\def\qeq{\stackrel{?}{=}}
\def\torder#1{\mathcal{T}\left(#1\right)}
\def\rorder#1{\mathcal{R}\left(#1\right)}
\def\contr#1#2{\contraction{}{#1}{}{#2}#1#2}
\def\trof#1{\mathrm{Tr}\left(#1\right)}
\def\trace{\mathrm{Tr}}
\def\comm#1{\ \ \ \left(\mathrm{used}\ #1\right)}
\def\tcomm#1{\ \ \ (\text{#1})}
\def\slp{\slashed{p}}
\def\slk{\slashed{k}}
\def\wulian{\includegraphics[width=0.18in]{emoji_wulian.jpg}}
\def\bye{\includegraphics[width=0.18in]{emoji_bye.jpg}}
\def\calp{{\mathfrak{p}}}
\def\veccalp{\mathbf{\mathfrak{p}}}
\def\atm{\,\mathrm{atm}}
\def\angstrom{\,\text{\AA}}
\def\Tthree{T_{\tiny \textcircled{3}}}
\def\pthree{p_{\tiny \textcircled{3}}}

\def\courseurl{http://zhiqihuang.top}

\def\tpage#1#2{
\begin{frame}
\bch
\begin{center}
\begin{large}
热学 \\
第#1讲 #2

\end{large}

\skiplines

黄志琦


\end{center}

\skiplines

{\small 
教材:《热学》第二版,赵凯华,罗蔚茵,高等教育出版社


课件下载
}
\courseurl 
\ech
\end{frame}
}

\def\bfr#1{
\begin{frame}
\chtitle{#1} 
\bch
}

\def\efr{
\ech 
\end{frame}
}

\title{Lesson 05 - Quantum Statistics}
  \author{}
  \date{}
\begin{document}
\tpage{5}{量子统计}

\section{Reivew}

\begin{frame}
\chtitle{上一讲内容回顾}
\bch
\bitem
\item{能均分定理}
\item{{\bf 理想}摩尔定体热容 = $\frac{R}{2}\times$ 二次型能量的项数}
\item{概率的熵,多可能性事件的熵,多次重复多可能性事件的熵 (理解即可)}
\item{孤立系统的熵增大原理$\Rightarrow$热平衡与热力学温度的数学定义(理解即可)}
\eitem
\ech
\end{frame}


\begin{frame}
\chtitle{本讲内容}
\bch
\bitem
\item{离散能级和实际摩尔定体热容}
\item{玻色气体和费米气体}
\eitem
\ech
\end{frame}

\section{Quantum States}

\begin{frame}
\chtitle{生活中有很多“理想很丰满,现实很骨感”的例子}
\bch
\includegraphics[height=1.5in]{expectation_nn.png}\includegraphics[height=1.52in]{reality_nn.png}
\ech
\end{frame}

\begin{frame}
\chtitle{理想很丰满,现实很骨感}
\bch
上节课由能均分定理计算了{\bf 理想}摩尔定体热容。

但实际气体或固体的摩尔定体热容往往比理想值要低很多,这是为什么呢?


\addfig{2}{expectation_and_reality.jpg}

\skipline

赶紧去翻翻上节课后习题11是不是做对了\bye

\ech
\end{frame}


\begin{frame}
\chtitle{量子力学的离散能级}
\bch
\bmini{0.45}
经典图像:粒子的能量可以连续变化 

\includegraphics[width=1.5in]{slope.jpg}
\emini
\hspace{0.2in}
\bmini{0.45}
量子力学:束缚态的能级一般是离散的

 \includegraphics[width=1.4in]{steps.jpg}  
\emini
\ech
\end{frame}



\begin{frame}
\chtitle{束缚系统量子态的离散性-谐振子}
\bch
{\small 
考虑一个质量为$m$,固有圆频率为$\omega$(即弹性系数为$m\omega^2$)的谐振子
\addfig{1.5}{harmonic_oscillator.png}
现在考虑所有能量小于$E$的谐振子的态的个数。若能量小于$E$,谐振子的位移$x$和动量$\calp$满足
$$\frac{1}{2}m \omega^2 x^2 + \frac{p^2}{2m} \le E $$
写成椭圆方程标准形式:
$$\frac{x^2} {a^2} + \frac{p^2}{b^2} \le 1 $$
其中$a = \frac{1}{\omega}\sqrt{\frac{2E}{m}}$, $b = \sqrt{2mE}$。故上述方程描述的相空间椭圆的面积为$ \pi a b = \frac{2\pi E}{\omega}$。
}
\ech
\end{frame}


\begin{frame}
\chtitle{束缚系统量子态的离散性-谐振子}
\bch
{\small 
谐振子能量小于$E$的相空间的态的个数为
$$n = \frac{2\pi E}{ h \omega}$$
若把态按能量从低到高排,则第$n$个态的能量为
$$ E_n \approx \frac{ n h\omega}{2\pi}$$
量子力学的严格计算结果为
$$ E_n = \frac{ (n+1/2) h\omega}{2\pi}\ \ (n = 0,1,2,\ldots)$$
量子力学里经常用一个符号{\blue $\hbar = \frac{h}{2\pi}$},则
$$ E_n = \left(n + \frac{1}{2}\right) \hbar\omega\ \ (n = 0,1,2,\ldots)$$

}
\ech
\end{frame}


\begin{frame}
\chtitle{振动特征温度}
\bch
{\small 
显然,要使处于能量最低($n=0$)态的谐振子发生振动(即到$n>0$的态去),最起码要给予它$\hbar \omega$的能量。

\skipline

如果$kT\ll \hbar\omega$ ,则谐振子无法被热运动激发(因分子碰撞只能把$\sim kT$的动能传递给谐振子)。我们把$\Theta_{\rm osc} = \frac{\hbar\omega}{k}$叫做振动特征温度。
}
\ech
\end{frame}


\begin{frame}
\chtitle{振动特征温度的数量级}
\bch
{\small
我们考虑一个“长度”为$1\angstrom$的共价键,典型的键能为$100\SIkcal/\SImol$,即每个键结合能为
$E_b\sim \frac{4.2\times 10^5J}{6.02\times 10^{23}} \sim 0.7\times 10^{-18}J$

我们近似取弹性系数
$m\omega^2 \sim  \frac{2E_b}{(1 \angstrom)^2} \sim 140 \SIN/\SIm$,并取等效质量$m = \frac{A_1A_2}{(A_1+A_2)\times 6.02\times 10^{26}} \SIkg$,则$\omega \sim 10^{14}\SIHz \times \sqrt{\frac{8.4(A_1+A_2)}{A_1A_2}} $

故振动特征温度
$$\Theta_{\rm osc} =\frac{\hbar \omega}{k} \sim 10^3\SIK \times \sqrt{\frac{5(A_1+A_2)}{A_1A_2}} $$
该估算比较粗糙,实际材料会有一个数量级的上下浮动。
\bitem
\item{大多数双原子气体的$\Theta_{\rm osc}$一般在若干个$10^3\SIK$之上。所以{\blue 室温下大多数双原子气体的振动自由度不被激发}。}
\item{固体的$\Theta_{\rm osc}$则通常要小于室温。所以{\blue 室温下大多数固体的振动自由度是被激发的}。}
\eitem
}
\ech
\end{frame}


\begin{frame}
\chtitle{转动的离散能级}
\bch
现在考虑绕固定轴的逆时针转动,设转动惯量为$I$。广义位置为转动角$\theta$,广义动量为$\calp = I\omega$ (这里$\omega = \frac{d\theta}{dt}$为转动角速度)。

若要转动能量低于$E$,则$\frac{1}{2}I\omega^2 \le E$,或者写成$0\le \calp \le \sqrt{2I E}$。又$0\le \theta \le 2\pi$。则相空间内能量低于$E$的态个数为
$$n = \frac{2\pi \sqrt{2IE}}{h} = \frac{\sqrt{2IE}}{\hbar}$$
若把态按能量从低到高排,则第$n$个态的能量为
$$E_n \approx \frac{n^2\hbar^2}{2I}$$
严格的量子力学计算结果为
$$E_n = \frac{n(n+1)\hbar^2}{2I}$$
\ech
\end{frame}


\begin{frame}
\chtitle{转动特征温度}
\bch
同样道理,若$kT\ll \frac{\hbar^2}{2I}$,则热运动无法激发转动,我们把$\Theta_{\rm rot} = \frac{\hbar^2}{2kI}$叫做转动特征温度。

\ech
\end{frame}

\begin{frame}
\chtitle{转动特征温度的数量级}
\bch
我们考虑一个双原子气体分子,设两个原子间的距离为$r$,两个原子的原子量分别为$A_1$和$A_2$,则转动惯量为
$$I = 10^{-46} \SIkg\cdot\SIm^2 \times \frac{ A_1A_2}{6.02(A_1+A_2)} \left(\frac{r}{1\angstrom}\right)^2 $$
 特征温度为
$$\Theta_{\rm rot} =  \frac{\hbar^2}{2kI} = 24\SIK \times \frac{(A_1+A_2)}{ A_1A_2} \left(\frac{1\angstrom}{r}\right)^2$$
一般都远远低于室温(特别是当$A_1$, $A_2$都较大时)。

所以{\blue 室温下气体分子的转动自由度一般是被激发的}。
\ech
\end{frame}


\begin{frame}
\chtitle{思考题}
\bch
先考虑简单的一维情形,设分子被限制在$0<x<L$的一维盒子里。
\bitem
\item{能量小于$E$的相空间的态有多少个?}
\item{估算第$n$个能级的能量。}
\eitem
\ech
\end{frame}

\begin{frame}
\chtitle{平动特征温度}
\bch
把思考题的结果推广到三维,就可以得到边长为$L$的容器内气体的平动特征温度:
$$\Theta_{\rm trans} = \frac{h^2}{8mkL^2} = 2.4\times 10^{-18}\SIK \times \frac{1}{A}\left(\frac{1\SIm}{L}\right)^2$$
可见,在{\blue 宏观尺度的容器里,平动自由度一般都是被激发的}。
\ech
\end{frame}


\begin{frame}
\chtitle{双原子气体的摩尔定体热容随温度的变化}
\bch
如果从低温开始到高温,双原子气体的转动自由度,振动自由度会被相继激发。摩尔定体热容就从$\frac{3}{2}R$增长到$\frac{5}{2}R$再到$\frac{7}{2}R$.

\addfig{2.5}{CVmol_T.png}

请再参考教材85页 图2-14
\ech
\end{frame}

\section{Bosons and Fermions}

\begin{frame}
\bch
需要理解的内容讲完了。

\skipline

下面我要开始乱讲,听不懂不要慌,因为——不——考。

\skiplines

\includegraphics[width=1.5in]{notlove.jpg} \hspace{0.25in} \includegraphics[width=1.5in]{notexam.jpg}

\ech
\end{frame}

\begin{frame}
\chtitle{玻尔兹曼不知道的绕口令}
\bch
在玻尔兹曼的时代,量子力学还没被发现,遑论后来量子力学里的再次革命——“二次量子化”。

\skiplines

在二次量子化之后,我们把“测量到a态上有两个粒子”这句话重新进行了描述:“测量到有两个a粒子的态”。

\skipline

\wulian绕口令说得不错
\ech
\end{frame}


\begin{frame}
\chtitle{玻尔兹曼的描述}
\bch
\addfig{1.2}{E_classic.png}

测到a态有3个光子,b态有2个光子,c态有1个光子。
\ech
\end{frame}


\begin{frame}
\chtitle{二次量子化描述}
\bch
\bmini{0.41}
\addfig{1.5}{E_quantum.png}
\emini
\bmini{0.55}
测到光子场处在(3a,2b, 1c)态。

\skiplines

我们以前描述的“粒子态”是用来描述单个粒子的(可以说某粒子处在某态)。而二次量子化之后的“场态”则是描述比粒子更基本的“场”的(\bye 你们已经穿越到研究生噩梦课程《量子场论》了)。

\emini

\ech
\end{frame}


\begin{frame}
\chtitle{玻色子和费米子}
\bch

根据二次量子化的对易假设不同,可以得到两种性质不同的粒子。

\bitem
\item{可以在一个“粒子态”上放任意个粒子,我们称之为{\blue 玻色子(Boson)}。例如光子。}
\item{最多只能在一个“粒子态”上放一个粒子,我们称之为{\blue 费米子(Fermion)}。例如电子,质子。}
\eitem

{\blue 两个全同费米子不能在完全相同的态上},这称为“{\blue 泡利(Pauli)不相容原理}”。

{\small
注意粒子往往有一些内禀自由度(自旋等)。之前我们讨论能级的时候,当一个自旋向上的电子和一个自旋向下的电子处于相空间的同一个“小块”时,我们并没有区分它们所处的态(因为并不对分布律的推导造成任何影响)。现在情况有所不同了:{\blue “泡利不相容原理”仅对“包含一切自由度的态”适用}。所以我们{\blue 计算态的个数(允许填充的费米子的个数)时,需要把内禀自由度算上}。}
\ech
\end{frame}


\begin{frame}
\chtitle{费米场的态}
\bch
例子:如果没有内禀自由度,则三个态上最多放三个费米子,一共有八种放法。

\addfig{1.5}{E_fermion.png}
\ech
\end{frame}


\begin{frame}
\chtitle{稍作修改的“万能法则”}
\bch
我们现在要对“万能法则”稍作修改:场态$(n_a a, n_bb, n_c c,\ldots)$出现的概率{\bf 正比于}
$$e^{-\frac{n_a(\varepsilon_a-\mu)+n_b(\varepsilon_b-\mu)+n_c(\varepsilon_c-\mu)+\ldots}{kT}} = e^{-\frac{\varepsilon_{\rm total} - \mu_{\rm total}}{kT}}$$

\skiplines

这里的$\mu$开始变得重要了,它称为化学势。

\ech
\end{frame}


\begin{frame}
\chtitle{费米-狄拉克分布(Fermi-Dirac distribution)}
\bch
现在考虑某粒子态a,对费米子而言,粒子态a上可以有$n=0,1$个粒子,出现概率依次正比于$e^{-\frac{n(\varepsilon_a-\mu)}{kT}}$。
平均粒子数为

\bea
\bar{n}_a &=& \frac{\sum_{n=0}^1 ne^{-\frac{n(\varepsilon_a-\mu)}{kT}}}{\sum_{n=0}^1 e^{-\frac{n(\varepsilon_a-\mu)}{kT}}}  \newl
&=& \frac{ e^{-\frac{\varepsilon_a-\mu}{kT}}}{1+ e^{-\frac{\varepsilon_a-\mu}{kT}}}  \newl
&=& \frac{ 1}{e^{\frac{\varepsilon_a-\mu}{kT}}+1}  
\eea

\ech
\end{frame}


\begin{frame}
\chtitle{玻色-爱因斯坦分布(Bose-Einstein distribution)}
\bch
玻色子的情况要稍复杂,对$n$的求和要从$0$到$\infty$。
{\scriptsize
\bea
\bar{n}_a &=& \frac{\sum_{n=0}^{\infty} ne^{-\frac{n(\varepsilon_a-\mu)}{kT}}}{\sum_{n=0}^1 e^{-\frac{n(\varepsilon_a-\mu)}{kT}}}  \newl
&=& \left. \frac{\sum_{n=0}^{\infty} ne^{-nx}}{\sum_{n=0}^1 e^{-nx}}\right\vert_{x = \frac{\varepsilon_a-\mu}{kT}}  \newl
&=& \left. \frac{-\frac{d}{d x}\sum_{n=0}^{\infty} e^{-nx}}{\sum_{n=0}^1 e^{-nx}}\right\vert_{x = \frac{\varepsilon_a-\mu}{kT}}  \newl
&=& \left. \frac{-\frac{d}{d x}\frac{1}{1- e^{-x}} }{ \frac{1}{1- e^{-x}} }\right\vert_{x = \frac{\varepsilon_a-\mu}{kT}}  \newl
&=& \frac{ 1}{e^{\frac{\varepsilon_a-\mu}{kT}}-1}  
\eea
}
\ech
\end{frame}

\begin{frame}
\chtitle{玻色子$\mu$不能高于基态能量}
\bch
玻色-爱因斯坦分布
\be
\bar{n}_a = \frac{ 1}{e^{\frac{\varepsilon_a-\mu}{kT}}-1}  
\ee
我们把{\blue 能量最低的态叫做基态}。显然要使基态上的平均粒子数为非负,{\blue 玻色子的化学势$\mu$不能高于基态的能量}。
\ech
\end{frame}

\begin{frame}
\chtitle{量子效应是“群鸟争巢”效应}
\bch
显然,当$\bar{n}\ll 1$时,FD分布和BE分布都趋向于经典分布(玻尔兹曼分布)。

\skiplines

{\bf \blue 量子效应是全同粒子“争抢”态造成的}
\ech
\end{frame}


\begin{frame}
\chtitle{等等,好像有bug!}
\bch

既然经典近似要求$\bar{n} \ll 1$,那么

\addfig{1.}{think.jpg}
我们之前推算经典粒子系统的熵的玻尔兹曼诠释的时候假设了$Np_i \gg 1 (i=1,2,\ldots, m)$,是自相矛盾了吗?

\ech
\end{frame}



\begin{frame}
\chtitle{量子动量}
\bch
设粒子的内禀态数为$g$,粒子动量的数量级为$\sim \calp$,则总态数数量级为$N_s \sim \frac{4\pi}{3} g \calp^3 V h^{-3}$。
\bitem
\item{当总粒子数$N\ll N_s$时,粒子几乎不需要“争抢”态,经典近似适用。}
\item{当总粒子数$N\gtrsim N_s$时,粒子之间需要“争抢”态,必须用量子图像。}
\eitem
也就是说,存在一个典型的{\blue
“量子动量”$$\calp_q=  \left(\frac{6\pi^2 n}{g}\right)^{1/3}\hbar$$
当粒子动量大多低于$\calp_q$时,系统的量子特性就开始显现。}

\ech
\end{frame}


\begin{frame}
\chtitle{非相对论粒子体系的量子能量和量子温度}
\bch
对非相对论粒子系统,$\varepsilon = \frac{\calp^2}{2m}$,所以“量子能量”
$$\varepsilon_q =  \frac{\calp_q^2}{2m} =  \left(\frac{6\pi^2 n}{g}\right)^{2/3}\frac{\hbar^2}{2m}$$
和量子温度
$$T_q=  \frac{\varepsilon_q}{k} =  \left(\frac{6\pi^2 n}{g}\right)^{2/3}\frac{\hbar^2}{2mk}$$
{\blue 当气体温度远高于量子温度时,可以忽略量子效应。}
\ech
\end{frame}


\begin{frame}
\chtitle{极端相对论粒子体系的量子能量和量子温度}
\bch
对极端相对论粒子系统,$\varepsilon = \frac{\calp}c$,
$$\varepsilon_q =  \calp_qc =  \left(\frac{6\pi^2 n}{g}\right)^{1/3}\hbar c$$
和量子温度
$$T_q =  \frac{\varepsilon_q}{k} =  \left(\frac{6\pi^2 n}{g}\right)^{1/3}\frac{\hbar c}{k}$$
{\blue 当气体温度远高于量子温度时,可以忽略量子效应。}
\ech
\end{frame}



\begin{frame}
\chtitle{$\mu$由归一化条件决定}
\bch
费米-狄拉克(FD)分布和玻色-爱因斯坦(BE)分布
\be
\bar{n}_a = \frac{ 1}{e^{\frac{\varepsilon_a-\mu}{kT}}\pm 1}  
\ee
中的化学势要由总能量来确定。

\be
\sum_a \frac{\varepsilon_a}{e^{\frac{\varepsilon_a-\mu}{kT}}\pm 1}   = U
\ee

实际材料低温时的内能都不大容易计算,所以上式并不如想象中那么好用。下面我们来考虑几种极限情况加深理解。

\ech
\end{frame}


\begin{frame}
\chtitle{经典极限下的$\mu$}
\bch
经典极限下
$$\bar{n}_a =  nh^3 \left(\frac{1}{2\pi mkT}\right)^{3/2} e^{-\frac{\calp_a^2}{2m kT}}$$
所以
$$\mu = kT \ln\left(nh^3 \left(\frac{1}{2\pi mkT}\right)^{3/2}\right) \sim -\frac{3}{2}kT\ln \frac{T}{T_q} $$
当然,上式仅当$T\gg T_q$时适用。
\ech
\end{frame}

\begin{frame}
\chtitle{固定粒子数的玻色理想气体的$\mu$-$T$图}
\bch
\bmini{0.5}
{\small
由于理想气体分子的基态($\calp = 0$) 能量为零。对玻色理想气体而言,必须有$\mu \le 0$。又$T \rightarrow 0\SIK$时,玻色子都涌往基态,形成一个“凝聚体”,称为“{\blue 玻色-爱因斯坦凝聚}”(BE condensation)。这时要使BE分布有意义,必须有$\mu \rightarrow 0$。

当我们从$0\SIK$逐渐升高温度,达到某一个临界温度$\Theta_B$(称为{\blue BE凝聚温度})时,大多数粒子都离开了基态,BE凝聚被破坏,$\mu$也开始向经典极限转变。

于是我们大致可以画出一个$\mu$-$T$关系图:}
\emini
\bmini{0.45}
\addfig{1.7}{boson_mu_T.png}
\emini

\ech
\end{frame}


\begin{frame}
\chtitle{固定粒子数的费米理想气体的$\mu$-$T$图}
\bch
\bmini{0.5}
{\small
当$T \rightarrow 0\SIK$时,费米子因为每个态最多只能放一个粒子,就只能按能量从低到高选择不同的态依次填充。当所有费米子都用完时,填到的那个态的能量$\varepsilon_F$称为{\blue 费米能}。

即能量低于$\varepsilon_F$的态均有$\bar{n} = 1$,否则$\bar{n} = 0$ (这种类似于一个平台的分布称为费米简并)。由FD分布公式可直接看出$\mu(T\rightarrow 0\SIK) \rightarrow \varepsilon_F$。


当我们从$0\SIK$逐渐升高温度,达到某一个临界温度$\Theta_F = \varepsilon_F/k $(称为{\blue 费米简并温度})时,费米简并被破坏,$\mu$也开始向经典极限转变。

根据刚才讨论的群鸟争巢图像,显然有$\varepsilon_F \sim \varepsilon_q$,  $\Theta_F \sim T_q$。

于是我们大致可以画出一个$\mu$-$T$关系图:}
\emini
\bmini{0.45}
\addfig{1.7}{fermion_mu_T.png}
\emini

\ech
\end{frame}


\begin{frame}
\chtitle{$T\lesssim \Theta_F$时的费米简并压}
\bch


费米气体在$T\lesssim \Theta_F$,甚至$T\rightarrow 0\SIK$时也会有大量粒子处在激烈运动的非基态,形成的动理压强称为{\blue 简并压}(degeneracy pressure)。这完全是一个量子力学效应。

\skipline

取决于我们讨论的是什么系统,$\Theta_F$未必是一个很低的温度。例如金属中的电子气,其$\Theta_F$可以高达数万$\SIK$。

下面我们来进行一些估算。
\ech
\end{frame}



\begin{frame}
\chtitle{思考题}
\bch
{\small

考虑非相对论的玻色理想气体和费米理想气体。

\skipline

我们根据量子力学的物理图像知道了玻色子的BE凝聚温度$\Theta_B \sim T_q$,费米子的简并温度$\Theta_F\sim T_q$。对玻色子,计算有些麻烦,我们这里仅仅给出结果:$\Theta_B \approx 0.44 T_q$。对费米子——

\addfig{0.8}{think2.jpg}

试证明费米子的$\varepsilon_F$严格等于$\varepsilon_q$,从而$\Theta_F = T_q$。

}
\ech
\end{frame}

\begin{frame}
\chtitle{思考题:费米理想气体简并压和费米能的关系}
\bch
设理想费米气体粒子数密度为$n$。

\skiplines

回忆气体动理压强公式:$p =\frac{2}{3}n\bar\varepsilon$ (非相对论情形)及$p =\frac{1}{3}n\bar\varepsilon$ (相对论情形)。对费米理想气体,证明简并压
\bitem
\item[(1)]{在非相对论的情形为$\frac{2}{5}n\varepsilon_F \propto n^{5/3}$}
\item[(2)]{在相对论的情形为$\frac{1}{4}n\varepsilon_F \propto n^{4/3}$}
\eitem
\ech
\end{frame}


\begin{frame}
\chtitle{金属内电子的简并温度和简并压}
\bch
金属中电子的数密度和原子数密度为同一数量级$n\sim 10^{29} \SIm^{-3}$ (还记得估算金属原子平均间距为$2\angstrom$左右吗)。
电子质量$m\approx 10^{-30} \SIkg$。电子内禀自由度为$g = 2$。由此估算出
$$\varepsilon_F = \left(\frac{6\pi^2 n}{ g}\right)^{2/3} \frac{\hbar^2}{2m} \sim 10^{-18}J $$
从而简并温度
$$\Theta_F \sim 8\times 10^4 \SIK$$


所以金属的电子在室温下有简并压。我们还可以估算出简并压为
$$ p = \frac{2}{5} n \varepsilon_F \sim 4\times 10^{10}\SIPa$$


金属抗压几乎全靠的就是电子简并压。参见教材101页表2-10。

\ech
\end{frame}


\begin{frame}
\chtitle{白矮星和中子星}
\bch
\bitem
\item{白矮星是靠电子简并压抗衡引力的高密度天体。}
\item{中子星是靠中子简并压抗衡引力的高密度天体。}
\eitem

学霸们请自己刷教材102-106页,课上不再展开讨论。

\ech
\end{frame}


\begin{frame}
\bch
{\Large
下面我们讨论一下光子气体
}

\addfig{2}{godsaylight.jpg}

{\scriptsize
神说:要有光。于是白昼和黑夜分开了…

GF说:要有钱。于是你和GF分开了…}
\ech
\end{frame}

\begin{frame}
\chtitle{斯特藩-玻尔兹曼(Stefan-Boltzmann)定律}
\bch
{\small
光子的能量$\varepsilon = \calp c$,基态能量为$0$。当有一个温度为$T$的辐射源和光子气体达到热平衡时,光子的化学势达到$\mu = 0$(当粒子数可以随意变化时化学势为零,这和我们之前讨论的粒子数固定系统的化学势不一样)。

可以计算热辐射的光子气体的能量为
$$E = \int_0^\infty \frac{\calp c}{e^{\frac{\calp c}{kT}}-1} \frac{g V 4\pi\calp^2 d\calp}{h^3} $$
其中$g=2$是光子的内禀态数(自旋为$1$和$-1$),$V$是体积。

上式稍加整理即得到光子气体的能量密度
$$u = \frac{E}{V}= a T^4$$
这称为斯特藩-玻尔兹曼定律,其中斯特藩-玻尔兹曼常数$a$为
$$a = \frac{k^4}{\pi^2c^3\hbar^3}\int_0^\infty  \frac{x^3 }{e^x-1} dx$$
}
\ech
\end{frame}


\begin{frame}
\chtitle{计算斯特藩-玻尔兹曼常数}
\bch
{\scriptsize
$$a = \frac{k^4}{\pi^2c^3\hbar^3}\int_0^\infty \frac{x^3 }{e^x-1} dx$$
利用级数展开以及反复分部积分的方法,可以把右边的积分化为一个级数的和:
\bea
\int_0^\infty  \frac{x^3 }{e^x-1} dx &=& \int_0^\infty\frac{x^3e^{-x}}{1-e^{-x}}dx \newl
&=& \sum_{n=1}^\infty  \int_0^\infty  x^3 e^{-nx} dx \newl
&=& \sum_{n=1}^\infty \frac{3}{n} \int_0^\infty  x^2 e^{-nx} dx \newl
&=& \sum_{n=1}^\infty \frac{6}{n^2} \int_0^\infty  x e^{-nx} dx \newl
&=& \sum_{n=1}^\infty \frac{6}{n^3} \int_0^\infty  e^{-nx} dx \newl
&=& \sum_{n=1}^\infty \frac{6}{n^4}
\eea
}
\ech
\end{frame}


\begin{frame}
\chtitle{计算斯特藩-玻尔兹曼常数(续)}
\bch
{\scriptsize
我们知道
$$\sin x = x-\frac{x^3}{6}+\frac{x^5}{120} - \ldots$$
把$\sin x$看成一个无穷阶多项式,它的根为$0, \pm\pi, \pm 2\pi, \ldots$。故乘积表达式可以写成
$$\sin x = x\prod_{n=1}^{\infty} \left(1-\frac{x}{n\pi}\right)\left(1+\frac{x}{n\pi}\right) = x\prod_{n=1}^{\infty} \left(1-\frac{x^2}{(n\pi)^2}\right)$$
对比$x^3$的系数,得到
\begin{equation}
\sum_{n=1} \frac{1}{n^2} = \frac{\pi^2}{6} \label{eq:n2}
\end{equation}
再对比$x^5$的系数,得到
\begin{equation}
\sum_{n>m\ge 1}\frac{1}{n^2}\frac{1}{m^2} = \frac{\pi^4}{120} \label{eq:n4}
\end{equation}
\eqref{eq:n2}平方再减去\eqref{eq:n4}的两倍,得到:
$$\sum_{n=1}^\infty \frac{1}{n^4} = \frac{\pi^4}{90} $$
}
\ech
\end{frame}


\begin{frame}
\chtitle{计算斯特藩-玻尔兹曼常数(续)}
\bch
{\small 
最后我们得到
$$a = \frac{\pi^2 k^4}{15c^3\hbar^3} = 7.566 \times 10^{-16} \SIJ/(\SIm^3\SIK^4) $$
由光子能量密度
$$u = a T^4$$
还能算出压强(单位时间穿过的动量)
$$p = \frac{1}{3}u = \frac{a}{3} T^4$$
}
\ech
\end{frame}


\begin{frame}
\chtitle{斯特藩-玻尔兹曼定律的另一种表述方式}
\bch
{\small 
如果在热辐射光子气体中放一个接收板,则单位时间单位面积接收的能量为
$$j = \frac{c}{4}aT^4 = \sigma T^4 $$
{\scriptsize (我们下节课会证明:对各向同性分布,泻流速率总是平均速率的$\frac{1}{4}$。)}

$\sigma = \frac{ac}{4} = 5.67 \times 10^{-8} \SIW/(\SIm^2 \SIK^4)$有时也叫做斯特藩-玻尔兹曼常数(\wulian反正$a$和$\sigma$都不是基本常数,别太在意)。
}
\ech
\end{frame}

\begin{frame}
\chtitle{估算太阳能辐射强度}
\bch

太阳表面温度约为$6000\SIK$。我们估算太阳表面辐射强度为
$$j = \sigma T^4 = 10^8 \SIW/\SIm^2 $$
因为中心辐射强度按距离平方反比衰减,从太阳表面(半径约2.3光秒)到地球(日地距离约500光秒),衰减了约5万倍。
地球上接收到的太阳辐射强度约为$2000\SIW/\SIm^2$。

\addfig{1.2}{sunshine.jpg}


\ech
\end{frame}

\begin{frame}
\chtitle{思考题}
\bch

\addfig{1.}{think.jpg}

为什么地面上接收到的实际可用的太阳能往往比上述计算结果几乎要小一个数量级。

\ech
\end{frame}

\begin{frame}
\chtitle{第五周作业(序号接第四周)}
\bch
\bitem
\item[13]{能均分定理也适用于水中的浮游微粒,试由此求解教材习题2-24。}
\item[14]{把室温下$1\SImol$稀薄氢气保持{\bf 压强不变}缓慢地加热,使得它温度升高$1\SIK$。在这个过程中氢气吸收了多少热量?

\scriptsize
提示:注意气体会膨胀做功,跟定体加热相比需要更多的热量。利用摩尔定体热容和理想气体状态方程。
}
\item[15]{计算温度为$T$的光子气体的光子数密度$n$。

\skipline

{\scriptsize 提示 : 要用到数学常数Ap{\'e}ry's constant
$$\zeta(3) = \sum_{n=1}^\infty \frac{1}{n^3} \approx 1.2020569\ldots $$}}
\eitem
\ech
\end{frame}

\end{document}

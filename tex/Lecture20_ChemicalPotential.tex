\documentclass[CJK,13pt]{beamer}
\usepackage{CJKutf8}
\usepackage{beamerthemesplit}
\usetheme{Malmoe}
\useoutertheme[footline=authortitle]{miniframes}
\usepackage{amsmath}
\usepackage{amssymb}
\usepackage{graphicx}
\usepackage{eufrak}
\usepackage{color}
\usepackage{slashed}
\usepackage{simplewick}
\usepackage{tikz}
\graphicspath{{../figures/}}
\def\addfig#1#2{\begin{center}\includegraphics[width=#1 in]{#2}\end{center}}
\def\blacktext#1{{\color{black}#1}}
\def\bluetext#1{{\color{blue}#1}}
\def\redtext#1{{\color{red}#1}}
\def\darkbluetext#1{{\color[rgb]{0,0.2,0.6}#1}}
\def\skybluetext#1{{\color[rgb]{0.2,0.7,1.}#1}}
\def\cyantext#1{{\color[rgb]{0.,0.5,0.5}#1}}
\def\greentext#1{{\color[rgb]{0,0.7,0.1}#1}}
\def\darkgray{\color[rgb]{0.2,0.2,0.2}}
\def\lightgray{\color[rgb]{0.6,0.6,0.6}}
\def\gray{\color[rgb]{0.4,0.4,0.4}}
\def\blue{\color{blue}}
\def\red{\color{red}}
\def\green{\color{green}}
\def\darkblue{\color[rgb]{0,0.2,0.6}}
\def\skyblue{\color[rgb]{0.2,0.7,1.}}
\def\fdeg{{^\circ \mathrm{F}}}
\def\cdeg{^\circ \mathrm{C}}
\def\be{\begin{equation}}
\def\ee{\nonumber\end{equation}}
\def\bea{\begin{eqnarray}}
\def\eea{\nonumber\end{eqnarray}}
\def\ii{{\dot{\imath}}}
\def\bch{\begin{CJK}{UTF8}{gbsn}}
\def\ech{\end{CJK}}
\def\bitem{\begin{itemize}}
\def\eitem{\end{itemize}}
\def\bcenter{\begin{center}}
\def\ecenter{\end{center}}
\def\bex{\begin{minipage}{0.3\textwidth}\includegraphics[width=1in]{jugelizi.png}\end{minipage}\begin{minipage}{0.6\textwidth}}
\def\eex{\end{minipage}}
\def\chtitle#1{\frametitle{\bch#1\ech}}
\def\skipline{{\vskip0.1in}}
\def\skiplines{{\vskip0.2in}}
\def\lagr{{\mathcal{L}}}
\def\hamil{{\mathcal{H}}}
\def\vecv{{\mathbf{v}}}
\def\vecx{{\mathbf{x}}}
\def\vecy{{\mathbf{y}}}
\def\veck{{\mathbf{k}}}
\def\vecp{{\mathbf{p}}}
\def\vecn{{\mathbf{n}}}
\def\vecA{{\mathbf{A}}}
\def\vecP{{\mathbf{P}}}
\def\vecsigma{{\mathbf{\sigma}}}
\def\hatJn{{\hat{J_\vecn}}}
\def\hatJx{{\hat{J_x}}}
\def\hatJy{{\hat{J_y}}}
\def\hatJz{{\hat{J_z}}}
\def\hatj#1{\hat{J_{#1}}}
\def\hatphi{{\hat{\phi}}}
\def\hatq{{\hat{q}}}
\def\hatpi{{\hat{\pi}}}
\def\vel{\upsilon}
\def\Dint{{\mathcal{D}}}
\def\adag{{\hat{a}^\dagger}}
\def\bdag{{\hat{b}^\dagger}}
\def\cdag{{\hat{c}^\dagger}}
\def\ddag{{\hat{d}^\dagger}}
\def\hata{{\hat{a}}}
\def\hatb{{\hat{b}}}
\def\hatc{{\hat{c}}}
\def\hatd{{\hat{d}}}
\def\hatN{{\hat{N}}}
\def\hatH{{\hat{H}}}
\def\hatp{{\hat{p}}}
\def\Fup{{F^{\mu\nu}}}
\def\Fdown{{F_{\mu\nu}}}
\def\newl{\nonumber \\}
\def\SIkm{\,\mathrm{km}}
\def\SIyr{\,\mathrm{yr}}
\def\SIGyr{\,\mathrm{Gyr}}
\def\SIeV{\,\mathrm{eV}}
\def\SIkeV{\,\mathrm{keV}}
\def\SIMeV{\,\mathrm{MeV}}
\def\SIGeV{\,\mathrm{GeV}}
\def\SIcal{\,\mathrm{cal}}
\def\SIkcal{\,\mathrm{kcal}}
\def\SImol{\,\mathrm{mol}}
\def\SIm{\,\mathrm{m}}
\def\SIcm{\,\mathrm{cm}}
\def\SIfm{\,\mathrm{fm}}
\def\SImm{\,\mathrm{mm}}
\def\SInm{\,\mathrm{nm}}
\def\SImum{\,\mathrm{\mu m}}
\def\SIJ{\,\mathrm{J}}
\def\SIkJ{\,\mathrm{kJ}}
\def\SIs{\,\mathrm{s}}
\def\SIkg{\,\mathrm{kg}}
\def\SIg{\,\mathrm{g}}
\def\SIK{\,\mathrm{K}}
\def\SImmHg{\,\mathrm{mmHg}}
\def\SIPa{\,\mathrm{Pa}}
\def\vece{\mathrm{e}}
\def\bmat#1{\left(\begin{array}{#1}}
\def\emat{\end{array}\right)}
\def\bcase#1{\left\{\begin{array}{#1}}
\def\ecase{\end{array}\right.}
\def\calM{{\mathcal{M}}}
\def\calT{{\mathcal{T}}}
\def\calR{{\mathcal{R}}}
\def\barpsi{\bar{\psi}}
\def\baru{\bar{u}}
\def\barv{\bar{\upsilon}}
\def\bmini#1{\begin{minipage}{#1\textwidth}}
\def\emini{\end{minipage}}
\def\qeq{\stackrel{?}{=}}
\def\torder#1{\mathcal{T}\left(#1\right)}
\def\rorder#1{\mathcal{R}\left(#1\right)}
\def\contr#1#2{\contraction{}{#1}{}{#2}#1#2}
\def\trof#1{\mathrm{Tr}\left(#1\right)}
\def\trace{\mathrm{Tr}}
\def\comm#1{\ \ \ \left(\mathrm{used}\ #1\right)}
\def\tcomm#1{\ \ \ (\text{#1})}
\def\slp{\slashed{p}}
\def\slk{\slashed{k}}
\def\wulian{\includegraphics[width=0.18in]{emoji_wulian.jpg}}
\def\bye{\includegraphics[width=0.18in]{emoji_bye.jpg}}
\def\calp{{\mathfrak{p}}}
\def\veccalp{\mathbf{\mathfrak{p}}}
\def\atm{\,\mathrm{atm}}
\def\angstrom{\,\text{\AA}}
\def\Tthree{T_{\tiny \textcircled{3}}}
\def\pthree{p_{\tiny \textcircled{3}}}

\def\courseurl{http://zhiqihuang.top}

\def\tpage#1#2{
\begin{frame}
\bch
\begin{center}
\begin{large}
热学 \\
第#1讲 #2

\end{large}

\skiplines

黄志琦


\end{center}

\skiplines

{\small 
教材:《热学》第二版,赵凯华,罗蔚茵,高等教育出版社


课件下载
}
\courseurl 
\ech
\end{frame}
}

\def\bfr#1{
\begin{frame}
\chtitle{#1} 
\bch
}

\def\efr{
\ech 
\end{frame}
}

\title{Lesson 20 - Non-Equilibrium Process}
  \author{}
  \date{}
\begin{document}
\bch
  
\tpage{20}{化学势和平衡判据}

\begin{frame}
\frametitle{本讲内容}

\bitem
\item{学渣守恒定理}
\item{从克拉珀龙方程说起}
\item{理想气体的化学势}
\item{平衡判据}
\eitem
  
\end{frame}

\section{Study God and Study Weak}
\secpage{学渣守恒定理}{学霸化学势 = 学渣化学势}

\begin{frame}
  \frametitle{学渣守恒定理}

  一个班级,总是有1/3的学霸和2/3的学渣。

  \addfig{4}{sanrenxing.jpg}
  
\end{frame}

\begin{frame}
  \frametitle{平衡是这样达到的}


  当学霸个数过多时,那部分挤不进前1/3的学霸会逐渐心灰意冷,逐渐转化为学渣。

  \addfig{2}{meiyisi.jpg}

\end{frame}

\begin{frame}
  \frametitle{平衡是这样达到的}
  
  当学渣个数过多时,有些学渣逐渐发现稍稍学一下就能成为学霸,就逐渐转化为学霸。
  
  \addfig{1.5}{yebucuo.jpg}
\end{frame}


\begin{frame}
  \frametitle{化学平衡}

  一般来说,当条件允许$A$和$B$互相转化时,
  $$A\leftrightarrow B$$
  $A$的量和$B$的量会自发地达到一定比例的平衡。

  这种平衡既不是{\blue 热平衡}(平衡态物质各处温度均匀),也不是{\blue 力学平衡}(平衡态物质各方向压强均匀),而是一种新的平衡,叫{\blue 化学平衡}。

\end{frame}


\begin{frame}
  \frametitle{化学势}

  正如热平衡时温度$T$相同,力学平衡时压强$p$相同一样,化学平衡时各种成分之间也有一个量相同:{\blue 化学势$\mu$}。
  
  $$A\leftrightarrow B$$
  达到平衡时:$\mu_A = \mu_B.$

    \skipline

  如果人为地增加一些$A$,则$\mu_A$上升,化学平衡被破坏,化学势大的成分转化为化学势小的成分,直至达到新的平衡。

  \skipline
  
  和温度,压强一样,{\blue 化学势也是个强度量}。

\end{frame}


\begin{frame}
\frametitle{总结}

一个孤立系统要达到平衡态,必须满足如下的平衡条件:
\bitem
\item[1]{\blue 热平衡条件:系统内部温度均匀}
\item[2]{\blue 力学平衡条件:系统内部压强均匀}
\item[3]{\blue 化学平衡条件:系统内各种可以互相转化的成分之间的化学势相等}
\eitem

相平衡条件(各种物态的化学势相等) 是化学平衡条件的一种特殊情形。

\end{frame}



\section{Review Clapeyron Eq.}
\secpage{说了半天,你倒是告诉我化学势怎么算啊!}{别急,让我们从克拉珀龙方程说起……}

\begin{frame}
  \frametitle{相变}
  相变描述的是两种物态的相互转化,也可以看作化学反应的一种特殊情形:

  \bcenter
  物态$\alpha\, \leftrightarrow\,$物态$\beta$
  
  \ecenter
\end{frame}


\begin{frame}
  \frametitle{克拉珀龙方程}
  $$ \frac{dp}{dT} = \frac{\Lambda^{\rm mol}}{T(V_\beta^{\rm mol} - V_\alpha^{\rm mol})}. $$
  上面是你们已经忘得差不多了(“一干二净”的委婉说法)的{\blue 克拉珀龙方程},描述的是每$\SImol$的$\alpha$相变为$\beta$相时吸收的潜热$\Lambda^{\rm mol}$和$p$-$T$图上相变曲线在相变发生点处的斜率$dp/dT$的关系。

  \skipline

  把$\alpha$态和$\beta$态的摩尔熵分别记作$S_\alpha^{\rm mol}$和$S_\beta^{\rm mol}$,则潜热 $\Lambda$也可以看成准静态过程吸热量
  $$\Lambda^{\rm mol} = T\left(S_\beta^{\rm mol} - S_\alpha^{\rm mol}\right)$$
  
\end{frame}


\begin{frame}
  \frametitle{克拉珀龙方程}
  于是突然间克拉珀龙方程变得十分工整:
  $$ \frac{dp}{dT} = \frac{S_\beta^{\rm mol} - S_\alpha^{\rm mol} }{V_\beta^{\rm mol} - V_\alpha^{\rm mol}}. $$
  再利用$dG = -SdT + Vdp$,上式可简写为
  $$ dG^{\rm mol}_\beta = dG^{\rm mol}_\alpha.$$
  由此可见,摩尔自由焓起到了类似化学势的作用:两相平衡时,摩尔自由焓(的增量)始终保持相同。
\end{frame}


\begin{frame}
  \frametitle{克拉珀龙方程}
  实际上,化学势正是利用自由焓的增量来定义的,只不过把$\SImol$换成一个粒子:

  \skipline
  
  {\blue 在定温定压条件下,每增加某个种类的一个粒子(并保持其他种类粒子数不变)时自由焓的增量称为该种类的化学势。
  $$ \mu_i = \pfrac{G}{N_i}{p,T,N_1,N_2,\ldots,N_{i-1},N_{i+1},\ldots}. $$}
\end{frame}


\section{Ideal Gas}
\secpage{平衡态理想气体的化学势}{$$ N = e^{-\frac{\varepsilon-\mu}{kT}}. $$}

\begin{frame}
\frametitle{平衡态理想气体的化学势}

考虑经典的平衡态理想气体,在第$i$($i=1,2,\ldots$)个微观态上的平均分子数
$$ N_i = c e^{-\frac{\varepsilon_i}{kT}} $$
我们把和$i$无关(但是可能依赖于$T$)的常数$c$写成$e^{\frac{\mu}{kT}}$。
$$ N_i = e^{-\frac{\varepsilon_i-\mu}{kT}},$$
下面证明:这里的$\mu$就是前面按自由焓定义的化学势。
\end{frame}


\begin{frame}
\frametitle{平衡态理想气体的化学势}
设$N$为总分子数,$p_i=\frac{N_i}{N}$为出现在第$i$个微观态的概率,假如按照概率熵的定义:
\bmini{0.65}
\bea
 S &=& N(-\sum_i p_i\ln p_i) \newl
  &=&  -\sum_i N_i\ln\frac{N_i}{N} \newl
  &=& \sum_i N_i\ln N -\sum_i N_i\ln N_i \newl
  &=& N\ln N - \sum_i N_i\ln N_i 
  \eea
  \emini
  \bmini{0.31} {\small \blue
  按量子力学“全同分子不可分辨”的观点,左边的结果(微观状态数的对数)要减去$$\ln (N!) \approx N \ln N - N. $$(以前不在意熵的零点时没有讨论这个修正)}
  \emini
\end{frame}

\begin{frame}
  \frametitle{平衡态理想气体的化学势}
  考虑了粒子不可分辨性的修正,并遵循热力学的熵多乘个$k$的习惯,  
  \bea
  S &=& k\left( N-\sum_i N_i\ln N_i\right) \newl
  &=& k\left(N + \sum_iN_i\frac{\varepsilon_i-\mu}{kT}\right) \newl
  &=& k N+\frac{U-\mu N}{T}.
  \eea
\end{frame}

\begin{frame}
  \frametitle{平衡态理想气体的化学势}
  两边同乘以$T$,并利用$pV = nkTV =NkT$
  即可得到
  $$TS = pV + U -  N \mu .$$
  也就是说自由焓
  $$ G = N\mu$$
  定温定压条件下增加一个分子,自由焓增加$\mu$,所以$\mu$就是化学势。
\end{frame}


\begin{frame}
  \frametitle{单原子理想气体的化学势}
  如果是单原子理想气体,$\varepsilon = \frac{m\upsilon^2}{2}$,就可以简单算出化学势的显示表达式。
  
  根据平衡态理想气体的麦克斯韦分布,在相空间的体积元内的分子数为
  $$ (n\, dx\, dy\, dz) \left(\frac{m}{2\pi kT}\right)^{3/2}e^{-\frac{m\upsilon^2}{2kT}}\, d\upsilon_xd\upsilon_yd\upsilon_z,$$
  其中$n$为分子数密度。因为每个可分辨微观态体积为$h^3$,所以在这个相空间体积元内共包含了
  $$ \frac{dx\,dy\,dz\, d(m\upsilon_x)d(m\upsilon_y)d(m\upsilon_z)}{h^3}$$
  个微观态。  
\end{frame}


\begin{frame}
  \frametitle{单原子理想气体的化学势}
  两式相除,得到了每个微观态上的粒子数为:
  $$  nh^3 \left(\frac{1}{2\pi m kT}\right)^{3/2}e^{-\frac{m\upsilon^2}{2kT}}. $$
  根据前面的讨论,上式又应该等于:
  $$ e^{-\frac{m\upsilon^2/2 - \mu}{kT}}.$$
  所以得到
  $$ \mu = kT \ln\left[ nh^3 \left(\frac{1}{2\pi m kT}\right)^{3/2} \right]. $$
\end{frame}

\begin{frame}
  \frametitle{单原子理想气体的化学势}
  乘以分子个数就得到自由焓
  $$ G = NkT \ln\left[ nh^3 \left(\frac{1}{2\pi m kT}\right)^{3/2} \right]. $$
  利用$NkT = nkTV = pV = \nu RT$,上式可以写为
  $$ G = \nu R T \ln\left[ nh^3 \left(\frac{1}{2\pi m kT}\right)^{3/2} \right]. $$
\end{frame}


\begin{frame}
  \frametitle{单原子理想气体的化学势}
  再利用$G = U + pV - TS = \frac{5}{2}\nu RT - TS$,  可以得到
  \bea
  S &=& \nu R\left\{\frac{5}{2} - \ln\left[ \frac{Nh^3}{V} \left(\frac{1}{2\pi m kT}\right)^{3/2} \right] \right\} \newl
  &=& \frac{3}{2}\nu R \ln T + \nu R \ln V + \nu R\left\{\frac{5}{2} - \ln\left[ Nh^3 \left(\frac{1}{2\pi m k}\right)^{3/2} \right] \right\}
  \eea
  

  另一方面,如果从熵的克劳修斯定义出发:
  \bea S &=& \int \frac{\dbar Q}{T} = \int \frac{dU + pdV }{T} = \int \frac{C_VdT + pdV}{T} \newl
  &=& \int \frac{C_VdT}{T} + \int\frac{\nu RdV}{V} = \frac{3}{2}\nu R \ln T + \nu R \ln V + S_0
  \eea

  两个结果显然是一致的,区别是前面的计算可以给出熵的零点。
\end{frame}



\begin{frame}
  \frametitle{思考题}
  \addfig{1}{think1.jpg}
  
  利用克劳修斯的熵公式,可以计算出$C_V$不变的理想气体的熵为
  $$S = C_V\ln T + \nu R \ln V + S_0 .$$
  那么,当$T\rightarrow 0$时,熵就发散了吗?
\end{frame}



\section{Equilibrium Conditions}

\secpage{平衡判据}{平衡判据,研究的其实是非平衡态。}

\begin{frame}
\frametitle{平衡判据}

对$pVT$系统,我们经常假想一些虚变动(偏离平衡态的微扰,实际上并不会自发产生),并使用下列平衡判据:
\bitem
\item{\blue 熵判据:在内能和体积不变的情况下,对于一切可能的变动来说,平衡态的熵最大。}
\item{\blue 自由能判据:在温度和体积不变的情况下,对于一切可能的变动来说,平衡态的自由能最小。}
\item{\blue 自由焓判据:在温度和压强不变的情况下,对于一切可能的变动来说,平衡态的自由焓最小。}
\eitem
\end{frame}


\begin{frame}

  看了这些判据后的感受是:

  \skipline

\addfig{1.2}{bzdsh.jpg}

\end{frame}




\begin{frame}
\frametitle{一分为二}
为了理解上述判据,我们假想有一个孤立系统处于平衡态,并把它人为地划分为两个宏观状态完全相同的子系统$\Sigma_1$和$\Sigma_2$。

\addfig{2}{sigma12.jpg}


\end{frame}


\begin{frame}
\frametitle{理解熵判据}
 {\blue 熵判据:在内能和体积不变的情况下,对于一切可能的变动来说,平衡态的熵最大。}

 \addfig{2}{sigma12_S.jpg}

 假设$\Sigma_1$突然抽风,内能和体积分别变化了$\Delta U, \Delta V$,则要保持系统的内能和体积不变,$\Sigma_2$的内能和体积必须变化
 $-\Delta U, -\Delta V$。


 在这样的非平衡态下,经过一段时间,孤立系统必将恢复到原先的平衡态。根据孤立系统的熵增大原理,(现在这样)经过(虚)变动的系统的熵一定小于平衡态的熵。

\end{frame}


\begin{frame}
\frametitle{理解熵判据}

\addfig{1}{think3.jpg}

如果明确地写出抽风后系统的熵的二阶近似,并和平衡态的熵比较,会得到什么结果?

\end{frame}



\begin{frame}
\frametitle{理解自由能判据}
{\blue 自由能判据:在温度和体积不变的情况下,对于一切可能的虚变动来说,平衡态的自由能最小。}


假设$\Sigma_1$突然抽风,体积膨胀了$\Delta V$,则要保持体积不变,$\Sigma_2$的体积必须收缩$-\Delta V$。(注意:温度是强度量,不能通过这种方式“补偿”。所谓温度不变,就是指温度处处不变。)

\skipline


$$ \Delta F_1 = \pfrac FVT \Delta V + \frac{1}{2} \left(\frac{\partial^2 F}{\partial V^2}\right)_T \Delta V^2 $$
$$ \Delta F_2 = -\pfrac FVT \Delta V + \frac{1}{2} \left(\frac{\partial^2 F}{\partial V^2}\right)_T \Delta V^2 $$
两式相加得到
$$\Delta F =  \left(\frac{\partial^2 F}{\partial V^2}\right)_T \Delta V^2. $$

\end{frame}

\begin{frame}
\frametitle{理解自由能判据}
注意由$dF = -SdT - pdV$可以得到$\pfrac FVT = -p$,所以

$$\left(\frac{\partial^2 F}{\partial V^2}\right)_T = -\pfrac pVT > 0.  $$
(最后的不等式是说:等温条件下,压缩物质会使压强增大。)

所以 $\Delta F>0$. 也就是说,抽风(偏离平衡态)的结果总是使自由能增大,所以平衡态的自由能最小。

\end{frame}


\begin{frame}
\frametitle{思考题}
{\blue 自由焓判据:在温度和压强不变的情况下,对于一切可能的虚变动来说,平衡态的自由焓最小。}

\addfig{1}{think3.jpg}

你能以前面“一分为二”的系统为例说说这个判据的物理涵义吗?

\skipline

{\small 题示:粒子数增多会导致化学势上升,失去化学平衡。}

\end{frame}





\section{Appendix}

\begin{frame}
\frametitle{附录:写平衡判据的套路}

热学书里的“$X$判据”可以这样快速地写出来:

\bitem
\item{第一步:把$X$写成全微分:$dX + \ldots dY + \ldots dZ = 0$}
\item{第二步:背书。“在$Y$和$Z$不变的情况下,对于一切可能的变动来说,平衡态的$X$最$\ldots$”}
\item{第三步:判定第二步中的$\ldots$应该写“大”还是“小”。$dX$, $dY$, $dZ$中必然有一个能量型的量(指$dU, dH, dF, dG$中的一个),它前面系数为正则为“小”,为负则为“大”。}
\eitem

请自行拿熵判据,自由能判据,自由焓判据验证。

除了这三个判据之外,还有各种形形色色的判据(详见王竹溪的《热学》),都可以这样搞定。
\end{frame}


\ech
\end{document}

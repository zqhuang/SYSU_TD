\documentclass[CJK]{beamer}
\usepackage{CJKutf8}
\usepackage{beamerthemesplit}
\usetheme{Malmoe}
\useoutertheme[footline=authortitle]{miniframes}
\usepackage{amsmath}
\usepackage{amssymb}
\usepackage{graphicx}
\usepackage{eufrak}
\usepackage{color}
\usepackage{slashed}
\usepackage{simplewick}
\usepackage{tikz}
\graphicspath{{../figures/}}
\def\addfig#1#2{\begin{center}\includegraphics[width=#1 in]{#2}\end{center}}
\def\blacktext#1{{\color{black}#1}}
\def\bluetext#1{{\color{blue}#1}}
\def\redtext#1{{\color{red}#1}}
\def\darkbluetext#1{{\color[rgb]{0,0.2,0.6}#1}}
\def\skybluetext#1{{\color[rgb]{0.2,0.7,1.}#1}}
\def\cyantext#1{{\color[rgb]{0.,0.5,0.5}#1}}
\def\greentext#1{{\color[rgb]{0,0.7,0.1}#1}}
\def\darkgray{\color[rgb]{0.2,0.2,0.2}}
\def\lightgray{\color[rgb]{0.6,0.6,0.6}}
\def\gray{\color[rgb]{0.4,0.4,0.4}}
\def\blue{\color{blue}}
\def\red{\color{red}}
\def\green{\color{green}}
\def\darkblue{\color[rgb]{0,0.2,0.6}}
\def\skyblue{\color[rgb]{0.2,0.7,1.}}
\def\fdeg{{^\circ \mathrm{F}}}
\def\cdeg{^\circ \mathrm{C}}
\def\be{\begin{equation}}
\def\ee{\nonumber\end{equation}}
\def\bea{\begin{eqnarray}}
\def\eea{\nonumber\end{eqnarray}}
\def\ii{{\dot{\imath}}}
\def\bch{\begin{CJK}{UTF8}{gbsn}}
\def\ech{\end{CJK}}
\def\bitem{\begin{itemize}}
\def\eitem{\end{itemize}}
\def\bcenter{\begin{center}}
\def\ecenter{\end{center}}
\def\bex{\begin{minipage}{0.3\textwidth}\includegraphics[width=1in]{jugelizi.png}\end{minipage}\begin{minipage}{0.6\textwidth}}
\def\eex{\end{minipage}}
\def\chtitle#1{\frametitle{\bch#1\ech}}
\def\skipline{{\vskip0.1in}}
\def\skiplines{{\vskip0.2in}}
\def\lagr{{\mathcal{L}}}
\def\hamil{{\mathcal{H}}}
\def\vecv{{\mathbf{v}}}
\def\vecx{{\mathbf{x}}}
\def\vecy{{\mathbf{y}}}
\def\veck{{\mathbf{k}}}
\def\vecp{{\mathbf{p}}}
\def\vecn{{\mathbf{n}}}
\def\vecA{{\mathbf{A}}}
\def\vecP{{\mathbf{P}}}
\def\vecsigma{{\mathbf{\sigma}}}
\def\hatJn{{\hat{J_\vecn}}}
\def\hatJx{{\hat{J_x}}}
\def\hatJy{{\hat{J_y}}}
\def\hatJz{{\hat{J_z}}}
\def\hatj#1{\hat{J_{#1}}}
\def\hatphi{{\hat{\phi}}}
\def\hatq{{\hat{q}}}
\def\hatpi{{\hat{\pi}}}
\def\vel{\upsilon}
\def\Dint{{\mathcal{D}}}
\def\adag{{\hat{a}^\dagger}}
\def\bdag{{\hat{b}^\dagger}}
\def\cdag{{\hat{c}^\dagger}}
\def\ddag{{\hat{d}^\dagger}}
\def\hata{{\hat{a}}}
\def\hatb{{\hat{b}}}
\def\hatc{{\hat{c}}}
\def\hatd{{\hat{d}}}
\def\hatN{{\hat{N}}}
\def\hatH{{\hat{H}}}
\def\hatp{{\hat{p}}}
\def\Fup{{F^{\mu\nu}}}
\def\Fdown{{F_{\mu\nu}}}
\def\newl{\nonumber \\}
\def\SIkm{\,\mathrm{km}}
\def\SIyr{\,\mathrm{yr}}
\def\SIGyr{\,\mathrm{Gyr}}
\def\SIeV{\,\mathrm{eV}}
\def\SIkeV{\,\mathrm{keV}}
\def\SIMeV{\,\mathrm{MeV}}
\def\SIGeV{\,\mathrm{GeV}}
\def\SIcal{\,\mathrm{cal}}
\def\SIkcal{\,\mathrm{kcal}}
\def\SImol{\,\mathrm{mol}}
\def\SIm{\,\mathrm{m}}
\def\SIcm{\,\mathrm{cm}}
\def\SIfm{\,\mathrm{fm}}
\def\SImm{\,\mathrm{mm}}
\def\SInm{\,\mathrm{nm}}
\def\SImum{\,\mathrm{\mu m}}
\def\SIJ{\,\mathrm{J}}
\def\SIkJ{\,\mathrm{kJ}}
\def\SIs{\,\mathrm{s}}
\def\SIkg{\,\mathrm{kg}}
\def\SIg{\,\mathrm{g}}
\def\SIK{\,\mathrm{K}}
\def\SImmHg{\,\mathrm{mmHg}}
\def\SIPa{\,\mathrm{Pa}}
\def\vece{\mathrm{e}}
\def\bmat#1{\left(\begin{array}{#1}}
\def\emat{\end{array}\right)}
\def\bcase#1{\left\{\begin{array}{#1}}
\def\ecase{\end{array}\right.}
\def\calM{{\mathcal{M}}}
\def\calT{{\mathcal{T}}}
\def\calR{{\mathcal{R}}}
\def\barpsi{\bar{\psi}}
\def\baru{\bar{u}}
\def\barv{\bar{\upsilon}}
\def\bmini#1{\begin{minipage}{#1\textwidth}}
\def\emini{\end{minipage}}
\def\qeq{\stackrel{?}{=}}
\def\torder#1{\mathcal{T}\left(#1\right)}
\def\rorder#1{\mathcal{R}\left(#1\right)}
\def\contr#1#2{\contraction{}{#1}{}{#2}#1#2}
\def\trof#1{\mathrm{Tr}\left(#1\right)}
\def\trace{\mathrm{Tr}}
\def\comm#1{\ \ \ \left(\mathrm{used}\ #1\right)}
\def\tcomm#1{\ \ \ (\text{#1})}
\def\slp{\slashed{p}}
\def\slk{\slashed{k}}
\def\wulian{\includegraphics[width=0.18in]{emoji_wulian.jpg}}
\def\bye{\includegraphics[width=0.18in]{emoji_bye.jpg}}
\def\calp{{\mathfrak{p}}}
\def\veccalp{\mathbf{\mathfrak{p}}}
\def\atm{\,\mathrm{atm}}
\def\angstrom{\,\text{\AA}}
\def\Tthree{T_{\tiny \textcircled{3}}}
\def\pthree{p_{\tiny \textcircled{3}}}

\def\courseurl{http://zhiqihuang.top}

\def\tpage#1#2{
\begin{frame}
\bch
\begin{center}
\begin{large}
热学 \\
第#1讲 #2

\end{large}

\skiplines

黄志琦


\end{center}

\skiplines

{\small 
教材:《热学》第二版,赵凯华,罗蔚茵,高等教育出版社


课件下载
}
\courseurl 
\ech
\end{frame}
}

\def\bfr#1{
\begin{frame}
\chtitle{#1} 
\bch
}

\def\efr{
\ech 
\end{frame}
}

\title{Lesson 05 - Maxwell Distribution}
  \author{}
  \date{}
\begin{document}
\tpage{5}{麦克斯韦分布}

\section{Reivew}

\begin{frame}
\chtitle{上讲内容回顾}
\bch
{\large
\bitem
\item{概率密度函数}
\item{高斯分布}
\eitem}
\ech
\end{frame}


\begin{frame}
\chtitle{思考题}
\bch
    {\large

      计算积分
      $$\int_0^\infty r^8e^{-r^2/2}dr. $$
      }
\ech
\end{frame}

\begin{frame}
\chtitle{本讲内容}
\bch
\bitem
\item{径向概率密度函数}
\item{麦克斯韦分布}
\eitem
\ech
\end{frame}

\section{Radial PDF}


\secpage{径向概率密度函数}{$$F(r) = \frac{G_1}{G_{n-1}} r^{n-1} e^{-r^2/2}.$$}

\begin{frame}
\bch
{\large
  我们先来讨论一下一个专题:{\bf 径向概率密度函数}。

  \skiplines
  
  简单讲,它就是{\bf 随机点到原点距离$r$的概率密度函数$F(r)$}。

  \skiplines

  显然对$r < 0$, $F(r)=0$。这个简单事实我们之后将不再重复,默认只讨论$r\ge 0$的情形。 
}
\ech
\end{frame}


\begin{frame}
\chtitle{径向概率密度函数:一维情形}
\bch
\addfig{0.6}{think1.jpg}
{\large 
  如果$x$服从标准正态分布
  $$f(x) = \frac{1}{\sqrt{2\pi}}e^{-\frac{x^2}{2}},$$
  定义新的随机变量$r=|x|$(即$r$是“到原点的距离”),那么$r$的概率密度函数$F(r)$是怎样的?}
\ech
\end{frame}


\begin{frame}
\chtitle{解答}
\bch
根据上节课所讲的随机变量替换法:
{\large
  $$F(r) |dr| = \sum_{x\rightarrow r} f(x) |dx| = f(r)|dr| + f(-r)|dr|,$$
  所以
  $$F(r) = f(r)+f(-r) = \sqrt{\frac{2}{\pi}}e^{-\frac{r^2}{2}}.$$
}
\ech
\end{frame}



\begin{frame}
\chtitle{讨论}
\bch
{\large
回顾一开始写出的等式:
$$F(r) |dr| = \sum_{x\rightarrow r} f(x) |dx| $$
等式左边的物理意义是:根据概率密度函数$F(r)$的定义计算随机点$x$落在离原点距离$r$和$r+dr$之间的概率。

\skipline

等式右边的物理意义是:根据概率密度函数$f(x)$的定义计算随机点$x$落在离原点距离$r$和
$r+dr$之间的概率 (通过列举所有满足条件的$x$得到结果)。

\skipline
因为同一个统计事件的概率是唯一的,所以左边等于右边。

\skipline

 这种计算方法可以推广到任意维: 
}
\ech
\end{frame}


\begin{frame}
\chtitle{思考题}
\bch
\addfig{0.6}{think1.jpg}
{\large
如果$x$, $y$分别都服从标准正态分布,且$x,y$之间没有统计关联,定义新的随机变量$r=\sqrt{x^2+y^2}$,那么$r$的概率密度函数是怎样的?
}
\ech
\end{frame}


\begin{frame}
\chtitle{解答}
    \bch
    {\large
      $x$的概率密度函数为$\frac{1}{\sqrt{2\pi}}e^{-x^2/2}$; $y$的概率密度函数为$\frac{1}{\sqrt{2\pi}}e^{-y^2/2}$。

      \skipline
      
      根据上节课所讲,{\bf 独立}随机变量的概率密度可以直接进行反投影,得到二元随机变量$(x,y)$的概率密度函数:

      $$ f(x,y) = \frac{1}{2\pi}e^{-(x^2+y^2)/2} = \frac{1}{2\pi}e^{-r^2/2}.$$

      {\bf 注意:这里为了偷懒把结果形式上写成了$r$的函数,但它是$(x,y)$的概率密度函数,不是$r$的概率密度函数。。}}
\ech
\end{frame}

\begin{frame}
\chtitle{解答(续)}
\bch
\bmini{0.5}
\lfig{2}{ringprob.png}
\emini
\bmini{0.45}
{\large
  如图,考虑距离原点为$r$,宽度为$dr\ll r$的一个圆环形区域。在这个区域内,二元随机变量$(x,y)$的概率密度处处都是$f(x,y)=\frac{1}{2\pi}e^{-r^2/2}$,所以$(x,y)$落在圆环内的概率为}
\emini
    {
     \large
  $$ f(x,y) (2\pi r dr) = \frac{1}{2\pi}e^{-r^2/2} (2\pi r dr) = re^{-r^2/2}dr. $$
  

  另一方面,根据$r$的概率密度函数$F(r)$的定义,上述事件的概率又等于 $F(r)dr$,根据统计事件概率唯一性得到
  $$F(r) = r e^{-r^2/2}. $$

  (注意:只要$f(x,y)$是归一化的,那么推导出来的$F(r)$的归一化是自然满足的,请自行验证。)
}
\ech
\end{frame}


\begin{frame}
\chtitle{思考题}
\bch
\addfig{0.6}{think1.jpg}
{\large
如果$x$, $y$, $z$分别都服从标准正态分布,且$x,y,z$之间没有统计关联,定义新的随机变量$r=\sqrt{x^2+y^2+z^2}$,那么$r$的概率密度函数是怎样的?
}
\ech
\end{frame}


\begin{frame}
\chtitle{解答}
    \bch
        {\large
          同样地,三元随机变量$(x,y,z)$的概率密度函数为

          $$ f(x,y,z) = \frac{1}{(2\pi)^{3/2}}e^{-r^2/2} $$

          考虑$(x,y,z)$落在距离原点为$r$,厚度$dr\ll r$的一个薄球壳内的概率,就有
          $$ \frac{1}{(2\pi)^{3/2}}e^{-r^2/2} (4\pi r^2dr) = F(r) dr $$
          即
          $$ F(r)=\sqrt{\frac{2}{\pi}} r^2 e^{-r^2/2}. $$

}
\ech
\end{frame}


\begin{frame}
\chtitle{一般维数}
\bch
{\large
在$n$维空间里,如果$n$维球的体积为$V_n(r)=S_n r^n$ ($S_n$为常数,例如$S_1 = 2$, $S_2 = \pi$, $S_3 = \frac{4\pi}{3}$\ldots),则$n$维“薄球壳”的体积为
$$ dV_n = n S_n r^{n-1} dr $$
用同样的方法可以得到
$$F(r) = \frac{nS_n}{(2\pi)^{n/2}} r^{n-1} e^{-r^2/2} \equiv C_n r^{n-1} e^{-r^2/2}. $$
其中$C_n$定义为$\frac{nS_n}{(2\pi)^{n/2}}$。

\lfig{0.5}{think1.jpg}对一般的$n$,你有什么办法可以求出$C_n$的确切值吗?}
\ech
\end{frame}


\begin{frame}
\chtitle{归一化}
\bch
{\large
  利用归一化条件
  $$C_n \int_0^\infty r^{n-1}e^{-r^2/2}dr = 1.$$
  以及上节课学习过的
  $$G_{n-1} = \int_{-\infty}^\infty \frac{1}{\sqrt{2\pi}}|x|^{n-1}e^{-x^2/2}dx =  \sqrt{\frac{2}{\pi}} \int_0^\infty x^{n-1}e^{-x^2/2}dx $$
  可以得到归一化系数
  \tbox{$$C_n = \sqrt{\frac{2}{\pi}}\frac{1}{G_{n-1}}. $$}
  }
\ech
\end{frame}

\begin{frame}
\chtitle{归一化}
\bch

      上面的结果也可以写成
        \tbox{$$C_n = \frac{G_1}{G_{n-1}}. $$}
      前几个$C_n$是:
      \bea
      C_1 &=& \sqrt{\frac{2}{\pi}} \newl
      C_2 &=& 1 \newl
      C_3 &=& \sqrt{\frac{2}{\pi}} \newl
      C_4 &=& \frac{1}{2} \nonumber
      \eea

      (我们还顺便得到了四维空间的球体积等于$\frac{\pi^2}{2}r^4$。)
\ech
\end{frame}


\begin{frame}
\chtitle{总结}
\bch
在$n$维空间的随机点,如果每个维度的分量均独立地满足标准正态分布,则径向概率密度函数为
\tbox{$$ F(r) = \frac{G_1}{G_{n-1}} r^{n-1} e^{-r^2/2}.$$}

当然,如果你不喜欢死记硬背,对比较小的$n$,直接进行推导也花不了多少功夫。
\ech
\end{frame}



\section{Maxwell Distribution}

\secpage{麦克斯韦分布}{以$\sqrt{\frac{kT}{m}}$为单位,每个维度上的速度分量满足标准正态分布}



\begin{frame}
\chtitle{相空间的概率密度函数(非相对论情形)}
\bch
{\large 


  根据我们第一讲给的“万能法则”,达到热平衡时,分子处在一个能量为$\varepsilon$的微观态的概率正比于$e^{-\frac{\varepsilon}{kT}}$。

  {\bf 因为相空间里微观态是均匀分布的,所以我们也可以说} 分子在相空间坐标$(x,y,z,\upsilon_x,\upsilon_y,\upsilon_z)$附近的(6维)体积元内出现的概率正比于
$$ e^{-\frac{\varepsilon}{kT}} dx dy dz d\upsilon_x d\upsilon_y d\upsilon_z$$
  这里的$\varepsilon$确切地说是$x,y,z,\upsilon_x,\upsilon_y,\upsilon_z$的一个函数$\varepsilon(x,y,z,\upsilon_x,\upsilon_y, \upsilon_z)$,本质上我们写出的是一个相空间的概率密度函数:
  $$ p(x,y,z,\upsilon_x,\upsilon_y, \upsilon_z) = C \exp\left(-\frac{\varepsilon(x,y,z,\upsilon_x,\upsilon_y, \upsilon_z)}{kT}\right).$$
$C$是待定的归一化常数。
}
\ech
\end{frame}


\begin{frame}
\chtitle{麦克斯韦分布}
\bch
    {\large
      当只需要考虑动能时,
      $$\varepsilon = \frac{m}{2}(\upsilon_x^2+\upsilon_y^2 + \upsilon_z^2)$$
      这时$p(x,y,z,\upsilon_x,\upsilon_y, \upsilon_z)$只依赖于速度$(\upsilon_x,\upsilon_y, \upsilon_z)$。因此我们只需要考虑投影到3维的速度的概率密度函数
      $$f(\upsilon_x,\upsilon_y, \upsilon_z) = C e^{-\frac{m}{2kT}(\upsilon_x^2+\upsilon_y^2 + \upsilon_z^2)}.$$
      为了增强可读性,我们还是用符号$C$来表示待定的归一化常数,但要注意它和之前相空间概率密度函数待定的归一化常数不一样。
      
      速度的概率密度函数称为{\blue 速度分布函数}。
      }
\ech
\end{frame}

\begin{frame}
\chtitle{麦克斯韦分布(续)}
\bch
{\large
速度分布函数$f(\upsilon_x,\upsilon_y, \upsilon_z)$可以继续进行投影,分别得到$\upsilon_x$, $\upsilon_y$, $\upsilon_z$的概率密度函数:

$$ f_{1D}(\upsilon_x) \propto e^{-\frac{m\upsilon_x^2}{2kT}}$$
$$ f_{1D}(\upsilon_y)  \propto  e^{-\frac{m\upsilon_y^2}{2kT}}$$
$$ f_{1D}(\upsilon_z)  \propto e^{-\frac{m\upsilon_z^2}{2kT}}$$
事实上,{\blue $\upsilon_x,\upsilon_y, \upsilon_z$这三个随机变量是独立的},可以把它们的概率密度函数进行反投影,还原到速度分布函数$f(\upsilon_x,\upsilon_y, \upsilon_z)$。
}
\ech
\end{frame}


\begin{frame}
\chtitle{麦克斯韦分布(续)}
\bch
    {\large
      上述讨论给出了一个非常重要的结论:
      \tbox{当只考虑动能时,理想气体分子的速度任何一个正交分量都独立地服从均值为零,标准差为$\sqrt{\frac{kT}{m}}$的正态分布。}
      这种速度分布称为{\blue 麦克斯韦分布 (Maxwell Distribution)},确切地写出来是
      {\blue $$f(\upsilon_x,\upsilon_y, \upsilon_z) = \left(\frac{m}{2\pi kT}\right)^{3/2} e^{-\frac{m}{2kT}(\upsilon_x^2+\upsilon_y^2 + \upsilon_z^2)}.$$}
      注意我们已经利用了正态分布的知识把归一化系数$C$确定了。
}
\ech
\end{frame}


\begin{frame}
\chtitle{快捷计算的办法}
\bch
    {\large
      当进行速度的各种统计计算时,有效的思考方式是换掉速度的单位:
      \tbox{当只考虑动能时,以$\upsilon_* = \sqrt{\frac{kT}{m}}$为速度单位,理想气体分子的速度任何一个正交分量都独立地服从标准正态分布。}

     当然,{\bf 完成计算时,要根据量纲把$\upsilon_*$的若干次方写回答案里去。}
}
\ech
\end{frame}


\begin{frame}
\chtitle{思考题}
\bch
\addfig{1}{think1.jpg}

对麦克斯韦分布计算$\upsilon_x^4+\upsilon_y^4+\upsilon_z^4$的期待值。
\ech
\end{frame}


\begin{frame}
\chtitle{速率分布函数}
\bch
速率$\upsilon$的概率密度函数称为{\blue 速率分布函数}。它可以看成“速度空间里的径向概率密度函数”。当以$\upsilon_* = \sqrt{\frac{kT}{m}}$为单位时,我们直接写出:
$$F(\upsilon) =\sqrt{\frac{2}{\pi}}\upsilon^2 e^{-\frac{1}{2}\upsilon^2}. $$
因为$F(\upsilon)d\upsilon$是无量纲的概率元,所以$F(\upsilon)$的量纲是$\upsilon^{-1}$,把上述结果写回普通单位制就是
$$ \frac{F(\upsilon)}{\upsilon_*^{-1}} = \sqrt{\frac{2}{\pi}}\left(\frac{\upsilon}{\upsilon_*}\right)^2 e^{-\frac{1}{2}\left(\frac{\upsilon}{\upsilon_*}\right)^2} .$$
化简得到{\blue 麦克斯韦分布的速率分布函数
$$F(\upsilon) = \sqrt{\frac{2}{\pi}}\left(\frac{m}{kT}\right)^{3/2}\upsilon^2e^{-\frac{m\upsilon^2}{2kT}} .$$}

\ech
\end{frame}

\begin{frame}
\chtitle{平均速率}
\bch
{\large
以$\upsilon_*$为单位,平均速率:
$$\overline{\upsilon} = \int_0^\infty \upsilon F(\upsilon)d\upsilon = \int_0^\infty \sqrt{\frac{2}{\pi}}\upsilon^3 e^{-\frac{1}{2}\upsilon^2} = G_3 =2G_1 = \sqrt{\frac{8}{\pi}} $$
补回量纲,得到
{\blue $$\overline{\upsilon} = \sqrt{\frac{8kT}{\pi m}}. $$}}
\ech
\end{frame}


\begin{frame}
\chtitle{速率的平方平均和方均根速率}
\bch
以$\upsilon_*$为单位,速率的平方平均:
$$\overline{\upsilon^2} = \int_0^\infty \upsilon^2 F(\upsilon)d\upsilon = \int_0^\infty \sqrt{\frac{2}{\pi}}\upsilon^4 e^{-\frac{1}{2}\upsilon^2} = G_4 =3G_2 = 3. $$
补回量纲,得到
 $$\overline{\upsilon^2} = \frac{3kT}{ m}. $$
它的平方根称为{\blue 方均根速率$\upsilon_{\rm rms}$
$$\upsilon_{\rm rms} \equiv \sqrt{\overline{\upsilon^2}} = \sqrt{ \frac{3kT}{ m}} .$$}
\ech
\end{frame}

\begin{frame}
\chtitle{最概然速率}
\bch
{\large 让速率分布函数取到最大值的速率称为{\blue 最概然速率$\upsilon_{\max}$}。
以$\upsilon_*$为单位,对$F(\upsilon)$进行求导并令之等于零:
$$F'(\upsilon_{\max}) = \sqrt{\frac{2}{\pi}}(2\upsilon_{\max} - \upsilon_{\max}^3)e^{-\upsilon_{\max}^2/2} = 0. $$
解得$\upsilon_{\max} = \sqrt{2}$。

补回量纲,得到
{\blue $$\upsilon_{\max} = \sqrt{\frac{2kT}{ m}}. $$}
}
\ech
\end{frame}

\begin{frame}
\chtitle{思考题}
\bch

\addfig{1}{think3.jpg}
对一维和二维的平衡态理想气体,分别计算速度分布函数,速率分布函数,平均速率,方均根速率和最概然速率。
\ech
\end{frame}

\end{document}

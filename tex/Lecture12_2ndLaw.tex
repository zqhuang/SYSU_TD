\documentclass[CJK]{beamer}
\usepackage{CJKutf8}
\usepackage{beamerthemesplit}
\usetheme{Malmoe}
\useoutertheme[footline=authortitle]{miniframes}
\usepackage{amsmath}
\usepackage{amssymb}
\usepackage{graphicx}
\usepackage{eufrak}
\usepackage{color}
\usepackage{slashed}
\usepackage{simplewick}
\usepackage{tikz}
\graphicspath{{../figures/}}
\def\addfig#1#2{\begin{center}\includegraphics[width=#1 in]{#2}\end{center}}
\def\blacktext#1{{\color{black}#1}}
\def\bluetext#1{{\color{blue}#1}}
\def\redtext#1{{\color{red}#1}}
\def\darkbluetext#1{{\color[rgb]{0,0.2,0.6}#1}}
\def\skybluetext#1{{\color[rgb]{0.2,0.7,1.}#1}}
\def\cyantext#1{{\color[rgb]{0.,0.5,0.5}#1}}
\def\greentext#1{{\color[rgb]{0,0.7,0.1}#1}}
\def\darkgray{\color[rgb]{0.2,0.2,0.2}}
\def\lightgray{\color[rgb]{0.6,0.6,0.6}}
\def\gray{\color[rgb]{0.4,0.4,0.4}}
\def\blue{\color{blue}}
\def\red{\color{red}}
\def\green{\color{green}}
\def\darkblue{\color[rgb]{0,0.2,0.6}}
\def\skyblue{\color[rgb]{0.2,0.7,1.}}
\def\fdeg{{^\circ \mathrm{F}}}
\def\cdeg{^\circ \mathrm{C}}
\def\be{\begin{equation}}
\def\ee{\nonumber\end{equation}}
\def\bea{\begin{eqnarray}}
\def\eea{\nonumber\end{eqnarray}}
\def\ii{{\dot{\imath}}}
\def\bch{\begin{CJK}{UTF8}{gbsn}}
\def\ech{\end{CJK}}
\def\bitem{\begin{itemize}}
\def\eitem{\end{itemize}}
\def\bcenter{\begin{center}}
\def\ecenter{\end{center}}
\def\bex{\begin{minipage}{0.3\textwidth}\includegraphics[width=1in]{jugelizi.png}\end{minipage}\begin{minipage}{0.6\textwidth}}
\def\eex{\end{minipage}}
\def\chtitle#1{\frametitle{\bch#1\ech}}
\def\skipline{{\vskip0.1in}}
\def\skiplines{{\vskip0.2in}}
\def\lagr{{\mathcal{L}}}
\def\hamil{{\mathcal{H}}}
\def\vecv{{\mathbf{v}}}
\def\vecx{{\mathbf{x}}}
\def\vecy{{\mathbf{y}}}
\def\veck{{\mathbf{k}}}
\def\vecp{{\mathbf{p}}}
\def\vecn{{\mathbf{n}}}
\def\vecA{{\mathbf{A}}}
\def\vecP{{\mathbf{P}}}
\def\vecsigma{{\mathbf{\sigma}}}
\def\hatJn{{\hat{J_\vecn}}}
\def\hatJx{{\hat{J_x}}}
\def\hatJy{{\hat{J_y}}}
\def\hatJz{{\hat{J_z}}}
\def\hatj#1{\hat{J_{#1}}}
\def\hatphi{{\hat{\phi}}}
\def\hatq{{\hat{q}}}
\def\hatpi{{\hat{\pi}}}
\def\vel{\upsilon}
\def\Dint{{\mathcal{D}}}
\def\adag{{\hat{a}^\dagger}}
\def\bdag{{\hat{b}^\dagger}}
\def\cdag{{\hat{c}^\dagger}}
\def\ddag{{\hat{d}^\dagger}}
\def\hata{{\hat{a}}}
\def\hatb{{\hat{b}}}
\def\hatc{{\hat{c}}}
\def\hatd{{\hat{d}}}
\def\hatN{{\hat{N}}}
\def\hatH{{\hat{H}}}
\def\hatp{{\hat{p}}}
\def\Fup{{F^{\mu\nu}}}
\def\Fdown{{F_{\mu\nu}}}
\def\newl{\nonumber \\}
\def\SIkm{\,\mathrm{km}}
\def\SIyr{\,\mathrm{yr}}
\def\SIGyr{\,\mathrm{Gyr}}
\def\SIeV{\,\mathrm{eV}}
\def\SIkeV{\,\mathrm{keV}}
\def\SIMeV{\,\mathrm{MeV}}
\def\SIGeV{\,\mathrm{GeV}}
\def\SIcal{\,\mathrm{cal}}
\def\SIkcal{\,\mathrm{kcal}}
\def\SImol{\,\mathrm{mol}}
\def\SIm{\,\mathrm{m}}
\def\SIcm{\,\mathrm{cm}}
\def\SIfm{\,\mathrm{fm}}
\def\SImm{\,\mathrm{mm}}
\def\SInm{\,\mathrm{nm}}
\def\SImum{\,\mathrm{\mu m}}
\def\SIJ{\,\mathrm{J}}
\def\SIkJ{\,\mathrm{kJ}}
\def\SIs{\,\mathrm{s}}
\def\SIkg{\,\mathrm{kg}}
\def\SIg{\,\mathrm{g}}
\def\SIK{\,\mathrm{K}}
\def\SImmHg{\,\mathrm{mmHg}}
\def\SIPa{\,\mathrm{Pa}}
\def\vece{\mathrm{e}}
\def\bmat#1{\left(\begin{array}{#1}}
\def\emat{\end{array}\right)}
\def\bcase#1{\left\{\begin{array}{#1}}
\def\ecase{\end{array}\right.}
\def\calM{{\mathcal{M}}}
\def\calT{{\mathcal{T}}}
\def\calR{{\mathcal{R}}}
\def\barpsi{\bar{\psi}}
\def\baru{\bar{u}}
\def\barv{\bar{\upsilon}}
\def\bmini#1{\begin{minipage}{#1\textwidth}}
\def\emini{\end{minipage}}
\def\qeq{\stackrel{?}{=}}
\def\torder#1{\mathcal{T}\left(#1\right)}
\def\rorder#1{\mathcal{R}\left(#1\right)}
\def\contr#1#2{\contraction{}{#1}{}{#2}#1#2}
\def\trof#1{\mathrm{Tr}\left(#1\right)}
\def\trace{\mathrm{Tr}}
\def\comm#1{\ \ \ \left(\mathrm{used}\ #1\right)}
\def\tcomm#1{\ \ \ (\text{#1})}
\def\slp{\slashed{p}}
\def\slk{\slashed{k}}
\def\wulian{\includegraphics[width=0.18in]{emoji_wulian.jpg}}
\def\bye{\includegraphics[width=0.18in]{emoji_bye.jpg}}
\def\calp{{\mathfrak{p}}}
\def\veccalp{\mathbf{\mathfrak{p}}}
\def\atm{\,\mathrm{atm}}
\def\angstrom{\,\text{\AA}}
\def\Tthree{T_{\tiny \textcircled{3}}}
\def\pthree{p_{\tiny \textcircled{3}}}

\def\courseurl{http://zhiqihuang.top}

\def\tpage#1#2{
\begin{frame}
\bch
\begin{center}
\begin{large}
热学 \\
第#1讲 #2

\end{large}

\skiplines

黄志琦


\end{center}

\skiplines

{\small 
教材:《热学》第二版,赵凯华,罗蔚茵,高等教育出版社


课件下载
}
\courseurl 
\ech
\end{frame}
}

\def\bfr#1{
\begin{frame}
\chtitle{#1} 
\bch
}

\def\efr{
\ech 
\end{frame}
}

\title{Lesson 12 - 2$^{\rm nd}$ Law of Thermodynamics}
  \author{}
  \date{}
\begin{document}
\tpage{12}{热力学第二定律}

\section{Review}

\begin{frame}
\chtitle{上讲内容回顾}
\bch 
\bitem
\item{孤立系统的熵增大原理}
\item{循环过程:可逆循环与不可逆循环;正循环与逆循环;}
\item{卡诺定理$\eta \le 1 -\frac{T_2}{T_1}$ (对制冷机则$\varepsilon \le \frac{T_2}{T_1-T_2}$)}
\item{理想气体循环效率的计算}
\eitem
\ech
\end{frame}

\begin{frame}
\chtitle{汤姆孙是谁}
\bch
学习焦耳-汤姆孙效应时,我有个疑问:

\skiplines

焦耳这位dalao我熟,但{\large \bf 汤姆孙}我怎么没听说过?
\addfig{1.3}{tom.jpg}

\ech
\end{frame}

\begin{frame}
\chtitle{汤姆孙就是开尔文\bigwulian}
\bch
\bitem
\item{James Prescott Joule (詹姆斯$\cdot$普雷斯科特 焦耳)

\addfig{0.5}{Joule.jpg}
}
\item{William Thomson, 1st Baron Kelvin (威廉姆$\cdot$汤姆孙,第一男爵开尔文)

\addfig{0.5}{Kelvin.jpg}
}
\eitem
这两位可不止是提出了焦耳-汤姆孙效应(\wulian也叫焦耳-开尔文效应)那么简单
\ech
\end{frame}



\begin{frame}
\chtitle{这两位dalao还摧毁了人类的梦想}
\bch
\bitem
\item{最早人们经常幻想可以{\bf 无限制产生能量的机器(第一类永动机)}。}
\item{人们逐渐熟悉能量守恒定律,特别是{\bf 焦耳确定了热功当量}之后,明白了{\bf 能量不能无中生有},第一类永动机只是痴人说梦。}
\item{梦想不能停:既然热量是能量的一种形式,海水中储藏着大量的热能(内能)。如果这些能量能够无条件地转化为有用的机械功,岂非几乎取之不尽?这类假想的{\bf 可以无条件从单一热源中吸收热量转化为机械功的机器称为第二类永动机。它并不违反能量守恒定律。}}
\item{最后,开尔文提出热力学第二定律的开尔文表述:{\bf 第二类永动机不可能}。}
\eitem

\ech
\end{frame}

\begin{frame}
\bch
\addfig{1}{xdldt.jpg}

为了纪念毁梦者,我们把能量的国际标准单位叫做焦耳($\SIJ$),把温度的国际标准单位叫做开尔文($\SIK$)。
\ech
\end{frame}

\begin{frame}
\chtitle{本讲内容}
\bch
\bitem
\item{热力学第二定律的多种表述方式}
\item{克劳修斯不等式和热力学第二定律的“微分”形式}
\item{自由能和自由焓名称的由来}
\item{熵的计算举例}
\item{热力学第二定律概念练习}
\eitem
\ech
\end{frame}

\section{statements of 2nd law}


\begin{frame}
\chtitle{热力学第二定律的开尔文表述}
\bch
热力学第二定律的{\bf 开尔文表述(Kelvin Statement)}简单地说就是:
\tbox{\bf 第二类永动机不可能。}

或者把第二类永动机的定义进行一番解释的说法就是:
\tbox{\bf 不可能从单一热源吸取热量,使之完全变为有用的功而不产生其他影响。}
\ech
\end{frame}

\begin{frame}
\chtitle{容易漏掉的一些关键词}
\bch
\bitem
\item{“单一热源”(指温度唯一的热源)不可省略。如果允许多热源的话,很容易找到反例,例如可逆卡诺循环。}
\item{“不产生其他影响”不可忽略。如果允许产生其他影响,则从单一热源给气体加热使之膨胀做功就是反例。}
\eitem

\ech
\end{frame}

\begin{frame}
\chtitle{我们再来谈谈另一位dalao}
\bch

\bcenter
克劳修斯(Rudolf Clausius)
\ecenter

\addfig{1.2}{Clausius.jpg}

\bcenter
$\uparrow$

这位dalao提出了熵
\ecenter
\ech
\end{frame}

\begin{frame}
\chtitle{热力学第二定律的克劳修斯表述}
\bch
热力学第二定律的{\bf 克劳修斯表述(Clausius Statement)}:
\tbox{\bf 不可能把热量从低温物体传到高温物体而不引起其他变化。}
\ech
\end{frame}

\begin{frame}
\chtitle{补充点逻辑学常识}
\bch
如果两个逻辑命题等价:$A=B$, 则两个命题的否命题也等价:$\mathrm{not}\ A = \mathrm{not}\ B$
\ech
\end{frame}

\begin{frame}
\chtitle{开尔文表述等价于克劳修斯表述的证明}
\bch
{\small
\bitem
\item{
开尔文表述的否命题是:

A)可以从单一热源吸热,使之完全变为有用的功而不产生其他影响。
}
\item{
克劳修斯表述的否命题是:

B) 可以把热量从低温物体传到高温物体而不引起其他变化。
}
\eitem

只要证明A和B等价,则开尔文表述和克劳修斯表述等价。

先证A $\rightarrow$ B

设A成立,则可以从高温物体吸热,转化为驱动(可逆)制冷机的功,从低温物体吸热传递给高温物体。其净效果为从低温热源吸热传递给高温热源。

\skipline

再证B $\rightarrow$ A

设B成立,则可以从低温物体传热到高温物体,然后在高温物体和低温物体之间通过一个(可逆)卡诺循环把热量返还给低温物体,并额外做功。净效果就是从高温物体吸热做功。
}
\ech
\end{frame}

\begin{frame}
\chtitle{孤立系统的熵增大原理和热力学第二定律等价}
\bch
克劳修斯表述实际上和孤立系统的熵增大原理是一回事情。因为热量$Q$($Q>0$)从低温物体(设温度为$T_2$)向高温物体(设温度为$T_1$)传递,等价于熵减过程
$$\Delta S = Q/T_1 - Q/T_2<0$$
\ech
\end{frame}

\begin{frame}
\chtitle{卡诺定理和热力学第二定律的等价}
\bch
卡诺定理是熵增大原理推出来的,故热二律$\rightarrow$卡诺定理。

反之,若热二律不成立,存在一个热机A可以从单一热源吸热做功而不引起其他变化,则该热机的效率为$1>1-T_2/T_1$,与卡诺定理矛盾。所以卡诺定理$\rightarrow$热二律。

\ech
\end{frame}

\begin{frame}
\chtitle{热力学温标的理想热机定义法}
\bch
如果一开始不引入热力学温标,而仅仅把卡诺定理表述为在两个恒温(“恒温”的定义可以由热力学第零定律得到,不需要具体温标)热源之间工作的热机具有相同效率。则我们可以用可逆热机来定义热力学温标。
即取定某标准点$T_1$后,通过可逆热机效率$\eta$来定义$T_2 \equiv (1-\eta)T_1$。这种定义热力学温标的方法和测温物质无关。
\ech
\end{frame}

\begin{frame}
\chtitle{总结}
\bch
热力学第二定律有多种表述方式:
\bitem
\item{开尔文表述:不可能从单一热源吸热转化为功而无其他影响 (即:第二类永动机不可能)}
\item{克劳修斯表述:不可能低温物体传热给高温物体而无其他影响}
\item{孤立系统的熵增大原理:孤立系统在非平衡态熵会持续增大,直到到达平衡态后熵取到极大值不再改变。}
\item{卡诺定理:所有工作于温度为$T_1$的高温热源和温度为$T_2$的低温热源之间的可逆热机效率均为$1-T_2/T_1$,不可逆热机的效率则低于这个值。}
\eitem
\ech
\end{frame}

\section{Clausius Inequality}

\begin{frame}
\bch
\wulian 感觉今天热学课变成了逻辑课。

\skiplines

下面我们来点干货。
\ech
\end{frame}

\begin{frame}
\chtitle{克劳修斯不等式}
\bch
设在一个循环中,系统依次和温度为$T_1, T_2, \ldots, T_n$的热源接触,分别吸收热量$\dbar Q_1, \dbar Q_2, \ldots, \dbar Q_n$(假设热源很大,始终处于热平衡,所以有明确的温度)。根据孤立系熵增大原理,系统和环境的总熵必须增大(如果循环不可逆)或者不变(如果循环可逆)。又,熵是态函数,故系统的熵经过循环后不变。那么环境的总熵必须增大(不可逆循环)或者不变(可逆循环)。写成数学表达式:
$$\sum_{i=1}^n \Delta S_i = \sum_{i=1}^n \frac{-\dbar Q_i}{T} \ge 0$$
其中等号对且仅对可逆循环成立。

当热源温度连续变化时,上述结果即称为积分形式的{\bf 克劳修斯不等式}:
{\blue $$\oint \frac{\dbar Q}{T} \le 0 $$}
{\bf 等号对且仅对可逆循环成立。}

\ech
\end{frame}

\begin{frame}
\chtitle{克劳修斯不等式的诠释}
\bch
克劳修斯不等式
{\blue $$\oint \frac{\dbar Q}{T} \le 0 $$}
中的{\bf $T$要理解为环境的温度}。

如果循环是可逆的,
$$\oint_{\rm reversible} \frac{\dbar Q}{T} = 0$$

在{\bf 可逆过程}中,热交换是在热平衡情形下进行的,{\bf $T$也可解读为系统的温度}。这时$\frac{\dbar Q}{T} = dS$。克劳修斯用下述(只依赖于起点和终点,与路径无关的)积分来定义熵
$$S_B-S_A = \int_A^B \frac{\dbar Q}{T} $$

\ech
\end{frame}

\begin{frame}
\chtitle{热力学第二定律的“微分”形式}
\bch
假设系统和温度为$T_{\rm env}$的环境发生接触(假设环境是个始终处于热平衡的热库)。在一个微过程中,
设系统的吸热量为$\dbar Q$,则环境熵变为
$$ dS_{\rm env} = \frac{-\dbar Q}{T_{\rm env}}$$
把系统和环境的总和看作一个孤立系统,设系统熵变为$dS$,则
$$dS +dS_{\rm env} \ge 0$$
综合上面两式即有
$$\dbar Q \le T_{\rm env} dS $$
等号当且仅当过程可逆时成立。
\ech
\end{frame}

\begin{frame}
\chtitle{热力学第二定律的“微分”形式(续)}
\bch
热力学第二定律常常被写作
{\blue $$\dbar Q \le TdS$$}
或者
{\blue $$ TdS \ge dU + pdV $$}
其中{\bf 等号对且仅对可逆过程成立}。

这个“微分”形式要打上引号,是因为它并不是描述系统的状态参量的一个微分表达式:$T$是环境的温度,而非系统的温度。

如果系统在热交换过程中是和热源处于准热平衡的(即温差足够小,传热过程足够缓慢),也就是{\bf 过程可逆}。即有我们一开始引入熵时的等式:
$$ TdS = dU + pdV$$
这时$T$可以解读为系统的温度(=热源的温度)。
\ech
\end{frame}

\begin{frame}
\chtitle{现在问题来了}
\bch
\addfig{0.8}{think2.jpg}

教材中证明克劳修斯不等式时,$T$指的是环境温度。证明结束之后(从184页讨论热温比开始)就默默地把$T$当成了系统温度。特别是描述$\dbar Q \le TdS$时毫无顾忌地对不可逆过程使用了温度的概念\wulian。

\skipline

如何帮教材自圆其说呢?

\ech
\end{frame}

\begin{frame}
\chtitle{自圆其说}
\bch
设热源恒温。在一个微过程开始和结束时系统都和环境处于热平衡,温度为$T$,则
$$\dbar Q \le TdS $$
中的$T$可以解读为系统在微过程开始和结束时的温度(=环境的温度)。

\skiplines

\bye 没毛病。


不过,真没必要为了向教材低头,非把$T$诠释为系统温度不可。
\addfig{0.8}{hekune.jpg}

\ech
\end{frame}

\section{Free Energy and Free Enthalpy}

\begin{frame}
\chtitle{自由能}
\bch
{\small
{\blue  设系统和温度恒为$T$的环境保持热接触。在某个过程中对环境做功$A'$,自由能改变了$\Delta F$。过程的开始和结束时系统都和环境处于热平衡。证明: $A'\le -\Delta F$,等号当且仅当过程可逆时成立。}

设系统从环境吸热$Q$,则由热一律有$Q = A' + \Delta U$。
由$F$的定义,有
$$\Delta F = \Delta U - \Delta (TS) = \Delta U - T\Delta S $$
又根据热二律的“微分”形式,有$T\Delta S \ge Q = A' +\Delta U$。故
$$A' \le - \Delta F$$
等号当且仅当过程可逆时成立。

也就是说,{\bf 定温过程中,对外做功不大于自由能的减少}。可以把自由能理解为恒温条件下“能自由对外做功”的那部分内能。这就是“自由能”的名称的由来。
}

{\scriptsize
注意:这里的系统可以是比$pVT$系统更为复杂的多参量系统。
}
\ech
\end{frame}

\begin{frame}
\chtitle{自由焓}
\bch

{\blue  设某系统处在温度恒为$T$,压强恒为$p$的大气中。在某个过程中系统对外功$A'$,体积改变了$\Delta V$,自由焓改变了$\Delta G$。过程的开始和结束时系统都和环境处于热平衡和力学平衡。我们往往对系统对大气做的功$p\Delta V$并不感兴趣,而只考虑系统做的其他有用功$A'_{\rm else} = A'-p\Delta V$。证明: $A'_{\rm else}  \le -\Delta G$,等号当且仅当过程可逆时成立。}

由于是定温条件下的过程,对外做功不大于自由能的减少:
$$ A' \le - \Delta F = -\Delta (G- pV) = - \Delta G + p \Delta V$$
两边减去$p\Delta V$即得结论。

也就是说,{\bf 定温定压过程中,系统做额外有用功不大于自由焓的减少}。可以把自由焓理解为恒温恒压条件下“能自由对外做的额外有用功”的那部分焓。这就是“自由焓”的名称的由来。

{\scriptsize
注意:这里的系统可以是比$pVT$系统更为复杂的多参量系统。
}
\ech
\end{frame}

\section{Applications}

\begin{frame}
\chtitle{熵的计算举例I}
\bch
{计算$1\SImol$冰在标准状态下熔解后的熵的变化,已知标准状态下冰的摩尔熔化热是$333\SIkJ$。}

\skipline
{
$$\Delta S = \frac{Q}{T} = \frac{333 \SIkJ}{273.15\SIK} = 1.22 \SIkJ/\SIK $$
}
\ech
\end{frame}


\begin{frame}
\chtitle{熵的计算举例II}
\bch
{\blue 设理想气体热容量为常量,它经历可逆等温过程从体积$V_1$膨胀到体积$V_2$,求熵的变化。}

\skipline

{
理想气体$dS = C_V d\ln T + \nu R d\ln V$。等温过程中:
\bea
\Delta S = \nu R \Delta \ln V = \nu R \ln\frac{V_2}{V_1}
\eea
}
\ech
\end{frame}

\begin{frame}
\chtitle{熵的计算举例III}
\bch
{\blue 体积为$V_1$的理想气体绝热自由膨胀,体积变为$V_2$。}

自由绝热膨胀既不做功,也不吸热,所以内能不变。又理想气体内能只跟温度有关,故温度不变。所以本题的初态,末态都和上题相同,熵的变化也相同(熵是态函数):

$$ \Delta S = \nu R \ln\frac{V_2}{V_1}$$


\ech
\end{frame}




\begin{frame}
\chtitle{熵的计算举例IV}
\bch
{\blue 光子气体的内能密度为$aT^4$,其中$a=\frac{\pi^2 k^4}{15\hbar^3c^3}$为常数,$T$为温度。已知$T=0$时光子气体的熵为零。试求温度为$T$的光子气体的熵密度。}

\skipline
{
内能$U=aVT^4$,定体热容$\pfrac UTV = 4aVT^3$。

利用$dU = TdS - p dV$得到$\pfrac USV = T$,故
$$\pfrac STV = \frac{\pfrac UTV}{\pfrac USV} = \frac{4aVT^3}{T} = 4aVT^2$$
固定体积,对温度从$0$到$T$积分,得到
$$ S(T, V) = \int_0^T \pfrac STV dT = \frac{4}{3}aVT^3 $$
即熵密度为
$$ \frac{S}{V}= \frac{4}{3}aT^3$$
}
\ech
\end{frame}


\begin{frame}
\chtitle{熵的计算举例V}
\bch
{\blue 把温度和压强都相同的,总摩尔数为$\nu$的$n$种不同的气体混合在一起。每种气体的摩尔分数(单种气体摩尔数/总摩尔数)分别为$c_1$, $c_2$, $\ldots$, $c_n$ ($\sum_i c_i = 1$)。试计算气体混合后相对于混合前的熵变。}

\skipline

{\small
每种气体温度不变,体积由混合前的$Vc_i$变为$V$,熵增为$(\nu c_i) R\ln \frac{V}{c_iV} = - \nu R c_i\ln c_i$,对$n$种气体求和即得
$$\Delta S =   -\nu R \sum_{i=1}^n c_i \ln c_i$$}

\skipline

{\scriptsize
另解:

在混合前,任取一个分子,我们知道它是属于那一类气体。混合后,我们只知道它是第$i$种气体的概率为$c_i$,故单个分子熵变为
$-\sum_{i=1}^{n} kc_i\ln c_i $。 总共有$ \nu N_A $个分子,故总熵变为
$$ \Delta S =  -\nu N_A k\sum_{i=1}^n c_i \ln c_i = -\nu R \sum_{i=1}^n c_i \ln c_i$$  
}
\ech
\end{frame}

\begin{frame}
\chtitle{熵的计算举例VI}
\bch
{\small \blue 某气体状态方程为$pV + f(V) = \nu RT$,其中$f$为某函数。气体经过准静态的等温加热膨胀体积从$V_1$变为$V_2$,这个过程的熵变。}

{\small
准静态过程吸热量为按定体热容计算的吸热量加上热压强做功消耗的能量,在等温过程中仅需计算后者。
\bea
dS &=& \frac{\dbar Q}{T} \newl
&=&   \frac{T\pfrac pTV dV}{T} \newl
&=& \pfrac pTV dV  \newl
&=& \frac{\nu R}{V} dV
\eea
所以等温膨胀后熵变为$\nu R \ln\frac{V_2}{V_1}$。
}
\ech
\end{frame}


\begin{frame}
\chtitle{热力学第二定律练习I}
\bch

\addfig{1.}{songfen.jpg}

\bitem
\item{论证摩擦生热过程是不可逆的}
\eitem

\ech
\end{frame}

\begin{frame}
\chtitle{热力学第二定律练习II}
\bch

\addfig{1.}{songfen.jpg}

\bitem
\item{有人想利用海洋不同深度处温度不同制造一种机器,把海水的内能转化为机械功,这是否违反热力学第二定律?}
\eitem

\ech
\end{frame}


\begin{frame}
\chtitle{热力学第二定律练习III}
\bch

\addfig{1.}{songfen.jpg}

\bitem
\item{给气筒里的气体加热,使它在保持内能不变的情况下膨胀推动活塞做功,这把热完全转化为了功,是否违反热力学第二定律?}
\eitem
\ech
\end{frame}


\begin{frame}
\chtitle{热力学第二定律练习IV}
\bch

\addfig{1.}{songfen.jpg}

\bitem
\item{论证绝热线和等温线不能有两个以上的交点。}
\eitem
\ech
\end{frame}


\begin{frame}
\chtitle{第十二周作业(序号接第十一周)}
\bch

\bitem
\item[30]{教材习题4-27}
\item[31]{某气体的状态方程为
$$p(V- \nu b) = \nu RT$$
该气体经过准静态的等温膨胀压强减少了一半。求这个过程的熵变。}
\item[32]{史瓦西黑洞是一个热力学系统,它的能量正比于质量,熵正比于质量的平方。用热力学第二定律证明一个史瓦西黑洞不可能分裂为两个史瓦西黑洞而不造成其他影响。}
\eitem

\ech
\end{frame}


\end{document}

\documentclass[CJK]{beamer}
\usepackage{CJKutf8}
\usepackage{beamerthemesplit}
\usetheme{Malmoe}
\useoutertheme[footline=authortitle]{miniframes}
\usepackage{amsmath}
\usepackage{amssymb}
\usepackage{graphicx}
\usepackage{eufrak}
\usepackage{color}
\usepackage{slashed}
\usepackage{simplewick}
\usepackage{tikz}
\graphicspath{{../figures/}}
\def\addfig#1#2{\begin{center}\includegraphics[width=#1 in]{#2}\end{center}}
\def\blacktext#1{{\color{black}#1}}
\def\bluetext#1{{\color{blue}#1}}
\def\redtext#1{{\color{red}#1}}
\def\darkbluetext#1{{\color[rgb]{0,0.2,0.6}#1}}
\def\skybluetext#1{{\color[rgb]{0.2,0.7,1.}#1}}
\def\cyantext#1{{\color[rgb]{0.,0.5,0.5}#1}}
\def\greentext#1{{\color[rgb]{0,0.7,0.1}#1}}
\def\darkgray{\color[rgb]{0.2,0.2,0.2}}
\def\lightgray{\color[rgb]{0.6,0.6,0.6}}
\def\gray{\color[rgb]{0.4,0.4,0.4}}
\def\blue{\color{blue}}
\def\red{\color{red}}
\def\green{\color{green}}
\def\darkblue{\color[rgb]{0,0.2,0.6}}
\def\skyblue{\color[rgb]{0.2,0.7,1.}}
\def\fdeg{{^\circ \mathrm{F}}}
\def\cdeg{^\circ \mathrm{C}}
\def\be{\begin{equation}}
\def\ee{\nonumber\end{equation}}
\def\bea{\begin{eqnarray}}
\def\eea{\nonumber\end{eqnarray}}
\def\ii{{\dot{\imath}}}
\def\bch{\begin{CJK}{UTF8}{gbsn}}
\def\ech{\end{CJK}}
\def\bitem{\begin{itemize}}
\def\eitem{\end{itemize}}
\def\bcenter{\begin{center}}
\def\ecenter{\end{center}}
\def\bex{\begin{minipage}{0.3\textwidth}\includegraphics[width=1in]{jugelizi.png}\end{minipage}\begin{minipage}{0.6\textwidth}}
\def\eex{\end{minipage}}
\def\chtitle#1{\frametitle{\bch#1\ech}}
\def\skipline{{\vskip0.1in}}
\def\skiplines{{\vskip0.2in}}
\def\lagr{{\mathcal{L}}}
\def\hamil{{\mathcal{H}}}
\def\vecv{{\mathbf{v}}}
\def\vecx{{\mathbf{x}}}
\def\vecy{{\mathbf{y}}}
\def\veck{{\mathbf{k}}}
\def\vecp{{\mathbf{p}}}
\def\vecn{{\mathbf{n}}}
\def\vecA{{\mathbf{A}}}
\def\vecP{{\mathbf{P}}}
\def\vecsigma{{\mathbf{\sigma}}}
\def\hatJn{{\hat{J_\vecn}}}
\def\hatJx{{\hat{J_x}}}
\def\hatJy{{\hat{J_y}}}
\def\hatJz{{\hat{J_z}}}
\def\hatj#1{\hat{J_{#1}}}
\def\hatphi{{\hat{\phi}}}
\def\hatq{{\hat{q}}}
\def\hatpi{{\hat{\pi}}}
\def\vel{\upsilon}
\def\Dint{{\mathcal{D}}}
\def\adag{{\hat{a}^\dagger}}
\def\bdag{{\hat{b}^\dagger}}
\def\cdag{{\hat{c}^\dagger}}
\def\ddag{{\hat{d}^\dagger}}
\def\hata{{\hat{a}}}
\def\hatb{{\hat{b}}}
\def\hatc{{\hat{c}}}
\def\hatd{{\hat{d}}}
\def\hatN{{\hat{N}}}
\def\hatH{{\hat{H}}}
\def\hatp{{\hat{p}}}
\def\Fup{{F^{\mu\nu}}}
\def\Fdown{{F_{\mu\nu}}}
\def\newl{\nonumber \\}
\def\SIkm{\,\mathrm{km}}
\def\SIyr{\,\mathrm{yr}}
\def\SIGyr{\,\mathrm{Gyr}}
\def\SIeV{\,\mathrm{eV}}
\def\SIkeV{\,\mathrm{keV}}
\def\SIMeV{\,\mathrm{MeV}}
\def\SIGeV{\,\mathrm{GeV}}
\def\SIcal{\,\mathrm{cal}}
\def\SIkcal{\,\mathrm{kcal}}
\def\SImol{\,\mathrm{mol}}
\def\SIm{\,\mathrm{m}}
\def\SIcm{\,\mathrm{cm}}
\def\SIfm{\,\mathrm{fm}}
\def\SImm{\,\mathrm{mm}}
\def\SInm{\,\mathrm{nm}}
\def\SImum{\,\mathrm{\mu m}}
\def\SIJ{\,\mathrm{J}}
\def\SIkJ{\,\mathrm{kJ}}
\def\SIs{\,\mathrm{s}}
\def\SIkg{\,\mathrm{kg}}
\def\SIg{\,\mathrm{g}}
\def\SIK{\,\mathrm{K}}
\def\SImmHg{\,\mathrm{mmHg}}
\def\SIPa{\,\mathrm{Pa}}
\def\vece{\mathrm{e}}
\def\bmat#1{\left(\begin{array}{#1}}
\def\emat{\end{array}\right)}
\def\bcase#1{\left\{\begin{array}{#1}}
\def\ecase{\end{array}\right.}
\def\calM{{\mathcal{M}}}
\def\calT{{\mathcal{T}}}
\def\calR{{\mathcal{R}}}
\def\barpsi{\bar{\psi}}
\def\baru{\bar{u}}
\def\barv{\bar{\upsilon}}
\def\bmini#1{\begin{minipage}{#1\textwidth}}
\def\emini{\end{minipage}}
\def\qeq{\stackrel{?}{=}}
\def\torder#1{\mathcal{T}\left(#1\right)}
\def\rorder#1{\mathcal{R}\left(#1\right)}
\def\contr#1#2{\contraction{}{#1}{}{#2}#1#2}
\def\trof#1{\mathrm{Tr}\left(#1\right)}
\def\trace{\mathrm{Tr}}
\def\comm#1{\ \ \ \left(\mathrm{used}\ #1\right)}
\def\tcomm#1{\ \ \ (\text{#1})}
\def\slp{\slashed{p}}
\def\slk{\slashed{k}}
\def\wulian{\includegraphics[width=0.18in]{emoji_wulian.jpg}}
\def\bye{\includegraphics[width=0.18in]{emoji_bye.jpg}}
\def\calp{{\mathfrak{p}}}
\def\veccalp{\mathbf{\mathfrak{p}}}
\def\atm{\,\mathrm{atm}}
\def\angstrom{\,\text{\AA}}
\def\Tthree{T_{\tiny \textcircled{3}}}
\def\pthree{p_{\tiny \textcircled{3}}}

\def\courseurl{http://zhiqihuang.top}

\def\tpage#1#2{
\begin{frame}
\bch
\begin{center}
\begin{large}
热学 \\
第#1讲 #2

\end{large}

\skiplines

黄志琦


\end{center}

\skiplines

{\small 
教材:《热学》第二版,赵凯华,罗蔚茵,高等教育出版社


课件下载
}
\courseurl 
\ech
\end{frame}
}

\def\bfr#1{
\begin{frame}
\chtitle{#1} 
\bch
}

\def\efr{
\ech 
\end{frame}
}

\title{Lesson 12 - 2$^{\rm nd}$ Law of Thermodynamics}
  \author{}
  \date{}
\begin{document}
\tpage{12}{热力学第二定律}

\section{Review}

\begin{frame}
\chtitle{上讲内容回顾}
\bch 
\bitem
\item{孤立系统的熵增大原理}
\item{循环过程:可逆循环与不可逆循环;正循环与逆循环;}
\item{卡诺定理$\eta \le 1 -\frac{T_2}{T_1}$ (对制冷机则$\varepsilon \le \frac{T_2}{T_1-T_2}$)}
\item{理想气体循环效率的计算}
\eitem
\ech
\end{frame}

\begin{frame}
\chtitle{汤姆孙是谁}
\bch
学习焦耳-汤姆孙效应时,我有个疑问:

\skiplines

焦耳这位dalao我熟,但{\large \bf 汤姆孙}我怎么没听说过?
\addfig{1.3}{tom.jpg}

\ech
\end{frame}

\begin{frame}
\chtitle{汤姆孙就是开尔文\bigwulian}
\bch
\bitem
\item{James Prescott Joule (詹姆斯$\cdot$普雷斯科特 焦耳)

\addfig{0.5}{Joule.jpg}
}
\item{William Thomson, 1st Baron Kelvin (威廉姆$\cdot$汤姆孙,第一男爵开尔文)

\addfig{0.5}{Kelvin.jpg}
}
\eitem
这两位可不止是提出了焦耳-汤姆孙效应(\wulian也叫焦耳-开尔文效应)那么简单
\ech
\end{frame}



\begin{frame}
\chtitle{这两位dalao还摧毁了人类的梦想}
\bch
\bitem
\item{最早人们经常幻想可以{\bf 无限制产生能量的机器(第一类永动机)}。}
\item{人们逐渐熟悉能量守恒定律,特别是{\bf 焦耳确定了热功当量}之后,明白了{\bf 能量不能无中生有},第一类永动机只是痴人说梦。}
\item{梦想不能停:既然热量是能量的一种形式,海水中储藏着大量的热能(内能)。如果这些能量能够无条件地转化为有用的机械功,岂非几乎取之不尽?这类假想的{\bf 可以无条件从单一热源中吸收热量转化为机械功的机器称为第二类永动机。它并不违反能量守恒定律。}}
\item{最后,开尔文提出热力学第二定律的开尔文表述:{\bf 第二类永动机不可能}。}
\eitem

\ech
\end{frame}

\begin{frame}
\bch
\addfig{1}{xdldt.jpg}

为了纪念毁梦者,我们把能量的国际标准单位叫做焦耳($\SIJ$),把温度的国际标准单位叫做开尔文($\SIK$)。
\ech
\end{frame}

\begin{frame}
\chtitle{本讲内容}
\bch
\bitem
\item{热力学第二定律的多种表述方式}
\item{克劳修斯不等式和热力学第二定律的“微分”形式}
\item{自由能和自由焓名称的由来}
\item{平衡判据}
\item{课程总结}
\eitem
\ech
\end{frame}

\section{2nd Law}


\begin{frame}
\chtitle{热力学第二定律的开尔文表述}
\bch
热力学第二定律的{\bf 开尔文表述(Kelvin Statement)}简单地说就是:
\tbox{\bf 第二类永动机不可能。}

或者把第二类永动机的定义进行一番解释的说法就是:
\tbox{\bf 不可能从单一热源吸取热量,使之完全变为有用的功而不产生其他影响。}
\ech
\end{frame}

\begin{frame}
\chtitle{容易漏掉的一些关键词}
\bch
\bitem
\item{“单一热源”(指温度唯一的热源)不可省略。如果允许多热源的话,很容易找到反例,例如可逆卡诺循环。}
\item{“不产生其他影响”不可忽略。如果允许产生其他影响,则从单一热源给气体加热使之膨胀做功就是反例。}
\eitem

\ech
\end{frame}

\begin{frame}
\chtitle{我们再来谈谈另一位dalao}
\bch

\bcenter
克劳修斯(Rudolf Clausius)
\ecenter

\addfig{1.2}{Clausius.jpg}

\bcenter
$\uparrow$

这位dalao提出了熵
\ecenter
\ech
\end{frame}

\begin{frame}
\chtitle{热力学第二定律的克劳修斯表述}
\bch
热力学第二定律的{\bf 克劳修斯表述(Clausius Statement)}:
\tbox{\bf 不可能把热量从低温物体传到高温物体而不引起其他变化。}
\ech
\end{frame}

\begin{frame}
\chtitle{补充点逻辑学常识}
\bch
如果两个逻辑命题等价:$A=B$, 则两个命题的否命题也等价:$\mathrm{not}\ A = \mathrm{not}\ B$
\ech
\end{frame}

\begin{frame}
\chtitle{开尔文表述等价于克劳修斯表述的证明}
\bch
{\small
\bitem
\item{
开尔文表述的否命题是:

A)可以从单一热源吸热,使之完全变为有用的功而不产生其他影响。
}
\item{
克劳修斯表述的否命题是:

B) 可以把热量从低温物体传到高温物体而不引起其他变化。
}
\eitem

只要证明A和B等价,则开尔文表述和克劳修斯表述等价。

先证A $\rightarrow$ B

设A成立,则可以从高温物体吸热,转化为驱动(可逆)制冷机的功,从低温物体吸热传递给高温物体。其净效果为从低温热源吸热传递给高温热源。

\skipline

再证B $\rightarrow$ A

设B成立,则可以从低温物体传热到高温物体,然后在高温物体和低温物体之间通过一个(可逆)卡诺循环把热量返还给低温物体,并额外做功。净效果就是从高温物体吸热做功。
}
\ech
\end{frame}

\begin{frame}
\chtitle{孤立系统的熵增大原理和热力学第二定律等价}
\bch
克劳修斯表述实际上和孤立系统的熵增大原理是一回事情。因为热量$Q$($Q>0$)从低温物体(设温度为$T_2$)向高温物体(设温度为$T_1$)传递,等价于熵减过程
$$\Delta S = Q/T_1 - Q/T_2<0$$
\ech
\end{frame}

\begin{frame}
\chtitle{卡诺定理和热力学第二定律的等价}
\bch
卡诺定理是熵增大原理推出来的,故热二律$\rightarrow$卡诺定理。

反之,若热二律不成立,存在一个热机A可以从单一热源吸热做功而不引起其他变化,则该热机的效率为$1>1-T_2/T_1$,与卡诺定理矛盾。所以卡诺定理$\rightarrow$热二律。

\ech
\end{frame}

\begin{frame}
\chtitle{热力学温标的理想热机定义法}
\bch
如果一开始不引入热力学温标,而仅仅把卡诺定理表述为在两个恒温(“恒温”的定义可以由热力学第零定律得到,不需要具体温标)热源之间工作的热机具有相同效率。则我们可以用可逆热机来定义热力学温标。
即取定某标准点$T_1$后,通过可逆热机效率$\eta$来定义$T_2 \equiv (1-\eta)T_1$。这种定义热力学温标的方法和测温物质无关。
\ech
\end{frame}

\begin{frame}
\chtitle{总结}
\bch
热力学第二定律有多种表述方式:
\bitem
\item{开尔文表述:不可能从单一热源吸热转化为功而无其他影响 (即:第二类永动机不可能)}
\item{克劳修斯表述:不可能低温物体传热给高温物体而无其他影响}
\item{孤立系统的熵增大原理:孤立系统在非平衡态熵会持续增大,直到到达平衡态后熵取到极大值不再改变。}
\item{卡诺定理:所有工作于温度为$T_1$的高温热源和温度为$T_2$的低温热源之间的可逆热机效率均为$1-T_2/T_1$,不可逆热机的效率则低于这个值。}
\eitem
\ech
\end{frame}


\begin{frame}
\chtitle{热力学第二定律练习I}
\bch

\bitem
\item{论证摩擦生热过程是不可逆的}
\eitem

\ech
\end{frame}

\begin{frame}
\chtitle{热力学第二定律练习II}
\bch

\addfig{1.}{songfen.jpg}

\bitem
\item{有人想利用海洋不同深度处温度不同制造一种机器,把海水的内能转化为机械功,这是否违反热力学第二定律?}
\eitem

\ech
\end{frame}


\begin{frame}
\chtitle{热力学第二定律练习III}
\bch

\addfig{1.}{songfen.jpg}

\bitem
\item{给气筒里的气体加热,使它在保持内能不变的情况下膨胀推动活塞做功,这把热完全转化为了功,是否违反热力学第二定律?}
\eitem
\ech
\end{frame}

\section{Clausius Inequality}

\begin{frame}
\bch
\wulian 感觉今天热学课变成了逻辑课。

\skiplines

下面我们来点干货。
\ech
\end{frame}


\begin{frame}
\chtitle{热库的概念}
\bch
热库是指比系统大得多的热源。热库和系统之间的热量交换对热库来说如九牛一毛,不影响热库的温度。

\addfig{1.8}{heatsource.png}
\ech
\end{frame}


\begin{frame}
\chtitle{克劳修斯不等式}
\bch
{\small
设在一个循环中,系统依次和温度为$T_1, T_2, \ldots, T_n$的环境(热库)接触,分别吸收热量$\dbar Q_1, \dbar Q_2, \ldots, \dbar Q_n$。根据孤立系熵增大原理,如果循环可逆,系统和环境的总熵不变;若循环不可逆,系统和环境的总熵必须增大。又,熵是态函数,故系统的熵经过循环后不变。那么环境的熵对可逆循环而言不变,对不可逆循环而言必须增大。写成数学表达式:
$$\sum_{i=1}^n \Delta S_i = \sum_{i=1}^n \frac{-\dbar Q_i}{T_i} \ge 0$$
其中等号对且仅对可逆循环成立。

当热源温度连续变化时,上述结果即称为积分形式的{\bf 克劳修斯不等式}:
{\blue $$\oint \frac{\dbar Q}{T_e} \le 0 $$}
{\bf 等号对且仅对可逆循环成立}。我们用符号$T_e$ 明确这里的温度为环境热源的温度(下标e指environment)。
}
\ech
\end{frame}


\begin{frame}
\chtitle{热力学第二定律的“微分”形式}
\bch
假设系统和温度为$T_e$的环境(热库)发生接触。在一个微过程中,
设系统的吸热量为$\dbar Q$,则环境熵变为
$$ dS_e = \frac{-\dbar Q}{T_e}$$
把系统和环境的总和看作一个孤立系统,设系统熵变为$dS$,则
$$dS +dS_e \ge 0$$
综合上面两式即有
{\blue $$\dbar Q \le T_e dS $$
等号当且仅当过程可逆时成立}。
\ech
\end{frame}


\begin{frame}
\chtitle{现在问题来了}
\bch
教材中证明克劳修斯不等式时,$T$指的是环境温度。证明结束之后(从184页讨论热温比开始)就默默地把$T$当成了系统温度。

\addfig{1.8}{huangainian.png}

\ech
\end{frame}

\begin{frame}
\chtitle{理想传导热库}
\bch
假设系统和温度为$T$的热库接触,且处于平衡态。当我们给系统一个扰动时,系统各处可能发生热不平衡(各处温度不同),力学不平衡(各处压强不同),和化学不平衡(各处化学势不同)。所谓理想传导热库是指和系统接触非常良好的热库,它使得系统各部分很快地恢复到温度$T$。而恢复力学平衡和化学平衡则要迟缓得多。

\bmini{0.5}
\addfig{1.5}{idealheatsource.png}
\emini
\bmini{0.46}
{\bf 和理想传导热库接触的系统,可处于非平衡态(力学不平衡或者化学不平衡)却仍具有确定的和热库一致的温度$T$。}
\emini
\ech
\end{frame}

\begin{frame}
\chtitle{就把$T$当系统温度吧,教材开心就好}
\bch
当我们{\blue 仅考虑和理想传导热库接触的系统}时,克劳修斯不等式就可以妥妥地写成
{\blue $$\oint \frac{\dbar Q}{T} \le 0 $$}
这时,$T$是指系统的温度了(反正等于热库温度\wulian)。

同样,热二律的“微分”形式也可以写成{\blue
$$ TdS \ge \dbar Q $$}
或者{\blue
$$ TdS \ge dU + pdV $$}
{\small 注意,当上式取大于号时,描述的是包含非平衡态(力学不平衡或者化学不平衡)的不可逆过程。这并不和我们之前对只包含平衡态的准静态过程引入的$ dS = \frac{1}{T}\left(dU + pdV\right)$矛盾。}

\ech
\end{frame}


\begin{frame}
\chtitle{总结}
\bch
克劳修斯不等式和热二律的“微分”形式:
{\blue $$\oint \frac{\dbar Q}{T_e} \le 0 $$
$$ T_edS \ge \dbar Q $$}
具有一般的意义:无须假设是理想传导热库也无须假设系统温度和热库一致。

\skipline

仅考虑和理想传导热库接触的系统时,可以把$T_e$替换为系统温度$T$(这是流行版本,也是教材采用的描述方式)。
\ech
\end{frame}


\section{Free Energy and Free Enthalpy}

\begin{frame}
\chtitle{自由能}
\bch
{\small
{\blue  设系统和温度恒为$T$的理想传导热库接触。在某个过程中系统对外做功$A'$,自由能改变了$\Delta F$。证明: $A'\le -\Delta F$,等号当且仅当过程可逆时成立。}

设系统从环境吸热$Q$,则由热一律有$Q = A' + \Delta U$。
由$F$的定义,有
$$\Delta F = \Delta U - \Delta (TS) = \Delta U - T\Delta S $$
又根据热二律的“微分”形式,有$T\Delta S \ge Q = A' +\Delta U$。故
$$A' \le - \Delta F$$
等号当且仅当过程可逆时成立。

也就是说,{\bf 定温过程中,对外做功不大于自由能的减少}。可以把自由能理解为恒温条件下“能自由对外做功”的那部分内能。这就是“自由能”的名称的由来。
}

\ech
\end{frame}

\begin{frame}
\chtitle{自由焓}
\bch

{\blue  设某系统处在温度恒为$T$,压强恒为$p$的大气中(当作理想传导热库)。在某个过程中系统对外功$A'$,体积改变了$\Delta V$,自由焓改变了$\Delta G$。我们往往对系统对大气做的功$p\Delta V$并不感兴趣,而只考虑系统做的其他有用功$A'_{\rm else} = A'-p\Delta V$ (例如,额外装了发电装置等)。证明: $A'_{\rm else}  \le -\Delta G$,等号当且仅当过程可逆时成立。}

由于是定温条件下的过程,对外做功不大于自由能的减少:
$$ A' \le - \Delta F = -\Delta (G- pV) = - \Delta G + p \Delta V$$
两边减去$p\Delta V$即得结论。

也就是说,{\bf 定温定压过程中,系统做额外有用功不大于自由焓的减少}。可以把自由焓理解为恒温恒压条件下“能自由对外做的额外有用功”的那部分焓。这就是“自由焓”的名称的由来。

\ech
\end{frame}

\section{Equilibrium Conditions}


\begin{frame}
\chtitle{眼花撩乱的平衡判据}
\bch
教材从198页开始叨念各种(完全看不懂的)“平衡判据”:

\bitem
\item{熵判据:在内能和体积不变的情况下,对于一切可能的变动来说,平衡态的熵最大。}
\item{自由能判据:在温度和体积不变的情况下,对于一切可能的变动来说,平衡态的自由能最小。}
\item{自由焓判据:在温度和压强不变的情况下,对于一切可能的变动来说,平衡态的自由焓最小。}
\eitem

\addfig{1.5}{bzdsh.jpg}
\ech
\end{frame}

\begin{frame}
\chtitle{背书宝的福音}
\bch
\addfig{1}{beishubao.jpg}

要写出“$X$判据”,

\bitem
\item{第一步:把$X$写成全微分:$dX + \ldots dY + \ldots dZ = 0$}
\item{第二步:背书。在$Y$和$Z$不变的情况下,对于一切可能的变动来说,平衡态的$X$最$\ldots$。}
\item{第三步:判定是最大还是最小。$dX$, $dY$, $dZ$中必然有一个能量型的量(指$dU, dH, dF, dG$中的一个),它前面系数为正则为最小,为负则为最大。}
\eitem

{\scriptsize
似乎有点理解了:

$Y$和$Z$不变,则$dX = 0$,代表$X$是某种极值?

能量型的量当然有下限?
}
\ech
\end{frame}


\begin{frame}
\chtitle{例子1:熵判据}
\bch
写出
$$dS - \frac{1}{T} dU - \frac{p}{T} dV = 0 $$
$dU$前系数为负,取最大。

在内能$U$和体积$V$不变的情况下,对于一切可能的变动来说,平衡态的熵$S$最大。

\skiplines

熵判据也常常说成,孤立系的平衡态熵最大。内能和体积不变,是孤立系的一种。
\ech
\end{frame}


\begin{frame}
\chtitle{例子2:自由能判据}
\bch
写出
$$dF  + S dT + p dV = 0 $$
$dF$前系数为正,取最小。


在温度$T$和体积$V$不变的情况下,对于一切可能的变动来说,平衡态的自由能$F$最小。

\ech
\end{frame}


\begin{frame}
\chtitle{例子3:自由焓判据}
\bch
写出
$$dG + S dT - V dp = 0$$
$dG$前系数为正,取最小。

在温度$T$和压强$p$不变的情况下,对于一切可能的变动来说,平衡态的自由焓$G$最小。

\ech
\end{frame}


\begin{frame}
\chtitle{例子4: 压强判据之一}
\bch
写出
$$dp - \frac{1}{V} dH + \frac{T}{V} dS = 0$$
$dH$前系数为负,取最大。

在焓$H$和熵$S$不变的情况下,对于一切可能的变动来说,平衡态的压强$p$最大。

\ech
\end{frame}


\begin{frame}
  \chtitle{练习一下}
  \bch
  \addfig{1}{songfen.jpg}
  补充完整下列平衡判据:
  \bitem
\item[(1)]{在\_\_和\_\_不变的情况下,平衡态的焓$H$最\_\_。}
\item[(2)]{在自由能$F$和\_\_不变的情况下,平衡态的温度$T$最\_\_。}
\item[(3)]{在焓$H$和\_\_不变的情况下,平衡态的熵$S$最\_\_。}
\item[(4)]{在自由焓$G$和\_\_不变的情况下,平衡态的压强$p$最\_\_。}
  \eitem
  \ech
\end{frame}

\begin{frame}
\chtitle{真懂了吗}
\bch
“一切可能的变动”是什么意思?

\addfig{1.5}{zhendongma.jpg}
\ech
\end{frame}

\begin{frame}
\chtitle{理解熵判据}
\bch
熵判据里“一切可能的变动”好理解:就是指孤立系的平衡态和一切非平衡态。对非平衡态,广延量内能和体积都有定义。


\addfig{1}{meimaobing.jpg}

证明:根据孤立系的熵增大原理,任何非平衡态的熵会持续增大直至到达平衡态。显然平衡态的熵是最大的。

\ech
\end{frame}


\begin{frame}
\chtitle{理解自由能判据}
\bch
自由能判据的条件是定温定体。“一切可能的变动”也包括非平衡态。这时需要假设系统和恒温的理想传导热库接触来理解非平衡态的温度。

\addfig{1}{meimaobing.jpg}

证明:从非平衡态到平衡态的过程为不可逆过程。定体过程不对外做功,又由于恒温条件下对外做功小于自由能的减少(对不可逆过程)。故自由能的减少大于零。即非平衡态的自由能大于平衡态的自由能。

\ech
\end{frame}

\begin{frame}
\chtitle{理解自由焓判据}
\bch
自由焓判据的条件是定温定压。“一切可能的变动”也包括非平衡态。这时需要假设系统和恒温的理想传导热库接触来理解非平衡态的温度。

\addfig{1}{meimaobing.jpg}

证明:从非平衡态到平衡态的过程为不可逆过程。定温定压条件下,对不可逆过程,其他形式的做功小于自由焓的减少,又这里不考虑其他形式的做功(即没有额外安装发电机什么的),故自由焓的减少大于零。即非平衡态的自由焓大于平衡态的自由焓。

\ech
\end{frame}

\begin{frame}
\chtitle{三大平衡条件}
\bch
\bitem
\item[1]{\blue 热平衡条件:系统内部温度均匀}
\item[2]{\blue 力学平衡条件:系统内部压强均匀}
\item[3]{\blue 相平衡条件:系统内各相化学势相等}
\eitem

{\scriptsize
注:第三条仅考虑了物态变化(如液态水和水蒸气之间的平衡),其实它可以推广到涉及化学反应的一般的化学平衡条件。化学反应比相变更为复杂,各种反应物或者产物有可能混合在一起产生额外的混合熵(请参考教材208页)。
}
\ech
\end{frame}


\begin{frame}
\chtitle{热平衡条件}
\bch
热平衡条件可以从孤立系的熵判据推出:如果有相邻的温度不相等的子系统,则由高温到低温的热量传输必然增大熵。这说明系统还未达到平衡态(熵最大的状态)。
\addfig{1.5}{eqc1.jpg}
$$\Delta S = \frac{-Q}{T_1} + \frac{Q}{T_2} > 0$$
\ech
\end{frame}

\begin{frame}
\chtitle{力学平衡条件}
\bch
力学平衡条件可以从定温定体系统的自由能判据推出:如果有相邻的压强不相等的子系统,则压强大的子系统等温膨胀$\Delta V$,压强小的子系统等温收缩$\Delta V$,可使系统温度和体积不变的情况下总自由能减小。这说明系统还未达到平衡态(自由能最低的态)。
\addfig{1.5}{eqc2.jpg}
$$\Delta F = -p_1 \Delta V + p_2 \Delta V < 0 $$
\ech
\end{frame}


\begin{frame}
\chtitle{相平衡条件}
\bch
设有种可以相互转化的物态。每种物态的粒子数分别为$N_1$, $N_2$, $\ldots$, $N_n$。每种物态成分的化学势即该成分的平均每个粒子的自由焓,即
$$ \mu_i \equiv \pfrac G{N_i}{T,p,N_1, N_2, \ldots, N_{i-1}, N_{i+1}, \ldots, N_n} $$

\addfig{1.5}{eqc3.jpg}

相平衡条件可以从定温定压系统的自由焓判据推出:如果定温定压条件下$ \mu_i > \mu_j$,则可以通过$i\rightarrow j$的相变降低系统的总自由焓。
,这说明系统还未达到平衡态。
\ech
\end{frame}


\begin{frame}
\chtitle{经典理想气体的化学势}
\bch
{\small
考虑经典的理想气体,在$i$态上的分子数为
$$ n_i = e^{-\frac{\varepsilon_i-\mu}{kT}}$$
下面我们证明,这里的$\mu$和用自由焓定义的化学势相同}

{\scriptsize
  我们来计算温度为$T$的处于平衡态的单一成分理想气体的自由焓。
  $$G = U + pV - TS$$
  设第i个态上有$n_i$个分子,总分子数$N=\sum_i n_i$。因为要用到熵的绝对数值,我们就不能像以前那样直接忽略粒子的全同性了(混合气体的熵不同于单一成分气体的熵)。因为置换$N$个粒子共有$N!$种可能性。按照熵是状态数的对数的理解,按可区分粒子计算出来的熵则要减去$\ln (N!) \approx N \ln N - N $。最后,按照热力学的单位制,熵以$k$为单位,于是
  $$\frac{S}{k} = N\sum_i\left(-\frac{n_i}{N}\ln\frac{n_i}{N}\right)  - (N\ln N - N) = N-\sum_i (n_i\ln n_i) = N - \sum_in_i\frac{\mu - \varepsilon_i}{kT} =N+\frac{U-\mu N}{kT}  $$
  即
  $$TS = kNT + U - \mu N = pV + U - \mu N$$
  移项即得结论$$ G = \mu N$$
}
\ech
\end{frame}

\begin{frame}
\chtitle{光子气体的化学势}
\bch
利用光子气体
$$U = aV T^4$$
以及熵
$$ S = \frac{4}{3}aV T^3$$
压强
$$ p = \frac{a}{3} T^4$$
很容易得到自由焓
$$ G = U + pV - TS = 0$$
即化学势为零。

但,光子气体的$U = aVT^4$本身就是我们用化学势为零的假设推导出来的,为了避免循环论证,我们必须用不同的方法进行证明。
\ech
\end{frame}


\begin{frame}
\chtitle{光子气体内能公式的独立推导}
\bch
{\small
把光子气体理解为温度为$T$的黑体辐射,则光子气体的能量密度只能是$T$的函数。我们假设
$$ U = V f(T)$$
其中$f$为待定函数。
我们曾经推导了极端相对论气体的压强公式,并得到了$p = \frac{1}{3} n\overline{\varepsilon} = \frac{U}{3V} $,即
$$p = \frac{1}{3} f(T)$$
由内能和状态方程的关系得到
$$ f(T) = \pfrac UVT = T\pfrac pTV - p = \frac{1}{3} \left[T f'(T) - f(T)\right]$$
化简即
$$ \frac{d \ln f}{d\ln T}  = 4 $$
即$ f = a T^4 $。这个证明无法确定积分常数$a$。但由此可以推出光子气体化学势为零,再按第5讲那样用玻色-爱因斯坦分布可以计算出$a$。
}
\ech
\end{frame}

\section{Review}

\begin{frame}
  \chtitle{知识结构总结}
\addfig{4.3}{summary.png}  
\end{frame}


\begin{frame}
\chtitle{补充知识:Ruchhardt测$\gamma$法}
\bch
参考教材150页图3-19

这是个力学问题,显然由$p$, $V$这些力学量来描述比较合理。选取绝热方程$pV^\gamma = \const$
$$ \frac{dp}{p} + \gamma \frac{dV}{V} = 0$$
把$dp = dF/S$,$dV = S dx$代入上式,
$$ dF = -\gamma \frac{S^2p}{V} dx $$
即等效回复系数$k =\gamma \frac{S^2p}{V} $ 简谐振动圆频率
$$\omega = \sqrt{\frac{k}{m}} = \sqrt{\frac{\gamma p S^2}{mV}}$$
\ech
\end{frame}

\begin{frame}
\chtitle{补充知识:大气垂直温差的修正}
\bch
{\small
对水蒸气饱和的空气,做功等于内能改变量加上汽化热:
$$ -pdV = C_V dT + \Lambda d(c\nu) $$
其中$\nu$为空气摩尔数,$c$为空气中水蒸气的摩尔百分比,$\Lambda$为摩尔汽化热。又
$$pdV + Vdp = \nu R dT$$
两式相加得到
$$ Vdp = C_p  dT + \nu \Lambda dc $$
再由力学平衡$ dp = -\rho g dz$,即有
$$ \frac{dT}{dz} =-\frac{\rho g V + \nu\Lambda\frac{dc}{dz}}{C_p} = -\frac{\gamma-1}{\gamma}\left(\frac{M^{\rm mol}g}{R}+\frac{\Lambda}{R}\frac{dc}{dz}\right)$$
因为饱和蒸气压随着温度降低而降低(请回忆三相图),在有饱和蒸汽的空气里,气团继续上升会造成水蒸气凝结,$ \frac{dc}{dz} < 0 $,垂直温度梯度就会小于$-10\SIK/\SIkm$。
}
\ech
\end{frame}


\begin{frame}
  \chtitle{后记:关于不考的一些内容}
  \bch
  问题1: 教材204页至208页在瞎扯什么,完全看不懂……

  \skiplines
  
  答:是这样的,教材在这节开头忘写了这么一段:

  {\small 假设气态和液态都能用范德瓦尔斯方程描述(看成一种假想的单相态),我们可以计算出理论上的单相$F$-$V$曲线等。实际发生的情况未必和这种假想的单相态理论预言相同,下面我们用自由能判据等来进行判断……}

  \ech
\end{frame}


\begin{frame}
  \chtitle{后记:关于不考的一些内容}
  \bch
  问题2: 直接跳过第五章的知识,是否会对学习后续课程造成影响。

  \skiplines
  
  答:本科阶段肯定不会。研究生阶段基本不会。研究生之后的阶段——你想太多了\bye

  \ech
\end{frame}


\begin{frame}
  \chtitle{后记:班长问我的一道题(非平衡过程)}
  \bch
  如果我们急于喝一杯奶茶,以下哪种冷却方法好?
  \bitem
\item[A]{先把热茶冷5分钟,加一匙冷牛奶。}
\item[B]{先把热茶冷2.5分钟,再加一匙冷牛奶,再冷2.5分钟。}
\item[C]{先将一匙冷牛奶加入热茶中,再冷却5分钟。}
\item[D]{以上效果一样}
  \eitem
  \ech
\end{frame}



\section{Homework}

\begin{frame}
\chtitle{第十二周作业(序号接第十一周)}
\bch
{\small
\bitem
\item[30]{某气体的状态方程为
$$p(V- \nu b) = \nu RT$$
该气体经过准静态的等温膨胀压强减少了一半。求这个过程气体的熵变。}
\item[31]{史瓦西黑洞是一个热力学系统,它的内能正比于质量,熵正比于质量的平方。用热力学第二定律证明一个史瓦西黑洞不可能分裂为两个史瓦西黑洞而不造成其他影响。}
\item[32]{无重力环境下,定温定体容器中装满水蒸气,水蒸气中凝结出一个半径为$R$的悬浮球状液态水滴。这个系统可以看作气相,液相,气液交界处的表面相三相共存。表面相的方程为
  $$ dU = TdS + \sigma dA $$
  其中$\sigma$为水的表面张力系数,$A = 4\pi R^2$为水滴表面积。利用自由能判据推导平衡态下液态水和气态水的压强之差(用$\sigma$, $R$表示)。该系统的相平衡条件还是液态水和气态水的化学势相等吗?为什么?}
\eitem
}
\ech
\end{frame}


\end{document}

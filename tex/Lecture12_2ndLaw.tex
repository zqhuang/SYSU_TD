\documentclass[CJK]{beamer}
\usepackage{CJKutf8}
\usepackage{beamerthemesplit}
\usetheme{Malmoe}
\useoutertheme[footline=authortitle]{miniframes}
\usepackage{amsmath}
\usepackage{amssymb}
\usepackage{graphicx}
\usepackage{eufrak}
\usepackage{color}
\usepackage{slashed}
\usepackage{simplewick}
\usepackage{tikz}
\graphicspath{{../figures/}}
\def\addfig#1#2{\begin{center}\includegraphics[width=#1 in]{#2}\end{center}}
\def\blacktext#1{{\color{black}#1}}
\def\bluetext#1{{\color{blue}#1}}
\def\redtext#1{{\color{red}#1}}
\def\darkbluetext#1{{\color[rgb]{0,0.2,0.6}#1}}
\def\skybluetext#1{{\color[rgb]{0.2,0.7,1.}#1}}
\def\cyantext#1{{\color[rgb]{0.,0.5,0.5}#1}}
\def\greentext#1{{\color[rgb]{0,0.7,0.1}#1}}
\def\darkgray{\color[rgb]{0.2,0.2,0.2}}
\def\lightgray{\color[rgb]{0.6,0.6,0.6}}
\def\gray{\color[rgb]{0.4,0.4,0.4}}
\def\blue{\color{blue}}
\def\red{\color{red}}
\def\green{\color{green}}
\def\darkblue{\color[rgb]{0,0.2,0.6}}
\def\skyblue{\color[rgb]{0.2,0.7,1.}}
\def\fdeg{{^\circ \mathrm{F}}}
\def\cdeg{^\circ \mathrm{C}}
\def\be{\begin{equation}}
\def\ee{\nonumber\end{equation}}
\def\bea{\begin{eqnarray}}
\def\eea{\nonumber\end{eqnarray}}
\def\ii{{\dot{\imath}}}
\def\bch{\begin{CJK}{UTF8}{gbsn}}
\def\ech{\end{CJK}}
\def\bitem{\begin{itemize}}
\def\eitem{\end{itemize}}
\def\bcenter{\begin{center}}
\def\ecenter{\end{center}}
\def\bex{\begin{minipage}{0.3\textwidth}\includegraphics[width=1in]{jugelizi.png}\end{minipage}\begin{minipage}{0.6\textwidth}}
\def\eex{\end{minipage}}
\def\chtitle#1{\frametitle{\bch#1\ech}}
\def\skipline{{\vskip0.1in}}
\def\skiplines{{\vskip0.2in}}
\def\lagr{{\mathcal{L}}}
\def\hamil{{\mathcal{H}}}
\def\vecv{{\mathbf{v}}}
\def\vecx{{\mathbf{x}}}
\def\vecy{{\mathbf{y}}}
\def\veck{{\mathbf{k}}}
\def\vecp{{\mathbf{p}}}
\def\vecn{{\mathbf{n}}}
\def\vecA{{\mathbf{A}}}
\def\vecP{{\mathbf{P}}}
\def\vecsigma{{\mathbf{\sigma}}}
\def\hatJn{{\hat{J_\vecn}}}
\def\hatJx{{\hat{J_x}}}
\def\hatJy{{\hat{J_y}}}
\def\hatJz{{\hat{J_z}}}
\def\hatj#1{\hat{J_{#1}}}
\def\hatphi{{\hat{\phi}}}
\def\hatq{{\hat{q}}}
\def\hatpi{{\hat{\pi}}}
\def\vel{\upsilon}
\def\Dint{{\mathcal{D}}}
\def\adag{{\hat{a}^\dagger}}
\def\bdag{{\hat{b}^\dagger}}
\def\cdag{{\hat{c}^\dagger}}
\def\ddag{{\hat{d}^\dagger}}
\def\hata{{\hat{a}}}
\def\hatb{{\hat{b}}}
\def\hatc{{\hat{c}}}
\def\hatd{{\hat{d}}}
\def\hatN{{\hat{N}}}
\def\hatH{{\hat{H}}}
\def\hatp{{\hat{p}}}
\def\Fup{{F^{\mu\nu}}}
\def\Fdown{{F_{\mu\nu}}}
\def\newl{\nonumber \\}
\def\SIkm{\,\mathrm{km}}
\def\SIyr{\,\mathrm{yr}}
\def\SIGyr{\,\mathrm{Gyr}}
\def\SIeV{\,\mathrm{eV}}
\def\SIkeV{\,\mathrm{keV}}
\def\SIMeV{\,\mathrm{MeV}}
\def\SIGeV{\,\mathrm{GeV}}
\def\SIcal{\,\mathrm{cal}}
\def\SIkcal{\,\mathrm{kcal}}
\def\SImol{\,\mathrm{mol}}
\def\SIm{\,\mathrm{m}}
\def\SIcm{\,\mathrm{cm}}
\def\SIfm{\,\mathrm{fm}}
\def\SImm{\,\mathrm{mm}}
\def\SInm{\,\mathrm{nm}}
\def\SImum{\,\mathrm{\mu m}}
\def\SIJ{\,\mathrm{J}}
\def\SIkJ{\,\mathrm{kJ}}
\def\SIs{\,\mathrm{s}}
\def\SIkg{\,\mathrm{kg}}
\def\SIg{\,\mathrm{g}}
\def\SIK{\,\mathrm{K}}
\def\SImmHg{\,\mathrm{mmHg}}
\def\SIPa{\,\mathrm{Pa}}
\def\vece{\mathrm{e}}
\def\bmat#1{\left(\begin{array}{#1}}
\def\emat{\end{array}\right)}
\def\bcase#1{\left\{\begin{array}{#1}}
\def\ecase{\end{array}\right.}
\def\calM{{\mathcal{M}}}
\def\calT{{\mathcal{T}}}
\def\calR{{\mathcal{R}}}
\def\barpsi{\bar{\psi}}
\def\baru{\bar{u}}
\def\barv{\bar{\upsilon}}
\def\bmini#1{\begin{minipage}{#1\textwidth}}
\def\emini{\end{minipage}}
\def\qeq{\stackrel{?}{=}}
\def\torder#1{\mathcal{T}\left(#1\right)}
\def\rorder#1{\mathcal{R}\left(#1\right)}
\def\contr#1#2{\contraction{}{#1}{}{#2}#1#2}
\def\trof#1{\mathrm{Tr}\left(#1\right)}
\def\trace{\mathrm{Tr}}
\def\comm#1{\ \ \ \left(\mathrm{used}\ #1\right)}
\def\tcomm#1{\ \ \ (\text{#1})}
\def\slp{\slashed{p}}
\def\slk{\slashed{k}}
\def\wulian{\includegraphics[width=0.18in]{emoji_wulian.jpg}}
\def\bye{\includegraphics[width=0.18in]{emoji_bye.jpg}}
\def\calp{{\mathfrak{p}}}
\def\veccalp{\mathbf{\mathfrak{p}}}
\def\atm{\,\mathrm{atm}}
\def\angstrom{\,\text{\AA}}
\def\Tthree{T_{\tiny \textcircled{3}}}
\def\pthree{p_{\tiny \textcircled{3}}}

\def\courseurl{http://zhiqihuang.top}

\def\tpage#1#2{
\begin{frame}
\bch
\begin{center}
\begin{large}
热学 \\
第#1讲 #2

\end{large}

\skiplines

黄志琦


\end{center}

\skiplines

{\small 
教材:《热学》第二版,赵凯华,罗蔚茵,高等教育出版社


课件下载
}
\courseurl 
\ech
\end{frame}
}

\def\bfr#1{
\begin{frame}
\chtitle{#1} 
\bch
}

\def\efr{
\ech 
\end{frame}
}

\title{Lesson 12 - 2$^{\rm nd}$ Law of Thermodynamics}
  \author{}
  \date{}
\begin{document}
\tpage{12}{热力学第二定律}

\section{Review}

\begin{frame}
\chtitle{上讲内容回顾}
\bch 
\bitem
\item{孤立系统的熵增大原理}
\item{循环过程:可逆循环与不可逆循环;正循环与逆循环;}
\item{卡诺定理$\eta \le 1 -\frac{T_2}{T_1}$ (对制冷机则$\varepsilon \le \frac{T_2}{T_1-T_2}$)}
\item{理想气体循环效率的计算}
\eitem
\ech
\end{frame}

\begin{frame}
\chtitle{汤姆孙是谁}
\bch
学习焦耳-汤姆孙效应时,我有个疑问:

\skiplines

焦耳这位dalao我熟,但汤姆孙我怎么没听说过?
\addfig{1.5}{tom.jpg}

\ech
\end{frame}

\begin{frame}
\chtitle{两位dalao}
\bch
\bitem
\item{James Prescott Joule (詹姆斯$\cdot$普雷斯科特 焦耳)

\addfig{0.5}{Joule.jpg}
}
\item{William Thomson, 1st Baron Kelvin (威廉姆$\cdot$汤姆孙,第一男爵开尔文)

\addfig{0.5}{Kelvin.jpg}
}
\eitem
这两位可不止是提出了焦耳-汤姆孙效应(\wulian也叫焦耳-开尔文效应)那么简单
\ech
\end{frame}



\begin{frame}
\chtitle{永动机的梦想}
\bch
\bitem
\item{最早人类并不知道能量守恒定律,总是幻想可以{\bf 无限制产生能量的机器(第一类永动机)}。}
\item{当人们逐渐熟悉能量守恒定律(特别是焦耳确定了热功当量)之后,明白了{\bf 能量不能无中生有},第一类永动机只是痴人说梦。}
\item{同时,人们了解到热量是能量的一种形式,大气,海水和地壳中储藏着大量的热能(内能)。如果这些能量能够无条件地转化为有用的机械功,岂非几乎取之不尽?这类假想的{\bf 可以无条件从单一热源中吸收热量转化为机械功的机器称为第二类永动机。它并不违反能量守恒定律。}}
\item{最后,开尔文提出热力学第二定律的开尔文表述:{\bf 第二类永动机不可能}。}
\eitem

简单讲,两位dalao摧毁了人类的梦想\wulian
\ech
\end{frame}

\begin{frame}
\bch
\addfig{1}{xdldt.jpg}

为了纪念摧毁两类永动机的两位dalao,我们把能量的国际标准单位叫做焦耳($\SIJ$),把温度的国际标准单位叫做开尔文($\SIK$)。
\ech
\end{frame}

\begin{frame}
\chtitle{本讲内容}
\bch
\bitem
\item{热力学第二定律的多种表述方式}
\item{克劳修斯不等式和热力学第二定律的“微分”形式}
\item{热力学第二定律应用举例}
\eitem
\ech
\end{frame}

\section{statements of 2nd law}


\begin{frame}
\chtitle{热力学第二定律的开尔文表述}
\bch
热力学第二定律的{\bf 开尔文表述(Kelvin Statement)}简单地说就是:
\tbox{\bf 第二类永动机不可能。}

或者把第二类永动机的定义进行一番解释的说法就是:
\tbox{\bf 不可能从单一热源吸取热量,使之完全变为有用的功而不产生其他影响。}
\ech
\end{frame}

\begin{frame}
\chtitle{容易漏掉的关键词}
\bch
注意开尔文表述中“单一热源”(指温度唯一的热源)是不可以漏掉的词。如果允许多热源的话,很容易找到反例,例如可逆卡诺循环。
\ech
\end{frame}

\begin{frame}
\chtitle{我们再来谈谈另一位dalao}
\bch

\bcenter
克劳修斯(Rudolf Clausius)
\ecenter

\addfig{1.2}{Clausius.jpg}

\bcenter
$\uparrow$

这位dalao提出了熵
\ecenter
\ech
\end{frame}

\begin{frame}
\chtitle{热力学第二定律的克劳修斯表述}
\bch
热力学第二定律的{\bf 克劳修斯表述(Clausius Statement)}:
\tbox{\bf 不可能把热量从低温物体传到高温物体而不引起其他变化。}
\ech
\end{frame}

\begin{frame}
\chtitle{补充点逻辑学常识}
\bch
如果两个逻辑命题等价:$A=B$, 则两个命题的否命题也等价:$\mathrm{not}\ A = \mathrm{not}\ B$
\ech
\end{frame}

\begin{frame}
\chtitle{开尔文表述等价于克劳修斯表述的证明}
\bch
{\small
\bitem
\item{
开尔文表述的否命题是:

A 可以从单一热源吸热,使之完全变为有用的功而不产生其他影响。
}
\item{
克劳修斯表述的否命题是:

B 可以把热量从低温物体传到高温物体而不引起其他变化。
}
\eitem

只要证明A和B等价,则开尔文表述和克劳修斯表述等价。

先证A $\rightarrow$ B

设A成立,则可以从高温物体吸热,转化为驱动(可逆)制冷机的功,从低温物体吸热传递给高温物体。其净效果为从低温热源吸热传递给高温热源。

\skipline

再证B $\rightarrow$ A

设B成立,则可以从低温物体传热到高温物体,然后在高温物体和低温物体之间通过一个(可逆)卡诺循环把热量返还给低温物体,并额外做功。净效果就是从高温物体吸热做功。
}
\ech
\end{frame}

\begin{frame}
\chtitle{孤立系统的熵增大原理和热力学第二定律等价}
\bch
克劳修斯表述实际上和孤立系统的熵增大原理是一回事情。因为热量$Q$($Q>0$)从低温物体(设温度为$T_2$)向高温物体(设温度为$T_1$)传递,等价于熵减过程
$$\Delta S = Q/T_1 - Q/T_2<0$$
\ech
\end{frame}

\begin{frame}
\chtitle{卡诺定理和热力学第二定律的等价}
\bch
卡诺定理是熵增大原理推出来的,故热二律$\rightarrow$卡诺定理。

反之,若热二律不成立,存在一个热机A可以从单一热源吸热做功而不引起其他变化,则该热机的效率为$1>1-T_2/T_1$,与卡诺定理矛盾。所以卡诺定理$\rightarrow$热二律。

\ech
\end{frame}

\begin{frame}
\chtitle{热力学温标的理想热机定义法}
\bch
如果一开始不引入热力学温标,而仅仅把卡诺定理表述为在两个恒温(“恒温”的定义可以由热力学第零定律得到,不需要具体温标)热源之间工作的热机具有相同效率。则我们可以用可逆热机来定义热力学温标。
即取定某标准点$T_1$后,通过可逆热机效率$\eta$来定义$T_2 \equiv (1-\eta)T_1$。这种定义热力学温标的方法和测温物质无关。
\ech
\end{frame}

\begin{frame}
\chtitle{总结}
\bch
热力学第二定律有多种表述方式:
\bitem
\item{开尔文表述:不可能从单一热源吸热转化为功而无其他影响 (即:第二类永动机不可能)}
\item{克劳修斯表述:不可能低温物体传热给高温物体而无其他影响}
\item{孤立系统的熵增大原理:孤立系统在非平衡态熵会持续增大,直到到达平衡态后熵取到极大值不再改变。}
\item{卡诺定理:所有工作于温度为$T_1$的高温热源和温度为$T_2$的低温热源之间的可逆热机效率均为$1-T_2/T_1$,不可逆热机的效率则低于这个值。}
\eitem
\ech
\end{frame}

\section{Clausius Inequality}

\begin{frame}
\bch
\wulian 感觉今天热学课变成了逻辑课。

\skiplines

下面我们来点干货。
\ech
\end{frame}

\begin{frame}
\chtitle{克劳修斯不等式}
\bch
设在一个循环中,系统依次和温度为$T_1, T_2, \ldots, T_n$的热源接触,分别吸收热量$\dbar Q_1, \dbar Q_2, \ldots, \dbar Q_n$(假设热源很大,始终处于热平衡,所以有明确的温度)。根据孤立系它熵增大原理,系统和环境的总熵必须增大(如果循环不可逆)或者不变(如果循环可逆)。又,熵是态函数,故系统的熵经过循环后不变。那么环境的总熵必须增大(不可逆循环)或者不变(可逆循环)。写成数学表达式:
$$\sum_{i=1}^n \Delta S_i = \sum_{i=1}^n \frac{-\dbar Q_i}{T} \ge 0$$
其中等号对且仅对可逆循环成立。

当热源温度连续变化时,上述结果即称为积分形式的{\bf 克劳修斯不等式}:
{\blue $$\oint \frac{\dbar Q}{T} \le 0 $$}
{\bf 等号对且仅对可逆循环成立。}

\ech
\end{frame}

\begin{frame}
\chtitle{克劳修斯不等式的诠释}
\bch
克劳修斯不等式
{\blue $$\oint \frac{\dbar Q}{T} \le 0 $$}
中的{\bf $T$要理解为环境的温度}。

如果循环是可逆的,
$$\roint \frac{\dbar Q}{T} = 0$$
我们用符号$\rint$来表示可逆过程的积分。

在可逆过程中,热交换是在热平衡情形下进行的,$T$也可解读为系统的温度。这时$\frac{\dbar Q}{T} = dS$。克劳修斯用下述(只依赖于起点和终点,与路径无关的)积分来定义熵
$$S_B-S_A = \rint_A^B \frac{\dbar Q}{T} $$

\ech
\end{frame}

\begin{frame}
\chtitle{热力学第二定律的“微分”形式}
\bch
假设系统和温度为$T_{\rm source}$的热源发生接触(假设热源很大,始终处于热平衡,所以有明确的温度)。在一个微过程中,
设系统的吸热量为$\dbar Q$,则热源吸热量为$-\dbar Q$,热源熵变为
$$ dS_{\rm source} = \frac{-\dbar Q}{T_{\rm source}}$$
把系统和热源的总和看作一个孤立系统,设系统熵变为$dS$,则
$$dS +dS_{\rm source} \ge 0$$
综合上面两式即有
$$\dbar Q \le T_{\rm source} dS $$
等号当且仅当过程可逆时成立。
\ech
\end{frame}

\begin{frame}
\chtitle{热力学第二定律的“微分”形式(续)}
\bch
热力学第二定律常常被写作
{\blue $$\dbar Q \le TdS$$}
或者
{\blue $$ TdS \ge dU + pdV $$}
其中{\bf 等号对且仅对可逆过程成立}。

这个“微分”形式要打上引号,是因为它并不是描述系统的状态参量的一个微分表达式:$T$是热源的温度,而非系统的温度。

如果系统在热交换过程中是和热源处于准热平衡的(即温差足够小,传热过程足够缓慢),也就是{\bf 过程可逆}。即有我们一开始引入熵时的等式:
$$ TdS = dU + pdV$$
这时$T$可以解读为系统的温度(=热源的温度)。
\ech
\end{frame}

\begin{frame}
\chtitle{现在问题来了}
\bch
\addfig{0.8}{think2.jpg}

如果你认真读了教材193页,你会发现,教材中描述$\dbar Q \le TdS$时,是把$T$当成系统温度的,并毫无顾忌地对不可逆过程使用了温度的概念\wulian。
那么如何帮教材自圆其说呢?

\ech
\end{frame}

\begin{frame}
\chtitle{自圆其说}
\bch
设热源恒温。在一个微过程开始和结束时系统都和热源处于准热平衡,温度为$T$,则
$$\dbar Q \le TdS $$
中的$T$可以解读为系统在微过程开始和结束时的温度。

\skiplines

\bye 没毛病。


不过,非把$T$诠释为系统温度,反而限制了这个不等式的应用范围。
\addfig{0.8}{hekune.jpg}
\ech
\end{frame}

\begin{frame}
\chtitle{热力学第二定律应用举例I}
\bch
{\blue 论证摩擦生热过程是不可逆的}

\skiplines

答:摩擦生热的逆过程即为从单一热源吸热,完全转化为机械功,违反热二律开尔文表述。

\ech
\end{frame}

\begin{frame}
\chtitle{热力学第二定律应用举例II}
\bch
{\blue 有人想利用海洋不同深度处温度不同制造一种机器,把海水的内能转化为机械功,这是否违反热力学第二定律?}


\skiplines

答:这是有可能的。这里有温度的不同,不是“单一热源”。
\ech
\end{frame}

\begin{frame}
\chtitle{热力学第二定律应用举例III}
\bch
{\blue 论证绝热线和等温线不能有两个以上的交点。}


\skiplines

答:若有等温线和绝热线有两个交点,则可以构造由一个等温过程过程和绝热过程构成的循环,这个循环中从单一热源吸热并对外做功,违反热二律开尔文表述。

\skipline

(另一种答法:在$TS$图中它们是垂直的两条直线,显然无法相交于两点。)
\ech
\end{frame}

\begin{frame}
\chtitle{热力学第二定律应用举例IV}
\bch
{\blue 恒温定体系统的自由能平衡判据:一个固定体积,并和恒温热源接触的系统,证明它的自由能当处于平衡态时最小。}


\ech
\end{frame}

\begin{frame}
\chtitle{热力学第二定律应用举例IV}
\bch
{\blue 恒温定压系统的自由焓平衡判据:一个固定压强,并和恒温热源接触的系统,证明它的自由焓当处于平衡态时最小。}


\ech
\end{frame}


\begin{frame}
\chtitle{第十二周作业(序号接第十一周)}
\bch

\bitem
\item[30]{教材习题4-16}
\item[31]{教材习题4-20}
\item[32]{教材习题4-27}
\eitem

\ech
\end{frame}


\end{document}

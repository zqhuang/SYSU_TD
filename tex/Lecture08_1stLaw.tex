\documentclass[CJK]{beamer}
\usepackage{CJKutf8}
\usepackage{beamerthemesplit}
\usetheme{Malmoe}
\useoutertheme[footline=authortitle]{miniframes}
\usepackage{amsmath}
\usepackage{amssymb}
\usepackage{graphicx}
\usepackage{eufrak}
\usepackage{color}
\usepackage{slashed}
\usepackage{simplewick}
\usepackage{tikz}
\graphicspath{{../figures/}}
\def\addfig#1#2{\begin{center}\includegraphics[width=#1 in]{#2}\end{center}}
\def\blacktext#1{{\color{black}#1}}
\def\bluetext#1{{\color{blue}#1}}
\def\redtext#1{{\color{red}#1}}
\def\darkbluetext#1{{\color[rgb]{0,0.2,0.6}#1}}
\def\skybluetext#1{{\color[rgb]{0.2,0.7,1.}#1}}
\def\cyantext#1{{\color[rgb]{0.,0.5,0.5}#1}}
\def\greentext#1{{\color[rgb]{0,0.7,0.1}#1}}
\def\darkgray{\color[rgb]{0.2,0.2,0.2}}
\def\lightgray{\color[rgb]{0.6,0.6,0.6}}
\def\gray{\color[rgb]{0.4,0.4,0.4}}
\def\blue{\color{blue}}
\def\red{\color{red}}
\def\green{\color{green}}
\def\darkblue{\color[rgb]{0,0.2,0.6}}
\def\skyblue{\color[rgb]{0.2,0.7,1.}}
\def\fdeg{{^\circ \mathrm{F}}}
\def\cdeg{^\circ \mathrm{C}}
\def\be{\begin{equation}}
\def\ee{\nonumber\end{equation}}
\def\bea{\begin{eqnarray}}
\def\eea{\nonumber\end{eqnarray}}
\def\ii{{\dot{\imath}}}
\def\bch{\begin{CJK}{UTF8}{gbsn}}
\def\ech{\end{CJK}}
\def\bitem{\begin{itemize}}
\def\eitem{\end{itemize}}
\def\bcenter{\begin{center}}
\def\ecenter{\end{center}}
\def\bex{\begin{minipage}{0.3\textwidth}\includegraphics[width=1in]{jugelizi.png}\end{minipage}\begin{minipage}{0.6\textwidth}}
\def\eex{\end{minipage}}
\def\chtitle#1{\frametitle{\bch#1\ech}}
\def\skipline{{\vskip0.1in}}
\def\skiplines{{\vskip0.2in}}
\def\lagr{{\mathcal{L}}}
\def\hamil{{\mathcal{H}}}
\def\vecv{{\mathbf{v}}}
\def\vecx{{\mathbf{x}}}
\def\vecy{{\mathbf{y}}}
\def\veck{{\mathbf{k}}}
\def\vecp{{\mathbf{p}}}
\def\vecn{{\mathbf{n}}}
\def\vecA{{\mathbf{A}}}
\def\vecP{{\mathbf{P}}}
\def\vecsigma{{\mathbf{\sigma}}}
\def\hatJn{{\hat{J_\vecn}}}
\def\hatJx{{\hat{J_x}}}
\def\hatJy{{\hat{J_y}}}
\def\hatJz{{\hat{J_z}}}
\def\hatj#1{\hat{J_{#1}}}
\def\hatphi{{\hat{\phi}}}
\def\hatq{{\hat{q}}}
\def\hatpi{{\hat{\pi}}}
\def\vel{\upsilon}
\def\Dint{{\mathcal{D}}}
\def\adag{{\hat{a}^\dagger}}
\def\bdag{{\hat{b}^\dagger}}
\def\cdag{{\hat{c}^\dagger}}
\def\ddag{{\hat{d}^\dagger}}
\def\hata{{\hat{a}}}
\def\hatb{{\hat{b}}}
\def\hatc{{\hat{c}}}
\def\hatd{{\hat{d}}}
\def\hatN{{\hat{N}}}
\def\hatH{{\hat{H}}}
\def\hatp{{\hat{p}}}
\def\Fup{{F^{\mu\nu}}}
\def\Fdown{{F_{\mu\nu}}}
\def\newl{\nonumber \\}
\def\SIkm{\,\mathrm{km}}
\def\SIyr{\,\mathrm{yr}}
\def\SIGyr{\,\mathrm{Gyr}}
\def\SIeV{\,\mathrm{eV}}
\def\SIkeV{\,\mathrm{keV}}
\def\SIMeV{\,\mathrm{MeV}}
\def\SIGeV{\,\mathrm{GeV}}
\def\SIcal{\,\mathrm{cal}}
\def\SIkcal{\,\mathrm{kcal}}
\def\SImol{\,\mathrm{mol}}
\def\SIm{\,\mathrm{m}}
\def\SIcm{\,\mathrm{cm}}
\def\SIfm{\,\mathrm{fm}}
\def\SImm{\,\mathrm{mm}}
\def\SInm{\,\mathrm{nm}}
\def\SImum{\,\mathrm{\mu m}}
\def\SIJ{\,\mathrm{J}}
\def\SIkJ{\,\mathrm{kJ}}
\def\SIs{\,\mathrm{s}}
\def\SIkg{\,\mathrm{kg}}
\def\SIg{\,\mathrm{g}}
\def\SIK{\,\mathrm{K}}
\def\SImmHg{\,\mathrm{mmHg}}
\def\SIPa{\,\mathrm{Pa}}
\def\vece{\mathrm{e}}
\def\bmat#1{\left(\begin{array}{#1}}
\def\emat{\end{array}\right)}
\def\bcase#1{\left\{\begin{array}{#1}}
\def\ecase{\end{array}\right.}
\def\calM{{\mathcal{M}}}
\def\calT{{\mathcal{T}}}
\def\calR{{\mathcal{R}}}
\def\barpsi{\bar{\psi}}
\def\baru{\bar{u}}
\def\barv{\bar{\upsilon}}
\def\bmini#1{\begin{minipage}{#1\textwidth}}
\def\emini{\end{minipage}}
\def\qeq{\stackrel{?}{=}}
\def\torder#1{\mathcal{T}\left(#1\right)}
\def\rorder#1{\mathcal{R}\left(#1\right)}
\def\contr#1#2{\contraction{}{#1}{}{#2}#1#2}
\def\trof#1{\mathrm{Tr}\left(#1\right)}
\def\trace{\mathrm{Tr}}
\def\comm#1{\ \ \ \left(\mathrm{used}\ #1\right)}
\def\tcomm#1{\ \ \ (\text{#1})}
\def\slp{\slashed{p}}
\def\slk{\slashed{k}}
\def\wulian{\includegraphics[width=0.18in]{emoji_wulian.jpg}}
\def\bye{\includegraphics[width=0.18in]{emoji_bye.jpg}}
\def\calp{{\mathfrak{p}}}
\def\veccalp{\mathbf{\mathfrak{p}}}
\def\atm{\,\mathrm{atm}}
\def\angstrom{\,\text{\AA}}
\def\Tthree{T_{\tiny \textcircled{3}}}
\def\pthree{p_{\tiny \textcircled{3}}}

\def\courseurl{http://zhiqihuang.top}

\def\tpage#1#2{
\begin{frame}
\bch
\begin{center}
\begin{large}
热学 \\
第#1讲 #2

\end{large}

\skiplines

黄志琦


\end{center}

\skiplines

{\small 
教材:《热学》第二版,赵凯华,罗蔚茵,高等教育出版社


课件下载
}
\courseurl 
\ech
\end{frame}
}

\def\bfr#1{
\begin{frame}
\chtitle{#1} 
\bch
}

\def\efr{
\ech 
\end{frame}
}

\title{Lesson 08 - 1st Law of Thermaldynamics}
  \author{}
  \date{}
\begin{document}
\tpage{8}{热力学第一定律}


\section{Review}

\begin{frame}
\chtitle{上讲内容回顾}
\bch
{\large
\bitem
\item{能均分定理的严谨表述.}
\item{气体/固体的$\cvmol$:单原子气体$\frac{3R}{2}$,双原子气体室温下一般是$\frac{5R}{2}$,固体一般是$3R$}
\eitem
}
\ech
\end{frame}


\begin{frame}
\chtitle{思考题}
\bch

\addfig{0.8}{songfen.jpg}

{\large
  为什么讨论气体摩尔热容时要强调是{\bf 定体}摩尔热容?如果不限定体积不变,热容会有所不同吗?
  
}
\ech
\end{frame}


\begin{frame}
\chtitle{定压摩尔热容$\cpmol$}
\bch
{\large
  {\blue 定压摩尔热容$\cpmol$定义为$1\SImol$物质在等压过程中升高$1\SIK$需要吸收的热量。}

  \addfig{1}{think3.jpg}
  
  对理想气体证明:$\cpmol = \cvmol + R$.
  
}
\ech
\end{frame}


\begin{frame}
\chtitle{本讲内容}
\bch
\bitem
\item{内能}
\item{热力学第一定律}
\eitem
\ech
\end{frame}


\section{Internal Energy}

\secpage{内能}{物质内部的热运动能量和分子间势能的总和}

\begin{frame}
\chtitle{内能是个态函数}
\bch
{\large
\bitem
\item{\bf 态函数里的“态”:指热平衡态。}
\item{\bf 态函数:由系统的宏观状态参量确定,和如何达到这个状态的过程无关。}
\item{\bf 内能:物质内部的热运动能量和分子间势能的总和。}  
\eitem
}
\ech
\end{frame}

\begin{frame}
\chtitle{思考题}
\bch
{\large
下面的量是不是态函数
\bitem
\item{温度$T$}
\item{热量$Q$}
\item{压强$p$}
\item{体积$V$}
\item{做功$A$}
\item{摩尔数$\nu$}
\item{平衡态的分子平均速率$\overline{\upsilon}$}
  \eitem
  }
\ech
\end{frame}

\begin{frame}
\chtitle{经典理想气体的内能只跟温度和摩尔数有关}
\bch

{\large
  在经典图像下,温度$T\rightarrow 0$时,物质的内能$U\rightarrow 0$,所以内能对摩尔数$\nu$,体积$V$和温度$T$的依赖关系为:

  {\blue  $$ U(\nu, V, T) = \nu \int_0^T\cvmol(T',V) dT' $$ }

  {\blue 理想气体}分子间势能为零,$\cvmol$(根据能均分定理)只和温度有关,因此内能
  {\blue  $$ U_{\rm ideal\ gas} = \nu \int_0^T\cvmol(T') dT' $$ }
    只依赖于摩尔数$\nu$和温度$T$。
}
\ech
\end{frame}

\begin{frame}
\chtitle{广延量和强度量}
\bch

{\large
  \bitem
\item{像内能这样和物质的量(摩尔数)成正比的量称为{\bf 广延量}}

\item{而像温度,压强这样和物质的量无关的称为{\bf 强度量}}

  \eitem

  
  判断小技巧:把物质砍掉一半(忽略两半之间的相互作用有可能带来的微弱影响),不变的量是强度量,变成一半的是广延量。
}
\ech
\end{frame}

\begin{frame}
\chtitle{思考题}
\bch
\addfig{1}{songfen.jpg}

{\large
  当温度$T<T_0$时,某双原子气体的定体摩尔热容近似为
  $$\cvmol = \left(\frac{3}{2} + \frac{T^2}{T_0^2}\right)R.$$
  按内能零点取在$T\rightarrow 0$的习惯,计算$1\SImol$该气体在$T<T_0$时的内能。
}
\ech
\end{frame}



\begin{frame}
\bch
    {\huge
因为内能和摩尔数成简单地线性关系,在下面的讨论中,我们默认摩尔数不变,只考虑内能对温度,压强,体积的依赖。
      }

\ech
\end{frame}

\begin{frame}
\chtitle{思考题}
\bch
\addfig{1}{songfen.jpg}

{\large
考虑实际气体分子之间有微弱的吸引力,实际气体的内能$U(V,T)$对体积有微弱的依赖。问:$\left(\frac{\partial U}{\partial V}\right)_T$一般是正的还是负的?
}
\skiplines

注:{\blue 偏微分$\pfrac ABC$表示保持$C$不变时,$A$和$B$的小变化量之比}。显然,这样写的前提是{\blue 默认了$A,B,C$受某个状态方程约束}。
\ech
\end{frame}


\begin{frame}
\chtitle{思考题}
\bch
\addfig{1}{songfen2.jpg}

{\large
由于$p, V, T$中只有两个是独立的,我们也可以把内能写成$U(p, T)$。那么实际气体的$\pfrac UpT$的符号一般是什么呢?
}

\ech
\end{frame}

\begin{frame}
\chtitle{百前的困惑}
\bch
我们用脚趾头就想出来了实际气体$\pfrac UVT > 0$(或者等价地$\pfrac UpT < 0$)

而在一两百年之前,物理学家们还在焦头烂额地做实验测量实际气体的$\pfrac UVT $:
\bmini{0.35}
盖吕萨克实验: 打开活塞,空气自由膨胀后测量温度变化。

请用脚趾头思考:气体自由绝热膨胀(假设瓶子是绝热的)后温度应该上升还是下降?
\emini
\bmini{0.6}
\addfig{1.8}{GLexperiment.jpg}
\emini

\ech
\end{frame}


\begin{frame}
\chtitle{盖吕萨克实验的结果}
\bch

\addfig{1.5}{cry2.jpg}

装置太简陋(漏),没测出温度变化 
\ech
\end{frame}

\begin{frame}
\chtitle{焦耳实验}
\bch
焦耳意识到瓶子可能漏热,就……

\bmini{0.35}
焦耳实验: 打开活塞,空气自由膨胀后测量水温的变化。
\emini
\bmini{0.6}
\addfig{2}{Joule_experiment.jpg}
\emini

\ech
\end{frame}


\begin{frame}
\chtitle{焦耳实验的结果}
\bch

\addfig{1.5}{cry1.png}

温度计不够准,没测出温度变化 
\ech
\end{frame}


\begin{frame}
\chtitle{后来}
\bch
1932年,Rossini和Frandsen改进了焦耳实验,终于测出了
$$\pfrac UpT < 0 $$

\skiplines

因为测出来了很没意思,细节不想讲\bye
\ech
\end{frame}

\begin{frame}
\chtitle{什么,你很想看?}
\bch
\bmini{0.42}
\addfig{1.8}{njg_experiment.png}
\emini
\bmini{0.53}
Rossini-Frandsen experiment

\lfig{2.2}{RossiniFrandsen.jpg}
\emini

\ech
\end{frame}

\section{1st Law}

\secpage{热力学第一定律}{能量守恒的一种形式}


\begin{frame}
\chtitle{热力学第一定律}
\bch
内能变化 = 吸收热量 + 环境对系统做功

$$ \Delta U =  Q + A $$

\skiplines

这里功$A$是{\blue 广义功},可以是机械功$-\int pdV$,也可以是电流做功$\int UIdt$等。
\ech
\end{frame}

\begin{frame}
\chtitle{准静态过程}
\bch
准静态过程的定义:{\blue 进行得足够缓慢,以至于系统连续经过的每个中间态都可以近似看成平衡态。}

\skiplines

{\small 例如:缓慢加热的过程,缓慢压缩气体的过程,政府工作人员的办公过程等等}

\addfig{2}{slowmotion.jpg}

\ech
\end{frame}


\begin{frame}
\chtitle{准静态过程的热力学第一定律表述}
\bch
{\large
准静态过程中内能在过程中间都是有定义的,所以:
{\blue
$$dU = \dbar A + \dbar Q$$
}
热学里特有的符号$\dbar$代表这个微元和过程有关。
}
\ech
\end{frame}


\begin{frame}
\chtitle{$p$-$V$图和做的功}
\bch
准静态过程可以用在$p$-$V$图上的一条曲线描述。气体对外界做功
$$ A' = \int p dV $$
等于$p$-$V$曲线下的面积
\addfig{2}{pdVwork.png}

\ech
\end{frame}

\section{Application}

\begin{frame}
\chtitle{理想气体等温过程}
\bch
理想气体的等温过程比较简单,利用理想气体状态方程$pV = \nu RT$即可求出对理想气体做的功
$$ A = -\int p dV = -\nu R T \int \frac{dV}{V} =- \nu R T\ln\frac{V_{\rm fin}}{V_{\rm ini}}$$
如果没有额外的自由度被激发,理想气体的内能不变。则可推算出等温过程吸收的热量为
$$ Q = - A $$
\ech
\end{frame}


\begin{frame}
\chtitle{理想气体绝热过程(adiabatic process)}
\bch
{\small
理想气体的绝热过程则稍显复杂,由$\dbar Q=0$得到
$$ dU = -p dV$$
又$$ dU = \nu \cvmol dT = \frac{\cvmol}{R}(p dV + V dp)$$
两式相减得到
$$ \cvmol Vdp + \cpmol p dV = 0$$
其中$\cpmol = \cvmol + R$是摩尔定压热容。
记$$\gamma = \frac{\cpmol}{\cvmol}$$
如果把$\gamma$近似当成常数(在日常条件下这个近似往往很不错),由上述方程可推出{\blue 理想气体绝热状态方程}(又称{\bf 泊松公式})
{\blue 
$$p V^\gamma = \const$$}
}
\ech
\end{frame}


\begin{frame}
\chtitle{各种气体的$\gamma$}
\bch
{\small
对单原子理想气体,$\gamma = \frac{5}{3}$。

\skipline

对室温下的双原子理想气体,$\gamma \approx \frac{7}{5}$。

\skipline

因空气大部分都是氮气和氧气(双原子分子),所以我们可以近似认为室温下空气的$\gamma = \frac{7}{5}$。
}

由$pV^\gamma = \const $以及理想气体状态方程可推出绝热状态方程的另外两个形式:

$$ TV^{\gamma-1} = \const$$

$$T p^{\frac{1}{\gamma}-1} = \const $$
\ech
\end{frame}

\begin{frame}
\chtitle{空气中声速}
\bch
\addfig{1.5}{soundwave.jpg}

{\scriptsize
力学里无敌公式$F = ma$:

$$ -dp dy dz = (\rho dx dy dz) \left(\frac{d\upsilon}{dx/\upsilon}\right) $$
即
$$ dp = - \rho \upsilon d\upsilon $$

然后根据物质流守恒:$ \rho \upsilon = (\rho + d\rho)(\upsilon+d\upsilon)$
忽略高阶小量即$ -\rho d\upsilon = \upsilon d\rho$
代入前面的$dp$表达式得到:
$$ dp = \upsilon ^2 d\rho$$
即
$$\upsilon = \sqrt{\frac{dp}{d\rho}}$$
}
\ech
\end{frame}

\begin{frame}
\chtitle{空气中声速(续)}
\bch
{\small
空气是热的不良导体,故做绝热近似 
$$p\rho^{-\gamma} = \const$$
即 $$\frac{dp}{d\rho} = \gamma \frac{p}{\rho} =  \frac{\gamma R T}{M^{\rm mol}}$$
其中空气摩尔质量$M^{\rm mol} = 0.0289 \SIkg/\SImol$
算出空气中声速为
$$c_s = \sqrt{\frac{\gamma RT}{M^{\rm mol}}} = 347 \sqrt{\frac{T}{300\SIK}} \SIm/\SIs$$
注意到$\frac{RT}{M^{\rm mol}} = \frac{kT}{m}$,所以声速也能写成$ \sqrt{\frac{\gamma kT}{m}}$。
因此声速和空气分子方均根速率之比为(见课本习题3-21)
$$\sqrt{\frac{\gamma}{3}} = \sqrt{\frac{7}{15}} = 0.683$$
}
\ech
\end{frame}

\begin{frame}
\chtitle{高处不胜寒}
\bch
我们以前计算大气压强梯度时把空气温度当成了常数,事实上我们都知道“高处不胜寒”。

\addfig{2}{icemountain.jpg}

因为大气是热的不良导体,绝热近似是更好的描述。
\ech
\end{frame}

\begin{frame}
  \chtitle{高处不胜寒}
\bch
由力学平衡有
$$dp = -\rho g dz$$
由$Tp^{\frac{1}{\gamma}-1}= \const$,可得$dT = \left(1-\frac{1}{\gamma}\right)\frac{T}{p} dp $,
故
$$\frac{dT}{dz} = -\frac{\gamma - 1}{\gamma} \frac{T}{p}\rho g= -\frac{\gamma - 1}{\gamma} \frac{V}{\nu R}\rho g =  -\frac{\gamma - 1}{\gamma} \frac{M^{\rm mol}}{ R} g$$
取$\gamma  = 7/5$, $M^{\rm mol} = 29\SIg/\SImol$,$g = 9.8 \mathrm{N}/\SIkg$得到
$$\frac{dT}{dz} \approx -10\SIK/\SIkm$$
这个结果的数量级是正确的,但实际温度梯度往往比它小。空气里的饱和水蒸气是一个重要的影响因素(见教材152页)。
\ech
\end{frame}






\end{document}

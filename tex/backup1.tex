\documentclass[CJK]{beamer}
\usepackage{CJKutf8}
\usepackage{beamerthemesplit}
\usetheme{Malmoe}
\useoutertheme[footline=authortitle]{miniframes}
\usepackage{amsmath}
\usepackage{amssymb}
\usepackage{graphicx}
\usepackage{eufrak}
\usepackage{color}
\usepackage{slashed}
\usepackage{simplewick}
\usepackage{tikz}
\graphicspath{{../figures/}}
\def\addfig#1#2{\begin{center}\includegraphics[width=#1 in]{#2}\end{center}}
\def\blacktext#1{{\color{black}#1}}
\def\bluetext#1{{\color{blue}#1}}
\def\redtext#1{{\color{red}#1}}
\def\darkbluetext#1{{\color[rgb]{0,0.2,0.6}#1}}
\def\skybluetext#1{{\color[rgb]{0.2,0.7,1.}#1}}
\def\cyantext#1{{\color[rgb]{0.,0.5,0.5}#1}}
\def\greentext#1{{\color[rgb]{0,0.7,0.1}#1}}
\def\darkgray{\color[rgb]{0.2,0.2,0.2}}
\def\lightgray{\color[rgb]{0.6,0.6,0.6}}
\def\gray{\color[rgb]{0.4,0.4,0.4}}
\def\blue{\color{blue}}
\def\red{\color{red}}
\def\green{\color{green}}
\def\darkblue{\color[rgb]{0,0.2,0.6}}
\def\skyblue{\color[rgb]{0.2,0.7,1.}}
\def\fdeg{{^\circ \mathrm{F}}}
\def\cdeg{^\circ \mathrm{C}}
\def\be{\begin{equation}}
\def\ee{\nonumber\end{equation}}
\def\bea{\begin{eqnarray}}
\def\eea{\nonumber\end{eqnarray}}
\def\ii{{\dot{\imath}}}
\def\bch{\begin{CJK}{UTF8}{gbsn}}
\def\ech{\end{CJK}}
\def\bitem{\begin{itemize}}
\def\eitem{\end{itemize}}
\def\bcenter{\begin{center}}
\def\ecenter{\end{center}}
\def\bex{\begin{minipage}{0.3\textwidth}\includegraphics[width=1in]{jugelizi.png}\end{minipage}\begin{minipage}{0.6\textwidth}}
\def\eex{\end{minipage}}
\def\chtitle#1{\frametitle{\bch#1\ech}}
\def\skipline{{\vskip0.1in}}
\def\skiplines{{\vskip0.2in}}
\def\lagr{{\mathcal{L}}}
\def\hamil{{\mathcal{H}}}
\def\vecv{{\mathbf{v}}}
\def\vecx{{\mathbf{x}}}
\def\vecy{{\mathbf{y}}}
\def\veck{{\mathbf{k}}}
\def\vecp{{\mathbf{p}}}
\def\vecn{{\mathbf{n}}}
\def\vecA{{\mathbf{A}}}
\def\vecP{{\mathbf{P}}}
\def\vecsigma{{\mathbf{\sigma}}}
\def\hatJn{{\hat{J_\vecn}}}
\def\hatJx{{\hat{J_x}}}
\def\hatJy{{\hat{J_y}}}
\def\hatJz{{\hat{J_z}}}
\def\hatj#1{\hat{J_{#1}}}
\def\hatphi{{\hat{\phi}}}
\def\hatq{{\hat{q}}}
\def\hatpi{{\hat{\pi}}}
\def\vel{\upsilon}
\def\Dint{{\mathcal{D}}}
\def\adag{{\hat{a}^\dagger}}
\def\bdag{{\hat{b}^\dagger}}
\def\cdag{{\hat{c}^\dagger}}
\def\ddag{{\hat{d}^\dagger}}
\def\hata{{\hat{a}}}
\def\hatb{{\hat{b}}}
\def\hatc{{\hat{c}}}
\def\hatd{{\hat{d}}}
\def\hatN{{\hat{N}}}
\def\hatH{{\hat{H}}}
\def\hatp{{\hat{p}}}
\def\Fup{{F^{\mu\nu}}}
\def\Fdown{{F_{\mu\nu}}}
\def\newl{\nonumber \\}
\def\SIkm{\,\mathrm{km}}
\def\SIyr{\,\mathrm{yr}}
\def\SIGyr{\,\mathrm{Gyr}}
\def\SIeV{\,\mathrm{eV}}
\def\SIkeV{\,\mathrm{keV}}
\def\SIMeV{\,\mathrm{MeV}}
\def\SIGeV{\,\mathrm{GeV}}
\def\SIcal{\,\mathrm{cal}}
\def\SIkcal{\,\mathrm{kcal}}
\def\SImol{\,\mathrm{mol}}
\def\SIm{\,\mathrm{m}}
\def\SIcm{\,\mathrm{cm}}
\def\SIfm{\,\mathrm{fm}}
\def\SImm{\,\mathrm{mm}}
\def\SInm{\,\mathrm{nm}}
\def\SImum{\,\mathrm{\mu m}}
\def\SIJ{\,\mathrm{J}}
\def\SIkJ{\,\mathrm{kJ}}
\def\SIs{\,\mathrm{s}}
\def\SIkg{\,\mathrm{kg}}
\def\SIg{\,\mathrm{g}}
\def\SIK{\,\mathrm{K}}
\def\SImmHg{\,\mathrm{mmHg}}
\def\SIPa{\,\mathrm{Pa}}
\def\vece{\mathrm{e}}
\def\bmat#1{\left(\begin{array}{#1}}
\def\emat{\end{array}\right)}
\def\bcase#1{\left\{\begin{array}{#1}}
\def\ecase{\end{array}\right.}
\def\calM{{\mathcal{M}}}
\def\calT{{\mathcal{T}}}
\def\calR{{\mathcal{R}}}
\def\barpsi{\bar{\psi}}
\def\baru{\bar{u}}
\def\barv{\bar{\upsilon}}
\def\bmini#1{\begin{minipage}{#1\textwidth}}
\def\emini{\end{minipage}}
\def\qeq{\stackrel{?}{=}}
\def\torder#1{\mathcal{T}\left(#1\right)}
\def\rorder#1{\mathcal{R}\left(#1\right)}
\def\contr#1#2{\contraction{}{#1}{}{#2}#1#2}
\def\trof#1{\mathrm{Tr}\left(#1\right)}
\def\trace{\mathrm{Tr}}
\def\comm#1{\ \ \ \left(\mathrm{used}\ #1\right)}
\def\tcomm#1{\ \ \ (\text{#1})}
\def\slp{\slashed{p}}
\def\slk{\slashed{k}}
\def\wulian{\includegraphics[width=0.18in]{emoji_wulian.jpg}}
\def\bye{\includegraphics[width=0.18in]{emoji_bye.jpg}}
\def\calp{{\mathfrak{p}}}
\def\veccalp{\mathbf{\mathfrak{p}}}
\def\atm{\,\mathrm{atm}}
\def\angstrom{\,\text{\AA}}
\def\Tthree{T_{\tiny \textcircled{3}}}
\def\pthree{p_{\tiny \textcircled{3}}}

\def\courseurl{http://zhiqihuang.top}

\def\tpage#1#2{
\begin{frame}
\bch
\begin{center}
\begin{large}
热学 \\
第#1讲 #2

\end{large}

\skiplines

黄志琦


\end{center}

\skiplines

{\small 
教材:《热学》第二版,赵凯华,罗蔚茵,高等教育出版社


课件下载
}
\courseurl 
\ech
\end{frame}
}

\def\bfr#1{
\begin{frame}
\chtitle{#1} 
\bch
}

\def\efr{
\ech 
\end{frame}
}

\title{Lesson 05 - Maxwell Distribution}
  \author{}
  \date{}
\begin{document}
\tpage{5}{麦克斯韦分布}

\section{Reivew}

\begin{frame}
\chtitle{上讲内容回顾}
\bch
{\large
\bitem
\item{概率密度函数}
\item{高斯分布}
\eitem}
\ech
\end{frame}

\begin{frame}
\chtitle{本讲内容}
\bch
\bitem
\item{径向概率密度函数}
\item{相空间的微观态}
\item{麦克斯韦分布}
\item{麦克斯韦-玻尔兹曼分布}  
\eitem
\ech
\end{frame}

\section{Radial PDF}


\secpage{径向概率密度函数}{$$F(r) = \frac{G_1}{G_{n-1}} r^{n-1} e^{-r^2/2}.$$}

\begin{frame}
\bch
{\large
  我们先来讨论一下一个专题:{\bf 径向概率密度函数}。

  \skiplines
  
  简单讲,它就是{\bf 随机点到原点距离$r$的概率密度函数$F(r)$}。

  \skiplines

  显然对$r < 0$, $F(r)=0$。这个简单事实我们之后将不再重复,默认只讨论$r\ge 0$的情形。 
}
\ech
\end{frame}


\begin{frame}
\chtitle{径向概率密度函数:一维情形}
\bch
\addfig{0.6}{think1.jpg}
{\large 
  如果$x$服从标准正态分布
  $$f(x) = \frac{1}{\sqrt{2\pi}}e^{-\frac{x^2}{2}},$$
  定义新的随机变量$r=|x|$(即$r$是“到原点的距离”),那么$r$的概率密度函数$F(r)$是怎样的?}
\ech
\end{frame}


\begin{frame}
\chtitle{解答}
\bch
根据上节课所讲的随机变量替换法:
{\large
  $$F(r) |dr| = \sum_{x\rightarrow r} f(x) |dx| = f(r)|dr| + f(-r)|dr|,$$
  所以
  $$F(r) = f(r)+f(-r) = \sqrt{\frac{2}{\pi}}e^{-\frac{r^2}{2}}.$$
}
\ech
\end{frame}



\begin{frame}
\chtitle{讨论}
\bch
{\large
回顾一开始写出的等式:
$$F(r) |dr| = \sum_{x\rightarrow r} f(x) |dx| $$
等式左边的物理意义是:根据概率密度函数$F(r)$的定义计算随机点$x$落在离原点距离$r$和$r+dr$之间的概率。

\skipline

等式右边的物理意义是:根据概率密度函数$f(x)$的定义计算随机点$x$落在离原点距离$r$和
$r+dr$之间的概率 (通过列举所有满足条件的$x$得到结果)。

\skipline
因为同一个统计事件的概率是唯一的,所以左边等于右边。

\skipline

 这种计算方法可以推广到任意维: 
}
\ech
\end{frame}


\begin{frame}
\chtitle{思考题}
\bch
\addfig{0.6}{think1.jpg}
{\large
如果$x$, $y$分别都服从标准正态分布,且$x,y$之间没有统计关联,定义新的随机变量$r=\sqrt{x^2+y^2}$,那么$r$的概率密度函数是怎样的?
}
\ech
\end{frame}


\begin{frame}
\chtitle{解答}
    \bch
    {\large
      $x$的概率密度函数为$\frac{1}{\sqrt{2\pi}}e^{-x^2/2}$; $y$的概率密度函数为$\frac{1}{\sqrt{2\pi}}e^{-y^2/2}$。

      \skipline
      
      根据上节课所讲,{\bf 独立}随机变量的概率密度可以直接进行反投影,得到二元随机变量$(x,y)$的概率密度函数:

      $$ f(x,y) = \frac{1}{2\pi}e^{-(x^2+y^2)/2} = \frac{1}{2\pi}e^{-r^2/2}.$$

      {\bf 注意:这里为了偷懒把结果形式上写成了$r$的函数,但它是$(x,y)$的概率密度函数,不是$r$的概率密度函数。。}}
\ech
\end{frame}

\begin{frame}
\chtitle{解答(续)}
\bch
\bmini{0.5}
\lfig{2}{ringprob.png}
\emini
\bmini{0.45}
{\large
  如图,考虑距离原点为$r$,宽度为$dr\ll r$的一个圆环形区域。在这个区域内,二元随机变量$(x,y)$的概率密度处处都是$f(x,y)=\frac{1}{2\pi}e^{-r^2/2}$,所以$(x,y)$落在圆环内的概率为}
\emini
    {
     \large
  $$ f(x,y) (2\pi r dr) = \frac{1}{2\pi}e^{-r^2/2} (2\pi r dr) = re^{-r^2/2}dr. $$
  

  另一方面,根据$r$的概率密度函数$F(r)$的定义,上述事件的概率又等于 $F(r)dr$,根据统计事件概率唯一性得到
  $$F(r) = r e^{-r^2/2}. $$

  (注意:只要$f(x,y)$是归一化的,那么推导出来的$F(r)$的归一化是自然满足的,请自行验证。)
}
\ech
\end{frame}


\begin{frame}
\chtitle{思考题}
\bch
\addfig{0.6}{think1.jpg}
{\large
如果$x$, $y$, $z$分别都服从标准正态分布,且$x,y,z$之间没有统计关联,定义新的随机变量$r=\sqrt{x^2+y^2+z^2}$,那么$r$的概率密度函数是怎样的?
}
\ech
\end{frame}


\begin{frame}
\chtitle{解答}
    \bch
        {\large
          同样地,三元随机变量$(x,y,z)$的概率密度函数为

          $$ f(x,y,z) = \frac{1}{(2\pi)^{3/2}}e^{-r^2/2} $$

          考虑$(x,y,z)$落在距离原点为$r$,厚度$dr\ll r$的一个薄球壳内的概率,就有
          $$ \frac{1}{(2\pi)^{3/2}}e^{-r^2/2} (4\pi r^2dr) = F(r) dr $$
          即
          $$ F(r)=\sqrt{\frac{2}{\pi}} r^2 e^{-r^2/2}. $$

}
\ech
\end{frame}


\begin{frame}
\chtitle{一般维数}
\bch
{\large
在$n$维空间里,如果$n$维球的体积为$V_n(r)=S_n r^n$ ($S_n$为常数,例如$S_1 = 2$, $S_2 = \pi$, $S_3 = \frac{4\pi}{3}$\ldots),则$n$维“薄球壳”的体积为
$$ dV_n = n S_n r^{n-1} dr $$
用同样的方法可以得到
$$F(r) = \frac{nS_n}{(2\pi)^{n/2}} r^{n-1} e^{-r^2/2} \equiv C_n r^{n-1} e^{-r^2/2}. $$
其中$C_n$定义为$\frac{nS_n}{(2\pi)^{n/2}}$。

\lfig{0.5}{think1.jpg}对一般的$n$,你有什么办法可以求出$C_n$的确切值吗?}
\ech
\end{frame}


\begin{frame}
\chtitle{归一化}
\bch
{\large
  利用归一化条件
  $$C_n \int_0^\infty r^{n-1}e^{-r^2/2}dr = 1.$$
  以及上节课学习过的
  $$G_{n-1} = \int_{-\infty}^\infty \frac{1}{\sqrt{2\pi}}|x|^{n-1}e^{-x^2/2}dx =  \sqrt{\frac{2}{\pi}} \int_0^\infty x^{n-1}e^{-x^2/2}dx $$
  可以得到归一化系数
  \tbox{$$C_n = \sqrt{\frac{2}{\pi}}\frac{1}{G_{n-1}}. $$}
  }
\ech
\end{frame}

\begin{frame}
\chtitle{归一化}
\bch

      上面的结果也可以写成
        \tbox{$$C_n = \frac{G_1}{G_{n-1}}. $$}
      前几个$C_n$是:
      \bea
      C_1 &=& \sqrt{\frac{2}{\pi}} \newl
      C_2 &=& 1 \newl
      C_3 &=& \sqrt{\frac{2}{\pi}} \newl
      C_4 &=& \frac{1}{2} \nonumber
      \eea

      (我们还顺便得到了四维空间的球体积等于$\frac{\pi^2}{2}r^4$。)
\ech
\end{frame}


\begin{frame}
\chtitle{总结}
\bch
在$n$维空间的随机点,如果每个维度的分量均独立地满足标准正态分布,则径向概率密度函数为
\tbox{$$ F(r) = \frac{G_1}{G_{n-1}} r^{n-1} e^{-r^2/2}.$$}

当然,如果你不喜欢死记硬背,对比较小的$n$,直接进行推导也花不了多少功夫。
\ech
\end{frame}


\section{Microstates in phase space}

\secpage{相空间的微观态}{相空间=坐标空间$\otimes$动量空间}

\begin{frame}
\chtitle{态的例子}
\bch
\bitem
\item{硬币有两种态:正面朝上,反面朝上

\addfig{0.7}{coins.jpg}
}
\item{骰子有六种态:1,2,3,4,5,6

\addfig{0.3}{touzi.png}
}
\item{同学们有很多种态:上课,吃饭,睡觉,和床引力战斗……

\addfig{1.5}{xbxstates.jpg}
}

\eitem
\ech
\end{frame}

\begin{frame}
\chtitle{有的人反对了: 骰子的六种态不一定是123456
}
\bch

\addfig{1.4}{touzi_work.jpg}

\skipline

\wulian好吧,这不是我们今天要讲的重点…
\ech
\end{frame}

\begin{frame}
\chtitle{相空间}
\bch
气体分子的“态”由位置$(x, y, z)$和速度$(\upsilon_x, \upsilon_y, \upsilon_z)$描述:
$$(x, y, z, \upsilon_x, \upsilon_y, \upsilon_z)$$
或者更专业的说法是由位置和动量描述:
$$(x, y, z, \calp_x, \calp_y, \calp_z)$$

由$x, y, z, \calp_x, \calp_y, \calp_z$这六个变量(坐标轴)张成的空间称为{\blue \bf 相空间}。

\bitem
\item{经典图像(错误):分子的态由相空间中的一个点来描述。}
\item{量子图像(正确):{\bf \blue 分子的态由相空间中的一个“小块”来描述。}}
\eitem

\ech
\end{frame}


\begin{frame}
\chtitle{相空间里的“小块”是什么鬼?}
\bch
假设位置空间$x$是一维的,则相空间是$(x, \calp)$构成的二维空间,相空间里的“小块”如下图所示:
\bmini{0.35}
\addfig{0.8}{phasespaceblock.png}
\emini
\bmini{0.6}
\bitem
\item{普朗克常数$h=6.63\times 10^{-34}\SIJ\cdot\SIs$}
\item{只要面积为$h$,即是合法的“小块”。如何取$\Delta x$或$\Delta \calp$视测量需要而定。}
\item{量子力学的测不准原理保证了“小块”内的点是无法通过测量区分的。}
\eitem
\emini

\skipline

在三维空间的情况,相空间为六维空间,“小块”体积为$h^3$ (六条边长满足$\Delta \calp_x\Delta x =\Delta \calp_y\Delta y =\Delta \calp_z\Delta z = h$)
\ech
\end{frame}


\begin{frame}
\chtitle{态是离散的相空间小块}
\bch
{\large
\bitem
\item{\bf $n$维位置空间$(x_1, x_2, \ldots, x_n)$对应$2n$维相空间$(x_1, x_2, \ldots, x_n, \calp_1, \calp_2, \ldots, \calp_n)$。}
\item{\bf 相空间内“小块”是一个假想的$2n$维体。在每个轴方向上的边长依次为$(\Delta x_1, \Delta x_2, \ldots, \Delta x_n, \Delta \calp_1, \Delta \calp_2, \ldots, \Delta \calp_n)$。边长两两成对满足$\Delta x_i \cdot \Delta \calp_i = h$ ($i=1, 2, \ldots, n$)。显然,“小块”的体积为$h^n$。}
\item{\bf 在每个维度上如何取$\Delta x_i$则视测量需求而定,有时甚至可以取$\Delta x_i$为宏观尺度。}
\item{\bf 相空间的每个“小块”对应一个微观粒子的态。虽然数学上“小块”仍然是无限可分的,在物理上却由于测不准原理而无法再划分更细的状态。}
  \eitem
  }
\ech
\end{frame}

\begin{frame}
\chtitle{思考题}
\bch
一个边长为$0.663\SIm$的正方体容器里,质量为$10^{-26}\SIkg$,速度不超过$10^3\SIm\SIs^{-1}$的粒子共有多少种可能的态?
\ech
\end{frame}

\begin{frame}
\chtitle{思考题}
\bch
一个边长为$0.663\SIm$的正方体容器里,质量为$10^{-26}\SIkg$,速度不超过$0.8$倍光速的粒子共有多少种可能的态?
\ech
\end{frame}


\begin{frame}
\chtitle{测量空气分子的动量}
\bch
考虑装在一个边长$L= 1\SIm$的正方体容器里的空气。取空气分子的平均质量,$\bar{m} = \frac{29\SIg\cdot\SImol^{-1}}{6.02\times 10^{23}\SImol^{-1}}\approx 5 \times 10^{-26}\SIkg$

取室温$T=300K$,则空气分子动量大小的数量级为
$$\calp \sim \sqrt{m^2\overline{\upsilon^2}} \sim \sqrt{3mkT} \approx 2\times 10^{-23} \SIkg \SIm\SIs^{-1} $$

我们准备测量空气分子的速度的分布规律,希望动量空间的分辨率尽可能地高,所以在位置空间尽可能地降低分辨率:取$\Delta x=\Delta y=\Delta z = L$(也就是说我们只要求分子在容器内,而不去测量它的具体位置)。那么动量空间的分辨率(即小块沿动量轴方向的边长)为
$$\Delta \calp_x = \Delta \calp_y = \Delta \calp_z= \frac{h}{L} = 6.63\times 10^{-34} \SIkg\SIm\SIs^{-1} \sim 10^{-11} \calp$$

\ech
\end{frame}


\begin{frame}
\chtitle{测量空气分子的动量(续)}
  \bch
  {\large
现在我们更贪心,希望把分子的位置确定到分子平均距离的数量级($d\sim 3\times 10^{-9}\SIm$),这时动量分辨率约为
$$\Delta \calp \sim \frac{h}{d} \sim 10^{-25} \SIkg\SIm\SIs^{-1} \sim 10^{-2} \calp$$

也就是说,课本上第二章讨论玻尔兹曼分布时的经典图像(同时知道分子的位置和动量),大致上可以实现。
}
\ech
\end{frame}


\begin{frame}
\chtitle{总结}
\bch
    {\large
      讨论了半天相空间的微观态,并不是真的要研究量子统计(毕竟这是养生热学课…)

      \skipline
      
      我们要知道的是:{\bf 粒子微观态在相空间均匀分布}。仅此而已。

}
\ech
\end{frame}


\section{Maxwell Distribution}

\secpage{麦克斯韦分布}{以$\sqrt{\frac{kT}{m}}$为单位,每个维度上的速度分量满足标准正态分布}



\begin{frame}
\chtitle{相空间的概率密度函数(非相对论情形)}
\bch
{\large 


  根据我们第一讲给的“万能法则”,达到热平衡时,分子处在一个能量为$\varepsilon$的微观态的概率正比于$e^{-\frac{\varepsilon}{kT}}$。

  {\bf 因为相空间里微观态是均匀分布的,所以我们也可以说} 分子在相空间坐标$(x,y,z,\upsilon_x,\upsilon_y,\upsilon_z)$附近的(6维)体积元内出现的概率正比于
$$ e^{-\frac{\varepsilon}{kT}} dx dy dz d\upsilon_x d\upsilon_y d\upsilon_z$$
  这里的$\varepsilon$确切地说是$x,y,z,\upsilon_x,\upsilon_y,\upsilon_z$的一个函数$\varepsilon(x,y,z,\upsilon_x,\upsilon_y, \upsilon_z)$,本质上我们写出的是一个相空间的概率密度函数:
  $$ p(x,y,z,\upsilon_x,\upsilon_y, \upsilon_z) = C \exp\left(-\frac{\varepsilon(x,y,z,\upsilon_x,\upsilon_y, \upsilon_z)}{kT}\right).$$

}
\ech
\end{frame}


\begin{frame}
\chtitle{麦克斯韦分布}
\bch
    {\large
      当只需要考虑动能时,
      f(x,y,z,
      }
\ech
\end{frame}

\begin{frame}
\chtitle{麦克斯韦分布的归一化系数}
\bch
设
$$F_M(\upsilon) = C \upsilon^2 e^{-\frac{m\upsilon^2}{2kT}}.$$
由归一化条件$\int_0^\infty F_M(\upsilon)d\upsilon = 1$得到
$$ C = \frac{1}{\int_0^\infty \upsilon^2 e^{-\frac{m\upsilon^2}{2kT}}d\upsilon} \, . $$
下面我们来计算这个积分。
\ech
\end{frame}


\begin{frame}
\chtitle{归一化的麦克斯韦速度分布函数}
\bch
{\small
于是我们得到
$$F_M(\upsilon) = 4\pi \upsilon^2 \left(\frac{m}{2\pi kT}\right)^{3/2} e^{-\frac{m\upsilon^2}{2kT}}\,.$$

写成{\bf \blue 麦克斯韦速度分布函数(即三维速度空间的概率密度):
$$f_M(\vecv) = \left(\frac{m}{2\pi kT}\right)^{3/2} e^{-\frac{m|\vecv|^2}{2kT}}\, .$$}
}
\ech
\end{frame}

\begin{frame}
\chtitle{另一种推导方法(更方便快捷)}
\bch
{\small
由于动能在三个自由度上的贡献是独立的,我们由“万能法则”可以得到$\upsilon_x$的概率密度函数为
$$ f_{1D}(\upsilon_x) =  C e^{-\frac{m\upsilon_x^2}{2kT}}$$
由归一化条件$\int_{-\infty}^\infty f(\upsilon_x)d\upsilon_x = 1$以及高斯积分公式($a = \frac{m}{2kT}$),得到

$$ f_{1D}(\upsilon_x) =   \sqrt{\frac{m}{2\pi kT}} e^{-\frac{m\upsilon_x^2}{2kT}}$$
显然对$\upsilon_y$, $\upsilon_z$可以得到一样的结果,因为综合事件的概率等于每一个独立子事件的概率的乘积,所以对$\vecv = (\upsilon_x, \upsilon_y, \upsilon_z)$有
$$f_M(\vecv) = f_{1D}(\upsilon_x)f_{1D}(\upsilon_y)f_{1D}(\upsilon_z)=  \left(\frac{m}{2\pi kT}\right)^{3/2} e^{-\frac{m(\upsilon_x^2+\upsilon_y^2+\upsilon_z^2)}{2kT}}   \, .$$
}
\ech
\end{frame}

\begin{frame}
\chtitle{统计平均量}
\bch
速度的任何函数$g(\vecv)$的统计平均显然为
{\blue $$\overline{g(\vecv)} = \int f_M(\vecv) g(\vecv) d^3\vecv$$}
如果$g$只是速率的函数,则其统计平均可以写成
{\blue $$\overline{g(\upsilon)} = \int_0^{\infty} F_M(\upsilon) g(\upsilon)d\upsilon.$$}
如果$g$只是单独一个自由度$\upsilon_x$的函数,则其统计平均
{\blue $$\overline{g(\upsilon_x)} = \int_{-\infty}^{\infty} f_{1D}(\upsilon_x) g(\upsilon_x)d\upsilon_x.$$}

\ech
\end{frame}

\begin{frame}
\chtitle{练习题}
\bch
{\small
按麦克斯韦分布的三维概率密度形式:
$$f_M(\upsilon_x, \upsilon_y, \upsilon_z) =\left(\frac{m}{2\pi kT}\right)^{3/2} e^{-\frac{m(\upsilon_x^2+\upsilon_y^2+\upsilon_z^2)}{2kT}}   \, .$$
或者其径向一维概率密度形式:
$$F_M(\upsilon) = 4\pi \upsilon^2 \left(\frac{m}{2\pi kT}\right)^{3/2} e^{-\frac{m\upsilon^2}{2kT}}\,\ \ \ \ (\upsilon>0)$$
或其单独一个自由度的一维概率密度形式:
$$ f_{1D}(\upsilon_x) =   \sqrt{\frac{m}{2\pi kT}} e^{-\frac{m\upsilon_x^2}{2kT}}$$

计算下列函数的统计平均量
\bitem
\item{ $\overline{\upsilon}$ }
\item{ $\overline{\upsilon^2}$ }
\item{ $\overline{\upsilon_x\theta(\upsilon_x)}$,其中台阶函数$\theta(x)$当$x\ge 0$时取值为$1$,否则为零。}
\eitem}
\ech
\end{frame}


\begin{frame}
\chtitle{方均根速率(root mean square velocity)}
\bch
{\small 方均根速率$\upsilon_{\rm rms}$定义为
$$\upsilon_{\rm rms} \equiv \sqrt{\overline{\upsilon^2}}$$
它表征的是分子平均动能的大小。麦克斯韦分布的方均根速率为$\upsilon_{\rm rms} = \sqrt{\frac{3kT}{m}}$,证明如下:}
{\tiny
\bmini{0.6}
\bea
\overline{\upsilon^2} &=& \int F_M(\upsilon) \upsilon^2d\upsilon \newl
&=& 4\pi  \left(\frac{m}{2\pi kT}\right)^{3/2} \int_0^\infty \upsilon^4 e^{-\frac{m\upsilon^2}{2kT}} d\upsilon \newl
&=& 4\pi  \left(\frac{\alpha}{\pi}\right)^{3/2}\left. \frac{d^2}{d\alpha^2}\left(\int_0^\infty e^{-\alpha \upsilon^2}d\upsilon \right)\right\vert_{\alpha = m/(2kT)} \newl
&=& 4\pi  \left(\frac{\alpha}{\pi}\right)^{3/2} \left. \frac{d^2}{d\alpha^2}\left(\frac{1}{2}\sqrt{\frac{\pi}{\alpha}}\right)\right\vert_{\alpha = m/(2kT)} \newl
&=& 4\pi  \left.\left(\frac{\alpha}{\pi}\right)^{3/2} \frac{3}{8\alpha^2} \sqrt{\frac{\pi}{\alpha}} \right\vert_{\alpha = m/(2kT)} \newl
&=& \frac{3kT}{m} 
\eea 
\emini
\bmini{0.35}
把$\upsilon^4$看成是两次求导的产物,并心安理得地交换了积分和求导次序

\skiplines

积分范围是$0$到$\infty$,所以是高斯积分的一半。
\vspace{0.05in}
\emini
}
\ech
\end{frame}


\begin{frame}
\chtitle{平均速率(mean velocity)}
\bch
{\small 
平均速率定义为所有分子的速率的平均值$$\overline{\upsilon}$$
它可以用来计算分子的平均自由程。麦克斯韦分布的平均速率为$\bar{\upsilon} = \sqrt{\frac{8kT}{\pi m}}$,证明如下:}
{\tiny
\bmini{0.6}
\bea
\bar{\upsilon} &=& \int_0^\infty F_M(\upsilon)\upsilon d\upsilon \newl
&=& 4\pi  \left(\frac{m}{2\pi kT}\right)^{3/2} \int_0^\infty \upsilon^3 e^{-\frac{m\upsilon^2}{2kT}} d\upsilon \newl
&=& 4\pi  \left(\frac{\alpha}{\pi}\right)^{3/2}\left. \frac{d}{d\alpha}\left(\int_0^\infty (-\upsilon e^{-\alpha \upsilon^2})d\upsilon \right)\right\vert_{\alpha = m/(2kT)} \newl
&=& 4\pi  \left(\frac{\alpha}{\pi}\right)^{3/2}\left. \frac{d}{d\alpha}\left(\left.\frac{1}{2\alpha} e^{-\alpha \upsilon^2}\right\vert_0^\infty \right)\right\vert_{\alpha = m/(2kT)} \newl
&=& 4\pi  \left(\frac{\alpha}{\pi}\right)^{3/2}\left. \frac{d}{d\alpha}\left(-\frac{1}{2\alpha} \right)\right\vert_{\alpha = m/(2kT)} \newl
&=& \sqrt{\frac{8kT}{\pi m}}
\eea
\emini
\bmini{0.35}
把$(-\upsilon^2)$看成是一次求导的产物,交换了积分和求导次序

\vspace{0.3in}
\emini
}
\ech
\end{frame}


\begin{frame}
\chtitle{泄流速率}
\bch
{\small
泄流速率定义为
$$\overline{\upsilon_x^+} \equiv \overline{\upsilon_x\theta(\upsilon_x)}$$
设分子数密度为$n$,在容器壁上挖个小孔,则单位时间从单位面积上泄出的分子数为
$n \overline{\upsilon_x^+}$。麦克斯韦分布下的泄流速率为$\sqrt{\frac{kT}{2\pi m}}$,证明如下:}
{\tiny
\bea
\overline{\upsilon_x^+} &=& \int_{-\infty}^\infty f_{1D}(\upsilon_x) \upsilon_x \theta(\upsilon_x) d\upsilon_x \newl
 &=& \int_0^\infty f_{1D}(\upsilon_x) \upsilon_x  d\upsilon_x \newl
 &=& \int_0^\infty \sqrt{\frac{m}{2\pi kT}} e^{-\frac{m\upsilon_x^2}{2kT}}\upsilon_x d\upsilon_x \newl
&=& \sqrt{\frac{m}{2\pi kT}} \left(-\frac{kT}{m}\right) \left. e^{-\frac{m\upsilon_x^2}{2kT}}\right\vert_0^{\infty} \newl
&=& \sqrt{\frac{kT}{2\pi m}}
\eea
}

\ech
\end{frame}

\begin{frame}
\chtitle{思考题}
\bch
\addfig{0.8}{think3.jpg}
“泄流”和“泻流”哪个词里的汉字用法更为正确?
\ech
\end{frame}


\section{MB distribution}

\begin{frame}
\chtitle{麦克斯韦-玻尔兹曼分布(Maxwell-Boltzmann Distribution)}
\bch
\bitem
\item{如果分子在不同位置有不同的势能,则分子在位置空间的分布就不是均匀的了。也就是说,气体各处会产生密度梯度。}
\item{如果我们希望同时描述分子在位置空间和速度空间的分布规律,划分相空间“小块”时就要两者兼顾了: 即$\Delta x$和
$\Delta \calp_x$都必须远小于我们测量的精度,才能用连续的概率密度函数来描述分子的分布。}
\eitem
\ech
\end{frame}


\begin{frame}
\chtitle{重力势能的情况}
\bch
假设沿$z$方向有强度为$g$的重力场,则分子的能量
$$\varepsilon = \frac{1}{2}m\upsilon^2 + mgz$$

按照“万能法则”,分子在相空间某小块(设坐标为$(x, y, z, \calp_x, \calp_y, \calp_z)$)出现的概率正比于:

$$f_{MB}(x, y, z, \upsilon_x, \upsilon_y,\upsilon_z) \propto e^{-\frac{\frac{1}{2}m\upsilon^2 + mgz}{kT}}$$

上式可以拆成两个独立分布的乘积:
$$f_{MB}(x, y, z, \upsilon_x, \upsilon_y,\upsilon_z) \propto e^{-\frac{m\upsilon^2}{2kT}} e^{-\frac{mgz}{kT}}$$

可以看到,在速度空间的分布和在位置空间的分布互不干扰,所以我们可以得到分子数密度
$$n = n_0 e^{-\frac{mgz}{kT}}\, ,$$
其中$n_0$为$z=0$处的分子数密度。
\ech
\end{frame}

\begin{frame}
\chtitle{离心力势能}
\bch
在旋转参照系中,离心力势能可以写成
$$U(r) = -\frac{1}{2}m\omega^2r^2$$
其中$r$为离旋转中心的径向距离,$\omega$为旋转系相对于惯性系的旋转角速度。

\skipline

根据一样的推导过程,可以得到
$$n = n_0 e^{\frac{m\omega^2r^2}{2kT}}$$
\ech
\end{frame}


\begin{frame}
\chtitle{思考题}
\bch
\addfig{0.8}{think.jpg}
你能否解释台风的“外围狂风暴雨,中心风和日丽”的奇特现象?
\ech
\end{frame}

\begin{frame}
\chtitle{麦克斯韦-玻尔兹曼能量分布律}
\bch
一般地,如果位置空间和速度空间的能量形式没有交叉项,我们可以分离位置空间和速度空间的分布(概率密度),得到分子在相空间的概率密度为:
$$f_{MB}(x, y, z,\upsilon_x, \upsilon_y,\upsilon_z) = n_0 \left(\frac{m}{2\pi kT}\right)^{3/2} e^{-\frac{\varepsilon}{kT}} $$
其中能量$\varepsilon$是动能与势能之和:
$$\varepsilon = \frac{1}{2}m\upsilon^2 + U(x,y,z)$$
这是著名的麦克斯韦-玻尔兹曼能量分布律(\wulian 好像跟我们反复念叨的万能法则是一回事)。
\ech
\end{frame}



\end{document}

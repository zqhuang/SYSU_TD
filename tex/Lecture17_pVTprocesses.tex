\documentclass[CJK]{beamer}
\usepackage{CJKutf8}
\usepackage{beamerthemesplit}
\usetheme{Malmoe}
\useoutertheme[footline=authortitle]{miniframes}
\usepackage{amsmath}
\usepackage{amssymb}
\usepackage{graphicx}
\usepackage{eufrak}
\usepackage{color}
\usepackage{slashed}
\usepackage{simplewick}
\usepackage{tikz}
\graphicspath{{../figures/}}
\def\addfig#1#2{\begin{center}\includegraphics[width=#1 in]{#2}\end{center}}
\def\blacktext#1{{\color{black}#1}}
\def\bluetext#1{{\color{blue}#1}}
\def\redtext#1{{\color{red}#1}}
\def\darkbluetext#1{{\color[rgb]{0,0.2,0.6}#1}}
\def\skybluetext#1{{\color[rgb]{0.2,0.7,1.}#1}}
\def\cyantext#1{{\color[rgb]{0.,0.5,0.5}#1}}
\def\greentext#1{{\color[rgb]{0,0.7,0.1}#1}}
\def\darkgray{\color[rgb]{0.2,0.2,0.2}}
\def\lightgray{\color[rgb]{0.6,0.6,0.6}}
\def\gray{\color[rgb]{0.4,0.4,0.4}}
\def\blue{\color{blue}}
\def\red{\color{red}}
\def\green{\color{green}}
\def\darkblue{\color[rgb]{0,0.2,0.6}}
\def\skyblue{\color[rgb]{0.2,0.7,1.}}
\def\fdeg{{^\circ \mathrm{F}}}
\def\cdeg{^\circ \mathrm{C}}
\def\be{\begin{equation}}
\def\ee{\nonumber\end{equation}}
\def\bea{\begin{eqnarray}}
\def\eea{\nonumber\end{eqnarray}}
\def\ii{{\dot{\imath}}}
\def\bch{\begin{CJK}{UTF8}{gbsn}}
\def\ech{\end{CJK}}
\def\bitem{\begin{itemize}}
\def\eitem{\end{itemize}}
\def\bcenter{\begin{center}}
\def\ecenter{\end{center}}
\def\bex{\begin{minipage}{0.3\textwidth}\includegraphics[width=1in]{jugelizi.png}\end{minipage}\begin{minipage}{0.6\textwidth}}
\def\eex{\end{minipage}}
\def\chtitle#1{\frametitle{\bch#1\ech}}
\def\skipline{{\vskip0.1in}}
\def\skiplines{{\vskip0.2in}}
\def\lagr{{\mathcal{L}}}
\def\hamil{{\mathcal{H}}}
\def\vecv{{\mathbf{v}}}
\def\vecx{{\mathbf{x}}}
\def\vecy{{\mathbf{y}}}
\def\veck{{\mathbf{k}}}
\def\vecp{{\mathbf{p}}}
\def\vecn{{\mathbf{n}}}
\def\vecA{{\mathbf{A}}}
\def\vecP{{\mathbf{P}}}
\def\vecsigma{{\mathbf{\sigma}}}
\def\hatJn{{\hat{J_\vecn}}}
\def\hatJx{{\hat{J_x}}}
\def\hatJy{{\hat{J_y}}}
\def\hatJz{{\hat{J_z}}}
\def\hatj#1{\hat{J_{#1}}}
\def\hatphi{{\hat{\phi}}}
\def\hatq{{\hat{q}}}
\def\hatpi{{\hat{\pi}}}
\def\vel{\upsilon}
\def\Dint{{\mathcal{D}}}
\def\adag{{\hat{a}^\dagger}}
\def\bdag{{\hat{b}^\dagger}}
\def\cdag{{\hat{c}^\dagger}}
\def\ddag{{\hat{d}^\dagger}}
\def\hata{{\hat{a}}}
\def\hatb{{\hat{b}}}
\def\hatc{{\hat{c}}}
\def\hatd{{\hat{d}}}
\def\hatN{{\hat{N}}}
\def\hatH{{\hat{H}}}
\def\hatp{{\hat{p}}}
\def\Fup{{F^{\mu\nu}}}
\def\Fdown{{F_{\mu\nu}}}
\def\newl{\nonumber \\}
\def\SIkm{\,\mathrm{km}}
\def\SIyr{\,\mathrm{yr}}
\def\SIGyr{\,\mathrm{Gyr}}
\def\SIeV{\,\mathrm{eV}}
\def\SIkeV{\,\mathrm{keV}}
\def\SIMeV{\,\mathrm{MeV}}
\def\SIGeV{\,\mathrm{GeV}}
\def\SIcal{\,\mathrm{cal}}
\def\SIkcal{\,\mathrm{kcal}}
\def\SImol{\,\mathrm{mol}}
\def\SIm{\,\mathrm{m}}
\def\SIcm{\,\mathrm{cm}}
\def\SIfm{\,\mathrm{fm}}
\def\SImm{\,\mathrm{mm}}
\def\SInm{\,\mathrm{nm}}
\def\SImum{\,\mathrm{\mu m}}
\def\SIJ{\,\mathrm{J}}
\def\SIkJ{\,\mathrm{kJ}}
\def\SIs{\,\mathrm{s}}
\def\SIkg{\,\mathrm{kg}}
\def\SIg{\,\mathrm{g}}
\def\SIK{\,\mathrm{K}}
\def\SImmHg{\,\mathrm{mmHg}}
\def\SIPa{\,\mathrm{Pa}}
\def\vece{\mathrm{e}}
\def\bmat#1{\left(\begin{array}{#1}}
\def\emat{\end{array}\right)}
\def\bcase#1{\left\{\begin{array}{#1}}
\def\ecase{\end{array}\right.}
\def\calM{{\mathcal{M}}}
\def\calT{{\mathcal{T}}}
\def\calR{{\mathcal{R}}}
\def\barpsi{\bar{\psi}}
\def\baru{\bar{u}}
\def\barv{\bar{\upsilon}}
\def\bmini#1{\begin{minipage}{#1\textwidth}}
\def\emini{\end{minipage}}
\def\qeq{\stackrel{?}{=}}
\def\torder#1{\mathcal{T}\left(#1\right)}
\def\rorder#1{\mathcal{R}\left(#1\right)}
\def\contr#1#2{\contraction{}{#1}{}{#2}#1#2}
\def\trof#1{\mathrm{Tr}\left(#1\right)}
\def\trace{\mathrm{Tr}}
\def\comm#1{\ \ \ \left(\mathrm{used}\ #1\right)}
\def\tcomm#1{\ \ \ (\text{#1})}
\def\slp{\slashed{p}}
\def\slk{\slashed{k}}
\def\wulian{\includegraphics[width=0.18in]{emoji_wulian.jpg}}
\def\bye{\includegraphics[width=0.18in]{emoji_bye.jpg}}
\def\calp{{\mathfrak{p}}}
\def\veccalp{\mathbf{\mathfrak{p}}}
\def\atm{\,\mathrm{atm}}
\def\angstrom{\,\text{\AA}}
\def\Tthree{T_{\tiny \textcircled{3}}}
\def\pthree{p_{\tiny \textcircled{3}}}

\def\courseurl{http://zhiqihuang.top}

\def\tpage#1#2{
\begin{frame}
\bch
\begin{center}
\begin{large}
热学 \\
第#1讲 #2

\end{large}

\skiplines

黄志琦


\end{center}

\skiplines

{\small 
教材:《热学》第二版,赵凯华,罗蔚茵,高等教育出版社


课件下载
}
\courseurl 
\ech
\end{frame}
}

\def\bfr#1{
\begin{frame}
\chtitle{#1} 
\bch
}

\def\efr{
\ech 
\end{frame}
}

\title{Lesson 17 Processes in pVT system }
  \author{}
  \date{}
\begin{document}
\tpage{17}{pVT系统的热力学过程}


\section{Introduction}

\begin{frame}
  \chtitle{本讲内容摘要}
  \bchL
  \bitem
  \item{准静态过程:等体,等压,等温,准静态绝热,多方过程}
  \item{非准静态过程:绝热自由膨胀,节流}
  \item{相变过程和克拉珀龙方程}
    \eitem
    \echL
\end{frame}


\begin{frame}
  \chtitle{$pVT$系统热力学过程大致分类}
  \bch
根据过程守恒量不同,$pVT$系统的过程大致有:
  \bitem
\item{$V$守恒的{\blue 等体过程}; $p$守恒的{\blue 等压过程};$T$守恒的{\blue 等温过程}; $S$守恒的{\blue 准静态绝热过程};}
\item{$pV^n$守恒的{\blue 多方过程}($n$称为多方指数);}
\item{$U$守恒的{\blue 绝热自由过程}(一般为非准静态);}
\item{$H$守恒的{\blue 节流过程} (一般为非准静态);}
  \eitem
  此外,还有{\blue 相变过程},涉及的方程叫{\blue 克拉珀龙(Clapeyron)方程}。

  \skiplines
  
  本讲中约定用{\blue 下标1表示初始状态,下标2表示末态}。
  \ech
\end{frame}

\section{QS process}

\secpage{准静态过程}{直接对过程进行积分}

\begin{frame}
  \chtitle{等体过程}
  \bchL
  定体过程显然有
  $$\Delta U = Q = \int C_V dT;\ \ A = 0.$$

  如果$C_V$没有给出,那么按理想气体近似,{\blue 单原子气体(如氦气)的定体热容 $C_V\approx \frac{3}{2}\nu R$;双原子气体在室温附近都默认$C_V\approx \frac{5}{2}\nu R$。}
  \echL
\end{frame}

\begin{frame}
  \chtitle{等压过程}
  \bchL
  定压过程显然有
  $$\Delta H = Q = \int C_p dT;\ \ A=-p\Delta V.$$
  对{\blue 理想气体, $C_p = C_V + \nu R$}。对一般的$pVT$系统,$C_p = C_V + T\pfrac pTV \pfrac VTp$ (证明参考附录1)。
  \echL
\end{frame}


\begin{frame}
  \chtitle{等温过程}
  \bchL
  {\blue 理想气体的等温过程内能不变},如果是准静态过程,则$Q = -A = \int p dV = \nu RT \ln \frac{V_2}{V_1}$。

  \skipline
  
  一般$pVT$系统的准静态等温过程则往往需要利用$\dbar Q$的两种形式:
$$\dbar Q = C_V dT + T \pfrac pTV  dV,$$

 $$\dbar Q = C_p dT - T \pfrac VTp  dp.$$

  \echL
\end{frame}


\begin{frame}
  \chtitle{实例与练习}
  \bchL
  等体,等压,等温过程一般比较简单,基本的计算技巧可以参考附录3-5。

  \addfig{0.6}{hugan.jpg}
  
  更有挑战性的一些练习可以在quiz题库中找到。

  \echL
\end{frame}


\begin{frame}
  \chtitle{准静态绝热过程}
  \bchL
  {\blue 准静态绝热过程是等熵过程}。另外,根据热一律,$\Delta U = A$。


  \skiplines
  
  $C_V$为常量的理想气体准静态{\blue 绝热}过程中,{\blue $pV^\gamma$是不变量},其中$\gamma = \frac{C_p}{C_V} = \frac{C_V+\nu R}{C_V}$. 再根据理想气体的状态方程,{\blue $TV^{\gamma-1}$和$p^{\frac{1}{\gamma}-1}T$都是不变量}。
  \echL
\end{frame}


\begin{frame}
  \chtitle{准静态绝热过程过程举例}
  \bchL
  $1\SImol$温度为$400\SIK$的单原子理想气体经历准静态绝热膨胀,压强减小了一半。该过程中气体对外做功多少?
  \echL
\end{frame}

\begin{frame}
  \chtitle{多方过程}
  \bchL
  \tbox{$pV^n$不变的过程称为多方过程,$n$称为多方指数。}
  理想气体的准静态多方过程中,外界对气体做功
  {\blue $$A = \frac{\nu R\Delta T}{n-1}$$}
  如果定体热容$C_V$为常量,则多方过程的热容:
  {\blue $$ C_n = \frac{n-\gamma}{n-1} C_V$$}
  上述结论的证明见附录2.
  \echL
\end{frame}


\begin{frame}
  \chtitle{思考题}
  \bchL
  很多常见的过程都是多方过程,说出下列过程对应的多方指数:
  \bitem
\item{定体过程}
\item{等压过程}
\item{理想气体的等温过程}
\item{  定体摩尔热容$\cvmol$不变的理想气体的绝热过程}
  \eitem

  \echL
\end{frame}


\begin{frame}
  \chtitle{多方过程举例}
  \bchL
  $1\SImol$温度为$400\SIK$的理想气体经历多方指数$n=2$的准静态多方过程,体积增大了一倍。该过程中气体对外做功多少?
  \echL
\end{frame}

\section{Non-QS process}

\secpage{非准静态过程}{研究方法:设计具有相同末态的准静态过程}

\begin{frame}
  \chtitle{绝热自由膨胀}
  \bchL
      {\blue 绝热自由膨胀(绝热: $Q=0$,  自由: $A=0$)是等内能降压过程,它不是准静态过程。}
      
  \bitem
\item{理想气体:内能和温度有一对一的映射关系,所以温度也不变。}
\item{实际气体,因为体积增大导致分子间势能增大,所以分子平均动能必须降低,所以温度下降。}
  \eitem

  \echL
\end{frame}

\begin{frame}
  \chtitle{思考题}
  \bchL
  范德瓦耳斯气体满足状态方程
  $$ \left(p+\frac{\nu^2 a}{V^2}\right)(V-\nu b) = \nu RT$$
  其中$a, b>0$为常量。证明:
  $$ \pfrac TVU < 0. $$
  从而说明该气体绝热自由膨胀后确实温度下降。
  \echL
\end{frame}



\begin{frame}
\chtitle{节流过程}
\bch

\bmini{0.48}
\addfig{2}{throttling_process.png}
\emini
\bmini{0.48}
用多孔塞把绝热箱隔开,两端分别施以恒定压强$p_1$, $p_2$ ($p_1>p_2$)。一开始气体都在左边,体积为$V_1$,内能为$U_1$。
\emini

经过一段时间,左边的气体都通过多孔塞被压到了右边,体积和内能变为$V_2$, $U_2$。

由于左边活塞对气体做功$p_1V_1$,右边活塞对气体做功$-p_2V_2$,气体内能变化为

$$U_2 - U_1  = p_1V_1 - p_2V_2$$

即
$$ U_1 + p_1V_1 = U_2 + p_2V_2 $$
{\blue 节流过程是等焓降压过程,它一般不是准静态过程。}
\ech
\end{frame}



\begin{frame}
\chtitle{焦耳-汤姆孙系数(Joule-Thomson coefficient)}
\bch

\bitem
\item{用{\bf 焦耳-汤姆孙系数$$\alpha \equiv \pfrac TpH$$}来计算节流过程温度的升降。}
\item{如果$\alpha>0$,则称为{\bf 正节流效应,温度随压强降低而降低}(注意节流过程总是压强降低的过程)。}
\item{反之若$\alpha<0$,则为{\bf 负节流效应,温度随压强降低而升高}。}
\item{(正/负)节流效应统称{\bf 焦耳-汤姆孙效应}。}
\eitem



\skiplines

{\scriptsize 注意:$\alpha$一般是$(p,T)$的函数,如果在节流过程对应的等焓线上,$\alpha$的符号发生变化,则节流效应的正负需要积分才能确定。}
\ech
\end{frame}

\begin{frame}
\chtitle{思考题}
\bch

既然绝热自由膨胀和节流过程都不是准静态过程,为什么分别能用等$U$线上的$\pfrac TVU$和等$H$线上的$\pfrac TpH$来计算的温度的升降呢?
\ech
\end{frame}

\begin{frame}
\chtitle{转换曲线图}
\bch

\bmini{0.4}
几种实际气体的节流方向如右图所示。

\skipline

正负节流的交界线称为转换曲线。

\skipline
    {\bf 常温常压下,氢气和氦气有负节流(降压升温)效应。而其他大部分气体(氮气,氧气等)有正节流(降压降温)效应}。

    \skipline
    
    {\scriptsize 转换曲线的大致形状可以用范德瓦耳斯模型推导出来,见附录6}
\emini
\bmini{0.55}
\lfig{2}{throttling_curve.png}
\emini



\ech
\end{frame}

\begin{frame}
\chtitle{节流液化气体的原理(预冷+节流)}
\bch
\bmini{0.43}
\lfig{1.7}{njg_throttling.png}
\emini
\bmini{0.52}
节流装置图
\addfig{1.8}{throttling_liquefy.jpg}
\emini

\ech
\end{frame}


\begin{frame}
\chtitle{思考题}
\bchL
为什么节流降温前一般都需要用其他方法(如绝热膨胀)进行预冷?
\echL
\end{frame}

\section{Phase Transition}

\secpage{相变过程和克拉珀龙方程}{$$\frac{dp}{dT} = \frac{\Lambda^{\rm mol}}{T\left(V_\beta^{\rm mol} - V_\alpha^{\rm mol}\right)}$$}

\begin{frame}
\chtitle{相变曲面的$p$, $T$合并现象}
\bch

在$pVT$系统的多相共存曲面上,$p$和$T$发生了自由度合并现象:$p$和$T$之间有一一对应关系;光用$p, T$无法完整地描述这个二自由度系统;类似$\pfrac VpT,\ \pfrac UTp$之类的符号也失去了物理意义。

\addfig{2.5}{PVTdiagram.png}
\ech
\end{frame}



\begin{frame}
\chtitle{相变曲面的投影}
\bchL
把体积自由度投影掉,相变曲面就成为相变曲线,反映的是$p$-$T$之间的一一映射关系。
\addfig{3}{PTdiagram.png}

\echL
\end{frame}

\begin{frame}
\chtitle{相变过程}
\bchL
通常的相变过程在定压定温条件下进行,体积随着不同物态成分的比例变化而变化。
\addfig{2.8}{phase_transition.png}
\echL
\end{frame}


\begin{frame}
\chtitle{相变潜热和相变曲面坡度的关系}
\bchL
虽然相变沿着图中“水平方向”进行,相变潜热却和相变曲面的“坡度”有数学关系:
\addfig{2.8}{phase_transition.png}
\echL
\end{frame}


\begin{frame}
\chtitle{相变潜热和相变曲面坡度的关系}
\bch
从$pT$图上看,相变在相变曲线上固定的一个点发生(如果想象垂直纸面还有个$V$轴,实际上相变沿着垂直纸面方向进行)。但相变潜热却和相变曲线在该点的斜率有关。
\addfig{2.8}{PTdiagram.png}
\ech
\end{frame}

\begin{frame}
\chtitle{相变潜热和相变曲面坡度的关系}
\bchL
这个关系可以用准静态过程吸热量的第一种形式导出:
$$\dbar Q = C_V dT + T\pfrac pTV dV $$
在相变过程中$dT=0$,$T\pfrac pTV$是固定不变的常量,所以直接积分得到:
$$ Q = T\pfrac pTV \Delta V $$
\echL
\end{frame}


\begin{frame}
\chtitle{克拉珀龙方程}
\bchL
考虑$1\SImol$的$\alpha$态物质变化到$\beta$态,吸收的潜热记为$\Lambda^{\rm mol}$,则有
$$ \Lambda^{\rm mol} = T \pfrac pTV \left(V_\beta^{\rm mol} - V_\alpha^{\rm mol}\right).$$
写成教材上的{\blue 克拉珀龙方程:
  $$\frac{dp}{dT} = \frac{\Lambda^{\rm mol}}{T\left(V_\beta^{\rm mol} - V_\alpha^{\rm mol}\right)}$$}
{\scriptsize 注意这里的$\frac{dp}{dT}$实际上是相变曲面上的$\pfrac pTV$,也可以理解为经过投影的相变曲线上的斜率。}
\echL
\end{frame}


\begin{frame}
\chtitle{克拉珀龙方程的蒸发/升华近似}
\bch
当考虑蒸发或者升华过程时,气态的摩尔体积远大于液态或者固态的摩尔体积,因此可以近似认为
$$ V_{\beta}^{\rm mol} - V_{\alpha}^{\rm mol} \approx V_{\beta}^{\rm mol} \approx \frac{RT}{p}. $$
代入克拉珀龙方程,得到:
$$\frac{dp}{dT} = \frac{p\Lambda^{\rm mol}}{RT^2}$$
稍作整理,即得到
{\blue 克拉珀龙方程的蒸发/升华近似
  $$\Lambda^{\rm mol} = -R\frac{d\ln p}{d\left(\frac{1}{T}\right)} $$
  }
\ech
\end{frame}

\begin{frame}
  \chtitle{练习}
  \bchL
  教材225页习题4-8:

  \skipline
  
  水从温度$99\cdeg$升高到$101\cdeg$时,饱和蒸气压从$733.7\SImmHg$增大到$788.0\SImmHg$。假定这时水蒸气可看作理想气体,求$100\cdeg$时水的汽化热。
  \echL
\end{frame}

\begin{frame}
  \chtitle{解答}
  \bch
  按照克拉珀龙方程的近似,
  
  $$\Lambda^{\rm mol} = -R \frac{\Delta \ln p}{\Delta\left(\frac{1}{T}\right)} = -8.314 \times \frac{\ln\frac{788}{733.7}}{\frac{1}{374.15}-\frac{1}{372.15}} \SIJ/\SImol = 4.13\times 10^4 \SIJ/\SImol$$
  又$1\SImol$水为$0.018\SIkg$,故按习惯把汽化热写成
  $$ \Lambda = \frac{4.13\times 10^4}{0.018} \SIJ/\SIkg = 2.29\times 10^6 \SIJ/\SIkg$$
  \ech
\end{frame}



\begin{frame}
\chtitle{附录1:$C_p$和$C_V$的关系}
\bch
{\small
\bea
C_p - C_V &=& \left(\frac{\dbar Q}{dT}\right)_p - \left(\frac{\dbar Q}{dT}\right)_V \newl
&=& \left(\frac{T dS}{dT}\right)_p - \left(\frac{T dS}{dT}\right)_V \newl
&=& T\left(\pfrac STp - \pfrac STV \right)
\eea
利用切换固定量公式,
$$ C_p-C_V = T \pfrac SVT \pfrac VTp . $$
最后,利用麦克斯韦关系$\pfrac SVT = \pfrac pTV$,得到
$$ C_p - C_V = T\pfrac pTV \pfrac VTp .$$}
\ech
\end{frame}


\begin{frame}
\chtitle{附录2:理想气体准静态多方过程}
\bch
设多方过程$pV^n = C$,则外界对气体做功:
\bea
A &=& -\int pdV \newl
&=& - \int C V^{-n} dV \newl
&=& \frac{1}{n-1} \Delta \left(CV^{1-n}\right) \newl
&=& \frac{\nu R\Delta T}{n-1} 
\eea
最后一步是利用了$\nu RT = pV = pV^n V^{1-n} = CV^{1-n}$.
\ech
\end{frame}

\begin{frame}
\chtitle{附录2:理想气体准静态多方过程}
\bch
如果$C_V$是常量,则利用$\nu R = C_p - C_V = (\gamma-1)C_V$,有
\bea
Q  &=& \Delta U - A \newl
&=& C_V\Delta T - \frac{\nu R\Delta T}{n-1} \newl
&=& C_V\Delta T\left(1 - \frac{\gamma -1}{n-1}\right) \newl
&=& \frac{n-\gamma}{n-1}C_V\Delta T
\eea
由此得到
$$C_n = \frac{n-\gamma}{n-1}C_V.$$
\ech
\end{frame}

\begin{frame}
  \chtitle{附录3:等体过程举例}
  \bch
  $2\SImol$氦气经历等体过程,温度从267K升高到300K,求这个过程中氦气的吸热量$Q$。

  \skipline
  
  解答:在题目未做说明的情况下,只能当成理想气体进行计算。氦气是单原子气体,定体热容$C_V = \frac{3\nu R}{2}$。按照定体热容的定义,
  $$Q = C_V\Delta T = \frac{3\nu R\Delta T}{2} = 823\SIJ $$
  \ech
\end{frame}



\begin{frame}
  \chtitle{附录4:等压过程举例}
  \bch
  $2\SImol$氮气经历等压过程,温度从286K升高到300K,求这个过程中氮气的吸热量。

  \skipline
  
  解答:在题目未做说明的情况下,只能当成理想气体进行计算。氮气是双原子气体,定体热容$C_V \approx \frac{5\nu R}{2}$,定压热容$C_p = C_V+\nu R \approx \frac{7\nu R}{2}$。按照定压热容的定义,
  $$Q = C_p\Delta T = \frac{7\nu R\Delta T}{2} = 815\SIJ $$
  \ech
\end{frame}


\begin{frame}
  \chtitle{附录5:等温过程举例}
  \bch
  压强为$10^5\SIPa$,体积为$1\SIL$理想气体经历准静态等温过程,压强减小了一半。求该过程中的吸热量$Q$。

  \skipline
  
  解答:在题目未做说明的情况下,只能当成理想气体进行计算。理想气体的内能只是温度的函数,所以等温过程中内能不变。根据热一律有
  $$Q = -A = \nu R T \ln \frac{V_2}{V_1}  =  pV \ln \frac{p_1}{p_2} $$
  代入数值得到
  $$ Q =  10^5\SIPa \times 10^{-3} \SIm^3 \times \ln 2 = 69.3\SIJ.$$
  \ech
\end{frame}

\begin{frame}
  \chtitle{附录6:焦耳-汤姆逊系数的符号}
  \bch
  利用三变量的偏微分链式法则以及上一讲提到过的$C_p = \pfrac HTp$,焦耳-汤姆逊系数可以写成:
  $$\alpha \equiv \pfrac TpH = - \pfrac THp \pfrac HpT = - \frac{\pfrac HpT}{C_p}$$
  也就是说,$\alpha$和$\pfrac HpT$符号相反,而$\pfrac HpT$可以由焓和状态方程的关系求出。下面我们用范德瓦耳斯模型计算$\pfrac HpT$。
  \ech
\end{frame}

  \begin{frame}
\chtitle{附录6:焦耳-汤姆逊系数的符号}
\bch
{\small

对范德瓦尔斯气体,由状态方程
$$\left(p + \frac{\nu^2 a}{V^2}\right) \left(V - \nu b\right) = \nu RT$$
以及焓和状态方程的关系,得到
$$\pfrac HpT = V \left(\delta_V - \frac{2\delta_p}{1+2\delta_p}\right) \frac{1+2\delta_p}{1-2\delta_p\left(1-\delta_V\right)}$$
其中$\delta_p \equiv \frac{\frac{\nu^2a}{V^2}}{p+\frac{\nu^2 a}{V^2}}$为压强修正的大小,$\delta_V\equiv \frac{\nu b}{V}$为体积修正的大小。

\bitem
\item{小修正情形:在$\delta_p \ll 1$, $\delta_V \ll 1$的情况下,很容易推算出:当$T>\frac{2a}{bR}$时,$\pfrac HpT>0$(负节流);反之则$\pfrac HpT<0$(正节流)}
\item{大修正情形:在极低温(接近液化温度)以及高压的情形,$\delta_V\rightarrow 1$,显然$\pfrac HpT>0$(负节流)。}
\eitem

}
\ech
\end{frame}



\begin{frame}
\chtitle{附录6:焦耳-汤姆逊系数的符号}
\bch
由前面的讨论,我们定性地知道负节流($\pfrac HpT>0$)区域近似为满足$T<\frac{2a}{bR}$的高温区域,以及体积修正比较大的低温高压区域。

下图是对某个范德瓦尔斯气体的数值计算结果:
\addfig{2}{dHdp.pdf}

\ech
\end{frame}


\begin{frame}
\chtitle{附录6:焦耳-汤姆逊系数的符号}
\bch
在非极限情形下,气体的体积修正和压强修正很小,气体的正节流条件为$T<\frac{2a}{Rb} \approx \frac{27}{4} T_K$,其中$T_K$为临界温度。

\skipline

{\scriptsize (对范德瓦耳斯气体,其临界温度$T_K$由条件
$$\left(\frac{d p}{d V}\right)_{T=T_K} = \left(\frac{d^2 p}{d V^2}\right)_{T=T_K} = 0$$ 
确定,计算结果为$T_K = \frac{8a}{27Rb}.$)}

\skipline

在教材52页查得,氢气和氦气的临界温度较低,$\frac{27}{4} T_K$在室温之下,而其他大部分气体的临界温度较高,$\frac{27}{4} T_K$在室温之上。所以在{\bf 常温常压下,氢气和氦气有负节流(升温)效应。而其他大部分气体(空气,氮气,氧气等)有正节流(降温)效应}。
\ech
\end{frame}


  

\end{document}

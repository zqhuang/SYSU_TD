\documentclass[10pt,CJK]{article}
\usepackage{geometry}
\input{reduced_macros.tex}
\geometry{tmargin=0.3in, bmargin=0.5in, lmargin=0.7in, rmargin=0.7in, nohead, nofoot}
\def\mark#1{{\color{blue} (#1分)}}
\renewcommand{\thepage}{}
\begin{document}
\bch
{\large 热学课堂练习 诸神黄昏版}

{\vskip 0.05in}

姓名 ....................... {\hskip 0.5in}    学号 .......................{\hskip 0.5in}  分数 ...................

\bitem

\item[(一)]{选择题,每题10分。

  \bitem

\item[(1)]{设实际气体的定体热容为$C_V$,定压热容为$C_p$。下列哪个表达式和焦耳汤姆孙系数$\pfrac TpH${\bf 不一定}等价 \bropt
  
  \optlist{$\frac{T\pfrac VTp  -V}{C_V+\nu R}$}{$-\frac{\pfrac HpT}{C_p}$}{$T \pfrac VHp - V\pfrac THp$}
}
  
\item[(2)]{处于热平衡的单一成分理想气体中任取一个分子,其速率超过方均根速率的$2$倍,但速度的$x$方向分量的大小不超过方均根速率(即$\upsilon>2\upsilon_{\rm rms}, |\upsilon_x|<\upsilon_{\rm rms}$)的概率为 \bropt
  
  \optlist{$\sqrt{\frac{3}{2\pi}}e^{-3/2}$}{$\sqrt{\frac{3}{2\pi}}e^{-6}$}{$\sqrt{\frac{6}{\pi}}e^{-6}$}

}    
\item[(3)]{某气体的状态方程为$\left(p+\frac{\nu^2 a}{V^2T}\right)V = \nu R T$,其中$\nu$为摩尔数,$p, V, T$分别为压强,体积和热力学温度,$a>0$为常量。则该气体的摩尔定体热容 \bropt
  
  \optlist{一定是常量}{不可能是常量}{可能是常量也可能是温度的函数}

}


\item[(4)]{  置于很大的真空室内的绝热容器里装有稀薄氦气。在容器壁上开一个小孔,经过一段时间后把小孔堵上,发现容器内氦气压强降低了$0.4\%$,问容器内氦气的分子数减少了多少?漏气的过程很缓慢,可以近似认为整个过程中容器内氦气一直处于热平衡。 \bropt

  \optlist{$0.4\%$}{$0.3\%$}{$0.1\%$}
}

\item[(5)]{在一个标准大气压下,某种气体从$T=250\SIK$准静态等压升温至$T=350\SIK$。在整个过程中该气体的焓和温度平方成正比。已知该气体在初始状态和末状态的化学势分别为$\mu_1$和$\mu_2$,则在过程中间$T=300\SIK$时,该气体的化学势为 \bropt
  
  \optlist{$\frac{\mu_1+\mu_2}{2}$}{$\frac{5}{7}\mu_1+\frac{2}{7}\mu_2$}{$\frac{3}{5}\mu_1+\frac{3}{7}\mu_2$}
}
  
  \eitem
}

  
\item[(二)]{
  对只有两个独立变量的$pVT$系统证明:
  $$\pfrac TSH +\frac{T^2}{V} \pfrac VHp = \frac{T}{C_p}, $$
  其中$C_p$为定压热容;$T, V, p, S, H$分别为温度,体积,压强,熵和焓。\mark{25}
  
  \vspace{5in}
  }

\item[(三)]{  某气体的焦耳-汤姆孙系数
  $$\alpha = \pfrac TpH =-\frac{a}{T^2},$$
  其中$a>0$为常量。
  
  当$p\rightarrow 0$时,该气体的定压比热容趋向于常量: $\lim_{p\rightarrow 0}C_p = c.$

  当温度趋向于零时,该气体的熵趋向于零: $ \lim_{T\rightarrow 0} S = 0.$
  
  在标准状态下的该气体,经过准静态绝热压缩,压强增强为$8\atm$。求末态气体温度。\mark{25}
  }
  
  
\eitem


\ech
\end{document}

\documentclass[CJK]{beamer}
\usepackage{CJKutf8}
\usepackage{beamerthemesplit}
\usetheme{Malmoe}
\useoutertheme[footline=authortitle]{miniframes}
\usepackage{amsmath}
\usepackage{amssymb}
\usepackage{graphicx}
\usepackage{eufrak}
\usepackage{color}
\usepackage{slashed}
\usepackage{simplewick}
\usepackage{tikz}
\graphicspath{{../figures/}}
\def\addfig#1#2{\begin{center}\includegraphics[width=#1 in]{#2}\end{center}}
\def\blacktext#1{{\color{black}#1}}
\def\bluetext#1{{\color{blue}#1}}
\def\redtext#1{{\color{red}#1}}
\def\darkbluetext#1{{\color[rgb]{0,0.2,0.6}#1}}
\def\skybluetext#1{{\color[rgb]{0.2,0.7,1.}#1}}
\def\cyantext#1{{\color[rgb]{0.,0.5,0.5}#1}}
\def\greentext#1{{\color[rgb]{0,0.7,0.1}#1}}
\def\darkgray{\color[rgb]{0.2,0.2,0.2}}
\def\lightgray{\color[rgb]{0.6,0.6,0.6}}
\def\gray{\color[rgb]{0.4,0.4,0.4}}
\def\blue{\color{blue}}
\def\red{\color{red}}
\def\green{\color{green}}
\def\darkblue{\color[rgb]{0,0.2,0.6}}
\def\skyblue{\color[rgb]{0.2,0.7,1.}}
\def\fdeg{{^\circ \mathrm{F}}}
\def\cdeg{^\circ \mathrm{C}}
\def\be{\begin{equation}}
\def\ee{\nonumber\end{equation}}
\def\bea{\begin{eqnarray}}
\def\eea{\nonumber\end{eqnarray}}
\def\ii{{\dot{\imath}}}
\def\bch{\begin{CJK}{UTF8}{gbsn}}
\def\ech{\end{CJK}}
\def\bitem{\begin{itemize}}
\def\eitem{\end{itemize}}
\def\bcenter{\begin{center}}
\def\ecenter{\end{center}}
\def\bex{\begin{minipage}{0.3\textwidth}\includegraphics[width=1in]{jugelizi.png}\end{minipage}\begin{minipage}{0.6\textwidth}}
\def\eex{\end{minipage}}
\def\chtitle#1{\frametitle{\bch#1\ech}}
\def\skipline{{\vskip0.1in}}
\def\skiplines{{\vskip0.2in}}
\def\lagr{{\mathcal{L}}}
\def\hamil{{\mathcal{H}}}
\def\vecv{{\mathbf{v}}}
\def\vecx{{\mathbf{x}}}
\def\vecy{{\mathbf{y}}}
\def\veck{{\mathbf{k}}}
\def\vecp{{\mathbf{p}}}
\def\vecn{{\mathbf{n}}}
\def\vecA{{\mathbf{A}}}
\def\vecP{{\mathbf{P}}}
\def\vecsigma{{\mathbf{\sigma}}}
\def\hatJn{{\hat{J_\vecn}}}
\def\hatJx{{\hat{J_x}}}
\def\hatJy{{\hat{J_y}}}
\def\hatJz{{\hat{J_z}}}
\def\hatj#1{\hat{J_{#1}}}
\def\hatphi{{\hat{\phi}}}
\def\hatq{{\hat{q}}}
\def\hatpi{{\hat{\pi}}}
\def\vel{\upsilon}
\def\Dint{{\mathcal{D}}}
\def\adag{{\hat{a}^\dagger}}
\def\bdag{{\hat{b}^\dagger}}
\def\cdag{{\hat{c}^\dagger}}
\def\ddag{{\hat{d}^\dagger}}
\def\hata{{\hat{a}}}
\def\hatb{{\hat{b}}}
\def\hatc{{\hat{c}}}
\def\hatd{{\hat{d}}}
\def\hatN{{\hat{N}}}
\def\hatH{{\hat{H}}}
\def\hatp{{\hat{p}}}
\def\Fup{{F^{\mu\nu}}}
\def\Fdown{{F_{\mu\nu}}}
\def\newl{\nonumber \\}
\def\SIkm{\,\mathrm{km}}
\def\SIyr{\,\mathrm{yr}}
\def\SIGyr{\,\mathrm{Gyr}}
\def\SIeV{\,\mathrm{eV}}
\def\SIkeV{\,\mathrm{keV}}
\def\SIMeV{\,\mathrm{MeV}}
\def\SIGeV{\,\mathrm{GeV}}
\def\SIcal{\,\mathrm{cal}}
\def\SIkcal{\,\mathrm{kcal}}
\def\SImol{\,\mathrm{mol}}
\def\SIm{\,\mathrm{m}}
\def\SIcm{\,\mathrm{cm}}
\def\SIfm{\,\mathrm{fm}}
\def\SImm{\,\mathrm{mm}}
\def\SInm{\,\mathrm{nm}}
\def\SImum{\,\mathrm{\mu m}}
\def\SIJ{\,\mathrm{J}}
\def\SIkJ{\,\mathrm{kJ}}
\def\SIs{\,\mathrm{s}}
\def\SIkg{\,\mathrm{kg}}
\def\SIg{\,\mathrm{g}}
\def\SIK{\,\mathrm{K}}
\def\SImmHg{\,\mathrm{mmHg}}
\def\SIPa{\,\mathrm{Pa}}
\def\vece{\mathrm{e}}
\def\bmat#1{\left(\begin{array}{#1}}
\def\emat{\end{array}\right)}
\def\bcase#1{\left\{\begin{array}{#1}}
\def\ecase{\end{array}\right.}
\def\calM{{\mathcal{M}}}
\def\calT{{\mathcal{T}}}
\def\calR{{\mathcal{R}}}
\def\barpsi{\bar{\psi}}
\def\baru{\bar{u}}
\def\barv{\bar{\upsilon}}
\def\bmini#1{\begin{minipage}{#1\textwidth}}
\def\emini{\end{minipage}}
\def\qeq{\stackrel{?}{=}}
\def\torder#1{\mathcal{T}\left(#1\right)}
\def\rorder#1{\mathcal{R}\left(#1\right)}
\def\contr#1#2{\contraction{}{#1}{}{#2}#1#2}
\def\trof#1{\mathrm{Tr}\left(#1\right)}
\def\trace{\mathrm{Tr}}
\def\comm#1{\ \ \ \left(\mathrm{used}\ #1\right)}
\def\tcomm#1{\ \ \ (\text{#1})}
\def\slp{\slashed{p}}
\def\slk{\slashed{k}}
\def\wulian{\includegraphics[width=0.18in]{emoji_wulian.jpg}}
\def\bye{\includegraphics[width=0.18in]{emoji_bye.jpg}}
\def\calp{{\mathfrak{p}}}
\def\veccalp{\mathbf{\mathfrak{p}}}
\def\atm{\,\mathrm{atm}}
\def\angstrom{\,\text{\AA}}
\def\Tthree{T_{\tiny \textcircled{3}}}
\def\pthree{p_{\tiny \textcircled{3}}}

\def\courseurl{http://zhiqihuang.top}

\def\tpage#1#2{
\begin{frame}
\bch
\begin{center}
\begin{large}
热学 \\
第#1讲 #2

\end{large}

\skiplines

黄志琦


\end{center}

\skiplines

{\small 
教材:《热学》第二版,赵凯华,罗蔚茵,高等教育出版社


课件下载
}
\courseurl 
\ech
\end{frame}
}

\def\bfr#1{
\begin{frame}
\chtitle{#1} 
\bch
}

\def\efr{
\ech 
\end{frame}
}

\title{Lesson 16 Two Forms of $\dbar Q$}
  \author{}
  \date{}
\begin{document}
\tpage{16}{准静态过程吸热量的两种表述}

\section{Review}


\begin{frame}
\chtitle{本讲内容提要}  
\bchL
\bitem
\item{类范德瓦耳斯方程}
\item{内能和状态方程的关系,准静态过程吸热量的第一种表述}
\item{焓和状态方程的关系,准静态过程吸热量的第二种表述}
\eitem
\echL
\end{frame}

\section{$\dbar Q$ 1$^{st}$ form}
\secpage{准静态过程吸热量的第一种表述}{从内能的角度出发$$ \dbar Q = C_V dT + T\pfrac pTV dV $$}
  
\begin{frame}
\chtitle{类范德瓦耳斯状态方程}  
\bchL
理想气体:
$$ p = nkT $$
范德瓦耳斯气体:
$$ p = \frac{V}{V-\nu b} n kT - \frac{\nu ^2 a}{V^2}.$$
一般性地,我们考虑状态方程:
$$p = n_{\rm eff}(V) kT + p_U(V) .$$
其中$n_{\rm eff}(V)$描述等效(可以运动)的分子数密度;$p_u$是分子间相互作用力导致的内压强。假设两者都只和体积有关。
\echL
\end{frame}


\begin{frame}
\chtitle{类范德瓦耳斯状态方程}  
\bchL
对类范德瓦耳斯状态方程,动理压强可以写成
$$ p_k = n_{\rm eff}kT = T \pfrac{p}{T}{V}. $$
于是,内压强
$$p_U = p-p_k = p - T \pfrac{p}{T}{V}. $$
是不是有些眼熟?
\echL
\end{frame}

\begin{frame}
\chtitle{类范德瓦耳斯状态方程:做功}  
\bchL
当体积变化时,物质对外界做功为$p dV$。这个功可以分为两部分:
\bitem
\item{动理压强做功$p_kdV$,即通过“分子碰撞”(宏观上体现为热量传输)对外界做功;}
\item{另一部分是内压强做功$p_UdV$,即内部分子间的吸引力做功,这是通过消耗分子间的势能来完成的。}
\eitem
\echL
\end{frame}



\begin{frame}
\chtitle{类范德瓦耳斯状态方程:吸热量}  
\bchL
当体积不变,物质吸收的热量用于改变分子的平均能量(能均分定理给出的$C_V dT$),以内能的形式存储在物质内部;当体积也同时发生变化时,吸收的热量还可以通过“分子碰撞力直接做功”,即动理压强直接做功($p_kdV$)的形式“挥霍掉”:

{\blue $$ \dbar Q = C_V dT + T\pfrac pTV dV. $$}

这是{\blue 准静态过程吸热量的第一种表述}。
\echL
\end{frame}


\begin{frame}
\chtitle{类范德瓦耳斯状态方程:内能的变化}  
\bchL
那“存起来”的能量($C_VdT$)是否代表了物质内能的全部变化呢?除了能均分定理给出的分子(独立)能量$C_V dT$之外,还有分子间相互作用的总势能,它会因为分子间相互作用力产生的内压强做功而被“挥霍掉”(即变化量为$-p_U dV$)。

{\blue $$ dU = C_V dT - \left(p - T\pfrac{p}{T}{V}\right) dV. $$}
当然,你也可以用前面$\dbar Q$的结果以及热力学第一定律($dU = \dbar Q-pdV$)得到一样的结果。
\echL
\end{frame}



\begin{frame}
\chtitle{思考题}  
\bchL

\addfig{1.2}{think5.jpg}

证明上述讨论结果:
{\blue $$ \dbar Q = C_V dT + T\pfrac pTV dV $$}
{\blue $$ dU = C_V dT  - \left(p - T\pfrac{p}{T}{V} \right) dV. $$}
对任意$pVT$系统的准静态过程成立。

\echL
\end{frame}


\section{$\dbar Q$ 2$^{nd}$ form}
\secpage{准静态过程吸热量的第二种表述}{从焓的角度出发 $$ \dbar Q = C_p dT - T\pfrac VTp dp.$$}


\begin{frame}
\chtitle{焓的物理意义}
\bchL

内能 $\Rightarrow$ 房子的价值

\skipline

焓 $\Rightarrow$ 房子的价值 + 房子所占空间的价值


\addfig{2}{houses.jpg}

(空间价值有时候可以占主导)


\echL
\end{frame}

\begin{frame}
\chtitle{焓的物理意义}
\bchL
很多过程都是等压过程(例如开放环境下的化学反应,自然界物态变化等)。

\lfig{1.1}{chemical_reaction.jpg}\hspace{0.1in}\lfig{1.4}{ice_melt.jpg}

当体积发生变化时就需要对环境做功以“获得生存空间”。这部分额外做的功做为“无形资产”加到内能上,就是焓$H$。

$$ H  \equiv U + pV$$ 

\echL
\end{frame}

\begin{frame}
\chtitle{定压热容}
\bchL
在{\blue 准静态等体过程}中,因为没有做功,所以{\blue 吸热量等于内能的变化}。定体热容
$$C_V \equiv \pfrac UTV. $$

在{\blue 准静态等压过程}中,一部分吸热量被消耗于做功,但是因为焓已经计入了做功所需的额外能量,所以{\blue 吸热量等于焓的变化}。定压热容
$$ C_p \equiv \pfrac HTp $$
\echL
\end{frame}


\begin{frame}
\chtitle{焓和状态方程的关系}
\bch

固定温度,变化压强时,
$$ (dH)_T = T(dS)_T + V(dp)_T = \left( T\pfrac SpT + V\right) (dp)_T$$
因为$d G = - SdT + Vdp$是全微分,所以有麦克斯韦关系 $ \pfrac SpT = - \pfrac VTp$,上式变为:
$$ (dH)_T = \left( V - T\pfrac VTp\right) (dp)_T$$
两边同除以$(dp)_T$,即得到{\blue 焓和状态方程的关系
$$\pfrac HpT = V - T\pfrac VTp$$  
}

\ech
\end{frame}


\begin{frame}
\chtitle{准静态过程吸热量的第二种表述}
\bchL
把焓看成$p, T$的函数,就有全微分:
{\blue $$ dH = C_p dT + \left(V - T\pfrac VTp\right) dp. $$}
然后利用$ dH = \dbar Q + Vdp$,即得到{\blue 准静态过程吸热量的第二种表述:
  $$ \dbar Q = C_p dT - T\pfrac VTp dp.$$}
\echL
\end{frame}

\begin{frame}
\chtitle{总结}
\bchL
内能和焓的全微分为:
$$ dU = T dS - p dV; \ dH = T dS + V dp.$$
似乎把内能看成$S, V$的函数,焓看成$S, p$的函数比较自然。

\skipline

但是实际状态方程往往以$p,V,T$之间的关系给出,$S$是一个“比较难用”的态函数。因此我们希望把内能看成$T, V$的函数,焓看成$T, p$的函数。
\echL
\end{frame}

\begin{frame}
\chtitle{总结}
\bchL
通过偏导数的一些运算技巧,我们成功地得到:
{\blue $$ dU = C_V dT  - \left(p - T\pfrac{p}{T}{V} \right) dV. $$}

{\blue $$ dH = C_p dT + \left(V - T\pfrac VTp\right) dp. $$}
\echL
\end{frame}

\begin{frame}
\chtitle{总结}
\bchL
最后根据$ \dbar Q = dU + pdV = dH - Vdp $,可以得到$\dbar Q$的两种表述:
\tbox{$$ \dbar Q = C_V dT + T\pfrac pTV dV $$}
\tbox{$$ \dbar Q = C_p dT - T\pfrac VTp dp.$$}
\echL
\end{frame}



\begin{frame}
\chtitle{思考题}
\bchL
范德瓦耳斯气体满足状态方程:
$$ p = \frac{\nu R T}{V-\nu b} - \frac{\nu^2a}{V^2}. $$ 
设该气体经历等温准静态过程,体积从$10\nu b$变大到 $20\nu b$,计算该过程气体的摩尔熵(即每摩尔气体的熵)$S^{\rm mol}$的变化。
\echL
\end{frame}

\end{document}

\documentclass[CJK]{beamer}
\usepackage{CJKutf8}
\usepackage{beamerthemesplit}
\usetheme{Malmoe}
\useoutertheme[footline=authortitle]{miniframes}
\usepackage{amsmath}
\usepackage{amssymb}
\usepackage{graphicx}
\usepackage{eufrak}
\usepackage{color}
\usepackage{slashed}
\usepackage{simplewick}
\usepackage{tikz}
\graphicspath{{../figures/}}
\def\addfig#1#2{\begin{center}\includegraphics[width=#1 in]{#2}\end{center}}
\def\blacktext#1{{\color{black}#1}}
\def\bluetext#1{{\color{blue}#1}}
\def\redtext#1{{\color{red}#1}}
\def\darkbluetext#1{{\color[rgb]{0,0.2,0.6}#1}}
\def\skybluetext#1{{\color[rgb]{0.2,0.7,1.}#1}}
\def\cyantext#1{{\color[rgb]{0.,0.5,0.5}#1}}
\def\greentext#1{{\color[rgb]{0,0.7,0.1}#1}}
\def\darkgray{\color[rgb]{0.2,0.2,0.2}}
\def\lightgray{\color[rgb]{0.6,0.6,0.6}}
\def\gray{\color[rgb]{0.4,0.4,0.4}}
\def\blue{\color{blue}}
\def\red{\color{red}}
\def\green{\color{green}}
\def\darkblue{\color[rgb]{0,0.2,0.6}}
\def\skyblue{\color[rgb]{0.2,0.7,1.}}
\def\fdeg{{^\circ \mathrm{F}}}
\def\cdeg{^\circ \mathrm{C}}
\def\be{\begin{equation}}
\def\ee{\nonumber\end{equation}}
\def\bea{\begin{eqnarray}}
\def\eea{\nonumber\end{eqnarray}}
\def\ii{{\dot{\imath}}}
\def\bch{\begin{CJK}{UTF8}{gbsn}}
\def\ech{\end{CJK}}
\def\bitem{\begin{itemize}}
\def\eitem{\end{itemize}}
\def\bcenter{\begin{center}}
\def\ecenter{\end{center}}
\def\bex{\begin{minipage}{0.3\textwidth}\includegraphics[width=1in]{jugelizi.png}\end{minipage}\begin{minipage}{0.6\textwidth}}
\def\eex{\end{minipage}}
\def\chtitle#1{\frametitle{\bch#1\ech}}
\def\skipline{{\vskip0.1in}}
\def\skiplines{{\vskip0.2in}}
\def\lagr{{\mathcal{L}}}
\def\hamil{{\mathcal{H}}}
\def\vecv{{\mathbf{v}}}
\def\vecx{{\mathbf{x}}}
\def\vecy{{\mathbf{y}}}
\def\veck{{\mathbf{k}}}
\def\vecp{{\mathbf{p}}}
\def\vecn{{\mathbf{n}}}
\def\vecA{{\mathbf{A}}}
\def\vecP{{\mathbf{P}}}
\def\vecsigma{{\mathbf{\sigma}}}
\def\hatJn{{\hat{J_\vecn}}}
\def\hatJx{{\hat{J_x}}}
\def\hatJy{{\hat{J_y}}}
\def\hatJz{{\hat{J_z}}}
\def\hatj#1{\hat{J_{#1}}}
\def\hatphi{{\hat{\phi}}}
\def\hatq{{\hat{q}}}
\def\hatpi{{\hat{\pi}}}
\def\vel{\upsilon}
\def\Dint{{\mathcal{D}}}
\def\adag{{\hat{a}^\dagger}}
\def\bdag{{\hat{b}^\dagger}}
\def\cdag{{\hat{c}^\dagger}}
\def\ddag{{\hat{d}^\dagger}}
\def\hata{{\hat{a}}}
\def\hatb{{\hat{b}}}
\def\hatc{{\hat{c}}}
\def\hatd{{\hat{d}}}
\def\hatN{{\hat{N}}}
\def\hatH{{\hat{H}}}
\def\hatp{{\hat{p}}}
\def\Fup{{F^{\mu\nu}}}
\def\Fdown{{F_{\mu\nu}}}
\def\newl{\nonumber \\}
\def\SIkm{\,\mathrm{km}}
\def\SIyr{\,\mathrm{yr}}
\def\SIGyr{\,\mathrm{Gyr}}
\def\SIeV{\,\mathrm{eV}}
\def\SIkeV{\,\mathrm{keV}}
\def\SIMeV{\,\mathrm{MeV}}
\def\SIGeV{\,\mathrm{GeV}}
\def\SIcal{\,\mathrm{cal}}
\def\SIkcal{\,\mathrm{kcal}}
\def\SImol{\,\mathrm{mol}}
\def\SIm{\,\mathrm{m}}
\def\SIcm{\,\mathrm{cm}}
\def\SIfm{\,\mathrm{fm}}
\def\SImm{\,\mathrm{mm}}
\def\SInm{\,\mathrm{nm}}
\def\SImum{\,\mathrm{\mu m}}
\def\SIJ{\,\mathrm{J}}
\def\SIkJ{\,\mathrm{kJ}}
\def\SIs{\,\mathrm{s}}
\def\SIkg{\,\mathrm{kg}}
\def\SIg{\,\mathrm{g}}
\def\SIK{\,\mathrm{K}}
\def\SImmHg{\,\mathrm{mmHg}}
\def\SIPa{\,\mathrm{Pa}}
\def\vece{\mathrm{e}}
\def\bmat#1{\left(\begin{array}{#1}}
\def\emat{\end{array}\right)}
\def\bcase#1{\left\{\begin{array}{#1}}
\def\ecase{\end{array}\right.}
\def\calM{{\mathcal{M}}}
\def\calT{{\mathcal{T}}}
\def\calR{{\mathcal{R}}}
\def\barpsi{\bar{\psi}}
\def\baru{\bar{u}}
\def\barv{\bar{\upsilon}}
\def\bmini#1{\begin{minipage}{#1\textwidth}}
\def\emini{\end{minipage}}
\def\qeq{\stackrel{?}{=}}
\def\torder#1{\mathcal{T}\left(#1\right)}
\def\rorder#1{\mathcal{R}\left(#1\right)}
\def\contr#1#2{\contraction{}{#1}{}{#2}#1#2}
\def\trof#1{\mathrm{Tr}\left(#1\right)}
\def\trace{\mathrm{Tr}}
\def\comm#1{\ \ \ \left(\mathrm{used}\ #1\right)}
\def\tcomm#1{\ \ \ (\text{#1})}
\def\slp{\slashed{p}}
\def\slk{\slashed{k}}
\def\wulian{\includegraphics[width=0.18in]{emoji_wulian.jpg}}
\def\bye{\includegraphics[width=0.18in]{emoji_bye.jpg}}
\def\calp{{\mathfrak{p}}}
\def\veccalp{\mathbf{\mathfrak{p}}}
\def\atm{\,\mathrm{atm}}
\def\angstrom{\,\text{\AA}}
\def\Tthree{T_{\tiny \textcircled{3}}}
\def\pthree{p_{\tiny \textcircled{3}}}

\def\courseurl{http://zhiqihuang.top}

\def\tpage#1#2{
\begin{frame}
\bch
\begin{center}
\begin{large}
热学 \\
第#1讲 #2

\end{large}

\skiplines

黄志琦


\end{center}

\skiplines

{\small 
教材:《热学》第二版,赵凯华,罗蔚茵,高等教育出版社


课件下载
}
\courseurl 
\ech
\end{frame}
}

\def\bfr#1{
\begin{frame}
\chtitle{#1} 
\bch
}

\def\efr{
\ech 
\end{frame}
}

\title{Lesson 08 Enthalpy}
  \author{}
  \date{}
\begin{document}
\tpage{8}{焓}

\section{Review}

\begin{frame}
\chtitle{上讲需要掌握的内容非常少}
\bch
\bitem
\item{内能是态函数}
\item{热力学第一定律:$\Delta U = A + Q$。}
\item{绝热过程状态方程:$pV^\gamma = \const$ ,其中$\gamma = C_p/C_V = (C_V + \nu R)/C_V$。}
\item{大气的绝热近似:空气中的声速$u_s =\sqrt{\gamma}$;干燥空气垂直温度梯度$\frac{dT}{dz} \approx -10 \SIK/\SIkm$。}
\item{多方过程气体对外做功$A' = -\frac{\nu R}{n-1}\Delta T$ (等温过程$n=1$则需要另算)。}
\item{多方过程热容$C_n =C_V -\frac{\nu R}{n-1}$ (等温过程$C = \infty$)。}
\item{吸热的本质;内能和状态方程的关系$\pfrac UVT = T\pfrac pTV - p$;克拉珀龙方程;范德瓦尔斯气体的内能;热力学第二定律;孤立系它熵不减原理;卡诺定理……}

\eitem
\ech
\end{frame}

\begin{frame}
\chtitle{关于大气垂直温差的补充知识}
\bch
{\small
对水蒸气饱和的空气,做功等于内能改变量加上汽化热:
$$ -pdV = C_V dT + \Lambda d(c\nu) $$
其中$\nu$为空气摩尔数,$c$为空气中水蒸气的摩尔百分比,$\Lambda$为摩尔汽化热。又
$$pdV + Vdp = \nu R dT + \nu \Lambda dc$$
两式相加得到
$$ Vdp = C_p  dT + \nu \Lambda dc $$
再由力学平衡$ dp = -\rho g dz$,即有
$$ \frac{dT}{dz} =-\frac{\rho g V + \nu\Lambda\frac{dc}{dz}}{C_p} = -\frac{\gamma-1}{\gamma}\left(\frac{M^{\rm mol}g}{R}+\frac{\Lambda}{R}\frac{dc}{dz}\right)$$
因为饱和蒸气压随着温度降低而降低(请回忆三相图),在有饱和蒸汽的空气里,气团继续上升会造成水蒸气凝结,$ \frac{dc}{dz} < 0 $,垂直温度梯度就会小于$-10\SIK/\SIs$。
}
\ech
\end{frame}


\begin{frame}
\chtitle{传热的本质}
\bch
初中物理复习完了,下面我们开始深邃的思考

\addfig{2}{think4.jpg}

热一律里$Q$和$A$的界限在哪里?“传热”难道不是很多微观的“做功”的和吗?

\ech
\end{frame}


\begin{frame}
\chtitle{吸热的本质(续)}
\bch
我们把热一律写成“收入=存款+开支”的形式:
$$ \dbar Q =  d U + \dbar A'$$
为了简化讨论,我们只考虑非相对论的粒子数守恒系统,并假设分子自由度只依赖于温度。先考虑准静态过程。

\skipline

先把内能分为两部分。我们知道每个分子单独具有的平均能量只是温度的函数,我们把这样的能量加起来,称之为{\blue 局部内能$U_{\rm local}$}。此外,分子之间还有相互作用的势能,我们把这部分能量称为{\blue 全局内能$U_{\rm global}$}。


再把物质对外做功也分为两部分。因为分子运动造成的{\blue 动理压强$p_k$对外做功记$ A_k'$},因为分子之间相互吸引造成的{\blue 内压强$p_U$对外做功记为$ A'_U$}。显然
{\blue
$$ d U_{\rm global} = - \dbar A'_U$$
}
\ech
\end{frame}


\begin{frame}
\chtitle{吸热的本质(续)}
\bch
因为$d U_{\rm global}$和$\dbar A'_U$互相抵消,热一律就可以写成:
$$ \dbar Q = d U_{\rm local}  +\dbar A'_k  $$
它揭示了准静态过程的吸热量$Q$的本质:

\tbox{\blue 准静态过程吸热量用来提高系统的局部内能和提供系统动理压强做功。}

更具体地说:吸热确实是很多次系统分子和环境分子的微观碰撞的总和。这些微观碰撞一方面改变了系统分子的平均能量(改变系统局部内能),另一方面又做了功(宏观上表现为动理压强做功)。


\ech
\end{frame}

\begin{frame}
\chtitle{热量的显式方程}
\bch
根据能均分定理,{\blue 局部内能只是温度的函数},
$$d U_{\rm local} = f(T) dT$$
其中$f(T)$是由能均分定理决定的函数。

\skipline

动理压强$p_k = \frac{2}{3}n_{\rm eff}\overline{\varepsilon}$,其中分子平均平动动能$\overline{\varepsilon}$和温度成正比,但这里的“有效分子数密度”$n_{\rm eff}$一般来讲不是真正的分子数密度,因为在大多数物质中分子并不能随心所欲地在整个体积内运动(只有理想气体分子可以)。我们可以一般地假设$n_{\rm eff}$是体积的函数,并把动理压强写成
{\blue
$$ p_k = g(V) T$$}
这样,动理压强做功为$g(V)T dV$。最后得到热量显式方程:$$\dbar Q = f(T)dT + Tg(V) dV$$
下面我们来求$f(T)$和$g(V)$。

\ech
\end{frame}



\begin{frame}
\chtitle{$f(T)$和$g(V)$的数学表达式}
\bch
显然,因为只和分子平均距离有关,{\blue 全局内能和内压强只是体积的函数}。所以固定体积时,

\bitem
\item{内能对温度的偏导数等于局部内能对温度的偏导数:
$$ \pfrac UTV = \pfrac {U_{\rm local}}TV = f(T) $$
也就是说$f(T)$其实就是$C_V$。}
\item{
总压强对温度的偏导数等于动理压强对温度的偏导数:
$$\pfrac pTV = \pfrac {p_k}TV = g(V)$$}
\eitem
\ech
\end{frame}

\begin{frame}
\chtitle{总结}
\bch
准静态过程热量的显式方程
\tbox
{$$\dbar Q = C_V(T)dT + T g(V) dV$$}
其中$C_V(T)$代表了分子自由度数目随温度变化的情况;$Tg(V)$是动理压强,其中
\tbox{
$$g(V) = \pfrac pTV $$}
代表了能够和环境发生碰撞的有效分子数密度随总体积变化的情况。

\ech
\end{frame}

\begin{frame}
\bch

下面我们用热量的显式方程来推导一系列超出初中水平的结果

\ech
\end{frame}


\begin{frame}
\chtitle{内能和物态方程的关系}
\bch
固定温度,变化体积时:
$$\dbar Q = T g(V)dV = T\pfrac pTV dV$$
$$\dbar A = -pdV$$
故
$$dU = \left(T \pfrac pTV - p\right)dV$$
即{\blue
$$\pfrac UVT = T \pfrac pTV - p$$
}
这是一个非常重要的方程,它把物质的内能和状态方程联系起来了。
\ech
\end{frame}

\begin{frame}
\chtitle{思考题}
\bch

\addfig{1}{think2.jpg}
$$\pfrac UVT = T \pfrac pTV - p$$
这个方程的右边是什么物理量?试直接给出该方程的物理诠释。

\ech
\end{frame}


\begin{frame}
\chtitle{相变温度和相变压强的关系}
\bch
{\small

在恒定压强$p$下,物质相变时保持相变温度$T$不变(不考虑非晶体等特殊情形),设每摩尔物质从$\alpha$态变到$\beta$态需要吸热量为$\Lambda^{\rm mol}$,$\alpha$态和$\beta$态的摩尔体积分别为$V_\alpha^{\rm mol}$和$V_\beta^{\rm mol}$。

\skipline

我们曾经定性地分析过,相变温度$T$对压强$p$有依赖性,相变过程体积变化($V_\beta^{\rm mol}-V_\alpha^{\rm mol}$)越大,相变温度对$p$的依赖性越强。

\skipline

下面我们来把这种依赖性定量化。
}
\ech
\end{frame}


\begin{frame}
\chtitle{图解}
\bch
先从图像上理解我们要计算的是什么量:

\lfig{1.8}{PVTdiagram.png}\hspace{0.1in}\lfig{1.8}{PTdiagram.png}

三相图上有三个两相共存的曲面(固液共存,气液共存,固气共存),这些曲面投影到三相图($PT$图)上后即为三条相变曲线。
这些曲线的斜率反映了相变温度随压强变化的情况。我们要求解的就是这些曲面上的$\pfrac pTV$。
\ech
\end{frame}


\begin{frame}
\chtitle{克拉珀龙(Clapeyron)方程}
\bch
假设有摩尔数为$d\nu$的$\alpha$态物质转化为$\beta$态物质,因为温度不变,吸热量等于$Tg(V)dV$
$$ \Lambda^{\rm mol} d\nu = Tg(V)dV = T\pfrac pTV dV $$
再根据$dV = \left(V_\beta^{\rm mol} - V_\alpha^{\rm mol}\right)d\nu$,即得到
{\blue 克拉珀龙(Clapeyron)方程
$$ \pfrac pTV = \frac{\Lambda^{\rm mol}}{T\left(V_\beta^{\rm mol} - V_\alpha^{\rm mol}\right)}$$}
\ech
\end{frame}


\begin{frame}
\chtitle{范德瓦尔斯气体的内能}
\bch
{\small
固定温度改变体积时,局部内能不变,全局内能因内压强做功而改变。

对范德瓦尔斯气体,我们假设了内压强$P_U = -\frac{a\nu^2}{V^2}$,故
$$\pfrac UVT = \frac{a\nu^2}{V^2}$$
由此可以积分得到
$$U(V, T) = \int_{T_0}^T C_V(T) dT -\frac{\nu^2a}{V} + U_0$$
}
\ech
\end{frame}

\begin{frame}
\chtitle{克劳修斯熵}
\bch
假设物质从状态$(T_1, V_1)$经过准静态过程变化到$(T_2, V_2)$,则可以定义{\blue(克劳修斯)熵变:
$$ \Delta S \equiv \int\frac{\dbar Q}{T} = \int_{T_1}^{T_2} \frac{C_V(T)}{T}dT + \int_{V_1}^{V_2} g(V)dV$$}
显然,{\blue $S$是和过程无关的态函数}。

准静态过程的吸热量可以写成
$$ \dbar Q  = T dS$$
热力学第一定律微分形式就可以写成:
{\blue $$ dU = TdS - pdV$$}
\ech
\end{frame}

\begin{frame}
\chtitle{偏导数的噩梦}
\bch
根据上面熵$S$的积分表达式,不难写出
\bea
\pfrac STV &=& \frac{C_V}{T} = \frac{1}{T} \pfrac UTV \newl
\pfrac SVT &=& g(V) = \pfrac pTV \newl
\eea
热学里可以写出几十个这样“莫名其妙”的偏导数等式,我们之后会系统地学习怎样推导它们(三四章的难点)。
\ech
\end{frame}


\begin{frame}
\chtitle{本讲内容}
\bch
\bitem
\item{数学准备知识:循环偏微分乘积定理}
\item{焓}
\item{节流过程(等焓的非准静态过程)}
\item{化学反应和生成焓}
\eitem
\ech
\end{frame}

\section{Math}

\begin{frame}
\chtitle{数学准备知识:循环偏微分乘积定理}
\bch
设三个变量$X$, $Y$, $Z$满足某状态方程,则
{\blue 
$$\pfrac XYZ  \pfrac YZX \pfrac ZXY = -1$$
}
当然,我们这是讨论物理问题,可以假设状态方程足够光滑,且不出现偏导数无穷大的情况。

\skipline

\bye不会证的请默默捧起数学书。
\ech
\end{frame}

\begin{frame}
\chtitle{什么,数学书早卖了?}
\bch
{\scriptsize
证明:设状态方程为$S(X, Y, Z) = 0$,取状态方程曲面上的一条等高线($dZ = 0$),在等高线上有
$$\pfrac SX{YZ} dX + \pfrac SY{ZX} dY = 0$$
又根据偏导数的定义,等高线上的$dX/dY$即$\pfrac XYZ$。故有
$$\pfrac XYZ = -\frac{\pfrac SY{ZX}}{\pfrac SX{ZY}}$$
同理可证
$$\pfrac YZX = -\frac{\pfrac SZ{XY}}{\pfrac SY{ZX}}$$
$$\pfrac ZXY = -\frac{\pfrac SX{YZ}}{\pfrac SZ{XY}}$$
三式相乘即得证
}
\ech
\end{frame}

\begin{frame}
\chtitle{思考题}
\bch
\addfig{1}{songfen2.jpg}

证明气体定体比热容可以写成:

$$
C_V =  - \pfrac UVT \pfrac VTU 
$$

\ech
\end{frame}


\section{Enthalpy}

\begin{frame}
\chtitle{焓}
\bch

内能 $\Rightarrow$ 你房子的价值

焓$\Rightarrow$你房子的价值 + 你房子所占空间的价值

\skipline

自然界很多过程都是等压过程而非等体过程,当体积发生变化时就需要对环境做功以“获得生存空间”。这部分额外做的功做为“无形资产”加到内能上,就是“焓”(enthalpy)。

$$ H \equiv U + pV$$


\ech
\end{frame}

\begin{frame}
\chtitle{思考题}
\bch
\addfig{0.8}{think3.jpg}
焓是态函数吗?
\ech
\end{frame}


\begin{frame}
\chtitle{理想气体的焓只是温度的函数}
\bch
固定摩尔数的理想气体的内能只是温度的函数,故焓
$$H = U(T) + pV = U(T) + \nu R T$$
也只是温度的函数。
\ech
\end{frame}

\begin{frame}
\chtitle{实际气体的焓}
\bch
由于理想气体近似已经给出
$$\pfrac HTp =  C_V+ \nu R > 0$$
实际气体一般只是对上式稍有偏离,并不会影响$\pfrac HTp $的正负符号。

比较不确定的是$H$对$p$的依赖关系,因为理想气体近似给出$\pfrac HpT = 0$,实际气体如果有所偏离的话,是往哪个方向偏离呢?
\ech
\end{frame}

\begin{frame}
\chtitle{焓和状态方程的关系}
\bch
{\small
我们上一讲中提到内能和状态方程的关系{\blue
$$\pfrac UVT = T\pfrac pTV - p$$}
由此可以得到
$$\pfrac UpT = \pfrac UVT\pfrac VpT= \left(T\pfrac pTV - p\right) \pfrac VpT$$
所以
$$\pfrac HpT = \pfrac UpT  + p\pfrac VpT + V = T\pfrac pTV \pfrac VpT + V$$
最后我们利用循环偏微分乘积定理,得到$\pfrac pTV \pfrac VpT = - \pfrac VTp$,即
{\blue
$$\pfrac HpT = - T\pfrac VTp + V$$
}}
\ech
\end{frame}



\begin{frame}
\chtitle{实际气体的$\pfrac HpT$}
\bch

对范德瓦尔斯气体,我们可以直接算出
$$\pfrac HpT = V\frac{\frac{\nu b}{V} + \frac{2p_U}{p_k}\left(1- \frac{\nu b}{V}\right)}{1 + \frac{2p_U}{p_k}\left(1-\frac{\nu b}{V}\right)}\approx V\left(\frac{\nu b}{V} +\frac{2p_U}{p_k}\right)$$

注意$p_U/p_k<0$,所以上式的符号取决于体积修正和压强修正的相对大小。

\bmini{0.48}
\lfig{1.9}{pfracHpT.jpg}
\emini
\bmini{0.48}
{\small 
实验表明,在常温常压下,氢气和氦气$\pfrac HpT > 0$,而大多数其他气体则有$\pfrac HpT < 0$。

在高温高压下,体积修正是主导项,所有气体均有$\pfrac HpT > 0$。
}
\emini

\ech
\end{frame}

\begin{frame}
\chtitle{思考题}
\bch
\addfig{0.8}{think.jpg}

根据范德瓦尔斯模型,$$\pfrac HpT\approx V\left(\frac{\nu b}{V} + \frac{2p_U}{p_k}\right)$$
可以看成体积修正项($>0$)与压强修正项($<0$)相互竞争的结果。在常温常压下,对小分子气体,为什么反而结果是$\pfrac HpT > 0$,即体积修正项为主导呢?

\ech
\end{frame}

\begin{frame}
\chtitle{节流过程和焦耳-汤姆孙效应}
\bch

\bmini{0.48}
\addfig{2}{throttling_process.png}
\emini
\bmini{0.48}
用多孔塞把绝热箱隔开,两端分别施以恒定压强$p_1$, $p_2$ ($p_1>p_2$)。
\emini

{\small

假设一开始气体都在左边,体积为$V_1$,内能为$U_1$。经过一段时间,左边的气体都通过多孔塞被压到了右边,体积和内能变为$V_2$, $U_2$。

那么由于左边活塞对气体做功$p_1V_1$,右边活塞对气体做功$-p_2V_2$,所以气体内能变化为

$$U_2 - U_1  = p_1V_1 - p_2V_2$$

即
$$ U_1 + p_1V_1 = U_2 + p_2V_2 $$
{\blue \bf 节流过程是等焓过程} (注意:节流过程一般不是准静态过程)

}
\ech
\end{frame}


\begin{frame}
\chtitle{焦耳-汤姆孙效应}
\bch
{\small
我们可以根据焦耳-汤姆孙系数$\alpha \equiv \pfrac TpH$来计算的温度的升降。

根据循环偏微分乘积定理:
$$\pfrac TpH \pfrac pHT \pfrac HTp = -1$$
和$\pfrac HTp > 0$,就知道$\pfrac TpH$和$\pfrac HpT$符号相反。

也就是说,在常温常压下,对氢气和氦气$\pfrac TpH < 0$, 对大多数其他气体$\pfrac TpH > 0$。

因为节流过程是降压过程,所以对{\bf 大部分气体(空气,氮气,氧气等)节流过程是降温过程(正节流效应),而对氢气和氦气是升温过程(负节流效应)}。这些效应统称{\bf 焦耳-汤姆孙效应}。
}
\ech
\end{frame}


\begin{frame}
\chtitle{节流液化气体}
\bch
{\small
根据范德瓦尔斯状态方程的定性分析,在高温高压下,气体呈现负节流效应,在低温低压下,气体呈现正节流效应。那么只要把气体降温到某个温度之下使得节流效应为正,则节流效应可以对气体无限降温使它最终液化。

\skiplines

请参考教材图3-15的节流液化装置。

\skiplines

\bitem
\item{实际气体在极低温时未必符合范德瓦尔斯模型,具体例子可见教材140页图3-13,氢气在极低温时也有一部分负节流效应区域。这样氢气的节流液化过程就会有一定的参数限制。}
\item{自由膨胀也能给气体降温。柯林斯液化机就是结合自由膨胀和节流两种方法。}
\eitem

}
\ech
\end{frame}

\section{Enthalpy of Reaction}


\begin{frame}
\chtitle{化学反应热和生成焓}
\bch
定义化学反应热 = 外界给系统的热量

\skipline

\bitem
\item{定体化学反应 $\Rightarrow$ 内能变化 = 反应热}
\item{定压化学反应 $\Rightarrow$ 焓变化 = 反应热}
\eitem

\skipline

定压反应的反应热又称为{\blue 反应焓}。
\ech
\end{frame}

\begin{frame}
\chtitle{焓的参考点:标准生成焓}
\bch
焓和势能一样,可以取任意参考点,{\blue 习惯上定义 $1\atm$, $25\cdeg$的纯元素的焓为零}。

\skipline

{\blue $1\atm$,$25\cdeg$下化合物的生成焓定义为化合物的标准生成焓(用$\stdHf$表示)}。

\skipline

{\blue $1\atm$,$25\cdeg$下,

 反应焓 = 合成物的标准生成焓 - 反应物的标准生成焓}

\skiplines

标准生成焓可以通过查表获得。见教材145页表3-3。

\ech
\end{frame}

\begin{frame}
\chtitle{反应焓计算举例}
\bch
\bex
教材习题3-23
\eex

\skipline

记给的三个燃烧反应为$A, B, C$。则 $2A + \frac{3}{2}B - \frac{1}{2}C$就等价于所需要的纯元素合成反应。故
{\small
\bea
\stdHf({\rm C_2H_6}) &=&\left[ 2\times (-393.5) + \frac{3}{2}\times (-571.7)-\frac{1}{2}\times(-1560)\right]\SIkJ/\SImol\newl
&=&-864.6\SIkJ/\SImol
\eea
}
\ech
\end{frame}

\begin{frame}
\chtitle{第九周作业(序号接第八周)}
\bch
\bitem
\item[22]{证明气体的定压比热容$C_p$可以写成
$$C_p = - \pfrac pTH \pfrac HpT $$}
\item[23]{教材习题3-14}
\item[24]{教材习题3-24}
\eitem
\ech
\end{frame}
\end{document}

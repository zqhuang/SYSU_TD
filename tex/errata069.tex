\documentclass[12pt,CJK]{article}
\usepackage{geometry}
\input{reduced_macros.tex}
%\geometry{tmargin=0.3in, bmargin=0.5in, lmargin=0.8in, rmargin=0.8in, nohead, nofoot}
\begin{document}
\bch
\title{\S2.1.3 (page 69) 麦克斯韦分布律的导出}
\author{Zhiqi Huang}
\date{}
\maketitle
教材上这一节的表述非常呵呵。


下面我用严谨的数学语言来重复这部分的推导。

{\bf 命题: 如果速度分布律满足如下两个条件
\bitem
\item{分布函数连续且各向同性}
\item{每个维度的速度分量的分布是独立的(即和其他分量没有关联)}
\eitem
则速度分布律一定是对应某个温度的麦克斯韦分布。}

证明:由各向同性条件,速度分布函数可以写成$\upsilon$的一个函数(不妨设为$g$)。再根据各维度独立条件,速度分布函数又可以写成各个分量的分布(不妨设为$f$)的乘积。
  \begin{equation}
  g(\upsilon) = f(\upsilon_x)f(\upsilon_y)f(\upsilon_z)
  \end{equation}
  两边取对数,得到
  \begin{equation}
  \ln g(\upsilon) =\ln f(\upsilon_x)+ \ln f(\upsilon_y)+ \ln f(\upsilon_z)
  \end{equation}
  令$\upsilon_x=\sqrt{a}$, $\upsilon_y=\upsilon_z=0$,得到
  \begin{equation}
  \ln g(\sqrt{a}) = \ln f(\sqrt{a}) + 2 \ln f(0). \label{a1}
  \end{equation}
  当然,我可以把$a\ge 0$换成任意$b\ge 0$
  \begin{equation}
  \ln g(\sqrt{b}) = \ln f(\sqrt{b}) + 2 \ln f(0). \label{a2}
  \end{equation}
  然后令$\upsilon_x= \sqrt{a}$, $\upsilon_y=\sqrt{b}$, $\upsilon_z=0$,则有
  \begin{equation}
  \ln g(\sqrt{a+b}) =  \ln f(\sqrt{a}) + \ln f(\sqrt{b}) + \ln f(0). \label{a3}
  \end{equation}
  在Eqs.~(\ref{a1}-\ref{a3})中消去$\ln f(\sqrt{a})$, $\ln f(\sqrt{b})$得到
  \begin{equation}
    \ln g(\sqrt{a+b}) = \ln g(\sqrt{a}) + \ln g(\sqrt{b})- 3\ln f(0) \label{eqadd}    
  \end{equation}
  
  再定义
  \begin{equation}
  G(a)\equiv \ln g(\sqrt{a}) - 3 \ln f(0) \label{eq:def}
  \end{equation}
  代入Eq.~\eqref{eqadd},得到
  \begin{equation}
  G(a+b) = G(a)+G(b) \label{eq0}
  \end{equation}
  对任意$a\ge 0, b\ge 0$成立。
  
  反复利用Eq.~\eqref{eq0}用归纳法容易证明,对任意正整数$n,m$,都有
  \begin{equation}
  G(n) = m G\left(\frac{n}{m}\right), \label{eq1}
  \end{equation}
  记$G(1)=c$,上式令$m=n$即得到
  \begin{equation}
  G(n) = n c
  \end{equation}
  代回Eq.~\eqref{eq1}得到
  \begin{equation}
  G\left(\frac{n}{m}\right) = c \frac{n}{m}
  \end{equation}
  也就是对任意正的有理数$x$都有$G(x) = cx$。由分布函数的连续性知道$G$也是连续的,所以对任意正实数$x$,都有
  \begin{equation}
  G(x) = cx.  
  \end{equation}
  那么就反解Eq.~\eqref{eq:def}得到
  \begin{equation}
  g(\upsilon) \propto e^{G(\upsilon^2)} = e^{c\upsilon^2} 
  \end{equation}
  由概率密度函数归一化条件可以确定$c<0$,得到麦克斯韦分布。

  
  这个推导方法显然可以推广到任意不低于二维的空间。

\ech
\end{document}

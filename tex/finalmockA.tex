\documentclass[10pt,CJK]{article}
\usepackage{geometry}
\input{reduced_macros.tex}
\geometry{tmargin=0.3in, bmargin=0.5in, lmargin=0.7in, rmargin=0.7in, nohead, nofoot}
\def\mark#1{{\color{blue} (#1分)}}
\renewcommand{\thepage}{}
\begin{document}
\bch
{\large 热学课堂练习 初入江湖版 }

{\vskip 0.05in}

姓名 ....................... {\hskip 0.5in}    学号 .......................{\hskip 0.5in}  分数 ...................


\bitem

\item[(一)]{选择题,每题5分,共30分。
\bitem
\item[(1)]{
  道尔顿分压定律指的是混合气体的 \bropt

  \optlist{体积等于各组分的体积之和}{压强等于各组分的压强之和}{温度等于各组分温度之和}

}

\item[(2)]{标准状态下,空气分子到离它最近的分子的典型距离大致量级是 \bropt

  \optlist{$10^{-9}\SIm$}{$10^{-7}\SIm$}{$10^{-5}\SIm$}
  
}

  
\item[(3)]{热力学第一定律   \bropt

  \optlist{只适用于可逆过程}{只适用于孤立体系}{是能量守恒定律的一种形式}
}

\item[(4)]{摩擦生热是不可逆过程,这是哪个物理定律决定的?   \bropt

  \optlist{热力学第零定律}{热力学第一定律}{热力学第二定律}
}


\item[(5)]{孤立平衡态纯物质的压强,体积,温度,熵分别记作$p,V,T,S$,则下列哪个等式恒成立? \bropt

  \optlist{$\pfrac TpS = \pfrac VSp$}{$\pfrac TVS = \pfrac pSV$ }{$\pfrac SpT = \pfrac VTp$}
}
  
  
  
\item[(6)]{分子质量为$m$,温度为$T$的热平衡理想气体,记分子速度$x$分量的绝对值为$u = |\upsilon_x|$。则$u$的分布函数为 \bropt
  
  \optlist{$\sqrt{\frac{m}{2\pi  kT}} e^{-\frac{mu^2}{2kT}}$}{ $\sqrt{\frac{2m}{\pi kT }} e^{-\frac{mu^2}{2kT}}$ }{ $\left(\frac{m}{2\pi kT}\right)^{3/2} e^{-\frac{mu^2}{2kT}}$ }

  
}
  



  \eitem
}


\item[(二)]{填空题,每题5分,共25分。

\bitem
\item[(1)]{某气体经历压强$p=10^5\SIPa$的准静态等压过程,体积增大了$1\SIL$,内能增大了$100\SIJ$。这个过程中气体吸收的热量为\uline{0.5}.}  
\item[(2)]{我们常说内能是态函数,这里的“态”是指 \uline{0.6}.}
\item[(3)]{孤立单一成分物质在温度和\uline{0.3}保持不变的情况下,对所有可能的变动,平衡态的自由焓最小。}
\item[(4)]{}
\item[(5)]{}    
\eitem
}
  
\item[(三)]{
  如图,粗细均匀的U型细管(请原谅我画得不是很细…)内装有左右高均为$25\SIcm$的贡柱,左边上端$19\SIcm$长的封闭端充满空气,右边和大气(压强为$76\,\mathrm{cmHg}$)达到平衡。当把右端与真空泵相接,抽空右端空气后,左端的汞柱会下降多少$\SIcm$?\mark{15}
    
    \lfig{1.5}{problem1_9.jpg}
    
    \vspace{3.5in}

}

\item[(四)]{$1\SImol$的温度为$300\SIK$的理想气体在准静态等体过程中吸收了$249.4\SIJ$的热量,温度变为$320\SIK$,然后经过准静态绝热膨胀温度又降回到$300\SIK$。设该气体的定体摩尔热容为常量。求过程中 (1) 气体对外做的功;\mark{10} (2) 末态体积和初始体积之比。\mark{5}

\vspace{3.5in}}
    
  


\item[(五)]{把100块质量为 $10\SIg$,温度为$0\cdeg$的冰块逐块投入初始时质量为$1\SIkg$,温度为$80\cdeg$的水中进行熔化。全部投完后,恰好得到$2\SIkg$温度为$0\cdeg$的水。把大气理想化为绝热的。试估算
  \bitem
\item[(1)]{冰的熔化热;\mark{5}}
\item[(2)]{整个过程的总熵变。\mark{10}}
  \eitem
    }
  

\eitem



\ech
\end{document}

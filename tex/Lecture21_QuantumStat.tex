\documentclass[CJK,14pt]{beamer}
\usepackage{CJKutf8}
\usepackage{beamerthemesplit}
\usetheme{Malmoe}
\useoutertheme[footline=authortitle]{miniframes}
\usepackage{amsmath}
\usepackage{amssymb}
\usepackage{graphicx}
\usepackage{eufrak}
\usepackage{color}
\usepackage{slashed}
\usepackage{simplewick}
\usepackage{tikz}
\graphicspath{{../figures/}}
\def\addfig#1#2{\begin{center}\includegraphics[width=#1 in]{#2}\end{center}}
\def\blacktext#1{{\color{black}#1}}
\def\bluetext#1{{\color{blue}#1}}
\def\redtext#1{{\color{red}#1}}
\def\darkbluetext#1{{\color[rgb]{0,0.2,0.6}#1}}
\def\skybluetext#1{{\color[rgb]{0.2,0.7,1.}#1}}
\def\cyantext#1{{\color[rgb]{0.,0.5,0.5}#1}}
\def\greentext#1{{\color[rgb]{0,0.7,0.1}#1}}
\def\darkgray{\color[rgb]{0.2,0.2,0.2}}
\def\lightgray{\color[rgb]{0.6,0.6,0.6}}
\def\gray{\color[rgb]{0.4,0.4,0.4}}
\def\blue{\color{blue}}
\def\red{\color{red}}
\def\green{\color{green}}
\def\darkblue{\color[rgb]{0,0.2,0.6}}
\def\skyblue{\color[rgb]{0.2,0.7,1.}}
\def\fdeg{{^\circ \mathrm{F}}}
\def\cdeg{^\circ \mathrm{C}}
\def\be{\begin{equation}}
\def\ee{\nonumber\end{equation}}
\def\bea{\begin{eqnarray}}
\def\eea{\nonumber\end{eqnarray}}
\def\ii{{\dot{\imath}}}
\def\bch{\begin{CJK}{UTF8}{gbsn}}
\def\ech{\end{CJK}}
\def\bitem{\begin{itemize}}
\def\eitem{\end{itemize}}
\def\bcenter{\begin{center}}
\def\ecenter{\end{center}}
\def\bex{\begin{minipage}{0.3\textwidth}\includegraphics[width=1in]{jugelizi.png}\end{minipage}\begin{minipage}{0.6\textwidth}}
\def\eex{\end{minipage}}
\def\chtitle#1{\frametitle{\bch#1\ech}}
\def\skipline{{\vskip0.1in}}
\def\skiplines{{\vskip0.2in}}
\def\lagr{{\mathcal{L}}}
\def\hamil{{\mathcal{H}}}
\def\vecv{{\mathbf{v}}}
\def\vecx{{\mathbf{x}}}
\def\vecy{{\mathbf{y}}}
\def\veck{{\mathbf{k}}}
\def\vecp{{\mathbf{p}}}
\def\vecn{{\mathbf{n}}}
\def\vecA{{\mathbf{A}}}
\def\vecP{{\mathbf{P}}}
\def\vecsigma{{\mathbf{\sigma}}}
\def\hatJn{{\hat{J_\vecn}}}
\def\hatJx{{\hat{J_x}}}
\def\hatJy{{\hat{J_y}}}
\def\hatJz{{\hat{J_z}}}
\def\hatj#1{\hat{J_{#1}}}
\def\hatphi{{\hat{\phi}}}
\def\hatq{{\hat{q}}}
\def\hatpi{{\hat{\pi}}}
\def\vel{\upsilon}
\def\Dint{{\mathcal{D}}}
\def\adag{{\hat{a}^\dagger}}
\def\bdag{{\hat{b}^\dagger}}
\def\cdag{{\hat{c}^\dagger}}
\def\ddag{{\hat{d}^\dagger}}
\def\hata{{\hat{a}}}
\def\hatb{{\hat{b}}}
\def\hatc{{\hat{c}}}
\def\hatd{{\hat{d}}}
\def\hatN{{\hat{N}}}
\def\hatH{{\hat{H}}}
\def\hatp{{\hat{p}}}
\def\Fup{{F^{\mu\nu}}}
\def\Fdown{{F_{\mu\nu}}}
\def\newl{\nonumber \\}
\def\SIkm{\,\mathrm{km}}
\def\SIyr{\,\mathrm{yr}}
\def\SIGyr{\,\mathrm{Gyr}}
\def\SIeV{\,\mathrm{eV}}
\def\SIkeV{\,\mathrm{keV}}
\def\SIMeV{\,\mathrm{MeV}}
\def\SIGeV{\,\mathrm{GeV}}
\def\SIcal{\,\mathrm{cal}}
\def\SIkcal{\,\mathrm{kcal}}
\def\SImol{\,\mathrm{mol}}
\def\SIm{\,\mathrm{m}}
\def\SIcm{\,\mathrm{cm}}
\def\SIfm{\,\mathrm{fm}}
\def\SImm{\,\mathrm{mm}}
\def\SInm{\,\mathrm{nm}}
\def\SImum{\,\mathrm{\mu m}}
\def\SIJ{\,\mathrm{J}}
\def\SIkJ{\,\mathrm{kJ}}
\def\SIs{\,\mathrm{s}}
\def\SIkg{\,\mathrm{kg}}
\def\SIg{\,\mathrm{g}}
\def\SIK{\,\mathrm{K}}
\def\SImmHg{\,\mathrm{mmHg}}
\def\SIPa{\,\mathrm{Pa}}
\def\vece{\mathrm{e}}
\def\bmat#1{\left(\begin{array}{#1}}
\def\emat{\end{array}\right)}
\def\bcase#1{\left\{\begin{array}{#1}}
\def\ecase{\end{array}\right.}
\def\calM{{\mathcal{M}}}
\def\calT{{\mathcal{T}}}
\def\calR{{\mathcal{R}}}
\def\barpsi{\bar{\psi}}
\def\baru{\bar{u}}
\def\barv{\bar{\upsilon}}
\def\bmini#1{\begin{minipage}{#1\textwidth}}
\def\emini{\end{minipage}}
\def\qeq{\stackrel{?}{=}}
\def\torder#1{\mathcal{T}\left(#1\right)}
\def\rorder#1{\mathcal{R}\left(#1\right)}
\def\contr#1#2{\contraction{}{#1}{}{#2}#1#2}
\def\trof#1{\mathrm{Tr}\left(#1\right)}
\def\trace{\mathrm{Tr}}
\def\comm#1{\ \ \ \left(\mathrm{used}\ #1\right)}
\def\tcomm#1{\ \ \ (\text{#1})}
\def\slp{\slashed{p}}
\def\slk{\slashed{k}}
\def\wulian{\includegraphics[width=0.18in]{emoji_wulian.jpg}}
\def\bye{\includegraphics[width=0.18in]{emoji_bye.jpg}}
\def\calp{{\mathfrak{p}}}
\def\veccalp{\mathbf{\mathfrak{p}}}
\def\atm{\,\mathrm{atm}}
\def\angstrom{\,\text{\AA}}
\def\Tthree{T_{\tiny \textcircled{3}}}
\def\pthree{p_{\tiny \textcircled{3}}}

\def\courseurl{http://zhiqihuang.top}

\def\tpage#1#2{
\begin{frame}
\bch
\begin{center}
\begin{large}
热学 \\
第#1讲 #2

\end{large}

\skiplines

黄志琦


\end{center}

\skiplines

{\small 
教材:《热学》第二版,赵凯华,罗蔚茵,高等教育出版社


课件下载
}
\courseurl 
\ech
\end{frame}
}

\def\bfr#1{
\begin{frame}
\chtitle{#1} 
\bch
}

\def\efr{
\ech 
\end{frame}
}

\title{Lesson 21 - Quantum Statistics and Radiation}
\author{}
\date{}
\begin{document}
  \bch
\tpage{21}{量子统计和辐射理论}

\begin{frame}
\frametitle{本讲内容}
\bitem
\item{量子场论和二次量子化}
\item{玻色子和费米子}
\item{光子气体}

\eitem
\end{frame}


\section{2nd Quantization}
\secpage{量子场论和二次量子化}{粒子是场的激发态}

\begin{frame}
  \frametitle{粒子是场的激发态}
  量子场论的基本观点:时空中充斥着各种场,各种粒子是各种场的激发态。

  \skiplines
  
  \bex
  
    电子和正电子是狄拉克旋量场的激发态。
    
    光子是电磁场的激发态。

    没有粒子的真空:所有场处于能量最低的基态。    
  \eex
  

\end{frame}


\begin{frame}
\frametitle{二次量子化}

在二次量子化之后,我们把“测量到a态上有两个光子”这句话重新进行了描述:“测量到光子对应的场(即电磁场)处于(2a)状态”。

\skipline

\wulian绕口令说得不错

\end{frame}


\begin{frame}
\frametitle{普通量子力学的描述}

\addfig{1.2}{E_classic.png}

测到a态有3个光子,b态有2个光子,c态有1个光子。

\end{frame}


\begin{frame}
\frametitle{二次量子化描述}

\bmini{0.41}
\addfig{1.5}{E_quantum.png}
\emini
\bmini{0.55}
测到电磁场处在(3a,2b, 1c)态。


\emini


\end{frame}


\begin{frame}
\frametitle{总结}
\bitem
\item{以前讨论的是单个粒子的状态,如果有多个粒子则要分别进行描述。}
\item{二次量子化之后讨论更为基本的“场”处于什么状态,如果有多个同种粒子一起出现,就要把它们“一锅端”地进行描述。}
\eitem
\end{frame}



\begin{frame}
以上我们快速地学习了研究生课《量子场论》。没听懂不要慌,因为——不——考。

\skiplines
\bcenter
\includegraphics[width=1.5in]{notlove.jpg} \hspace{0.25in} \includegraphics[width=1.5in]{notexam.jpg}
\ecenter
\end{frame}


\section{Bosons and Fermions}
\secpage{玻色子和费米子}{玻色子和平相处,费米子一山不容二虎}

\begin{frame}
\frametitle{玻色子和费米子}


根据二次量子化的对易假设不同,可以得到两种性质不同的粒子。

\bitem
\item{可以在一个“粒子态”上放任意多个粒子,称之为{\blue 玻色子(Boson)}。例如光子。}
\item{最多只能在一个“粒子态”上放一个粒子,称之为{\blue 费米子(Fermion)}。例如电子,质子。}
\eitem

{\blue 两个全同费米子不能具有完全相同的微观状态},这称为“{\blue 泡利(Pauli)不相容原理}”。

{\small 注意:除了普通相空间的位置,动量外,粒子往往还具有一些内禀自由度(如自旋)。自旋向上的电子和自旋向下的电子可以具有相同的位置和动量。}

\end{frame}


\begin{frame}
\frametitle{费米场的态}

\bmini{0.5}
例子:三个(由位置,动量,自旋等决定的)微观态上最多放三个电子,一共有八种放法。
\emini
\bmini{0.45}
\lfig{1.5}{E_fermion.png}
\emini
\end{frame}



\begin{frame}
\frametitle{稍作修改的“万能法则”}

我们现在要对“万能法则”稍作修改:场态$(n_a a, n_bb, n_c c,\ldots)$出现的概率{\bf 正比于}
$$e^{-\frac{n_a(\varepsilon_a-\mu)+n_b(\varepsilon_b-\mu)+n_c(\varepsilon_c-\mu)+\ldots}{kT}} = e^{-\frac{\varepsilon_{\rm total} - \mu_{\rm total}}{kT}}$$

这里的$\mu$是化学势。
\end{frame}


\begin{frame}
\frametitle{费米-狄拉克分布(Fermi-Dirac distribution)}

现在考虑某粒子态a,对费米子而言,粒子态a上可以有$n=0$或$n=1$个粒子,出现概率依次正比于$e^{-\frac{n(\varepsilon_a-\mu)}{kT}}$。
平均粒子数为

{\small
\bea
\bar{n}_a &=& \frac{\sum_{n=0}^1 ne^{-\frac{n(\varepsilon_a-\mu)}{kT}}}{\sum_{n=0}^1 e^{-\frac{n(\varepsilon_a-\mu)}{kT}}}  \newl
&=& \frac{ e^{-\frac{\varepsilon_a-\mu}{kT}}}{1+ e^{-\frac{\varepsilon_a-\mu}{kT}}}  \newl
&=& \frac{ 1}{e^{\frac{\varepsilon_a-\mu}{kT}}+1}  
\eea
}

\end{frame}


\begin{frame}
\frametitle{玻色-爱因斯坦分布(Bose-Einstein distribution)}

玻色子的情况要稍复杂,对$n$的求和要从$0$到$\infty$。

{\small
\bea
\bar{n}_a &=& \frac{\sum_{n=0}^{\infty} ne^{-\frac{n(\varepsilon_a-\mu)}{kT}}}{\sum_{n=0}^\infty e^{-\frac{n(\varepsilon_a-\mu)}{kT}}}  
= \left. \frac{\sum_{n=0}^{\infty} ne^{-nx}}{\sum_{n=0}^\infty e^{-nx}}\right\vert_{x = \frac{\varepsilon_a-\mu}{kT}}  \newl
&=& \left. \frac{-\frac{d}{d x}\sum_{n=0}^{\infty} e^{-nx}}{\sum_{n=0}^\infty e^{-nx}}\right\vert_{x = \frac{\varepsilon_a-\mu}{kT}}  
= \left. \frac{-\frac{d}{d x}\frac{1}{1- e^{-x}} }{ \frac{1}{1- e^{-x}} }\right\vert_{x = \frac{\varepsilon_a-\mu}{kT}}  \newl
&=& \frac{ 1}{e^{\frac{\varepsilon_a-\mu}{kT}}-1}  
\eea
}

\end{frame}



\begin{frame}
\frametitle{总结}


费米子FD分布
\tbox{$$\bar{n} = \frac{ 1}{e^{\frac{\varepsilon-\mu}{kT}}+1}$$ }

玻色子BE分布
\tbox{$$\bar{n} = \frac{ 1}{e^{\frac{\varepsilon-\mu}{kT}}-1}$$ }

在$\bar{n}\ll 1$时,两种分布都趋向于经典分布$$\bar{n} = e^{-\frac{\varepsilon-\mu}{kT}} $$

\end{frame}

\section{Radiation}

\begin{frame}

\frametitle{
下面我们讨论一下光子气体
}


\addfig{2}{godsaylight.jpg}

神说:要有光。于是白昼和黑夜分开了…

GF说:要有钱。于是你和GF分开了…

\end{frame}


\begin{frame}
  \frametitle{封闭窖内的热辐射(光子气体)}
  假想有一个封闭的恒温窖,内壁上的原子可以通过吸收和发出光子和窖内的热辐射(即光子气体)达到热平衡。两者温度同为$T$。

  \addfig{3}{radiation_in_cave.jpg}
  
\end{frame}


\begin{frame}
  \frametitle{封闭窖内的热辐射(光子气体)}
  由于光子{\blue 没有静止质量},动量$\rightarrow 0$ 的光子,其能量也$\rightarrow 0$。恒温窖会毫不吝啬地向窖内大量发射这种即没有能量,也不带任何电荷(或其他守恒荷)的光子。导致
  $$ n_0 = \frac{1}{e^{\frac{0-\mu}{kT}}-1} \rightarrow \infty. $$
  即$\mu \rightarrow 0$。

  \skiplines
  
  一般性地,当达到平衡态时,{\blue 静质量为零,且为自身反粒子的的粒子的化学势为零。}
\end{frame}



\begin{frame}
\frametitle{光子气体的粒子数}
记光子动量大小为$\calp$,则其能量为$\calp c$。

利用玻色子在每个微观态的平均粒子数公式,可以算出光子气体的总粒子数为:

$$N = \int_0^\infty \frac{1}{e^{\frac{\calp c}{kT}}-1} \frac{g V 4\pi\calp^2 d\calp}{h^3}, $$
其中$g=2$是光子的内禀态数(光子有两种偏振方式,或者说自旋为$\pm 1$),$V$是体积。因子$\frac{g V 4\pi\calp^2 d\calp}{h^3}$是动量大小在$\calp$和$\calp+d\calp$之间时的微观态个数。
\end{frame}

\begin{frame}
  \frametitle{光子气体的粒子数密度}

稍加整理即得到光子气体的粒子数密度
$$n = \frac{N}{V}= b T^3$$
其中常数
$$ b = \frac{k^3}{\pi^2c^3\hbar^3}\int_0^\infty  \frac{x^2 }{e^x-1} dx = \frac{2k^3}{\pi^2c^3\hbar^3} \zeta(3)  $$
其中$\zeta(3) = \frac{1}{1^3}+ \frac{1}{2^3}+\frac{1}{3^3}+\frac{1}{4^3}+\ldots\approx  1.202$

\end{frame}


\begin{frame}
  \frametitle{光子气体的能量密度}
同样可以写出光子气体总能量为
$$E = \int_0^\infty \frac{\calp c}{e^{\frac{\calp c}{kT}}-1} \frac{g V 4\pi\calp^2 d\calp}{h^3}. $$
能量密度
$$u = \frac{E}{V}= a T^4$$  
这称为斯特藩-玻尔兹曼定律,其中斯特藩-玻尔兹曼常数$a$为
$$a = \frac{k^4}{\pi^2c^3\hbar^3}\int_0^\infty  \frac{x^3 }{e^x-1} dx = \frac{\pi^2k^4}{15c^3\hbar^3}$$
\end{frame}




\begin{frame}
  \frametitle{光子气体的压强}
  在课程开始(Lecture 1, page 29)我们曾经推导出理想气体压强
  $$ p = \frac{1}{3} n \overline{\calp \cdot \vecv} $$
  对光子气体,$\overline{\calp \cdot \vecv} = \overline{pc} = \overline{\varepsilon}$。所以{\blue 光子气体的压强是能量密度的$1/3$.}

  $$p = \frac{1}{3}u = \frac{a}{3} T^4$$

\end{frame}


\begin{frame}
\frametitle{斯特藩-玻尔兹曼定律的另一种表述方式}

如果在热辐射中放一个接收板,则单位时间单位面积接收的能量可以由泻流速率算出:
$$j = \frac{c}{4}aT^4 = \sigma T^4 $$
{\small (各向同性情况下,泻流速率是平均速率的$\frac{1}{4}$。)}

$\sigma = \frac{ac}{4} = 5.67 \times 10^{-8} \SIW/(\SIm^2 \SIK^4)$有时也叫做斯特藩-玻尔兹曼常数\wulian

\end{frame}

\begin{frame}
\frametitle{估算太阳能辐射强度}

太阳表面温度约为$6000\SIK$。我们估算太阳表面辐射强度为
$$j = \sigma T^4 = 10^8 \SIW/\SIm^2 $$
因为中心辐射强度按距离平方反比衰减,从太阳表面(半径约2.3光秒)到地球(日地距离约500光秒),衰减了约5万倍。
地球上接收到的太阳辐射强度约为$2000\SIW/\SIm^2$。

\addfig{1.2}{sunshine.jpg}

\end{frame}


\begin{frame}
  \frametitle{思考题(热力学方法研究光子气体)}

  \addfig{0.5}{think.jpg}  
  用热力学第二定律证明:平衡态光子气体的内能密度只能是温度的函数。
\end{frame}


\begin{frame}
  \frametitle{思考题(热力学方法研究光子气体)}

  \addfig{0.5}{think1.jpg}  

  在前题基础上,并利用光子气体压强是内能密度的$1/3$,以及内能和状态方程的关系,证明
  $$ u = a T^4,$$
  其中$a$为常量(这里我们无法确定$a$是多少)。
\end{frame}


\begin{frame}
  \frametitle{思考题(热力学方法研究光子气体)}

  \addfig{0.5}{think2.jpg}

  在前题基础上,证明光子气体的熵密度

  $$ s = \frac{4}{3} aT^3.$$

\end{frame}


\begin{frame}
  \frametitle{思考题(热力学方法研究光子气体)}

  \addfig{1}{think3.jpg}

  在前题基础上,证明光子气体的化学势为零。

\end{frame}



\begin{frame}
\frametitle{附录1:计算斯特藩-玻尔兹曼常数}

$$a = \frac{k^4}{\pi^2c^3\hbar^3}\int_0^\infty \frac{x^3 }{e^x-1} dx$$
利用级数展开以及反复分部积分的方法,可以把右边的积分化为一个级数的和:
{\small
\bea
\int_0^\infty  \frac{x^3 }{e^x-1} dx &=& \int_0^\infty\frac{x^3e^{-x}}{1-e^{-x}}dx 
= \sum_{n=1}^\infty  \int_0^\infty  x^3 e^{-nx} dx \newl
&=& \sum_{n=1}^\infty \frac{3}{n} \int_0^\infty  x^2 e^{-nx} dx 
= \sum_{n=1}^\infty \frac{6}{n^2} \int_0^\infty  x e^{-nx} dx \newl
&=& \sum_{n=1}^\infty \frac{6}{n^3} \int_0^\infty  e^{-nx} dx 
= \sum_{n=1}^\infty \frac{6}{n^4}
\eea
}
\end{frame}


\begin{frame}
\frametitle{附录1:计算斯特藩-玻尔兹曼常数}

我们知道
$$\sin x = x-\frac{x^3}{6}+\frac{x^5}{120} - \ldots$$
把$\sin x$看成一个无穷阶多项式,它的根为$0, \pm\pi, \pm 2\pi, \ldots$。故乘积表达式可以写成
$$\sin x = x\prod_{n=1}^{\infty} \left(1-\frac{x}{n\pi}\right)\left(1+\frac{x}{n\pi}\right) = x\prod_{n=1}^{\infty} \left(1-\frac{x^2}{(n\pi)^2}\right)$$
\end{frame}

\begin{frame}
  \frametitle{附录1:计算斯特藩-玻尔兹曼常数}
对比$x^3$的系数,得到
\begin{equation}
\sum_{n=1}^\infty \frac{1}{n^2} = \frac{\pi^2}{6} \label{eq:n2}
\end{equation}
再对比$x^5$的系数,得到
\begin{equation}
\sum_{n>m\ge 1}\frac{1}{n^2}\frac{1}{m^2} = \frac{\pi^4}{120} \label{eq:n4}
\end{equation}

\end{frame}


\begin{frame}
\frametitle{附录1:计算斯特藩-玻尔兹曼常数}

\eqref{eq:n2}平方再减去\eqref{eq:n4}的两倍,得到:
$$\sum_{n=1}^\infty \frac{1}{n^4} = \frac{\pi^4}{90} $$
最后我们得到
$$a = \frac{\pi^2 k^4}{15c^3\hbar^3} = 7.566 \times 10^{-16} \SIJ/(\SIm^3\SIK^4) $$
\end{frame}

\ech
\end{document}

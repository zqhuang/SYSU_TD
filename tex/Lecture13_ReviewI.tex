\documentclass[CJK]{beamer}
\usepackage{CJKutf8}
\usepackage{beamerthemesplit}
\usetheme{Malmoe}
\useoutertheme[footline=authortitle]{miniframes}
\usepackage{amsmath}
\usepackage{amssymb}
\usepackage{graphicx}
\usepackage{eufrak}
\usepackage{color}
\usepackage{slashed}
\usepackage{simplewick}
\usepackage{tikz}
\graphicspath{{../figures/}}
\def\addfig#1#2{\begin{center}\includegraphics[width=#1 in]{#2}\end{center}}
\def\blacktext#1{{\color{black}#1}}
\def\bluetext#1{{\color{blue}#1}}
\def\redtext#1{{\color{red}#1}}
\def\darkbluetext#1{{\color[rgb]{0,0.2,0.6}#1}}
\def\skybluetext#1{{\color[rgb]{0.2,0.7,1.}#1}}
\def\cyantext#1{{\color[rgb]{0.,0.5,0.5}#1}}
\def\greentext#1{{\color[rgb]{0,0.7,0.1}#1}}
\def\darkgray{\color[rgb]{0.2,0.2,0.2}}
\def\lightgray{\color[rgb]{0.6,0.6,0.6}}
\def\gray{\color[rgb]{0.4,0.4,0.4}}
\def\blue{\color{blue}}
\def\red{\color{red}}
\def\green{\color{green}}
\def\darkblue{\color[rgb]{0,0.2,0.6}}
\def\skyblue{\color[rgb]{0.2,0.7,1.}}
\def\fdeg{{^\circ \mathrm{F}}}
\def\cdeg{^\circ \mathrm{C}}
\def\be{\begin{equation}}
\def\ee{\nonumber\end{equation}}
\def\bea{\begin{eqnarray}}
\def\eea{\nonumber\end{eqnarray}}
\def\ii{{\dot{\imath}}}
\def\bch{\begin{CJK}{UTF8}{gbsn}}
\def\ech{\end{CJK}}
\def\bitem{\begin{itemize}}
\def\eitem{\end{itemize}}
\def\bcenter{\begin{center}}
\def\ecenter{\end{center}}
\def\bex{\begin{minipage}{0.3\textwidth}\includegraphics[width=1in]{jugelizi.png}\end{minipage}\begin{minipage}{0.6\textwidth}}
\def\eex{\end{minipage}}
\def\chtitle#1{\frametitle{\bch#1\ech}}
\def\skipline{{\vskip0.1in}}
\def\skiplines{{\vskip0.2in}}
\def\lagr{{\mathcal{L}}}
\def\hamil{{\mathcal{H}}}
\def\vecv{{\mathbf{v}}}
\def\vecx{{\mathbf{x}}}
\def\vecy{{\mathbf{y}}}
\def\veck{{\mathbf{k}}}
\def\vecp{{\mathbf{p}}}
\def\vecn{{\mathbf{n}}}
\def\vecA{{\mathbf{A}}}
\def\vecP{{\mathbf{P}}}
\def\vecsigma{{\mathbf{\sigma}}}
\def\hatJn{{\hat{J_\vecn}}}
\def\hatJx{{\hat{J_x}}}
\def\hatJy{{\hat{J_y}}}
\def\hatJz{{\hat{J_z}}}
\def\hatj#1{\hat{J_{#1}}}
\def\hatphi{{\hat{\phi}}}
\def\hatq{{\hat{q}}}
\def\hatpi{{\hat{\pi}}}
\def\vel{\upsilon}
\def\Dint{{\mathcal{D}}}
\def\adag{{\hat{a}^\dagger}}
\def\bdag{{\hat{b}^\dagger}}
\def\cdag{{\hat{c}^\dagger}}
\def\ddag{{\hat{d}^\dagger}}
\def\hata{{\hat{a}}}
\def\hatb{{\hat{b}}}
\def\hatc{{\hat{c}}}
\def\hatd{{\hat{d}}}
\def\hatN{{\hat{N}}}
\def\hatH{{\hat{H}}}
\def\hatp{{\hat{p}}}
\def\Fup{{F^{\mu\nu}}}
\def\Fdown{{F_{\mu\nu}}}
\def\newl{\nonumber \\}
\def\SIkm{\,\mathrm{km}}
\def\SIyr{\,\mathrm{yr}}
\def\SIGyr{\,\mathrm{Gyr}}
\def\SIeV{\,\mathrm{eV}}
\def\SIkeV{\,\mathrm{keV}}
\def\SIMeV{\,\mathrm{MeV}}
\def\SIGeV{\,\mathrm{GeV}}
\def\SIcal{\,\mathrm{cal}}
\def\SIkcal{\,\mathrm{kcal}}
\def\SImol{\,\mathrm{mol}}
\def\SIm{\,\mathrm{m}}
\def\SIcm{\,\mathrm{cm}}
\def\SIfm{\,\mathrm{fm}}
\def\SImm{\,\mathrm{mm}}
\def\SInm{\,\mathrm{nm}}
\def\SImum{\,\mathrm{\mu m}}
\def\SIJ{\,\mathrm{J}}
\def\SIkJ{\,\mathrm{kJ}}
\def\SIs{\,\mathrm{s}}
\def\SIkg{\,\mathrm{kg}}
\def\SIg{\,\mathrm{g}}
\def\SIK{\,\mathrm{K}}
\def\SImmHg{\,\mathrm{mmHg}}
\def\SIPa{\,\mathrm{Pa}}
\def\vece{\mathrm{e}}
\def\bmat#1{\left(\begin{array}{#1}}
\def\emat{\end{array}\right)}
\def\bcase#1{\left\{\begin{array}{#1}}
\def\ecase{\end{array}\right.}
\def\calM{{\mathcal{M}}}
\def\calT{{\mathcal{T}}}
\def\calR{{\mathcal{R}}}
\def\barpsi{\bar{\psi}}
\def\baru{\bar{u}}
\def\barv{\bar{\upsilon}}
\def\bmini#1{\begin{minipage}{#1\textwidth}}
\def\emini{\end{minipage}}
\def\qeq{\stackrel{?}{=}}
\def\torder#1{\mathcal{T}\left(#1\right)}
\def\rorder#1{\mathcal{R}\left(#1\right)}
\def\contr#1#2{\contraction{}{#1}{}{#2}#1#2}
\def\trof#1{\mathrm{Tr}\left(#1\right)}
\def\trace{\mathrm{Tr}}
\def\comm#1{\ \ \ \left(\mathrm{used}\ #1\right)}
\def\tcomm#1{\ \ \ (\text{#1})}
\def\slp{\slashed{p}}
\def\slk{\slashed{k}}
\def\wulian{\includegraphics[width=0.18in]{emoji_wulian.jpg}}
\def\bye{\includegraphics[width=0.18in]{emoji_bye.jpg}}
\def\calp{{\mathfrak{p}}}
\def\veccalp{\mathbf{\mathfrak{p}}}
\def\atm{\,\mathrm{atm}}
\def\angstrom{\,\text{\AA}}
\def\Tthree{T_{\tiny \textcircled{3}}}
\def\pthree{p_{\tiny \textcircled{3}}}

\def\courseurl{http://zhiqihuang.top}

\def\tpage#1#2{
\begin{frame}
\bch
\begin{center}
\begin{large}
热学 \\
第#1讲 #2

\end{large}

\skiplines

黄志琦


\end{center}

\skiplines

{\small 
教材:《热学》第二版,赵凯华,罗蔚茵,高等教育出版社


课件下载
}
\courseurl 
\ech
\end{frame}
}

\def\bfr#1{
\begin{frame}
\chtitle{#1} 
\bch
}

\def\efr{
\ech 
\end{frame}
}

\title{Lesson 13 Course Review I}
  \author{}
  \date{}
\begin{document}
\tpage{13}{课程复习I}

\begin{frame}
  \chtitle{复习纲要}
  \bch
  \bitem
\item{第一篇:理想气体状态方程}
\item{第二篇:麦克斯韦分布}
\item{第三篇:$pVT$系统的态函数}
\item{第四篇:热力学过程和循环(期末考重点)}
\item{第五篇:克拉珀龙方程的应用}
\item{第六篇:熵的计算}
\item{第七篇:热力学第二定律的应用}
  \eitem
  \ech
\end{frame}

\newcounter{chap}
\newcounter{problem}[chap]
\def\proid{{Problem \thechap.\theproblem}}

\section{Ideal Gas}
\setcounter{chap}{1}
\setcounter{problem}{0}


\begin{frame}
  \chtitle{第一篇 理想气体状态方程}
  \bch
  \tbox{$$ pV = \nu RT$$}
  可以写成
  $$p = n k T$$
  或者
  $$ pV = N k T$$
  \ech
\end{frame}

\stepcounter{problem}
\begin{frame}
  \chtitle{\proid (\stwo)}
  \bch
一个体积为$0.02494\SIm^3$的恒温容器里的$1\SImol$气体压强为$101200 \SIPa$。抽去$\frac{1}{3}\SImol$气体后,容器内气体达到热平衡时压强变为$67200\SIPa$。再抽去$\frac{1}{3}\SImol$气体,容器内气体达到热平衡时压强变为$33467\SIPa$。试计算恒温容器的温度。 
  \ech
\end{frame}

\begin{frame}
  \chtitle{\proid 解答}
  \bch
  如果是理想气体,$\frac{p}{\nu} = \frac{RT}{V}$应该为常数。题目中讨论的是对理想气体有偏离的实际气体。
  
  我们把题目所给的$\frac{p}{\nu}$对摩尔数的依赖列出来:
  $$ \nu = 1\SImol,\ \frac{p}{\nu } = 1.012\times 10^5 \SIPa/\SImol$$
  $$ \nu = \frac{2}{3}\SImol,\ \frac{p}{\nu } = 1.008\times 10^5 \SIPa/\SImol $$
  $$ \nu = \frac{1}{3}\SImol,\ \frac{p}{\nu } = 1.004\times 10^5 \SIPa/\SImol$$
  因为气体越稀薄越接近理想气体,作线性外推得到
  $ \nu \rightarrow 0$时,  $\frac{p}{\nu} = 1.000\times 10^5\SIPa/\SImol$。
  所以
  $$ T = \frac{p}{\nu} \frac{V}{R} = 10^5\times \frac{0.02494}{8.314} \SIK = 300\SIK$$
  
  \ech
\end{frame}

\stepcounter{problem}
\begin{frame}
  \chtitle{\proid (\stwo)}
  \bch
  在标准状态下,某气球里充有$\nu=1\SImol$气体。气球近似可看作半径$r=0.15\SIm$的均匀球面。问它的张力系数(单位长度受力)为多少。
  \ech
\end{frame}

\begin{frame}
  \chtitle{\proid 解答}
  \bch
  {\small
  把气球划分为上下半球,上半球受气体净推力为
  $$F=(p -p_0)(\pi r^2)$$
  其中$p$为气球内气压,$p_0=1\atm$为外界大气压。

  上半球受到的气体推力必须和下半球拉它的张力平衡
  $$ F = \sigma (2\pi r)$$
  其中$\sigma$为所求的张力系数

  结合上面两式得到熟知的:
  $$ p-p_0 = \frac{2\sigma}{r} $$
  再由理想气体状态方程
  $$\sigma = \frac{r}{2}(p-p_0) = \frac{r}{2}\left(\frac{\nu R T_0}{\frac{4\pi}{3}r^3} - p_0\right) = 4.45\times 10^3 \SIN/\SIm $$

  }
  \ech
\end{frame}


\section{Maxwell Distribution}

\setcounter{chap}{2}
\setcounter{problem}{0}

\begin{frame}
  \chtitle{第二篇 麦克斯韦分布}
  \bch
  {\small
  只需要记住{\bf 无量纲速度的每个分量服从标准正态分布}:
  $$\tilde{f}_{\rm 1D}(u_x) = \frac{1}{\sqrt{2\pi}}e^{-\frac{u_x^2}{2}}$$
  可秒推麦克斯韦分布的所有公式:
  
  三维无量纲速度分布
  $$ \tilde{f}(u_x, u_y, u_z) = \tilde{f}_{\rm 1D}(u_x)\tilde{f}_{\rm 1D}(u_y)\tilde{f}_{\rm 1D}(u_z) =\left(\frac{1}{2\pi}\right)^{3/2}e^{-u^2/2}$$
  
  速率分布
  $$ \tilde{F}(u) = 4\pi u^2 \tilde{f}(u_x, u_y, u_z) = \sqrt{\frac{2}{\pi}}u^2e^{-u^2/2}$$

  要换回普通单位制,只需要分析量纲,配以$\upsilon_c$的若干次幂。}
  \ech
\end{frame}


\begin{frame}
  \chtitle{第二篇 麦克斯韦分布(续)}
  \bch
  {\small
  服从标准正态分布的变量$x$的绝对值$n$次平均可以写成两种积分式:
  $$\overline{|x|^n}=\frac{1}{\sqrt{2\pi}}\int_{-\infty}^\infty |x|^n e^{-x^2/2}dx = \sqrt{\frac{2}{\pi}}\int_0^\infty x^n e^{-x^2/2}dx $$

  我们往往是把某个积分式转化为$\overline{|x|^n}$,然后用下面的方法计算:
  $$\overline{|x|^0} = 1,$$
  $$\overline{|x|} = \sqrt{\frac{2}{\pi}},$$
  再利用递推式
  $$\overline{|x|^n} = (n-1) \overline{|x|^{n-2}}\ \  (n\ge 2),$$
  可以对任意正整数$n$迅速求出$\overline{|x|^n}$。
  }
  \ech
\end{frame}


\begin{frame}
  \chtitle{第二篇 麦克斯韦分布(续)}
  \bch
      {\small
        或者记住高斯积分公式
        $$\int_{-\infty}^{\infty} e^{-ax^2}dx = \sqrt{\frac{\pi}{a}}$$
        用积分号下求导或者分部积分的方式可以求$\int x^n e^{-ax^2}$类的积分。
  }
  \ech
\end{frame}

\begin{frame}
  \chtitle{第二篇 麦克斯韦分布(续)}
  \bch
      {\small
        当然,最重要的是对速度分布函数和速率分布函数有清晰的图像理解。这是写出积分式的前提,然后才是计算积分的问题。
  }
  \ech
\end{frame}

\stepcounter{problem}
\begin{frame}
  \chtitle{\proid (\stwo)}
  \bch
  一个装有某常见双原子气体(分子式为$X_2$)的圆柱形恒温容器绕中心轴以每秒$30$转的速度旋转。已知容器的半径为$0.4\SIm$,温度为$300\SIK$。等容器内气体达到热平衡后,测得容器中心轴附近的气体压强为容器壁附近的气体压强的$96.4\%$。问$X$是什么元素?
  \ech
\end{frame}

\begin{frame}
  \chtitle{\proid 解答}
  \bch
  根据$p = nkT$知道压强正比于分子数密度,而分子数密度根据玻尔兹曼分布为
  $$ n(r) = n_0e^{\frac{m\omega^2 r^2}{2kT}}$$
  其中$\omega = 60\pi \SIs^{-1}, r= 0.4\SIm$。题目所给条件即
  $$e^{-\frac{m\omega^2 r^2}{2kT}}=0.964$$
  由此得气体摩尔质量
  $$ M^{\rm mol} = N_Am = -\ln 0.964\times \frac{2RT}{\omega^2r^2} = 32\SIg/\SImol$$
  所以$X$的原子量为16,是氧元素。
  \ech
\end{frame}


\stepcounter{problem}
\begin{frame}
  \chtitle{\proid (\sthree)}
  \bch
  假设有某种二维气体的速度分布函数为
  $$f(\upsilon_x,\upsilon_y)\propto e^{-\left[\left(\frac{m\upsilon^2}{2kT}\right)^2\right]}$$
  其中$m$为分子质量,$T$为温度。

  求该气体分子的平均动能。
  \ech
\end{frame}

\begin{frame}
  \chtitle{\proid 解答}
  \bch
      {\small
        无量纲速率的分布函数为(注意速率的分布函数要乘以$2\pi u$):
        $$F(u) \propto u e^{-u^4/4}$$
$$        \overline{u^2} = \frac{\int_0^\infty u^3e^{-u^4/4}du}{\int_0^\infty ue^{-u^4/4}du} =      \frac{ -e^{-u^4/4}|_0^\infty }{\int_0^\infty e^{-x^2}dx} =\frac{1}{\frac{1}{2}\sqrt{\pi}} = \frac{2}{\sqrt{\pi}} $$
        换算回普通单位制即
        $$\overline{\upsilon^2} = \frac{2}{\sqrt{\pi}} \frac{kT}{m}$$
        即平均动能
        $$\frac{1}{2}m \overline{\upsilon^2} = \frac{kT}{\sqrt{\pi}}$$
  }
  \ech
\end{frame}


\stepcounter{problem}
\begin{frame}
  \chtitle{\proid (\sfour)}
  \bch
  置于很大的真空室内的绝热容器里装有稀薄氦气。在容器壁上开一个小孔,经过一段时间后把小孔堵上,发现容器内氦气压强降低了$0.4\%$,问容器内氦气的分子数减少了百分之多少?漏气的过程很缓慢,可以近似认为整个过程中容器内氦气一直处于热平衡。
  \ech
\end{frame}

\begin{frame}
  \chtitle{\proid 解答}
  \bch
        稀薄氦气可以看成单原子理想气体。
        由内能$U = \frac{3}{2}\nu RT = \frac{3}{2}pV$。压强降低$0.4\%$即内能降低$0.4\%$。对单原子理想气体,泻能速率为$\frac{1}{3}\overline{\upsilon}$,泻流速率为$\frac{1}{4}\overline{\upsilon}$,故分子数减少百分比是内能减少百分比的$3/4$,即为$0.3\%$。        
  \ech
\end{frame}

\stepcounter{problem}
\begin{frame}
  \chtitle{\proid (\sfour)}
  \bch
  假设内径为$5\SIm$的封闭球形飞船绕中子星做每秒一周的匀速圆周运动。飞船的质心在球心,且自转和公转同步(即保持同一面对着中子星)。飞船内有温度为$285\SIK$的氧气。问:氧气的压强是均匀的吗?如果不均匀,最小压强和最大压强之比为多少?
  \ech
\end{frame}

\begin{frame}
  \chtitle{\proid 解答}
  \bch
  {\small
  取旋转参考系。氧气分子的离心力势能为$-\frac{1}{2}mr^2\omega^2$,重力势能为$-GMm/r$。两者之和为
  $$\varepsilon(r) = -\frac{1}{2}mr^2\omega^2 -\frac{GMm}{r}$$
  其中$M$为中子星质量,$r$为氧气分子距离中子星的距离,$\omega = 2\pi \SIs^{-1}$。
  
  当然,我们考察的是飞船内部不同位置的势能差,我们取质心位置$r_0$为基准把上述势能进行泰勒展开近似。注意到势能的一阶导数即质心处离心力与引力之和为零(即$mr_0\omega^2 = \frac{GMm}{r_0^2}$)。所以考虑二阶项
  $$\varepsilon(r_0+\delta r) = \varepsilon(r_0) + \frac{1}{2}\left(-m\omega^2 -\frac{2GMm}{r_0^3}\right)\delta r^2 =  \varepsilon(r_0) - \frac{3}{2}m\omega^2\delta r^2 $$
  由此得出飞船边缘($\delta r = 5\SIm$)处分子数密度和中心分子数密度差别最大,两者之比为
  $$e^{\frac{3m \omega^2\delta r^2}{2kT}}= 1.02 $$
    }
  \ech
\end{frame}


\section{PDEs}
\setcounter{chap}{3}
\setcounter{problem}{0}

\begin{frame}
  \chtitle{第三篇:令人头疼的偏导数}
  \bch
  对{\bf 仅有两个独立变量}的态函数们,有普通链式法则
  $$\pfrac XYS \pfrac YZS = \pfrac XZS$$

  三变量新链式法则
  $$\pfrac XYZ \pfrac YZX = -\pfrac XZY$$

  和四变量新链式法则
  $$ \pfrac WZX \pfrac ZXY = \pfrac WXY - \pfrac WXZ  $$
  
  \ech
\end{frame}


\begin{frame}
  \chtitle{第三篇:全微分成立条件}
  \bch
  $ f dX + g dY$为全微分的充要条件为
  $$\pfrac fYX = \pfrac gXY $$
  
  \ech
\end{frame}


\begin{frame}
  \chtitle{第三篇:令人头疼的偏导数}
  \bch
  {\small
  内能和状态方程的关系:
    $$\pfrac UVT = \pfrac p{\ln T}V - p $$
  焓和状态方程的关系
  $$\pfrac HpT = V - \pfrac V{\ln T}p  $$
  上面两式分别可推导出准静态过程中吸热量的两种计算方式:
    $$ \dbar Q = C_VdT + \pfrac p{\ln T}V dV $$
  $$ \dbar Q = C_pdT - \pfrac V{\ln T}p dp $$  
}
  \ech
\end{frame}


\begin{frame}
  \chtitle{第三篇:令人头疼的偏导数(续)}
  \bch
  当然,最重要的是对偏导数有清晰的图像理解。这样才能打开思路,不拘泥于公式。
  \ech
\end{frame}


\stepcounter{problem}
\begin{frame}
  \chtitle{\proid (\sone)}
  \bch
  证明焦耳汤姆孙系数$$ \alpha$$可以写成
  $$ \alpha = -\frac{\pfrac HpT}{C_p} $$
\ech
\end{frame}

\begin{frame}
  \chtitle{\proid 解答}
  \bch
  焓的物理意义为计入了等压过程做功的“总能量”,所以根据热力学第一定律,等压过程中
  $$C_p dT = \dbar Q = dH$$
  即$$C_p = \pfrac HTp$$
  因此
  $$ \frac{\pfrac HpT}{C_p} = \pfrac HpT \pfrac THp = -\pfrac TpH  = -\alpha$$
  上述推导中我们用了三变量的新链式法则。
  
\ech
\end{frame}


\stepcounter{problem}
\begin{frame}
  \chtitle{\proid (\stwo)}
  \bch
  对$pVT$系统证明下述Maxwell关系:
\bitem
\item{$\pfrac TVS = -\pfrac pSV$}
\item{$\pfrac TpS = \pfrac VSp$}
\item{$\pfrac SVT = \pfrac pTV$}
\item{$\pfrac SpT = -\pfrac VTp$}
\eitem
\ech
\end{frame}

\begin{frame}
  \chtitle{\proid 解答}
  \bch
\bitem
\item{由$dU = TdS - pdV$为全微分以及全微分成立条件得到$\pfrac TVS = -\pfrac pSV$。}
\item{由$dH = TdS + Vdp$为全微分以及全微分成立条件得到$\pfrac TpS = \pfrac VSp$。}
\item{由$dF = -SdT - pdV$为全微分以及全微分成立条件得到$\pfrac SVT = \pfrac pTV$。}
\item{由$dG = -SdT + Vdp$为全微分以及全微分成立条件得到$\pfrac SpT = -\pfrac VTp$。}
\eitem
  \ech
\end{frame}

\stepcounter{problem}
\begin{frame}
  \chtitle{\proid (\sthree)}
  \bch
某气体的物态方程为
$$p(V-\nu b) = \nu R T$$
其中$b$为常量。

证明该气体的定压热容和定体热容之差为
$$ C_p - C_V = \nu R $$

  \ech
\end{frame}

\begin{frame}
  \chtitle{\proid 解法1}
  \bch
由$C_p  =\pfrac HTp = T \pfrac STp$, $C_V = \pfrac UTp = T\pfrac STV$
得到
$$\frac{C_p - C_V}{T} = \pfrac STp - \pfrac STV = \pfrac SVT  \pfrac VTp $$
最后一步我们用了四变量的偏微分链式法则。然后根据$dF = - SdT - pdV$为全微分得到$\pfrac SVT = \pfrac pTV$,代入上式
$$ C_p - C_V = T\pfrac pTV \pfrac VTp = T \frac{\nu R}{V-\nu b} \frac{\nu R}{p} = \nu R$$

  \ech
\end{frame}


\begin{frame}
  \chtitle{\proid 解法2}
  \bch
由内能和状态方程的关系得到(准静态过程吸热量的第一种表达方式)
{\scriptsize 
$$ T dS  = dU + pdV = C_V dT + \left(T\pfrac pTV -p\right)dV+pdV = C_V dT + T\pfrac pTV dV  $$}
由焓和状态方程的关系得到(准静态过程吸热量的第二种表达方式)
{\scriptsize 
$$ TdS = dH - Vdp = C_p dT + (V-T\pfrac VTp)dp - Vdp = C_p dT - T\pfrac VTp dp$$}
两式相减得到
$$ (C_p- C_V)dT =  T\pfrac VTp dp + T\pfrac pTV dV = (V-\nu b)dp + p dV $$
又根据状态方程有$ \nu R dT = (V-\nu b) dp + p dV$,与上式比较即得证$C_p - C_V = \nu R$。
  \ech
\end{frame}


\stepcounter{problem}
\begin{frame}
  \chtitle{\proid (\sthree)}
  \bch
某固定体积的$pVT$系统,在一定温度范围内($200\SIK < T< 400 \SIK$)内能和温度有如下关系:
  $$U =  aT^2 $$
  其中$a=1 \SIJ/\SIK^2$。
  初始时该系统温度$250\SIK$,自由能$F = 100 \SIkJ$。
  当系统升温到$350\SIK$,求系统自由能$F$。
  \ech
\end{frame}

\begin{frame}
  \chtitle{\proid 解法1}
  \bch
  初始状态时,
  $$ S_{\rm ini} = \frac{U_{\rm ini}-F_{\rm ini}}{T_{\rm ini}} = \frac{1\times 250^2 - 10^5}{250} \SIJ/\SIK = -150\SIJ/\SIK$$
  在体积不变的情况下
  $$ dU = T\pfrac STV dT  =  2aTdT$$
  故
  $$ \pfrac STV = 2a $$
  积分得到
  $$ S_{\rm final}  = S_{\rm ini} + 2a\Delta T =  \left[-150 + 2\times 1\times (350-250) \right] \SIJ/\SIK = 50 \SIJ/\SIK $$
  $$ F_{\rm final} = U_{\rm final}-T_{\rm final}S_{\rm final} = \left[1\times 350^2 - 350\times 50\right] \SIJ = 105\SIkJ$$
  
  \ech
\end{frame}

\begin{frame}
  \chtitle{\proid 解法2}
  \bch
  $$\pfrac {(F/T)}TV = \frac{1}{T}\pfrac FTV - \frac{F}{T^2} = -\frac{F+TS}{T^2} = -\frac{U}{T^2} = -a $$
  积分得到
  $$\frac{F}{350\SIK} = \frac{10^5\SIJ}{250\SIK} - 1\SIJ/\SIK^2\times(350\SIK - 250\SIK) = 300 \SIJ/\SIK$$
  即
  $$F = 105 \SIkJ$$

  \ech
\end{frame}


\stepcounter{problem}
\begin{frame}
  \chtitle{\proid (\stwo)}
  \bch
  $pVT$系统中,证明
  $$ \pfrac TpV \pfrac SVp - \pfrac TVp \pfrac SpV = 1 $$
  \ech
\end{frame}

\begin{frame}
  \chtitle{\proid 解答}
  \bch
  等式左边是$(T,S)\rightarrow (p, V)$的Jaccobi行列式,代表$(T,S)$和$(p,V)$图上的面积元之比。在研究准静态循环时,我们根据热力学第一定律已经知道了$(T,S)$和$(p,V)$图上的对应的任何闭合曲线包围的面积相等(这包括面积的符号,即闭合曲线的绕行方向)。面积元只是很小的闭合曲线围绕的面积而已,所以得证。

  \skipline

  补充$T$-$S$图和$p$-$V$图上闭合曲线包围面积相等的证明:
$$ \oint TdS - \oint pdV = \oint dU = 0$$
  
  \ech
\end{frame}


\stepcounter{problem}
\begin{frame}
  \chtitle{\proid  (\sthree)}
  \bch
  $pVT$系统中,设$x, y$为任意两个独立变量,证明:
      {\small $$ \pfrac Txy \pfrac Syx - \pfrac Tyx \pfrac Sxy = \pfrac pxy \pfrac Vyx - \pfrac pyx \pfrac Vxy $$}  
  \ech
\end{frame}


\begin{frame}
  \chtitle{\proid 解答}
  \bch
等式左边是$(T,S)\rightarrow (x, y)$的Jaccobi行列式,代表$(T,S)$和$(x,y)$图上的面积元之比。等式右边是$(p,V)\rightarrow (x, y)$的Jaccobi行列式,代表$(p,V)$和$(x,y)$图上的面积元之比。由于$(T,S)$和$(p,V)$图上面积元相等,故得证。
  \ech
\end{frame}

\stepcounter{problem}
\begin{frame}
  \chtitle{习题\proid  (\sthree)}
  \bch
  $pVT$系统中,设$x, y$为任意两个独立变量,证明:
      {\scriptsize $$ \pfrac {\left(\frac{1}{T}\right)}xy \pfrac Uyx - \pfrac {\left(\frac{1}{T}\right)}yx \pfrac Uxy = \pfrac Vxy \pfrac {\left(\frac{p}{T}\right)}yx - \pfrac Vyx \pfrac {\left(\frac{p}{T}\right)}xy $$}  
  \ech
\end{frame}

\begin{frame}
  \chtitle{习题\proid 解答}
  \bch
  同样只要证明$\frac{1}{T}$-$U$图上的闭合曲线包围的面积和$V$-$\frac{p}{T}$图上对应的闭合曲线包围的面积相等。
  注意应用环路积分计算$x$-$y$图上闭合曲线包围的面积时既可以用$\oint ydx$,也可以用$\oint -x dy$。对$V$-$\frac{p}{T}$图我们采取后者。
  
  $$\oint \frac{dU}{T} - \oint \frac{-pdV}{T} = \oint \frac{\dbar Q}{T} = \oint dS = 0$$
  
  \ech
\end{frame}

\stepcounter{problem}
\begin{frame}
  \chtitle{习题\proid  (\sthree)}
  \bch
  对$pVT$系统证明
  $$\pfrac UpV = -T\pfrac VTS$$
  \ech
\end{frame}


\begin{frame}
  \chtitle{习题\proid 解答}
  \bch
  证明:固定熵时
  $$ \dbar Q = C_V dT + T \pfrac pTV dV = 0 $$
  即
  $$ \pfrac VTS = -\frac{C_V}{T\pfrac pTV} = -\frac{\pfrac UTV}{T\pfrac pTV} = -\frac{1}{T} \pfrac UpV $$
  两边乘以$-T$即得证。
  \ech
\end{frame}

\stepcounter{problem}
\begin{frame}
  \chtitle{习题\proid  (\sthree)}
  \bch
  对$pVT$系统证明
  $$\pfrac TVU = p\pfrac TUV - T\pfrac pUV $$
  \ech
\end{frame}


\begin{frame}
  \chtitle{习题\proid解答}
  \bch
  证明:固定内能时
  $$ dU = C_V dT + \left(T\pfrac pTV - p\right)dV = 0$$
  即
  {\scriptsize
  $$ \pfrac TVU = \frac{p-T\pfrac pTV }{C_V} = p\pfrac TUV - T\frac{\pfrac pTV}{\pfrac UTV}  = p\pfrac TUV - T\pfrac pUV $$}
  \ech
\end{frame}

\stepcounter{problem}
\begin{frame}
  \chtitle{习题\proid   (\sfour)}
  \bch
  对$pVT$系统证明
  $$\pfrac TSH +\frac{T^2}{V} \pfrac VHp = \frac{T}{C_p} $$
  \ech
\end{frame}

\begin{frame}
  \chtitle{习题\proid 解答}
  \bch
  {\scriptsize
  证明:由$C_p = T\pfrac STp$得到
  $$ \frac{T}{C_p} - \pfrac TSH = \pfrac TSp - \pfrac TSH = \pfrac THS \pfrac HSp $$
  最后一步我们使用了四变量的链式法则。由$dH = TdS + Vdp$即知$\pfrac HSp = T$,所以
  \begin{equation}
    \frac{T}{C_p} - \pfrac TSH = T \pfrac THS = \frac{T}{V} \pfrac TpS \label{eq1}
  \end{equation}
  上面最后一步我们利用了$S$不变时,$ dH = Vdp$。
  最后,利用$dH  = TdS +Vdp$为全微分,得到
  \begin{equation}
    \pfrac TpS = \pfrac VSp = T\pfrac VHp \label{eq2}
  \end{equation}
  其中最后一步利用了$p$不变时,$dH = TdS$。

  把\eqref{eq2}带入 \eqref{eq1}即证毕。
  
  }
  \ech
\end{frame}

\stepcounter{problem}
\begin{frame}
  \chtitle{\proid (\sfour)}
  \bch
  某物质在一定范围内的物态方程为
  $$ pV = \nu R T\left(1+\ln\frac{T}{T_0}\right) $$
  $1\SImol$的该物质经过等温膨胀之后,体积增大一倍,问定体热容增加了多少?
  \ech
\end{frame}

\begin{frame}
  \chtitle{\proid 解答}
  \bch
  {\small
    本题考察的是$C_V$在固定温度时对体积的依赖。先利用内能和状态方程的关系得到
    $$\pfrac UVT = T\pfrac pTV - p = \frac{\nu RT}{V} $$
    
  $$\pfrac {C_V}VT = \frac{\partial^2U}{\partial T\partial V} = \frac{\partial^2U}{\partial V\partial T} = \pfrac {\left(\frac{\nu RT}{V}\right)}{T}V =\frac{\nu R}{V} $$

    积分得到

    $$ C_V(T, V) = C_V(T, V_0) + \nu R \ln\frac{V}{V_0}$$
    因此在等温过程$V$增大一倍时,$C_V$增大$\nu R \ln 2 = 5.76 \SIJ/\SIK$。

  }
  \ech
\end{frame}

\section{Homework}

\begin{frame}
  \chtitle{第13周作业(序号接第12周)}
  \bch
  {\small
  \bitem
\item[33]{ 考虑温度为$T$,分子质量为$m$的热平衡理想气体。
  \bitem
\item[(1)]{写出分子速率大于$5\upsilon_{\rm rms}$的概率的积分表达式,其中$\upsilon_{\rm rms}$为方均根速率。}
\item[(2)] {写出所有速率超过$5\upsilon_{\rm rms}$的分子的平均速率的积分表达式。}
\item[(3)] {由于分子的速率分布是指数衰减的,所以大多数速率大于$5\upsilon_{\rm rms}$的分子的速率都会集中在$5\upsilon_{\rm rms}$附近。我们近似认为速率大于$5\upsilon_{\rm rms}$的分子的平均速率为$5\upsilon_{\rm rms}$。由此估算分子速率大于$5\upsilon_{\rm rms}$的概率。}
          \eitem
}
\item[34]{对$pVT$系统证明  $$\pfrac HVp = T\pfrac pTS $$}
\item[35]{对$pVT$系统证明焦耳-汤姆孙系数可以写成$$\alpha = T \pfrac VHp - V\pfrac THp$$}
  \eitem
}
  
  \ech
\end{frame}

\end{document}

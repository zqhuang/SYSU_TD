\documentclass[CJK]{beamer}
\usepackage{CJKutf8}
\usepackage{beamerthemesplit}
\usetheme{Malmoe}
\useoutertheme[footline=authortitle]{miniframes}
\usepackage{amsmath}
\usepackage{amssymb}
\usepackage{graphicx}
\usepackage{eufrak}
\usepackage{color}
\usepackage{slashed}
\usepackage{simplewick}
\usepackage{tikz}
\graphicspath{{../figures/}}
\def\addfig#1#2{\begin{center}\includegraphics[width=#1 in]{#2}\end{center}}
\def\blacktext#1{{\color{black}#1}}
\def\bluetext#1{{\color{blue}#1}}
\def\redtext#1{{\color{red}#1}}
\def\darkbluetext#1{{\color[rgb]{0,0.2,0.6}#1}}
\def\skybluetext#1{{\color[rgb]{0.2,0.7,1.}#1}}
\def\cyantext#1{{\color[rgb]{0.,0.5,0.5}#1}}
\def\greentext#1{{\color[rgb]{0,0.7,0.1}#1}}
\def\darkgray{\color[rgb]{0.2,0.2,0.2}}
\def\lightgray{\color[rgb]{0.6,0.6,0.6}}
\def\gray{\color[rgb]{0.4,0.4,0.4}}
\def\blue{\color{blue}}
\def\red{\color{red}}
\def\green{\color{green}}
\def\darkblue{\color[rgb]{0,0.2,0.6}}
\def\skyblue{\color[rgb]{0.2,0.7,1.}}
\def\fdeg{{^\circ \mathrm{F}}}
\def\cdeg{^\circ \mathrm{C}}
\def\be{\begin{equation}}
\def\ee{\nonumber\end{equation}}
\def\bea{\begin{eqnarray}}
\def\eea{\nonumber\end{eqnarray}}
\def\ii{{\dot{\imath}}}
\def\bch{\begin{CJK}{UTF8}{gbsn}}
\def\ech{\end{CJK}}
\def\bitem{\begin{itemize}}
\def\eitem{\end{itemize}}
\def\bcenter{\begin{center}}
\def\ecenter{\end{center}}
\def\bex{\begin{minipage}{0.3\textwidth}\includegraphics[width=1in]{jugelizi.png}\end{minipage}\begin{minipage}{0.6\textwidth}}
\def\eex{\end{minipage}}
\def\chtitle#1{\frametitle{\bch#1\ech}}
\def\skipline{{\vskip0.1in}}
\def\skiplines{{\vskip0.2in}}
\def\lagr{{\mathcal{L}}}
\def\hamil{{\mathcal{H}}}
\def\vecv{{\mathbf{v}}}
\def\vecx{{\mathbf{x}}}
\def\vecy{{\mathbf{y}}}
\def\veck{{\mathbf{k}}}
\def\vecp{{\mathbf{p}}}
\def\vecn{{\mathbf{n}}}
\def\vecA{{\mathbf{A}}}
\def\vecP{{\mathbf{P}}}
\def\vecsigma{{\mathbf{\sigma}}}
\def\hatJn{{\hat{J_\vecn}}}
\def\hatJx{{\hat{J_x}}}
\def\hatJy{{\hat{J_y}}}
\def\hatJz{{\hat{J_z}}}
\def\hatj#1{\hat{J_{#1}}}
\def\hatphi{{\hat{\phi}}}
\def\hatq{{\hat{q}}}
\def\hatpi{{\hat{\pi}}}
\def\vel{\upsilon}
\def\Dint{{\mathcal{D}}}
\def\adag{{\hat{a}^\dagger}}
\def\bdag{{\hat{b}^\dagger}}
\def\cdag{{\hat{c}^\dagger}}
\def\ddag{{\hat{d}^\dagger}}
\def\hata{{\hat{a}}}
\def\hatb{{\hat{b}}}
\def\hatc{{\hat{c}}}
\def\hatd{{\hat{d}}}
\def\hatN{{\hat{N}}}
\def\hatH{{\hat{H}}}
\def\hatp{{\hat{p}}}
\def\Fup{{F^{\mu\nu}}}
\def\Fdown{{F_{\mu\nu}}}
\def\newl{\nonumber \\}
\def\SIkm{\,\mathrm{km}}
\def\SIyr{\,\mathrm{yr}}
\def\SIGyr{\,\mathrm{Gyr}}
\def\SIeV{\,\mathrm{eV}}
\def\SIkeV{\,\mathrm{keV}}
\def\SIMeV{\,\mathrm{MeV}}
\def\SIGeV{\,\mathrm{GeV}}
\def\SIcal{\,\mathrm{cal}}
\def\SIkcal{\,\mathrm{kcal}}
\def\SImol{\,\mathrm{mol}}
\def\SIm{\,\mathrm{m}}
\def\SIcm{\,\mathrm{cm}}
\def\SIfm{\,\mathrm{fm}}
\def\SImm{\,\mathrm{mm}}
\def\SInm{\,\mathrm{nm}}
\def\SImum{\,\mathrm{\mu m}}
\def\SIJ{\,\mathrm{J}}
\def\SIkJ{\,\mathrm{kJ}}
\def\SIs{\,\mathrm{s}}
\def\SIkg{\,\mathrm{kg}}
\def\SIg{\,\mathrm{g}}
\def\SIK{\,\mathrm{K}}
\def\SImmHg{\,\mathrm{mmHg}}
\def\SIPa{\,\mathrm{Pa}}
\def\vece{\mathrm{e}}
\def\bmat#1{\left(\begin{array}{#1}}
\def\emat{\end{array}\right)}
\def\bcase#1{\left\{\begin{array}{#1}}
\def\ecase{\end{array}\right.}
\def\calM{{\mathcal{M}}}
\def\calT{{\mathcal{T}}}
\def\calR{{\mathcal{R}}}
\def\barpsi{\bar{\psi}}
\def\baru{\bar{u}}
\def\barv{\bar{\upsilon}}
\def\bmini#1{\begin{minipage}{#1\textwidth}}
\def\emini{\end{minipage}}
\def\qeq{\stackrel{?}{=}}
\def\torder#1{\mathcal{T}\left(#1\right)}
\def\rorder#1{\mathcal{R}\left(#1\right)}
\def\contr#1#2{\contraction{}{#1}{}{#2}#1#2}
\def\trof#1{\mathrm{Tr}\left(#1\right)}
\def\trace{\mathrm{Tr}}
\def\comm#1{\ \ \ \left(\mathrm{used}\ #1\right)}
\def\tcomm#1{\ \ \ (\text{#1})}
\def\slp{\slashed{p}}
\def\slk{\slashed{k}}
\def\wulian{\includegraphics[width=0.18in]{emoji_wulian.jpg}}
\def\bye{\includegraphics[width=0.18in]{emoji_bye.jpg}}
\def\calp{{\mathfrak{p}}}
\def\veccalp{\mathbf{\mathfrak{p}}}
\def\atm{\,\mathrm{atm}}
\def\angstrom{\,\text{\AA}}
\def\Tthree{T_{\tiny \textcircled{3}}}
\def\pthree{p_{\tiny \textcircled{3}}}

\def\courseurl{http://zhiqihuang.top}

\def\tpage#1#2{
\begin{frame}
\bch
\begin{center}
\begin{large}
热学 \\
第#1讲 #2

\end{large}

\skiplines

黄志琦


\end{center}

\skiplines

{\small 
教材:《热学》第二版,赵凯华,罗蔚茵,高等教育出版社


课件下载
}
\courseurl 
\ech
\end{frame}
}

\def\bfr#1{
\begin{frame}
\chtitle{#1} 
\bch
}

\def\efr{
\ech 
\end{frame}
}

\title{Lesson 10 Enthalpy}
  \author{}
  \date{}
\begin{document}
\tpage{10}{焓和定压热容}

\section{Review}

\begin{frame}
\chtitle{上讲内容回顾}
\bch 
\bitem
\item{态函数的大家庭:温度$T$,熵$S$,压强$p$,体积$V$,内能$U$,焓$H=U+pV$,自由能$F=U-TS$和自由焓$G=H-TS$}
\item{内能的物理内涵:
 $$\pfrac UVT = \pfrac p{\ln T}V - p $$
其中$\pfrac p{\ln T}V$称为热压强}
\item{准静态过程吸热量为按照定体比热容计算的吸热量加上热压强做功消耗的能量:
$$ \dbar Q = C_V dT + \pfrac p{\ln T}V dV$$}
\eitem
\ech
\end{frame}

\begin{frame}
\chtitle{百前的困惑}
\bch
在第七讲介绍内能时,我们用脚趾头就想出来了实际气体$\pfrac UVT > 0$(或者等价地$\pfrac UpT < 0$)

而在一两百年之前,物理学家们还在焦头烂额地做实验测量实际气体的$\pfrac UVT $:
\bmini{0.35}
盖吕萨克实验: 打开活塞,空气自由膨胀后测量温度变化。
\emini
\bmini{0.6}
\addfig{1.8}{GLexperiment.jpg}
\emini

\ech
\end{frame}


\begin{frame}
\chtitle{送分题}
\bch
\addfig{0.9}{songfen.jpg}
请用脚趾头思考:盖吕萨克实验中气体自由绝热膨胀(假设瓶子是绝热的)后温度应该上升还是下降?
\ech
\end{frame}


\begin{frame}
\chtitle{送分题}
\bch
\addfig{0.9}{songfen2.jpg}
实际气体$\pfrac UVT>0$(因增大体积使得分子间势能增大)。绝热自由膨胀既无吸热也无做功,故内能不变。根据循环偏微分乘积定理有
$$\pfrac TVU = -\pfrac TUV \pfrac UVT = -\frac{1}{C_V} \pfrac UVT < 0$$
即得出自由绝热膨胀后温度应该降低。

\skipline

但是,绝热自由膨胀不是准静态过程,上面这样推导是否有bug?

\ech
\end{frame}


\begin{frame}
\chtitle{盖吕萨克实验的结果}
\bch

\addfig{1.5}{cry2.jpg}

装置太简陋(漏),没测出温度变化 
\ech
\end{frame}

\begin{frame}
\chtitle{焦耳实验}
\bch
焦耳意识到瓶子可能漏热,就……

\bmini{0.35}
焦耳实验: 打开活塞,空气自由膨胀后测量水温的变化。
\emini
\bmini{0.6}
\addfig{2}{Joule_experiment.jpg}
\emini

\ech
\end{frame}


\begin{frame}
\chtitle{焦耳实验的结果}
\bch

\addfig{1.5}{cry1.png}

温度计不够准,没测出温度变化 
\ech
\end{frame}


\begin{frame}
\chtitle{后来}
\bch
1932年,Rossini和Frandsen改进了焦耳实验,终于测出了
$$\pfrac UpT < 0 $$
(教材138-139页)

因为测出来了很没意思,细节不讲了\bye
\ech
\end{frame}


\begin{frame}
\chtitle{脚趾头想不出来的问题}
\bch
实际气体固定温度增大压强(减小体积)时,因分子间的吸引势能变低,故内能变少。

\addfig{0.5}{think2.jpg}

那么,当固定温度增大压强(减小体积)时,焓$H\equiv U + pV$会如何变化呢?

\ech
\end{frame}


\begin{frame}
\chtitle{本讲内容}
\bch
\bitem
\item{焓和定压热容}
\item{焦耳-汤姆逊效应}
\item{节流膨胀液化气体}
\item{化学反应和生成焓}
\item{关于内能和焓的一些练习}
\eitem
\ech
\end{frame}

\section{Enthalpy}

\begin{frame}
\chtitle{本讲仍然考虑$pVT$系统,挖掘一下焓的物理内涵}
\bch

\addfig{3.2}{wajueji_H.png}

\ech
\end{frame}

\begin{frame}
\chtitle{焓的物理意义}
\bch

内能 $\Rightarrow$ 房子的价值

\skipline

焓 $\Rightarrow$ 房子的价值 + 房子所占空间的价值


\addfig{2}{houses.jpg}

(空间价值有时候可以占主导)


\ech
\end{frame}

\begin{frame}
\chtitle{焓的物理意义}
\bch


很多过程都是等压过程(例如开放环境下的化学反应,自然界物态变化等)。

\lfig{1.1}{chemical_reaction.jpg}\hspace{0.1in}\lfig{1.4}{ice_melt.jpg}

当体积发生变化时就需要对环境做功以“获得生存空间”。这部分额外做的功做为“无形资产”加到内能上,就是焓$H$。

$$ H  \equiv U + pV$$ 

\ech
\end{frame}



\begin{frame}
\chtitle{定压热容}
\bch
\addfig{0.8}{songfen.jpg}

证明定压热容和焓之间的关系:
$$ C_p = \pfrac HTp $$
并解释其物理意义。
\ech
\end{frame}

\begin{frame}
\chtitle{定压热容}
\bch
\addfig{0.8}{songfen2.jpg}

证明定压热容和熵之间的关系:
$$ C_p = T\pfrac STp $$

\ech
\end{frame}


\begin{frame}
\chtitle{定体摩尔热容和定压摩尔热容的普适关系}
\bch

\addfig{1.}{think3.jpg}

证明定压摩尔热容$C_p$和定体摩尔热容$C_V$之间有如下关系
{\blue
$$\frac{C_p-C_V}{T} =  \pfrac VTp \pfrac pTV$$
}

\ech
\end{frame}




\begin{frame}
\chtitle{焓和状态方程的关系}
\bch
因为$d G = - SdT + Vdp$是全微分,所以
{\blue
$$ \pfrac SpT = - \pfrac VTp$$
}

固定温度,变化压强时,
$$ dH = TdS + Vdp = \left( T\pfrac SpT + V\right) dp = \left( V - T\pfrac VTp\right) dp$$
即

{\blue
$$\pfrac HpT = V - \pfrac V{\ln T}p$$  
}

下面我们用范德瓦尔斯模型来研究实际气体$\pfrac HpT$的符号。

\ech
\end{frame}



\begin{frame}
\chtitle{实际气体的$\pfrac HpT$}
\bch
{\small

对范德瓦尔斯气体,由状态方程
$$\left(p + \frac{\nu^2 a}{V^2}\right) \left(V - \nu b\right) = \nu RT$$
以及焓和状态方程的关系,得到
$$\pfrac HpT = V \left(\delta_V - \frac{2\delta_p}{1+2\delta_p}\right) \frac{1+2\delta_p}{1-2\delta_p\left(1-\delta_V\right)}$$
其中$\delta_p \equiv \frac{\frac{\nu^2a}{V^2}}{p+\frac{\nu^2 a}{V^2}}$为压强修正的大小,$\delta_V\equiv \frac{\nu b}{V}$为体积修正的大小。

\bitem
\item{小修正情形:在$\delta_p \ll 1$, $\delta_V \ll 1$的情况下,很容易推算出:当$T>\frac{2a}{bR}$时,$\pfrac HpT>0$;反之则$\pfrac HpT<0$}
\item{大修正情形:在极低温(接近液化温度)以及高压的情形,$\delta_V\rightarrow 1$,显然$\pfrac HpT>0$。}
\eitem

}
\ech
\end{frame}



\begin{frame}
\chtitle{实际气体的$\pfrac HpT$(续)}
\bch
由前面的讨论,我们定性地知道$\pfrac HpT>0$的区域近似为满足$T<\frac{2a}{bR}$的高温区域,以及体积修正比较大的低温高压区域。

下图是对某个范德瓦尔斯气体的数值计算结果:
\addfig{2}{dHdp.pdf}

\ech
\end{frame}

\section{Throttling Process}


\begin{frame}
\chtitle{等焓过程}
\bch
\addfig{0.8}{songfen.jpg}
证明等焓过程中温度随压强的变化为
{\blue
$$\pfrac TpH = - \frac{\pfrac HpT}{C_p}$$
}
\ech
\end{frame}



\begin{frame}
\chtitle{节流过程}
\bch

\bmini{0.48}
\addfig{2}{throttling_process.png}
\emini
\bmini{0.48}
用多孔塞把绝热箱隔开,两端分别施以恒定压强$p_1$, $p_2$ ($p_1>p_2$)。
\emini

{\small

假设一开始气体都在左边,体积为$V_1$,内能为$U_1$。经过一段时间,左边的气体都通过多孔塞被压到了右边,体积和内能变为$V_2$, $U_2$。

那么由于左边活塞对气体做功$p_1V_1$,右边活塞对气体做功$-p_2V_2$,所以气体内能变化为

$$U_2 - U_1  = p_1V_1 - p_2V_2$$

即
$$ U_1 + p_1V_1 = U_2 + p_2V_2 $$
{\blue \bf 节流过程是等焓过程} (注意:节流过程一般不是准静态过程)

}
\ech
\end{frame}



\begin{frame}
\chtitle{焦耳-汤姆孙系数}
\bch
我们用焦耳-汤姆孙系数$\alpha \equiv \pfrac TpH$来计算的温度的升降。如果$\alpha>0$,则称为正节流效应,温度随压强降低而降低(注意节流过程总是压强降低的过程)。反之若$\alpha<0$,则为负节流效应,温度随压强降低而升高。

\ech
\end{frame}

\begin{frame}
\chtitle{思考题}
\bch
\addfig{1}{think3.jpg}

既然节流过程不是准静态过程,为什么能用$\pfrac TpH$来计算的温度的升降呢?
\ech
\end{frame}

\begin{frame}
\chtitle{转换曲线图}
\bch

\bmini{0.4}
{
\small

根据我们刚刚证明的
$$\pfrac TpH = - \frac{\pfrac HpT}{C_p}$$
知道$\pfrac TpH$和$\pfrac HpT$符号相反,实际气体如右图所示。

\skipline

正负节流的交界线称为转换曲线。

}
\emini
\bmini{0.55}
\lfig{2}{throttling_curve.png}
\emini
\ech
\end{frame}


\begin{frame}
\chtitle{焦耳-汤姆孙效应}
\bch

在常温常压下,气体的体积修正和压强修正总是很小。气体的正节流条件$\pfrac TpH>0$等价于$T<\frac{2a}{Rb} \approx \frac{27}{4} T_K$ (我们曾推导了范德瓦尔斯气体的临界温度$T_K = \frac{8a}{27Rb}$)。

\skipline

在教材52页查得,氢气和氦气的临界温度较低,$\frac{27}{4} T_K$在室温之下,而其他大部分气体的临界温度较高,$\frac{27}{4} T_K$在室温之上。所以在{\bf 常温常压下,氢气和氦气有负节流(升温)效应。而其他大部分气体(空气,氮气,氧气等)有正节流(降温)效应}。这些效应统称{\bf 焦耳-汤姆孙效应}。
\ech
\end{frame}

\begin{frame}
\chtitle{思考题}
\bch
\addfig{1.}{think3.jpg}

节流转变温度$\frac{2a}{Rb}$可以看成压强修正和体积修正的比值,为什么小分子气体的节流转变温度反而更低呢?

\ech
\end{frame}

\begin{frame}
\chtitle{节流液化气体的原理}
\bch
{\bf (绝热膨胀或其他方法)预冷到节流转变温度之下(进入正节流区域)$\rightarrow$反复节流降温$\rightarrow$液化}

\skiplines

请参考教材P143图3-15 

\skiplines

目前定型的液化机:柯林斯机(原理:绝热膨胀法+节流)
\ech
\end{frame}

\section{Chemical Reaction}

\begin{frame}
\chtitle{化学反应焓}
\bch
很多化学反应是在大气中(定压)进行的,所以吸(放)热量等于焓的增(减)量。“反应热”其实应该叫做“反应焓”。

\skiplines

化学反应焓的计算{\bf \Huge 不考}。
\ech
\end{frame}


\begin{frame}
\chtitle{第十周作业(序号接第九周)}
\bch
\bitem
\item[24]{ 教材习题3-13}
\item[25]{ 对$pVT$系统,利用焓和状态方程的关系证明准静态过程中的吸热量为
$$ \dbar Q = C_p dT - \pfrac V{\ln T}p dp$$}
\item[26]{ 对$pVT$系统,证明
$$C_p \pfrac TpS = T \pfrac VTp $$ }
\eitem
\ech
\end{frame}

\end{document}

\documentclass[12pt,CJK]{article}
\usepackage{geometry}
\input{reduced_macros.tex}
\geometry{tmargin=0.3in, bmargin=0.5in, lmargin=0.7in, rmargin=0.7in, nohead,nofoot}
\def\mark#1{{\color{blue} (#1分)}}
\renewcommand{\thepage}{}
\begin{document}
\bch
{\large 快期末考试了,小测一下吧 (你不是梅西,你慌也没用)}

{\vskip 0.1in}

姓名 ....................... {\hskip 0.5in}    学号 .......................{\hskip 0.5in}  分数 ...................

\ \\
一、选择题,每题3分,共30分。

{\vskip 0.05in}

\bitem


\item[1.]{平衡态的理想气体绝热自由膨胀后重新达到平衡态,下列哪个态函数变小了?\bropt
  
  \ooptlist{温度}{内能}{焓}{自由能}

}

\item[2.]{对二自由度的$pVT$系统,分别用$p,V,T,S,G$表示压强,体积,温度,熵和自由焓。那么$\pfrac GTp = $ \bropt

  \ooptlist{$S$}{$-S$}{$V$}{$-V$}
}
  

  
\item[3.]{平衡态的{\bf 实际气体}绝热自由膨胀后重新达到平衡,温度和压强一般如何变化? \bropt
  
  \ooptlist{温度变低,压强变小 }{温度不变,压强变小}{温度升高,压强变小}{都不变}

}

  
\item[4.]{用某恒温热源加热气体使之膨胀做功,为什么没有违反热力学第二定律的开尔文表述(不可能从单一热源吸热,使之完全转变成有用的功并不产生其他影响)? \bropt
  
  \ooptlist{气体和热源温度不同,不满足“单一热源”}{气体膨胀做功不能算“有用的功” \\}{气体状态发生了变化,算是产生了“其他影响”}{热源熵发生了变化,算是产生了“其他影响” }

}
  
  
\item[5.]{
  在大气中进行的化学反应过程中的吸热量等于生成物和反应物的 \bropt

  \ooptlist{内能差}{焓差}{自由能差}{自由焓差}
}
  

\item[6.]{
  处于热平衡的氢气中随机抽取一个分子,其速率不小于方均根速率的10倍的概率和下列哪个数的数量级最接近?\bropt

  \ooptlist{$1/10$}{$e^{-10}$}{$e^{-50}$}{$e^{-150}$}
}


\item[7.]{当氮气温度和压强为下列那组数值时,氮气的节流效应是致冷的? \bropt
  
  \ooptlist{$T=3000\SIK, p=100\atm$ }{$T=5000\SIK, p = 1\atm $ \\ }{$T = 300\SIK, p = 0.5\atm $}{$T=10\SIK, p = 100\atm$}

}
    
\item[8.]{一个均匀金属球的热容为$200\SIJ/\SIK$。在初始时刻,上半球面和温度为$280\SIK$的恒温热库接触,下半球面和温度为$320\SIK$的恒温热库接触,金属球内温度分布处于稳恒状态。突然把两个热库都撤去,孤立的金属球经过一段时间后达到了热平衡。估算这个过程中金属球的熵变$\Delta S$和下列哪个数量级最接近? \bropt

  \ooptlist{$0.1\SIJ/\SIK$}{$10\SIJ/\SIK$}{$10^3\SIJ/\SIK$}{$10^5\SIJ/\SIK$}
}


\item[9.]{某气体的状态方程为$\left(p+\frac{\nu^2 a}{V^2T}\right)V = \nu R T$,其中$\nu$为摩尔数,$p, V, T$分别为压强,体积和热力学温度,$a>0$为常量。则该气体的定体摩尔热容$C_V^{\rm mol}$ \bropt
  
  \ooptlist{一定是常量}{不可能是常量}{可能是常量也可能依赖于$T$}{可能是常量也可能依赖于$V$}
}


\item[10.]{把某平衡态实际气体的摩尔数,压强,体积,温度,焓,定体热容,定压热容分别记作$\nu, p, V, T, H,C_V, C_p$。下列哪个选项{\bf 有可能}和其他三个选项不等价 \bropt
  
  \ooptlist{$\pfrac TpH$}{$-\frac{\pfrac HpT}{C_p}$}{$\frac{T\pfrac VTp  -V}{C_V+\nu R}$}{$T \pfrac VHp - V\pfrac THp$}
}


    

\eitem


{\vskip 0.5in}

\ \\
二、填空题,每题7分,共70分。

{\vskip 0.05in}


\bitem
\item[1.]{把标准状态下的$1\SImol$氮气和$2\SImol$氧气混合,成为标准状态的$3\SImol$混合气体,该过程中气体的总熵变化了\uline{0.5}.}
  
\item[2.]{ 在一个大气压下,把$100$块质量为$10\SIg$,温度为$0\cdeg$的小冰块逐块投入质量为$1\SIkg$,初始温度为$80\cdeg$的水中进行熔化。最后一块冰块熔化完时,恰好得到质量为$2\SIkg$,温度为$0\cdeg$的水。把大气当成绝热的,忽略冰和水的体积变化,则这个过程中冰和水(最后变成全是水)的总熵变化了\uline{0.5}.}
  
\item[3.]{ 处于热平衡的单一成分理想气体中任取一个分子,其速率超过方均根速率的$2$倍,但速度的$x$方向分量的大小不超过方均根速率(即$\upsilon>2\upsilon_{\rm rms}, |\upsilon_x|<\upsilon_{\rm rms}$)的概率为 \uline{0.5}.}

\item[4.]{某种气体的状态方程为$\left(p+\frac{a\nu^2}{V^3}\right)V = \nu R T$,其中$\nu$为摩尔数,$p$为压强,$V$为体积,$T$为热力学温度,$a>0$为固定常量。 $1\SImol$该气体经过等温过程体积增大了$10\%$,它的熵变化了\uline{0.5}.}  
  

\item[5.]{置于很大的真空室内的绝热容器里装有稀薄氦气。在容器壁上开一个小孔,经过一段时间缓慢漏气后把小孔堵上,发现容器内氦气压强降低了$0.4\%$,则容器内的分子数减少了\uline{0.5}\%. }

\item[6.]{实验室里有很多相同的高压气体样本,初始体积都是$0.5\SIL$,定压热容为$20\SIJ/\SIK$。取两份样本,分别用准静态绝热膨胀法和绝热节流法进行降温,都使气体压强降低了$1\%$。结果发现准静态绝热膨胀法的降温效果要稍好一点,比绝热节流法多降低了$0.3K$。由此可知,这些气体样本初始时的压强大约是\uline{0.3}$\atm$.}  
  

\item[7.]{某可逆理想气体热机按下述循环工作:
      \bitem
    \item{以多方指数$n=\frac{5}{3}$的准静态多方过程从$423\SIK$升温到$777\SIK$;}
    \item{准静态绝热膨胀,降温到$200\SIK$;}
    \item{以$200\SIK$的温度准静态等温压缩;}
    \item{准静态绝热压缩,回到初始状态。}      
      \eitem
      在循环过程的温度范围内,该理想气体的定体摩尔热容随温度变化规律为
      $$C_V^{\rm mol} = \left(\frac{3}{2} + \frac{T}{T_0}\right)R,$$
      其中$T_0$为某固定常量。 该热机的效率是\uline{1}. }

\item[8.]{某干燥无风地区的地面大气的压强为$1\atm$,温度为$300\SIK$。在该地区的一处山顶,水的沸点为$90\cdeg$。已知水的汽化热为$4.1\times 10^4\SIJ/\SImol$,估算山的高度为\uline{0.5}$\mathrm{km}$。 }  


\item[9.]{在一个标准大气压下,某种气体从$T=250\SIK$准静态等压升温至$T=350\SIK$。在整个过程中该气体的焓和温度平方成正比。已知该气体在初始状态和末状态的化学势分别为$\mu_1$和$\mu_2$,则在过程中间$T=300\SIK$时,该气体的化学势为\uline{1} (用$\mu_1,\mu_2$的线性组合表示). } 

\item[10.]{某气体的焦耳-汤姆孙系数
  $$\alpha = \pfrac TpH =-\frac{a}{T^2},$$
  其中$a>0$为常量。
  
  当$p\rightarrow 0$时,该气体的定压比热容趋向于常量: $\lim_{p\rightarrow 0}C_p = c.$

  当温度趋向于零时,该气体的熵趋向于零: $ \lim_{T\rightarrow 0} S = 0.$
  
  在标准状态下的该气体,经过准静态绝热压缩,压强变为$2\atm$,这时气体的温度为\uline{0.5}. 
}
  \eitem  

\ech
\end{document}

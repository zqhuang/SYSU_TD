\documentclass[12pt,CJK]{article}
\usepackage{geometry}
\input{reduced_macros.tex}
\geometry{tmargin=0.3in, bmargin=0.5in, lmargin=0.8in, rmargin=0.8in, nohead, nofoot}
\def\mark#1{{\color{blue} (#1分)}}
\renewcommand{\thepage}{}
\begin{document}
\bch


{\blue 纸上得来终觉浅,每逢考试就不行} {\hskip 1.in} 姓名\uline{1} {\hskip 0.5in} 分数\uline{1}

{\vskip 0.3in}
\ \\
(一) 假设某气体的分子速率分布函数为
$$F(\upsilon) = \frac{c\upsilon_0^n}{\upsilon_0^m + \upsilon^4},$$
其中$\upsilon_0 = 100\SIm/\SIs$;$c, m,n$是无量纲常数。
进行量纲分析可以得出$m = $\uline{0.5},$n=$\uline{0.5},从而计算出$c=$\uline{0.5},以及平均速率$\overline{\upsilon} = $\uline{0.5}.  \mark{600}

{\vskip 0.1in}
\ \\
(二) $0.112\SIkg$的氮气初始温度为$400\SIK$,经历准静态绝热膨胀,体积增大为初始状态的$32$倍。估算这个过程中气体对外做功多少。\mark{200}

{\vskip 2.5in}
\ \\
(三) 对$pVT$系统证明 
  $$\pfrac TSH +\frac{T^2}{V} \pfrac VHp = \frac{T}{C_p}, $$
  \mark{800}



\ech
\end{document}

\documentclass[12pt,CJK]{article}
\usepackage{geometry}
\input{reduced_macros.tex}
\geometry{tmargin=0.3in, bmargin=0.5in, lmargin=0.8in, rmargin=0.8in, nohead, nofoot}
\def\mark#1{{\color{blue} (#1分)}}
\renewcommand{\thepage}{}
\begin{document}
\bch
{\large 热学课堂练习 I (总分$\times 0.02$换算成平时分);}

%{\vskip 0.3in}

姓名 ....................... {\hskip 0.5in}    学号 .......................{\hskip 0.5in}  分数 ...................

%{\vskip 0.1in}


\item[(一)]{选择题,每题5分,共30分。

  \bitem
  \item[(1)]{
  下列对准静态过程描述最准确的是 \bropt

  \optlist{缓慢的过程}{每一时刻都近似可以看成平衡态的缓慢过程}{熵不变的过程}

}

\item[(2)]{
  下列哪一个表述是错误的? \bropt
  
  \optlist{第二类永动机不可能}{不可能从低温热源传热给高温热源}{孤立系统的熵不会减少}

}


\item[(3)]{
  固定摩尔数的理想气体的内能 \bropt

  \optlist{只是温度的函数}{只是体积的函数}{同时依赖于体积和温度}
  
}

\item[(4)]{
  工作于温度为$27\cdeg$的高温热源和温度为$227\cdeg$的低温热源之间的卡诺热机的效率{\bf 不可能是} \bropt

  \optlist{0.7}{0.5}{0.3}
}
  \eitem
  }
\item[(三)]{设氮气可以看成理想气体,定体摩尔热容为$C_V^{\rm mol}=\frac{5}{2}R$。温度为300$\SIK$,压强为$1\atm$的$1\SImol$氮气经过准静态绝热压缩,压强变为$1.14\atm$。末态温度为多少?(10分) 这个过程中外界对氮气做了多少功?(10分)

    \vspace{3.5in}


  }
\item[(四)]{某液态物质在$1\atm$附近的的蒸气压方程为: $$ \ln\frac{p}{1\atm} = 4 -\left(\frac{1000\SIK}{T}\right)^2.$$试计算该物质在$1\atm$下的沸点(5分)和摩尔汽华热(10分)。


    \vspace{2.3in}
  }

  \item[(五)]{某可逆理想气体热机的循环过程如下:
      \bitem
    \item{以$600\SIK$的温度准静态等温膨胀,体积增大$37.3\%$。}
    \item{准静态绝热膨胀,温度降到$400\SIK$。}
    \item{以$400\SIK$的温度准静态等温膨胀,体积增大$37.3\%$。}      
    \item{准静态绝热膨胀,温度降到$200\SIK$。}
    \item{以$200\SIK$的温度等温压缩。}
    \item{准静态绝热压缩回到初始状态。}            
      \eitem
      选择你喜欢的变量为坐标画出循环的大致示意图(5分),并求该热机的效率(10分)。
    }
\eitem


\ech
\end{document}

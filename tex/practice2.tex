\documentclass[12pt,CJK]{article}
\usepackage{geometry}
\input{reduced_macros.tex}
\geometry{tmargin=0.3in, bmargin=0.5in, lmargin=0.8in, rmargin=0.8in, nohead, nofoot}
\def\mark#1{{\color{blue} (#1分)}}
\renewcommand{\thepage}{}
\begin{document}
\bch


{\blue 纸上得来终觉浅,一到考试就不行} {\hskip 1.in} 姓名\uline{1} {\hskip 0.5in} 分数\uline{1}

{\vskip 0.3in}
\ \\
(一) 假设某容器内的单一成分的(非平衡态)气体的分子速度分布函数为
$$f(\upsilon_x, \upsilon_y,\upsilon_z) = \frac{A}{B + \upsilon^n},$$
其中$A, B>0$为常量,$n$为固定的正整数;$\upsilon=\sqrt{\upsilon_x^2+\upsilon_y^2+\upsilon_z^2}$是速率。
那么,最小可能的$n$值为\uline{0.5}. \mark{500}

{\vskip 0.1in}
\ \\
(二) $2\SImol$的氦气初始温度为$400\SIK$,经历准静态绝热膨胀,体积增大为初始状态的$32$倍。估算这个过程中气体对外做功多少。\mark{200}

{\vskip 2.5in}
\ \\
(三) 对$pVT$系统证明不等式
  $$\pfrac TpH < T \pfrac VHp $$
  \mark{900}



\ech
\end{document}

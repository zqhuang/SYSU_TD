\documentclass[12pt,CJK]{article}
\usepackage{geometry}
\input{reduced_macros.tex}
\geometry{tmargin=0.3in, bmargin=0.5in, lmargin=0.7in, rmargin=0.7in, nohead, nofoot}
\def\mark#1{{\color{blue} (#1分)}}
\renewcommand{\thepage}{}
\begin{document}
\bch
{\large 热学课堂练习 II 诸神黄昏版 (总分$\times 0.05$换算成平时分);}

%{\vskip 0.3in}

姓名 ....................... {\hskip 0.5in}    学号 .......................{\hskip 0.5in}  分数 ...................

%{\vskip 0.1in}
\bitem

\item[(一)]{选择题,每题10分。

  \bitem

\item[(1)]{设实际气体的定体热容为$C_V$,定压热容为$C_p$。下列哪个表达式和焦耳汤姆孙系数$\pfrac TpH${\bf 不一定}等价 \bropt
  
  \optlist{$-\frac{\pfrac HpT}{C_p}$}{$T \pfrac VHp - V\pfrac THp$}{$\frac{T\pfrac VTp  -V}{C_V+\nu R}$}
}
  
\item[(2)]{处于热平衡的单一成分理想气体中任取一个分子,其速率超过方均根速率的$\sqrt{6}$倍,且速度的三个分量满足$\upsilon_x>\upsilon_y>\upsilon_z$的概率约为 \bropt
  
  \optlist{$2.21\times 10^{-4}$}{$1.47\times 10^{-4}$}{$7.35\times 10^{-5}$}

}    
\item[(3)]{某气体的状态方程为$\left(p+\frac{\nu^2 a}{V^2T}\right)V = \nu R T$,其中$\nu$为摩尔数,$p, V, T$分别为压强,体积和热力学温度,$a>0$为常量。则该气体的摩尔定体热容 \bropt
  
  \optlist{一定是常量}{不可能是常量}{可能是常量也可能是温度的函数}

}


\item[(4)]{  置于很大的真空室内的绝热容器里装有稀薄氦气。在容器壁上开一个小孔,经过一段时间后把小孔堵上,发现容器内氦气压强降低了$0.4\%$,问容器内氦气的分子数减少了多少?漏气的过程很缓慢,可以近似认为整个过程中容器内氦气一直处于热平衡。 \bropt

  \optlist{$0.4\%$}{$0.3\%$}{$0.1\%$}
}

\item[(5)]{在一个标准大气压下,某种气体从$T=250\SIK$准静态等压升温至$T=350\SIK$。在整个过程中该气体的焓和温度平方成正比。已知该气体在初始状态和末状态的化学势分别为$\mu_1$和$\mu_2$,则在过程中间$T=300\SIK$时,该气体的化学势为 \bropt
  
  \optlist{$\frac{\mu_1+\mu_2}{2}$}{$\frac{5}{7}\mu_1+\frac{2}{7}\mu_2$}{$\frac{3}{5}\mu_1+\frac{3}{7}\mu_2$}
}
  
  \eitem
}

\item[(二)]{
  对只有两个独立变量的$pVT$系统证明:
  $$\pfrac TSH +\frac{T^2}{V} \pfrac VHp = \frac{T}{C_p}, $$
  其中$C_p$为定压热容。(25分)
  
  \vspace{5in}
  }

\item[]{\

  \vspace{2 in}}  
\item[(三)]{ 假设内径为$5\SIm$的封闭球形飞船绕中子星做每秒一周的匀速圆周运动。飞船的质心在球心,且自转和公转同步(即保持同一面对着中子星)。飞船内有温度为$285\SIK$的氧气。问:氧气的压强是均匀的吗?如果不均匀,最小压强和最大压强之比为多少? (25分)   
  }
  
  
\eitem


\ech
\end{document}

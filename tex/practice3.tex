\documentclass[12pt,CJK]{article}
\usepackage{geometry}
\input{reduced_macros.tex}
\geometry{tmargin=0.3in, bmargin=0.5in, lmargin=0.8in, rmargin=0.8in, nohead, nofoot}
\def\mark#1{{\color{blue} (#1分)}}
\renewcommand{\thepage}{}
\begin{document}
\bch


{\blue 纸上得来终觉浅,一到考试就不行} {\hskip 1.in} 姓名\uline{1} {\hskip 0.5in} 分数\uline{1}

{\vskip 0.3in}
\ \\
(一) 一个圆柱形容器一端和温度为$T$的热库接触,另一端和温度为$2T$的热库接触,并达到稳恒状态。容器内装有每个分子质量为$m$的理想气体,其分子平均速率是多少? \mark{700}

{\vskip 2.5in}
\ \\
(二) $0.112\SIkg$的氮气初始温度为$200\SIK$,经历等压过程,体积增大为初始状态的$2$倍。估算这个过程中气体吸热量$Q=$\uline{0.8}。 \mark{200}

{\vskip 0.1in}
\ \\
(三) 某$pVT$系统的状态方程为
$$\left(p+\frac{\nu^2 a}{V^2T}\right)V = \nu R T,$$
其中常量$a>0$,$\nu$是摩尔数。 证明定体摩尔热容$\cvmol$不可能只依赖于温度。 \mark{700}



\ech
\end{document}

\documentclass[12pt,CJK]{article}
\usepackage{geometry}
\input{reduced_macros.tex}
\geometry{tmargin=0.3in, bmargin=0.5in, lmargin=0.7in, rmargin=0.7in, nohead, nofoot}
\def\mark#1{{\color{blue} (#1分)}}
\renewcommand{\thepage}{}
\begin{document}
\bch
{\large 热学课堂练习 II 华山论剑版 (总分$\times 0.03$换算成平时分);}

%{\vskip 0.3in}

姓名 ....................... {\hskip 0.5in}    学号 .......................{\hskip 0.5in}  分数 ...................

%{\vskip 0.1in}

\bitem
\item[(一)]{判断题,请用$\checkmark$表示正确,$\times$表示错误,每题5分。

  \bitem
\item[(1)]{定义温标必须用到固定标准点,测温物质和测温属性,缺一不可。\bropt}
\item[(2)]{热力学第一定律对不可逆过程也适用。\bropt}
\item[(3)]{对任何平衡态物质均有$dU = C_VdT$,其中$U$为内能,$T$为温度,$C_V$为定体热容。\bropt}
\item[(4)]{液体的特点是分子排列短程无序,长程有序。\bropt}
\item[(3)]{经典气体的特点是每个微观态上的平均粒子数远小于1。\bropt}  

  \eitem
}

\item[(二)]{选择题,每题5分。

  \bitem

\item[(1)]{按照能均分定理,每一个 \bropt 贡献$\frac{1}{2}kT$的分子平均能量
  
  \optlist{独立自由度}{动量分量或坐标分量的二次型能量}{任意变量的二次型能量}

}    
\item[(2)]{当氮气温度和压强为下列那组数值时,氮气的节流效应是致冷的? \bropt
  
  \optlist{$T=3000\SIK, p = 1\atm $ }{$T=10\SIK, p = 100\atm$}{$T = 300\SIK, p = 0.5\atm $}

}


\item[(3)]{已知气体分子的速率分布函数为$F(\upsilon)$,下列哪个表达式是速率不超过$\upsilon_0$的所有分子的平均速率?
  \bropt

  \optlist{$\upsilon_0 \int_0^{\upsilon_0}  F(\upsilon)d\upsilon$}{$\int_0^{\upsilon_0}\upsilon F(\upsilon)d\upsilon$}{$\frac{\int_0^{\upsilon_0}\upsilon F(\upsilon)d\upsilon}{\int_0^{\upsilon_0} F(\upsilon)d\upsilon}$}
}
  

  \item[(4)]{
  在大气中进行的化学反应过程中的吸热量等于生成物和反应物的 \bropt

  \optlist{内能差}{焓差}{自由焓差}

}

\item[(5)]{
  处于热平衡的氢气中随机抽取一个分子,其速率不小于方均根速率的10倍的概率和下列哪个数的数量级最接近?\bropt

  \optlist{$1/10$}{$e^{-50}$}{$e^{-150}$}
}
  \eitem
}

  
\item[(三)]{某种气体的状态方程为$\left(p+\frac{a\nu^2}{V^3}\right)V = \nu R T$,其中$\nu$为摩尔数,$p$为压强,$V$为体积,$T$为热力学温度,$a>0$为固定常量。 $1\SImol$该气体经过等温过程体积增大了$10\%$。它的内能变大还是变小?(10分) 熵变化了多少? (10分)
    \vspace{3.4in}
  }


\item[(四)]{在日常环境有总质量为$1\SIkg$,温度为$0\cdeg$的很多小冰块。把它们慢慢逐块投入初始时质量为$1\SIkg$,温度为$80\cdeg$的水中进行熔化。全部投完后,恰好得到$2\SIkg$温度为$0\cdeg$的水。把大气理想化为绝热的。试估算 (1)冰的熔化热。(5分) (2)这个过程的熵变。(10分)
\vspace{3.5in}
}
  
  
  \item[(五)]{某可逆理想气体热机按下述循环工作:
      \bitem
    \item{以多方指数$n=\frac{5}{3}$的准静态多方过程从$423\SIK$升温到$777\SIK$;}
    \item{准静态绝热膨胀,降温到$200\SIK$;}
    \item{以$200\SIK$的温度准静态等温压缩;}
    \item{准静态绝热压缩,回到初始状态。}      
      \eitem
      在循环过程的温度范围内,该理想气体的定体摩尔热容随温度变化规律为
      $$C_V^{\rm mol} = \left(\frac{3}{2} + \frac{T}{T_0}\right)R,$$
      其中$T_0$为某固定常量。选择你喜欢的变量为坐标画出该循环的大致示意图(5分), 并求该热机的效率 (10分)。
  }  

\eitem


\ech
\end{document}

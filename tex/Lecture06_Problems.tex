\documentclass[CJK]{beamer}
\usepackage{CJKutf8}
\usepackage{beamerthemesplit}
\usetheme{Malmoe}
\useoutertheme[footline=authortitle]{miniframes}
\usepackage{amsmath}
\usepackage{amssymb}
\usepackage{graphicx}
\usepackage{eufrak}
\usepackage{color}
\usepackage{slashed}
\usepackage{simplewick}
\usepackage{tikz}
\graphicspath{{../figures/}}
\def\addfig#1#2{\begin{center}\includegraphics[width=#1 in]{#2}\end{center}}
\def\blacktext#1{{\color{black}#1}}
\def\bluetext#1{{\color{blue}#1}}
\def\redtext#1{{\color{red}#1}}
\def\darkbluetext#1{{\color[rgb]{0,0.2,0.6}#1}}
\def\skybluetext#1{{\color[rgb]{0.2,0.7,1.}#1}}
\def\cyantext#1{{\color[rgb]{0.,0.5,0.5}#1}}
\def\greentext#1{{\color[rgb]{0,0.7,0.1}#1}}
\def\darkgray{\color[rgb]{0.2,0.2,0.2}}
\def\lightgray{\color[rgb]{0.6,0.6,0.6}}
\def\gray{\color[rgb]{0.4,0.4,0.4}}
\def\blue{\color{blue}}
\def\red{\color{red}}
\def\green{\color{green}}
\def\darkblue{\color[rgb]{0,0.2,0.6}}
\def\skyblue{\color[rgb]{0.2,0.7,1.}}
\def\fdeg{{^\circ \mathrm{F}}}
\def\cdeg{^\circ \mathrm{C}}
\def\be{\begin{equation}}
\def\ee{\nonumber\end{equation}}
\def\bea{\begin{eqnarray}}
\def\eea{\nonumber\end{eqnarray}}
\def\ii{{\dot{\imath}}}
\def\bch{\begin{CJK}{UTF8}{gbsn}}
\def\ech{\end{CJK}}
\def\bitem{\begin{itemize}}
\def\eitem{\end{itemize}}
\def\bcenter{\begin{center}}
\def\ecenter{\end{center}}
\def\bex{\begin{minipage}{0.3\textwidth}\includegraphics[width=1in]{jugelizi.png}\end{minipage}\begin{minipage}{0.6\textwidth}}
\def\eex{\end{minipage}}
\def\chtitle#1{\frametitle{\bch#1\ech}}
\def\skipline{{\vskip0.1in}}
\def\skiplines{{\vskip0.2in}}
\def\lagr{{\mathcal{L}}}
\def\hamil{{\mathcal{H}}}
\def\vecv{{\mathbf{v}}}
\def\vecx{{\mathbf{x}}}
\def\vecy{{\mathbf{y}}}
\def\veck{{\mathbf{k}}}
\def\vecp{{\mathbf{p}}}
\def\vecn{{\mathbf{n}}}
\def\vecA{{\mathbf{A}}}
\def\vecP{{\mathbf{P}}}
\def\vecsigma{{\mathbf{\sigma}}}
\def\hatJn{{\hat{J_\vecn}}}
\def\hatJx{{\hat{J_x}}}
\def\hatJy{{\hat{J_y}}}
\def\hatJz{{\hat{J_z}}}
\def\hatj#1{\hat{J_{#1}}}
\def\hatphi{{\hat{\phi}}}
\def\hatq{{\hat{q}}}
\def\hatpi{{\hat{\pi}}}
\def\vel{\upsilon}
\def\Dint{{\mathcal{D}}}
\def\adag{{\hat{a}^\dagger}}
\def\bdag{{\hat{b}^\dagger}}
\def\cdag{{\hat{c}^\dagger}}
\def\ddag{{\hat{d}^\dagger}}
\def\hata{{\hat{a}}}
\def\hatb{{\hat{b}}}
\def\hatc{{\hat{c}}}
\def\hatd{{\hat{d}}}
\def\hatN{{\hat{N}}}
\def\hatH{{\hat{H}}}
\def\hatp{{\hat{p}}}
\def\Fup{{F^{\mu\nu}}}
\def\Fdown{{F_{\mu\nu}}}
\def\newl{\nonumber \\}
\def\SIkm{\,\mathrm{km}}
\def\SIyr{\,\mathrm{yr}}
\def\SIGyr{\,\mathrm{Gyr}}
\def\SIeV{\,\mathrm{eV}}
\def\SIkeV{\,\mathrm{keV}}
\def\SIMeV{\,\mathrm{MeV}}
\def\SIGeV{\,\mathrm{GeV}}
\def\SIcal{\,\mathrm{cal}}
\def\SIkcal{\,\mathrm{kcal}}
\def\SImol{\,\mathrm{mol}}
\def\SIm{\,\mathrm{m}}
\def\SIcm{\,\mathrm{cm}}
\def\SIfm{\,\mathrm{fm}}
\def\SImm{\,\mathrm{mm}}
\def\SInm{\,\mathrm{nm}}
\def\SImum{\,\mathrm{\mu m}}
\def\SIJ{\,\mathrm{J}}
\def\SIkJ{\,\mathrm{kJ}}
\def\SIs{\,\mathrm{s}}
\def\SIkg{\,\mathrm{kg}}
\def\SIg{\,\mathrm{g}}
\def\SIK{\,\mathrm{K}}
\def\SImmHg{\,\mathrm{mmHg}}
\def\SIPa{\,\mathrm{Pa}}
\def\vece{\mathrm{e}}
\def\bmat#1{\left(\begin{array}{#1}}
\def\emat{\end{array}\right)}
\def\bcase#1{\left\{\begin{array}{#1}}
\def\ecase{\end{array}\right.}
\def\calM{{\mathcal{M}}}
\def\calT{{\mathcal{T}}}
\def\calR{{\mathcal{R}}}
\def\barpsi{\bar{\psi}}
\def\baru{\bar{u}}
\def\barv{\bar{\upsilon}}
\def\bmini#1{\begin{minipage}{#1\textwidth}}
\def\emini{\end{minipage}}
\def\qeq{\stackrel{?}{=}}
\def\torder#1{\mathcal{T}\left(#1\right)}
\def\rorder#1{\mathcal{R}\left(#1\right)}
\def\contr#1#2{\contraction{}{#1}{}{#2}#1#2}
\def\trof#1{\mathrm{Tr}\left(#1\right)}
\def\trace{\mathrm{Tr}}
\def\comm#1{\ \ \ \left(\mathrm{used}\ #1\right)}
\def\tcomm#1{\ \ \ (\text{#1})}
\def\slp{\slashed{p}}
\def\slk{\slashed{k}}
\def\wulian{\includegraphics[width=0.18in]{emoji_wulian.jpg}}
\def\bye{\includegraphics[width=0.18in]{emoji_bye.jpg}}
\def\calp{{\mathfrak{p}}}
\def\veccalp{\mathbf{\mathfrak{p}}}
\def\atm{\,\mathrm{atm}}
\def\angstrom{\,\text{\AA}}
\def\Tthree{T_{\tiny \textcircled{3}}}
\def\pthree{p_{\tiny \textcircled{3}}}

\def\courseurl{http://zhiqihuang.top}

\def\tpage#1#2{
\begin{frame}
\bch
\begin{center}
\begin{large}
热学 \\
第#1讲 #2

\end{large}

\skiplines

黄志琦


\end{center}

\skiplines

{\small 
教材:《热学》第二版,赵凯华,罗蔚茵,高等教育出版社


课件下载
}
\courseurl 
\ech
\end{frame}
}

\def\bfr#1{
\begin{frame}
\chtitle{#1} 
\bch
}

\def\efr{
\ech 
\end{frame}
}

\title{Lesson 05 - Problems}
  \author{}
  \date{}
\begin{document}
\tpage{6}{一二章知识的应用(题霸版)}

\section{Temperature}

\begin{frame}
\chtitle{温标之间的非线性关系例1}
\bch
{\blue 教材习题1-4}

{\scriptsize
\bitem
\item[(1)]{$\mathcal{E}(-100\cdeg) = [0.21\times (-100) - 10^{-4}\times (-100)^2] \SImV = -22 \SImV$

其余以及图略。}
\item[(2)]{ $\mathcal{E}(0^\circ) =\mathcal{E}(0\cdeg) = 0 \SImV$, $\mathcal{E}(100^\circ) = \mathcal{E}(100\cdeg) = 20 \SImV$
$$a = \frac{100^\circ - 0^\circ}{20\SImV} =5 ^\circ/\SImV,\ b = 0$$ 图略}
\item[(3)]{$t = -100\cdeg$时, $t^*=-22\SImV \times 5^\circ/\SImV = -110^\circ$,其余类似,略。}
\item[(4)]{$t^*$和$t$除了在两个固定标准点被规定相等,两者成非线性关系,一般来说并不相等。}
\eitem
}
\ech
\end{frame}


\begin{frame}
\chtitle{温标之间的非线性关系例2}
\bch
{\blue 教材习题1-6}

{\scriptsize
\bitem
\item[(1)]{由于定体温度计的理想气体温度$T$(或热力学温度)和压强$p$成正比,所以 $t^* = \left[\ln (T/\SIK) + c\right]^\circ$ ($c$为常数)。由$T=273.16\SIK$时$t^* = 273.16^\circ$可以确定$c=273.16-\ln 273.16$。所以 $t^* = \left[\ln \frac{T}{273.16\SIK} + 273.16\right]^\circ$}
\item[(2)]{冰点$T = 273.15\SIK$,$t^* = \left[\ln \frac{273.15}{273.16}+273.16\right]^\circ = 273.159963^\circ$;

沸点$T=373.15\SIK$, $t^* =  \left[\ln \frac{373.15}{273.16}+273.16\right]^\circ = 273.471923^\circ$。}
\item[(3)]{存在,当$T=273.16\times e^{-273.16}\SIK$时,$t^* = 0^\circ$。}
\eitem
}
\ech
\end{frame}


\begin{frame}
\chtitle{定体气体温度计外推法例1}
\bch
{\blue 教材习题1-2}

{\scriptsize
我们先假设理想气体状态方程,在不同压强下计算待测温度,然后用外推的方法得到$p=0$时的待测温度:
\bea
(p_1, T_1) &=&  (734 \SImmHg, \frac{734}{500}\times 273.16 \SIK) = (734\SImmHg, 400.999\SIK) \newl
(p_2, T_2) &=&  (293.4 \SImmHg, \frac{293.4}{200}\times 273.16 \SIK) = (293.4\SImmHg, 400.726\SIK) \newl
(p_3, T_3) &=&  (146.68\SImmHg, \frac{146.68}{100}\times 273.16\SIK) = (146.68\SImmHg, 400.671\SIK)
\eea
求平均,并每个数据点减去平均:
\bea
(\bar{p},\bar{T}) &=& \left(\frac{p_1+p_2+p_3}{3},\frac{T_1+T_2+T_3}{3}\right) = (391.36\SImmHg, 400.799\SIK) \newl
(\Delta p_1, \Delta T_1) &=& (p_1-\bar{p}, T_1-\bar{T})=(342.64 \SImmHg, 0.200\SIK), \newl
(\Delta p_2, \Delta T_2) &=& (p_2-\bar{p}, T_2-\bar{T})= (-97.96 \SImmHg, -0.073\SIK), \newl
(\Delta p_3, \Delta T_3) &=& (p_3-\bar{p}, T_3-\bar{T})= (-244.68 \SImmHg, -0.128\SIK)
\eea
设拟合直线为$T = ap + b$,则
\be
a = \frac{\sum \Delta p \Delta T}{\sum \Delta p^2}= 0.0005726 \SImmHg/\SIK, \ b = \bar{T} - k\bar{p} = 400.575 \SIK
\ee
所以外推到$p=0$时$T = b = 400.575\SIK$,这就是所求的温度。
}
\ech
\end{frame}



\begin{frame}
\chtitle{定体温度计外推法例2}
\bch
{\blue 教材习题1-3}

{\scriptsize
定体气体温度计的$T/p$随着$p\rightarrow 0$而趋向于一个常数$\frac{V}{\nu R}$。
由数据点

$(p_1, T_1/p_1) = (0.400\atm, 682.875 \SIK/\atm)$

$(p_2, T_2/p_2) = (0.546\atm, 683.425 \SIK/\atm)$

拟合直线为
$$\frac{T}{p} = \left[3.767\frac{p}{\atm} + 681.368\right]\SIK/\atm $$

\bitem
\item[(1)]{$p=0.100\atm$时,$T=0.100\times(3.767\times 0.1 + 681.368)\SIK = 68.17\SIK$}
\item[(2)]{$T=444.60\cdeg = 717.75\SIK$,可以先忽略$3.767\frac{p}{\atm} $这项修正项,估算出
$$p \approx {717.75\over 681.368} \atm = 1.053\atm$$
然后再迭代计算更精确的解:
$$ p = \frac{717.75}{681.368 + 3.767\times 1.053} \atm = 1.047 \atm $$
}
\eitem
}
\ech
\end{frame}

\begin{frame}
\chtitle{定体温度计外推法例2:另一种解法}
\bch
{\blue 教材习题1-3}

{\scriptsize
我们考虑另一种拟合方案,拟合$p/T$为$p$的线性函数:

$(p_1, p_1/T_1) = (0.400\atm, 0.00146440 \atm/\SIK)$

$(p_2, p_2/T_2) = (0.546\atm, 0.00146322  \atm/\SIK)$

拟合直线为
$$\frac{p}{T} = \left[-8.082\times 10^{-6}\frac{p}{\atm} + 0.00146763\right]\atm/\SIK $$

\bitem
\item[(1)]{$p=0.100\atm$时,
$$T  = \frac{0.1}{-8.082\times 10^{-6}\times 0.1 + 0.00146763} = 68.17\SIK$$
}
\item[(2)]{$T=444.60\cdeg = 717.75\SIK$,可以先忽略$-8.082\times 10^{-6}\frac{p}{\atm} $这项修正项,估算出
$$p \approx 0.00146763\times 717.75 \atm = 1.053\atm$$
然后再迭代计算更精确的解:
$$ p = 717.75\times(-8.082\times 10^{-6}\times 1.053 + 0.00146763) \atm = 1.047 \atm $$
}
\eitem

{\bf 虽然各种线性拟合的假设不同,但因为实际气体偏离理想气体较小,得到的结果往往是一致的。}

}

\ech
\end{frame}


\begin{frame}
\chtitle{定体温度计外推法例2:应该扣一半分的解法}
\bch
{\blue 教材习题1-3}

\skipline

{\scriptsize
我们考虑直接使用那个啥定律(忘了是查理还是波意儿还是…):

直接拟合$T$为$p$的线性函数:
$$ T = (684.932\frac{p}{\atm} -0.8228)\SIK$$
(虽然我们明知这当$p\rightarrow 0$时误差较大)

\bitem
\item[(1)]{$p=0.100\atm$时,$T=67.67\SIK$}
\item[(2)]{$T=444.60\cdeg = 717.75\SIK$,$p  = 1.049\atm$}
\eitem

{\bf 这种拟合方法相当于没有使用$(p=0, T=0)$这个隐藏数据点,所以比较不精确 (不要迷信教材给的答案)。}

}

\ech
\end{frame}


\section{Gas EOS}



\begin{frame}
\chtitle{$pV =\nu RT$的简单应用例1}
\bch
{\blue 教材习题1-7}

\skipline

{\small
固定温度时,$pV$正比于摩尔数,所以我们可以用$pV$来代表“氧气的量”。可以用的天数为
$$\frac{130\times 32 - 10\times 32}{1\times 400} = 9.6 $$
所以每隔9天就要去充气。
}
\ech
\end{frame}



\begin{frame}
\chtitle{$pV =\nu RT$的简单应用例2}
\bch
{\blue 教材习题1-10}

{\scriptsize
由于空气比水银密度小,灌入水银时右侧管以及底管的空气都会直接或者以气泡的方式漏出。而左侧管的空气则会被压缩。由理想气体状态方程得到
$$ p_0 h_1 = \left(p_0+\rho g (h_2-h)\right) (h_1-h)$$
其中$p_0 =750 \SImmHg$为大气压强。上式可以化简为
$$ 1500 = (275- \frac{h}{\SIcm}) (20 - \frac{h}{\SIcm})$$

可以直接由二次方程求根公式求解上式得到$h=14.24742\SIcm$。下面介绍一个利用物理近似迭代求解的方法,求解更复杂的方程时它往往非常有用:

先由近似$275-\frac{h}{\SIcm}\approx 275$得到零级近似$h \approx (20-1500/275)\SIcm=14.55\SIcm$,然后迭代:

一级近似$h\approx \left(20 - \frac{1500}{275-14.55}\right)\SIcm = 14.241\SIcm$

二级近似$h\approx \left(20 - \frac{1500}{275-14.241}\right)\SIcm = 14.2476\SIcm$

三级近似$h\approx \left(20 - \frac{1500}{275-14.2476}\right)\SIcm = 14.24742\SIcm$

对结果要求不太精确的问题,往往一两次迭代就足够精确了。
}
\ech
\end{frame}

\begin{frame}
\chtitle{$pV =\nu RT$的简单应用例3}
\bch
{\blue 教材习题1-13}

\skipline

{\scriptsize
取大气压为$76\SIcm$汞柱, 步骤(1), (2)可以得到体积比为
$$\frac{V_{AC}}{V_{ABC}} = \frac{76}{76+12.5}$$
由$V_{ABC} = 1000\SIcm^3$即得$V_{AC} = 858.76 \SIcm^3$。

设矿物体积为$V_m$,由步骤(3),(4)可以得到
$$\frac{V_{AC}-V_{\rm m}}{V_{ABC}-V_m} = \frac{76}{76+23.7}$$
代入$V_{AC}$, $V_{ABC}$的值即得$V_m = 405.84\SIcm^3$。

故密度$\rho = 400\SIg/(405.84\SIcm^3) = 0.986 \SIg/\SIcm^3$
}
\ech
\end{frame}

\begin{frame}
\chtitle{阿伏伽德罗定律}
\bch
{\blue 教材习题1-15}

\skipline

{\scriptsize
氮气的摩尔质量$28\SIg/\SImol$, 氧气的摩尔质量$32\SIg/\SImol$,氩气的摩尔质量$40\SIg/\SImol$。

空气的摩尔质量
$$\frac{1\SIg}{\frac{0.76\SIg}{28\SIg/\SImol}+\frac{0.23\SIg}{32\SIg/\SImol}+\frac{0.01\SIg}{40\SIg/\SImol}} = 28.9\SIg/\SImol $$

标准状态下的空气密度
$$\rho = \frac{28.9\SIg}{22.4\mathrm{L}} = 1.29\SIkg/\SIm^3$$

{\bf 当然,阿伏伽德罗定律只是理想气体状态方程在$T=273.15\SIK$, $p = 1\atm$时的特殊情形,并不需要额外记忆。}
}
\ech
\end{frame}




\begin{frame}
\chtitle{道尔顿分压定律例1}
\bch
{\blue 教材习题1-17}

\skipline

{\scriptsize
氮气压强变为
$$p_{N_2} = \frac{0.5}{0.2}\times 1.0\times 10^5\SIPa = 2.5\times 10^5\SIPa$$
混合气体压强为
$$p = p_{N_2} + p_{O_2} = 2.5\times 10^5 \SIPa + 1.0\times 10^5\SIPa = 3.5\times 10^5 \SIPa$$
}
\ech
\end{frame}

\begin{frame}
\chtitle{道尔顿分压定律例2}
\bch
{\blue 教材习题1-16}

\skipline

{\scriptsize
收集的气体分压为$p_0=767.5\SImmHg - 17.5\SImmHg = 750\SImmHg$, 体积$V_0=150\SIcm^3$,温度$T_0=293.15\SIK$。
在$0\cdeg$干燥时,压强$p_1 = 767.5\SImmHg$,温度$T_1 = 273.15\SIK$。故体积
$$V_1 = \frac{\nu R T_1}{p_1} = \frac{p_0V_0 T_1}{T_0 p_1} = 150\SIcm^3 \times \frac{750}{767.5}\frac{273.15}{293.15} = 136.64 \SIcm^3 $$
}
\ech
\end{frame}


\begin{frame}
\chtitle{缓慢状态变化做功$= -\int p dV$}
\bch
{\blue 作业题3: 把1 mol的理想气体保持恒温300K进行等温压缩,使得体积变为原来一半,最少要做多少功?}

\skipline

{\scriptsize
“保持恒温”意味着一直处于热平衡(至少就我们目前接触的温度定义而言),由理想气体状态方程得到做的功为:
\bea
 W &=& -\int_{V_0}^{V_0/2} p dV \newl
&=& -\nu R T\int_{V_0}^{ V_0/2} \frac{dV}{V} \newl
&=& \nu R T \ln 2 \newl
&=& 1.73\times 10^3 \SIJ
\eea
}
\ech
\end{frame}


\begin{frame}
\chtitle{范德瓦尔斯方程}
\bch
{\blue 教材习题1-20}

\skipline

{\scriptsize
$CO_2$的分子量为$44$,故摩尔数为
$$\nu = \frac{1.1\SIkg}{44\SIg/\SImol} = 25\SImol$$
温度$T = 286.15\SIK$, 体积$V=0.02\SIm^3$

按范德瓦尔斯方程
$$ p = \frac{\nu RT}{V-\nu b} -a\left(\frac{\nu}{V}\right)^2 = 2.573\times 10^6\SIPa$$
按理想气体状态方程
$$ p = \frac{\nu R T}{V} = 2.974\times 10^6\SIPa$$
}
\ech
\end{frame}

\section{Probablity Density}

\begin{frame}
\chtitle{无量纲速度的概率密度}
\bch

我们定义温度$T$对应的特征速率$\upsilon_c = \sqrt{\frac{kT}{m}}$,并对理想气体分子定义无量纲速度$\mathbf{u} \equiv \frac{\vecv}{\upsilon_c}$。试写出下列概率密度函数:
\bitem
\item{以$u_x$为变量的概率密度$\tilde{f}_{\rm 1D}(u_x) $}
\item{以$u_x, u_y, u_z$为变量的三维概率密度$\tilde{f}_{\rm M}(u_x, u_y, u_z)$}
\item{以$u = |\mathbf{u}|$为变量的概率密度$\tilde{F}_{M}(u)$ ($u\ge 0$)}
\eitem

\ech
\end{frame}

\begin{frame}
\chtitle{无量纲速度的概率密度(续)}
\bch
\bmini{0.55}
{\scriptsize
因为是一一映射,换算概率密度就比较容易。由
$$\tilde{f}_{\rm 1D}(u_x) |d u_x| = f_{\rm 1D}(\upsilon_x) |d\upsilon_x|$$
得到
\bea
\tilde{f}_{\rm 1D}(u_x) &=& f_{\rm 1D}(\upsilon_x) \left\vert\frac{d\upsilon_x}{du_x}\right\vert \newl
&=& \frac{1}{\sqrt{2\pi}\upsilon_c} e^{-\frac{\upsilon_x^2}{2 \upsilon_c^2}} \upsilon_c \newl
&=& \frac{1}{\sqrt{2\pi}} e^{-\frac{u_x^2}{2}}
\eea
}
\emini
\bmini{0.4}
\addfig{1.5}{f1D.pdf}
\emini

{\scriptsize 概率密度$ \frac{1}{\sqrt{2\pi}} e^{-\frac{u_x^2}{2}}$称为标准正则分布,记作$N(u_x)$。

容易计算出下列平均值:
$$\bar{u_x} = \bar{u_x^3} = 0,\ \overline{|u_x|} = \sqrt{\frac{2}{\pi}} ,\ \bar{u_x^2} = 1, \ \bar{u_x^4} = 3$$
}

\ech
\end{frame}


\begin{frame}
\chtitle{无量纲速度的概率密度(续)}
\bch
{\scriptsize
因为$u_x$, $u_y$, $u_z$的分布相互独立,就有
$$\tilde{f}_M(u_x, u_y, u_z) =  N(u_x)N(u_y) N(u_z) = \left(\frac{1}{2\pi}\right)^{\frac{3}{2}}e^{-\frac{u_x^2+u_y^2+u_z^2}{2}}$$ 
转换到球坐标即可求出无量纲速率分布在$u$和$u+du$之间的概率,除以$du$得到:
$$\tilde{F}_M(u) =  \sqrt{\frac{2}{\pi}}u^2 e^{-\frac{u^2}{2}} $$
}
\ech
\end{frame}


\begin{frame}
\chtitle{正则分布的区间积分}
\bch
{\scriptsize
下面我们计算$u_x$在[-a,a]内的概率($a>0$)。
$$P(-a<u_x<a) = \frac{1}{\sqrt{2\pi}} \int_{-a}^a e^{-\frac{u_x^2}{2}}du_x \approx  \sqrt{1-e^{-\frac{a^2}{2}\left(\frac{\frac{4}{\pi}+ca^2}{1+ca^2}\right)}}$$
最后一步只是一个好用的近似公式,其中$c = \frac{4(\pi-3)}{3\pi(4-\pi)} \approx 0.07$。易算出
$$P(-1<u_x<1) =  0.683$$
$$P(-2<u_x<2) = 0.954$$
$$P(-3<u_x<3) = 0.9973$$
$$P(-4<u_x<4) = 0.999937$$
$$P(-5<u_x<5) = 0.99999943$$
在科学研究中,常常要用到上述置信区间。在论文中常可以看到"68.3\% confidence level" , "95.4\% confidence level" 等术语。粒子物理实验往往要求达到$5\sigma$精度,就是指可信度$99.999943\%$。
}
\ech
\end{frame}



\section{Maxwell Distribution}


\begin{frame}
\chtitle{Maxwell分布例1}
\bch
{\blue 教材习题2-8}

{\scriptsize
\bitem
\item[(1)]{由$\overline{u_x^2} = 1$ 得到$\overline{\upsilon_x^2} = \upsilon_c^2$,即一维的$\upsilon_{\rm rms} = \upsilon_c = \sqrt{\frac{kT}{m}}$}
\item[(2)]{由$\overline{|u_x|} = \sqrt{\frac{2}{\pi}}$  得到一维的$\bar{\upsilon} =  \sqrt{\frac{2}{\pi}}\upsilon_c = \sqrt{\frac{2kT}{\pi m}}$}
\item[(3)]{显然一维情况$\upsilon_{\max} = 0$}
\eitem

}
\ech
\end{frame}


\begin{frame}
\chtitle{Maxwell分布例2}
\bch
{\blue 教材习题2-7}

{\scriptsize
\bitem
\item[(1)]{由$\overline{u_x^2} = \overline{u_y^2} = 1$ 得到$\overline{\upsilon_x^2+\upsilon_y^2} = 2\upsilon_c^2$,即二维的$\upsilon_{\rm rms} = \sqrt{2}\upsilon_c = \sqrt{\frac{2kT}{m}}$}
\item[(2)]{二维的无量纲速率的概率密度为 $\tilde{F}_M(u) = 2\pi u \left(\frac{1}{2\pi} e^{-u^2/2}\right)= u e^{-u^2/2}$,所以平均无量纲速率:
$$\bar{u} = \int_0^\infty u^2 e^{-u^2/2} du = -\left.\frac{d}{d\alpha}\int_0^\infty e^{-\alpha u^2}du\right\vert_{\alpha=1/2}= \sqrt{\frac{\pi}{2}}$$ 
即$\bar{\upsilon} = \bar{u}\upsilon_c = \sqrt{\frac{\pi kT}{2m}}$
}
\item[(3)]{对$\tilde{F}_M(u)$求导并令其为零,得到$u_{\max} = 1$。 故$\upsilon_{\max} = \upsilon_c = \sqrt{\frac{kT}{m}}$}
\eitem

}
\ech
\end{frame}


\begin{frame}
\chtitle{Maxwell分布例3}
\bch
{\blue 教材习题2-3}

{\scriptsize
\bea
\overline{\left(\frac{1}{\upsilon}\right)} &=& \frac{1}{\upsilon_c} \overline{\left(\frac{1}{u}\right)} \newl
&=& \frac{1}{\upsilon_c} \int_0^\infty \frac{1}{u}\tilde{F}_M(u) du \newl
&=& \sqrt{\frac{2}{\pi}}\frac{1}{\upsilon_c} \int_0^\infty \frac{1}{u}  u^2e^{-u^2/2} du \newl
&=& \sqrt{\frac{2}{\pi}}\frac{1}{\upsilon_c}\left. e^{-u^2/2}\right\vert_0^\infty \newl
&=& \sqrt{\frac{2}{\pi}}\frac{1}{\upsilon_c} \newl
&=& \sqrt{\frac{2m}{\pi kT}} 
\eea

又由 $\bar{\upsilon} = \sqrt{\frac{8kT}{\pi m}}$,
$$\overline{\left(\frac{1}{\upsilon}\right)} = \frac{4}{\pi} \frac{1}{\bar{\upsilon}}>\frac{1}{\bar{\upsilon}} $$  
}
\ech
\end{frame}



\begin{frame}
\chtitle{Maxwell分布例4}
\bch
{\blue 计算使 $F_M(\upsilon)$最大的“最概然速率”$\upsilon_{\max}$}

\skipline

{\small
设$u_{\max}$使$\tilde{F}_M(u)$最大, 显然$\upsilon_{\max} = \upsilon_c u_{\max}$

由
$$\frac{d \tilde{F}_M(u)}{du}\vert_{u=u_{\max}} = 0$$
就可以解出$u_{\max} = \sqrt{2}$,故
$$\upsilon_{\max} = \sqrt{2}\upsilon_c = \sqrt{\frac{2kT}{m}}$$ }

\ech
\end{frame}

\begin{frame}
\chtitle{Maxwell分布例5}
\bch
{\blue 教材习题 2-14}

{\scriptsize
条件$\upsilon> \upsilon_{\max}$等同于 $u>\sqrt{2}$。其概率为
\bea
\int_{\sqrt{2}}^\infty \tilde{F}_M(u) du &=& \sqrt{\frac{2}{\pi}} \int_{\sqrt{2}}^\infty u^2e^{-\frac{u^2}{2}} du \newl
 &=& -\left.\sqrt{\frac{2}{\pi}} u e^{-\frac{u^2}{2}}\right\vert_{\sqrt{2}}^\infty +\sqrt{\frac{2}{\pi}} \int_{\sqrt{2}}^\infty e^{-\frac{u^2}{2}} du \newl
 &=& \frac{2}{e\sqrt{\pi}} + 1 - \sqrt{\frac{1}{2\pi}} \int_{-\sqrt{2}}^{\sqrt{2}} e^{-\frac{u^2}{2}} du \newl
&\approx & \frac{2}{e\sqrt{\pi}} + 1- \sqrt{1-e^{-\left(\frac{\frac{4}{\pi}+ 0.14}{1+0.14}\right)}} \newl
&=& 0.572\newl
\eea
同样的方法可以求出$\upsilon>2\upsilon_{\max}$的概率为 $0.0459$。
}
\ech
\end{frame}


\begin{frame}
\chtitle{Maxwell分布例6}
\bch
{\blue 教材思考题2-10}

{\scriptsize
设$\upsilon_0 = u_0 \upsilon_c$,则用和上例同样的方法可以算出

\bea
\frac{1}{2} &=& \int_{0}^{u_0} \tilde{F}_M(u) du \newl
&\approx & -\sqrt{\frac{2}{\pi}} u_0 e^{-u_0^2/2} +  \sqrt{1-e^{-\frac{u_0^2}{2}\left(\frac{\frac{4}{\pi}+ 0.07 u_0^2}{1+ 0.07 u_0^2}\right)}}  \newl
&\equiv& G(u_0)
\eea
把$u_0=1$作为零级近似,迭代计算出

一级近似$u_0 = 1 + \frac{0.5-G(1)}{\tilde{F}_M(1)} = 1.6223 $

二级近似$u_0 = 1.6223+\frac{0.5-G(1.6223)}{\tilde{F}_M(1.6223)} = 1.5364$

三级近似$u_0 = 1.5364+\frac{0.5-G(1.5364)}{\tilde{F}_M(1.5364)} = 1.5377$


从而$\upsilon_0 = 1.5377\sqrt{\frac{kT}{m}}$

显然,$\upsilon_0$和方均根速率$1.73\sqrt{\frac{kT}{m}}$,平均速率$1.60\sqrt{\frac{kT}{m}}$,泻流速率$0.40\sqrt{\frac{kT}{m}}$都不相同。
}
\ech
\end{frame}

\begin{frame}
\chtitle{方均根速率例1}
\bch
{\blue 教材习题2-24}

\skipline

{\small
温度$T = 273.15\SIK$。

如果认为灰尘的自由度为2(仅在水面运动),则方均根速率为
$$\sqrt{\frac{2kT}{m}} = 2.7\times 10^{-5} \SIm/\SIs$$  
}

\ech
\end{frame}


\begin{frame}
\chtitle{方均根速率例2}
\bch
{\blue 教材习题2-24}

\skipline

{\small
温度$T = 300.15\SIK$。

浮游微粒的的自由度为3,则方均根速率为
$$\sqrt{\frac{3kT}{m}} = 3.5\times 10^{-4} \SIm/\SIs$$  
}

\ech
\end{frame}


\begin{frame}
\chtitle{方均根速率$\ge$平均速率}
\bch
{\blue 教材思考题2-9}

{\small
\bea
0 &\le& \overline{\left(\upsilon- \overline{\upsilon}\right)^2} \newl
&=& \overline{\upsilon^2} + \overline{\upsilon}^2 - 2\overline{\upsilon\bar{\upsilon}} \newl 
&=& \overline{\upsilon^2} + \overline{\upsilon}^2 - 2\overline{\upsilon}^2 \newl 
&=& \overline{\upsilon^2} - \overline{\upsilon}^2 
\eea
从而有
$$ \overline{\upsilon^2} \ge \overline{\upsilon}^2 $$
两边开平方即有
$$ \upsilon_{\rm rms} \ge \overline{\upsilon}$$
}

\ech
\end{frame}

\begin{frame}
\chtitle{方均根速率$\ge$平均速率(另一种证明方法)}
\bch
{\blue 教材思考题2-9}

{\small
令$f(x) = - x^2$,则$f''(x) = -2<0$。故$f(x)$为凸函数。根据琴生不等式得到
$$ \overline{f(\upsilon)} \le f(\overline{\upsilon})$$
即
$$  - \overline{\upsilon^2} \le - \overline{\upsilon}^2$$
即
$$  \overline{\upsilon^2} \ge \overline{\upsilon}^2$$
等号当且仅当所有速率均相等时才能取到。
上式开平方即得
$$ \upsilon_{\rm rms} \ge \overline{\upsilon}$$

利用这种证明方法还可以迅速得到一系列和分布函数无关的不等式:
$$\overline{\left(\frac{1}{\upsilon}\right)} \ge \frac{1}{\bar{\upsilon}},\ \ \ \overline{\upsilon^4} \ge \bar{\upsilon}^4, \ \ \overline{\sqrt{\upsilon}} \le \sqrt{\bar{\upsilon}} \ldots $$

}

\ech
\end{frame}


\begin{frame}
\chtitle{各向同性分布:泻流速率和平均速率的关系}
\bch
{\blue 当速率分布各向同性时,泻流速率总是平均速率的$\frac{1}{4}$}

\skipline

{\scriptsize
设三维概率密度函数为$f(\upsilon_x, \upsilon_y,\upsilon_z) = g(\upsilon)$ (因各向同性所以可以这样设),在球坐标里计算泻流速率,体积元为$\upsilon^2 d\upsilon \sin\theta d\theta d\varphi$,$\upsilon_z = \upsilon\cos\theta$


\bea
\overline{\upsilon^+_z} &=& \int_0^\infty \upsilon^2 d\upsilon \int_0^{\frac{\pi}{2}} \sin\theta d\theta \int_0^{2\pi} d\varphi \,(\upsilon \cos\theta) g(\upsilon) \newl
&=& \pi \int_0^\infty \upsilon^2 d\upsilon\,  \upsilon g(\upsilon) \newl
&=& \frac{1}{4} \bar{\upsilon}
\eea

}

\ech
\end{frame}


\begin{frame}
\chtitle{泻流法提纯}
\bch
{\blue 教材习题2-5}

\skipline

{\scriptsize
因泻流分子数正比于$n\overline{\upsilon^+_x} \propto \frac{n}{\sqrt{m}}$每次泻流之后$U^{235}$和$U^{238}$分子数密度之比提高了
$$\sqrt{\frac{m_{U^{238}F_6}}{m_{U^{238}F_6}}} = \sqrt{\frac{238+19\times 6}{235+19\times 6}} = 1.0042888$$
倍。
所以需要提纯
$$ n = \frac{\ln \frac{\frac{0.99}{0.01}}{\frac{0.007}{0.993}}}{\ln 1.0042888} = 2232$$
次  
}
\ech
\end{frame}

\section{MB Distribution}

\begin{frame}
\chtitle{压强随高度变化}
\bch
{\blue 教材习题2-18}

\skipline

{\scriptsize
由“万能法则”(或按2,3班的叫法:玻尔兹曼分布),分子数密度正比于$e^{-\frac{mgh}{kT}}$,又由理想气体状态方程,压强正比于分子数密度。故
$$ e^{-\frac{mgh}{kT}} = \frac{0.8\atm}{1\atm} = 0.8$$
空气分子平均质量
$$m = \frac{29\SIg}{6.02\times 10^{23}} = 4.82\times 10^{-26}\SIkg$$
即
$$h = -\frac{kT\ln 0.8}{mg} = 1.96\times 10^3 \SIm$$ 
}
\ech
\end{frame}

\section{Heat Capacity}


\begin{frame}
\chtitle{气体的定压比热容}
\bch
{\blue 证明理想气体的摩尔定压比热容$C_p^{\rm mol}$和摩尔定体比热容$C_V^{\rm mol}$相差$R$}

\skipline
{\scriptsize
气体的内能$U(T)$ 满足
$$\frac{dU}{dT} = \nu C_V^{\rm mol}$$
在固定压强时,当气体温度变化$dT$,体积变化$dV = \frac{ \nu R dT}{p}$
故对气体做功为
$$-p dV = - \nu R dT$$
设气体吸收热量$dQ$,则由能量守恒,有
$$dQ - pdV = dU$$
(我们以后会叫它热力学第一定律),即
$$dQ = dU + pdV =  \nu C_V^{\rm mol}dT + \nu RdT$$
所以
$$C_p^{\rm mol} = \frac{1}{\nu} \frac{dQ}{dT} = C_V + R$$

}
\ech
\end{frame}

\end{document}

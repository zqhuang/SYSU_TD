\documentclass[CJK]{beamer}
\usepackage{CJKutf8}
\usepackage{beamerthemesplit}
\usetheme{Malmoe}
\useoutertheme[footline=authortitle]{miniframes}
\usepackage{amsmath}
\usepackage{amssymb}
\usepackage{graphicx}
\usepackage{eufrak}
\usepackage{color}
\usepackage{slashed}
\usepackage{simplewick}
\usepackage{tikz}
\graphicspath{{../figures/}}
\def\addfig#1#2{\begin{center}\includegraphics[width=#1 in]{#2}\end{center}}
\def\blacktext#1{{\color{black}#1}}
\def\bluetext#1{{\color{blue}#1}}
\def\redtext#1{{\color{red}#1}}
\def\darkbluetext#1{{\color[rgb]{0,0.2,0.6}#1}}
\def\skybluetext#1{{\color[rgb]{0.2,0.7,1.}#1}}
\def\cyantext#1{{\color[rgb]{0.,0.5,0.5}#1}}
\def\greentext#1{{\color[rgb]{0,0.7,0.1}#1}}
\def\darkgray{\color[rgb]{0.2,0.2,0.2}}
\def\lightgray{\color[rgb]{0.6,0.6,0.6}}
\def\gray{\color[rgb]{0.4,0.4,0.4}}
\def\blue{\color{blue}}
\def\red{\color{red}}
\def\green{\color{green}}
\def\darkblue{\color[rgb]{0,0.2,0.6}}
\def\skyblue{\color[rgb]{0.2,0.7,1.}}
\def\fdeg{{^\circ \mathrm{F}}}
\def\cdeg{^\circ \mathrm{C}}
\def\be{\begin{equation}}
\def\ee{\nonumber\end{equation}}
\def\bea{\begin{eqnarray}}
\def\eea{\nonumber\end{eqnarray}}
\def\ii{{\dot{\imath}}}
\def\bch{\begin{CJK}{UTF8}{gbsn}}
\def\ech{\end{CJK}}
\def\bitem{\begin{itemize}}
\def\eitem{\end{itemize}}
\def\bcenter{\begin{center}}
\def\ecenter{\end{center}}
\def\bex{\begin{minipage}{0.3\textwidth}\includegraphics[width=1in]{jugelizi.png}\end{minipage}\begin{minipage}{0.6\textwidth}}
\def\eex{\end{minipage}}
\def\chtitle#1{\frametitle{\bch#1\ech}}
\def\skipline{{\vskip0.1in}}
\def\skiplines{{\vskip0.2in}}
\def\lagr{{\mathcal{L}}}
\def\hamil{{\mathcal{H}}}
\def\vecv{{\mathbf{v}}}
\def\vecx{{\mathbf{x}}}
\def\vecy{{\mathbf{y}}}
\def\veck{{\mathbf{k}}}
\def\vecp{{\mathbf{p}}}
\def\vecn{{\mathbf{n}}}
\def\vecA{{\mathbf{A}}}
\def\vecP{{\mathbf{P}}}
\def\vecsigma{{\mathbf{\sigma}}}
\def\hatJn{{\hat{J_\vecn}}}
\def\hatJx{{\hat{J_x}}}
\def\hatJy{{\hat{J_y}}}
\def\hatJz{{\hat{J_z}}}
\def\hatj#1{\hat{J_{#1}}}
\def\hatphi{{\hat{\phi}}}
\def\hatq{{\hat{q}}}
\def\hatpi{{\hat{\pi}}}
\def\vel{\upsilon}
\def\Dint{{\mathcal{D}}}
\def\adag{{\hat{a}^\dagger}}
\def\bdag{{\hat{b}^\dagger}}
\def\cdag{{\hat{c}^\dagger}}
\def\ddag{{\hat{d}^\dagger}}
\def\hata{{\hat{a}}}
\def\hatb{{\hat{b}}}
\def\hatc{{\hat{c}}}
\def\hatd{{\hat{d}}}
\def\hatN{{\hat{N}}}
\def\hatH{{\hat{H}}}
\def\hatp{{\hat{p}}}
\def\Fup{{F^{\mu\nu}}}
\def\Fdown{{F_{\mu\nu}}}
\def\newl{\nonumber \\}
\def\SIkm{\,\mathrm{km}}
\def\SIyr{\,\mathrm{yr}}
\def\SIGyr{\,\mathrm{Gyr}}
\def\SIeV{\,\mathrm{eV}}
\def\SIkeV{\,\mathrm{keV}}
\def\SIMeV{\,\mathrm{MeV}}
\def\SIGeV{\,\mathrm{GeV}}
\def\SIcal{\,\mathrm{cal}}
\def\SIkcal{\,\mathrm{kcal}}
\def\SImol{\,\mathrm{mol}}
\def\SIm{\,\mathrm{m}}
\def\SIcm{\,\mathrm{cm}}
\def\SIfm{\,\mathrm{fm}}
\def\SImm{\,\mathrm{mm}}
\def\SInm{\,\mathrm{nm}}
\def\SImum{\,\mathrm{\mu m}}
\def\SIJ{\,\mathrm{J}}
\def\SIkJ{\,\mathrm{kJ}}
\def\SIs{\,\mathrm{s}}
\def\SIkg{\,\mathrm{kg}}
\def\SIg{\,\mathrm{g}}
\def\SIK{\,\mathrm{K}}
\def\SImmHg{\,\mathrm{mmHg}}
\def\SIPa{\,\mathrm{Pa}}
\def\vece{\mathrm{e}}
\def\bmat#1{\left(\begin{array}{#1}}
\def\emat{\end{array}\right)}
\def\bcase#1{\left\{\begin{array}{#1}}
\def\ecase{\end{array}\right.}
\def\calM{{\mathcal{M}}}
\def\calT{{\mathcal{T}}}
\def\calR{{\mathcal{R}}}
\def\barpsi{\bar{\psi}}
\def\baru{\bar{u}}
\def\barv{\bar{\upsilon}}
\def\bmini#1{\begin{minipage}{#1\textwidth}}
\def\emini{\end{minipage}}
\def\qeq{\stackrel{?}{=}}
\def\torder#1{\mathcal{T}\left(#1\right)}
\def\rorder#1{\mathcal{R}\left(#1\right)}
\def\contr#1#2{\contraction{}{#1}{}{#2}#1#2}
\def\trof#1{\mathrm{Tr}\left(#1\right)}
\def\trace{\mathrm{Tr}}
\def\comm#1{\ \ \ \left(\mathrm{used}\ #1\right)}
\def\tcomm#1{\ \ \ (\text{#1})}
\def\slp{\slashed{p}}
\def\slk{\slashed{k}}
\def\wulian{\includegraphics[width=0.18in]{emoji_wulian.jpg}}
\def\bye{\includegraphics[width=0.18in]{emoji_bye.jpg}}
\def\calp{{\mathfrak{p}}}
\def\veccalp{\mathbf{\mathfrak{p}}}
\def\atm{\,\mathrm{atm}}
\def\angstrom{\,\text{\AA}}
\def\Tthree{T_{\tiny \textcircled{3}}}
\def\pthree{p_{\tiny \textcircled{3}}}

\def\courseurl{http://zhiqihuang.top}

\def\tpage#1#2{
\begin{frame}
\bch
\begin{center}
\begin{large}
热学 \\
第#1讲 #2

\end{large}

\skiplines

黄志琦


\end{center}

\skiplines

{\small 
教材:《热学》第二版,赵凯华,罗蔚茵,高等教育出版社


课件下载
}
\courseurl 
\ech
\end{frame}
}

\def\bfr#1{
\begin{frame}
\chtitle{#1} 
\bch
}

\def\efr{
\ech 
\end{frame}
}

\title{Lesson 06 - Problems}
  \author{}
  \date{}
\begin{document}
\tpage{6}{一二章知识的应用(题霸版)}

\section{Temperature}

\begin{frame}
\chtitle{温标之间的非线性关系例1}
\bch
{\blue 教材习题1-4}

{\scriptsize
\bitem
\item[(1)]{$\mathcal{E}(-100\cdeg) = [0.21\times (-100) - 10^{-4}\times (-100)^2] \SImV = -22 \SImV$

其余以及图略。}
\item[(2)]{ $\mathcal{E}(0^\circ) =\mathcal{E}(0\cdeg) = 0 \SImV$, $\mathcal{E}(100^\circ) = \mathcal{E}(100\cdeg) = 20 \SImV$
$$a = \frac{100^\circ - 0^\circ}{20\SImV} =5 ^\circ/\SImV,\ b = 0$$ 图略}
\item[(3)]{$t = -100\cdeg$时, $t^*=-22\SImV \times 5^\circ/\SImV = -110^\circ$,其余类似,略。}
\item[(4)]{$t^*$和$t$除了在两个固定标准点被规定相等,两者成非线性关系,一般来说并不相等。}
\eitem
}
\ech
\end{frame}


\begin{frame}
\chtitle{温标之间的非线性关系例2}
\bch
{\blue 教材习题1-6}

{\scriptsize
\bitem
\item[(1)]{由于定体温度计的理想气体温度$T$(或热力学温度)和压强$p$成正比,所以 $t^* = \left[\ln (T/\SIK) + c\right]^\circ$ ($c$为常数)。由$T=273.16\SIK$时$t^* = 273.16^\circ$可以确定$c=273.16-\ln 273.16$。所以 $t^* = \left[\ln \frac{T}{273.16\SIK} + 273.16\right]^\circ$}
\item[(2)]{冰点$T = 273.15\SIK$,$t^* = \left[\ln \frac{273.15}{273.16}+273.16\right]^\circ = 273.159963^\circ$;

沸点$T=373.15\SIK$, $t^* =  \left[\ln \frac{373.15}{273.16}+273.16\right]^\circ = 273.471923^\circ$。}
\item[(3)]{存在,当$T=273.16\times e^{-273.16}\SIK$时,$t^* = 0^\circ$。}
\eitem
}
\ech
\end{frame}


\begin{frame}
\chtitle{定体气体温度计外推法例1}
\bch
{\blue 教材习题1-2}

{\scriptsize
我们先假设理想气体状态方程,在不同压强下计算待测温度,然后用外推的方法得到$p=0$时的待测温度:
\bea
(p_1, T_1) &=&  (734 \SImmHg, \frac{734}{500}\times 273.16 \SIK) = (734\SImmHg, 400.999\SIK) \newl
(p_2, T_2) &=&  (293.4 \SImmHg, \frac{293.4}{200}\times 273.16 \SIK) = (293.4\SImmHg, 400.726\SIK) \newl
(p_3, T_3) &=&  (146.68\SImmHg, \frac{146.68}{100}\times 273.16\SIK) = (146.68\SImmHg, 400.671\SIK)
\eea
求平均,并每个数据点减去平均:
\bea
(\bar{p},\bar{T}) &=& \left(\frac{p_1+p_2+p_3}{3},\frac{T_1+T_2+T_3}{3}\right) = (391.36\SImmHg, 400.799\SIK) \newl
(\Delta p_1, \Delta T_1) &=& (p_1-\bar{p}, T_1-\bar{T})=(342.64 \SImmHg, 0.200\SIK), \newl
(\Delta p_2, \Delta T_2) &=& (p_2-\bar{p}, T_2-\bar{T})= (-97.96 \SImmHg, -0.073\SIK), \newl
(\Delta p_3, \Delta T_3) &=& (p_3-\bar{p}, T_3-\bar{T})= (-244.68 \SImmHg, -0.128\SIK)
\eea
设拟合直线为$T = ap + b$,则
\be
a = \frac{\sum \Delta p \Delta T}{\sum \Delta p^2}= 0.0005726 \SImmHg/\SIK, \ b = \bar{T} - k\bar{p} = 400.575 \SIK
\ee
所以外推到$p=0$时$T = b = 400.575\SIK$,这就是所求的温度。
}
\ech
\end{frame}



\begin{frame}
\chtitle{定体温度计外推法例2}
\bch
{\blue 教材习题1-3}

{\scriptsize
定体气体温度计的$T/p$随着$p\rightarrow 0$而趋向于一个常数$\frac{V}{\nu R}$。
由数据点

$(p_1, T_1/p_1) = (0.400\atm, 682.875 \SIK/\atm)$

$(p_2, T_2/p_2) = (0.546\atm, 683.425 \SIK/\atm)$

拟合直线为
$$\frac{T}{p} = \left[3.767\frac{p}{\atm} + 681.368\right]\SIK/\atm $$

\bitem
\item[(1)]{$p=0.100\atm$时,$T=0.100\times(3.767\times 0.1 + 681.368)\SIK = 68.17\SIK$}
\item[(2)]{$T=444.60\cdeg = 717.75\SIK$,可以先忽略$3.767\frac{p}{\atm} $这项修正项,估算出
$$p \approx {717.75\over 681.368} \atm = 1.053\atm$$
然后再迭代计算更精确的解:
$$ p = \frac{717.75}{681.368 + 3.767\times 1.053} \atm = 1.047 \atm $$
}
\eitem
}
\ech
\end{frame}

\begin{frame}
\chtitle{定体温度计外推法例2:另一种解法}
\bch
{\blue 教材习题1-3}

{\scriptsize
我们考虑另一种拟合方案,拟合$p/T$为$p$的线性函数:

$(p_1, p_1/T_1) = (0.400\atm, 0.00146440 \atm/\SIK)$

$(p_2, p_2/T_2) = (0.546\atm, 0.00146322  \atm/\SIK)$

拟合直线为
$$\frac{p}{T} = \left[-8.082\times 10^{-6}\frac{p}{\atm} + 0.00146763\right]\atm/\SIK $$

\bitem
\item[(1)]{$p=0.100\atm$时,
$$T  = \frac{0.1}{-8.082\times 10^{-6}\times 0.1 + 0.00146763} = 68.17\SIK$$
}
\item[(2)]{$T=444.60\cdeg = 717.75\SIK$,可以先忽略$-8.082\times 10^{-6}\frac{p}{\atm} $这项修正项,估算出
$$p \approx 0.00146763\times 717.75 \atm = 1.053\atm$$
然后再迭代计算更精确的解:
$$ p = 717.75\times(-8.082\times 10^{-6}\times 1.053 + 0.00146763) \atm = 1.047 \atm $$
}
\eitem

{\bf 虽然各种线性拟合的假设不同,但因为实际气体偏离理想气体较小,得到的结果往往是一致的。}

}

\ech
\end{frame}


\begin{frame}
\chtitle{定体温度计外推法例2:比较粗糙的解法}
\bch
{\blue 教材习题1-3}

\skipline

{\scriptsize
我们考虑直接使用那个啥定律(忘了是查理还是波意儿还是…):

直接拟合$T$为$p$的线性函数:
$$ T = (684.932\frac{p}{\atm} -0.8228)\SIK$$
(虽然我们明知这当$p\rightarrow 0$时误差较大)

\bitem
\item[(1)]{$p=0.100\atm$时,$T=67.67\SIK$}
\item[(2)]{$T=444.60\cdeg = 717.75\SIK$,$p  = 1.049\atm$}
\eitem

{\bf 这种拟合方法相当于没有使用$(p=0, T=0)$这个隐藏数据点,所以比较不精确。}

}

\ech
\end{frame}


\section{Gas EOS}



\begin{frame}
\chtitle{$pV =\nu RT$的简单应用例1}
\bch
{\blue 教材习题1-7}

\skipline

{\small
固定温度时,$pV$正比于摩尔数,所以我们可以用$pV$来代表“氧气的量”。可以用的天数为
$$\frac{130\times 32 - 10\times 32}{1\times 400} = 9.6 $$
所以每隔9天就要去充气。
}
\ech
\end{frame}



\begin{frame}
\chtitle{$pV =\nu RT$的简单应用例2}
\bch
{\blue 教材习题1-10}

{\scriptsize
由于空气比水银密度小,灌入水银时右侧管以及底管的空气都会直接或者以气泡的方式漏出。而左侧管的空气则会被压缩。由理想气体状态方程得到
$$ p_0 h_1 = \left(p_0+\rho g (h_2-h)\right) (h_1-h)$$
其中$p_0 =750 \SImmHg$为大气压强。上式可以化简为
$$ 1500 = (275- \frac{h}{\SIcm}) (20 - \frac{h}{\SIcm})$$

可以直接由二次方程求根公式求解上式得到$h=14.24742\SIcm$。下面介绍一个利用物理近似迭代求解的方法,求解更复杂的方程时它往往非常有用:

先由近似$275-\frac{h}{\SIcm}\approx 275$得到零级近似$h \approx (20-1500/275)\SIcm=14.55\SIcm$,然后迭代:

一级近似$h\approx \left(20 - \frac{1500}{275-14.55}\right)\SIcm = 14.241\SIcm$

二级近似$h\approx \left(20 - \frac{1500}{275-14.241}\right)\SIcm = 14.2476\SIcm$

三级近似$h\approx \left(20 - \frac{1500}{275-14.2476}\right)\SIcm = 14.24742\SIcm$

对结果要求不太精确的问题,往往一两次迭代就足够精确了。
}
\ech
\end{frame}

\begin{frame}
\chtitle{$pV =\nu RT$的简单应用例3}
\bch
{\blue 教材习题1-13}

\skipline

{\scriptsize
取大气压为$76\SIcm$汞柱, 步骤(1), (2)可以得到体积比为
$$\frac{V_{AC}}{V_{ABC}} = \frac{76}{76+12.5}$$
由$V_{ABC} = 1000\SIcm^3$即得$V_{AC} = 858.76 \SIcm^3$。

设矿物体积为$V_m$,由步骤(3),(4)可以得到
$$\frac{V_{AC}-V_{\rm m}}{V_{ABC}-V_m} = \frac{76}{76+23.7}$$
代入$V_{AC}$, $V_{ABC}$的值即得$V_m = 405.84\SIcm^3$。

故密度$\rho = 400\SIg/(405.84\SIcm^3) = 0.986 \SIg/\SIcm^3$
}
\ech
\end{frame}

\begin{frame}
\chtitle{阿伏伽德罗定律}
\bch
{\blue 教材习题1-15}

\skipline

{\scriptsize
氮气的摩尔质量$28\SIg/\SImol$, 氧气的摩尔质量$32\SIg/\SImol$,氩气的摩尔质量$40\SIg/\SImol$。

空气的摩尔质量
$$\frac{1\SIg}{\frac{0.76\SIg}{28\SIg/\SImol}+\frac{0.23\SIg}{32\SIg/\SImol}+\frac{0.01\SIg}{40\SIg/\SImol}} = 28.9\SIg/\SImol $$

标准状态下的空气密度
$$\rho = \frac{28.9\SIg}{22.4\mathrm{L}} = 1.29\SIkg/\SIm^3$$

{\bf 当然,阿伏伽德罗定律只是理想气体状态方程在$T=273.15\SIK$, $p = 1\atm$时的特殊情形,并不需要额外记忆。}
}
\ech
\end{frame}




\begin{frame}
\chtitle{道尔顿分压定律例1}
\bch
{\blue 教材习题1-17}

\skipline

{\scriptsize
氮气压强变为
$$p_{N_2} = \frac{0.5}{0.2}\times 1.0\times 10^5\SIPa = 2.5\times 10^5\SIPa$$
混合气体压强为
$$p = p_{N_2} + p_{O_2} = 2.5\times 10^5 \SIPa + 1.0\times 10^5\SIPa = 3.5\times 10^5 \SIPa$$
}
\ech
\end{frame}

\begin{frame}
\chtitle{道尔顿分压定律例2}
\bch
{\blue 教材习题1-16}

\skipline

{\scriptsize
收集的气体分压为$p_0=767.5\SImmHg - 17.5\SImmHg = 750\SImmHg$, 体积$V_0=150\SIcm^3$,温度$T_0=293.15\SIK$。
在$0\cdeg$干燥时,压强$p_1 = 767.5\SImmHg$,温度$T_1 = 273.15\SIK$。故体积
$$V_1 = \frac{\nu R T_1}{p_1} = \frac{p_0V_0 T_1}{T_0 p_1} = 150\SIcm^3 \times \frac{750}{767.5}\frac{273.15}{293.15} = 136.64 \SIcm^3 $$
}
\ech
\end{frame}


\begin{frame}
\chtitle{缓慢状态变化做功$= -\int p dV$}
\bch
{\blue 作业题3: 把1 mol的理想气体保持恒温300K进行等温压缩,使得体积变为原来一半,最少要做多少功?}

\skipline

{\scriptsize
“保持恒温”意味着一直处于热平衡(至少就我们目前接触的温度定义而言),由理想气体状态方程得到做的功为:
\bea
 W &=& -\int_{V_0}^{V_0/2} p dV \newl
&=& -\nu R T\int_{V_0}^{ V_0/2} \frac{dV}{V} \newl
&=& \nu R T \ln 2 \newl
&=& 1.73\times 10^3 \SIJ
\eea
}
\ech
\end{frame}


\begin{frame}
\chtitle{范德瓦尔斯方程}
\bch
{\blue 教材习题1-20}

\skipline

{\scriptsize
$CO_2$的分子量为$44$,故摩尔数为
$$\nu = \frac{1.1\SIkg}{44\SIg/\SImol} = 25\SImol$$
温度$T = 286.15\SIK$, 体积$V=0.02\SIm^3$

按范德瓦尔斯方程
$$ p = \frac{\nu RT}{V-\nu b} -a\left(\frac{\nu}{V}\right)^2 = 2.573\times 10^6\SIPa$$
按理想气体状态方程
$$ p = \frac{\nu R T}{V} = 2.974\times 10^6\SIPa$$
}
\ech
\end{frame}

\section{General Distribution}

\begin{frame}
\bch
{\Large
一般性分布的例子
}
\ech
\end{frame}


\begin{frame}
\chtitle{方均根速率$\ge$平均速率}
\bch
{\blue 教材思考题2-9}

{\small
\bea
0 &\le& \overline{\left(\upsilon- \overline{\upsilon}\right)^2} \newl
&=& \overline{\upsilon^2} + \overline{\upsilon}^2 - 2\overline{\upsilon\bar{\upsilon}} \newl 
&=& \overline{\upsilon^2} + \overline{\upsilon}^2 - 2\overline{\upsilon}^2 \newl 
&=& \overline{\upsilon^2} - \overline{\upsilon}^2 
\eea
从而有
$$ \overline{\upsilon^2} \ge \overline{\upsilon}^2 $$
两边开平方即有
$$ \upsilon_{\rm rms} \ge \overline{\upsilon}$$
}

\ech
\end{frame}

\begin{frame}
\bch
下面再介绍两种“超纲”的解法,如果觉得学习难度太大可以暂时跳过。
\ech
\end{frame}

\begin{frame}
\chtitle{方均根速率$\ge$平均速率(第二种证明方法及推广)}
\bch
{\blue 教材思考题2-9}

{\small
令$f(x) = - x^2$,则$f''(x) = -2<0$。故$f(x)$为凸函数。根据琴生不等式得到
$$ \overline{f(\upsilon)} \le f(\overline{\upsilon})$$
即
$$  - \overline{\upsilon^2} \le - \overline{\upsilon}^2$$
即
$$  \overline{\upsilon^2} \ge \overline{\upsilon}^2$$
等号当且仅当所有速率均相等时才能取到。
上式开平方即得
$$ \upsilon_{\rm rms} \ge \overline{\upsilon}$$

利用这种证明方法还可以迅速得到一系列和分布函数无关的不等式:
$$\overline{\left(\frac{1}{\upsilon}\right)} \ge \frac{1}{\bar{\upsilon}},\ \ \ \overline{\upsilon^4} \ge \bar{\upsilon}^4, \ \ \overline{\sqrt{\upsilon}} \le \sqrt{\bar{\upsilon}} \ldots $$

}

\ech
\end{frame}


\begin{frame}
\chtitle{方均根速率$\ge$平均速率 (第三种证明方法及推广)}
\bch
{\blue 教材思考题2-9}

{\scriptsize
设有$N$个分子,速率分别为$\upsilon_i (i = 1, 2, \ldots, N)$。对任何$1\le i, j\le N$,显然有
$$(\upsilon_i - \upsilon_j)^2 > 0$$
上式展开
$$ \upsilon_i^2 +  \upsilon_j^2 - 2 \upsilon_i\upsilon_j \ge 0$$
上式对$i, j$求和,即
$$ N \sum_{i=1}^N \upsilon_i^2 + N \sum_{j=1}^n \upsilon_j^2  - 2\sum_{i=1}^N \upsilon_i \sum_{j=1}^N\upsilon_j \ge 0$$
或者写成
$$2 N^2 \overline{\upsilon^2} - 2\left(N\overline{\upsilon}\right)^2 \ge 0$$
即 
$$ \overline{\upsilon^2} \ge \left(\overline{\upsilon}\right)^2 $$
两边开平方即得证。
}

\ech
\end{frame}

\begin{frame}
\chtitle{思考题}
\bch
{\scriptsize
\bitem
\item[1]{对同号的$p, q$,显然有$(\upsilon_i^p - \upsilon_j^p)(\upsilon_i^q - \upsilon_j^q) \ge 0$,试证明
$$\overline{\upsilon^{p+q}} \ge \overline{\upsilon^p}\; \overline{\upsilon^q} $$}
\item[2]{
对异号的$p, q$,显然有$(\upsilon_i^p - \upsilon_j^p)(\upsilon_i^q - \upsilon_j^q) \le 0$,试证明
$$\overline{\upsilon^{p+q}} \le \overline{\upsilon^p}\,\overline{\upsilon^q} $$}
\item[3]{对同号的$p, q$ 以及任意$m$,显然有$\upsilon_i^m\upsilon_j^m(\upsilon_i^p - \upsilon_j^p)(\upsilon_i^q - \upsilon_j^q) \ge 0$,试证明
$$\overline{\upsilon^{p+q+m}} \;\overline{\upsilon^{m}} \ge \overline{\upsilon^{p+m}}\; \overline{\upsilon^{q+m}} $$}
\item[4]{对异号的$p, q$ 以及任意$m$,显然有$\upsilon_i^m\upsilon_j^m(\upsilon_i^p - \upsilon_j^p)(\upsilon_i^q - \upsilon_j^q) \le 0$,试证明
$$\overline{\upsilon^{p+q+m}} \;\overline{\upsilon^{m}} \le \overline{\upsilon^{p+m}}\; \overline{\upsilon^{q+m}} $$}
\item[5]{对任意个同号的数$p_1, p_2, \ldots, p_n$,用归纳法证明
$$\overline{\upsilon^{p_1+p_2+\ldots+p_n}} \ge \overline{\upsilon^{p_1}}\; \overline{\upsilon^{p_2}}\, \ldots\,\overline{\upsilon^{p_n}} $$
}
\item[6]{对同号的$p_1,p_2,\ldots, p_n$ 以及任意$m$,用归纳法证明
$$\overline{\upsilon^{p_1+p_2+\ldots+p_n+m}}\; \left(\overline{\upsilon^m}\right)^{n-1}  \ge \overline{\upsilon^{p_1+m}}\; \overline{\upsilon^{p_2+m}}\, \ldots\,\overline{\upsilon^{p_n+m}} $$
}
\eitem
}
\ech
\end{frame}



\begin{frame}
\chtitle{各向同性分布:对称性考虑}
\bch
{\blue \small 设气体是各向同性的,对气体分子的速度$\vecv=(\upsilon_x, \upsilon_y, \upsilon_z)$和速率$\upsilon = \sqrt{\upsilon_x^2+\upsilon_y^2+\upsilon_z^2}$求下列事件的概率:
\bitem
\item[1]{\blue $\upsilon_x>\upsilon_y$}
\item[2]{\blue $\upsilon_x>0$且$\upsilon_y>0$且$\upsilon_z>0$}
\item[3]{\blue $|\upsilon_x|>\frac{1}{3}\upsilon$}
\eitem
}

{\scriptsize

\bitem
\item[1]{由对称性,$P(\upsilon_x>\upsilon_y) = P(\upsilon_x<\upsilon_y)=\frac{1}{2}$}
\item[2]{由对称性,$P(\upsilon_x>0) = P(\upsilon_y>0) = P(\upsilon_z>0) = \frac{1}{2}$,又$\upsilon_x$, $\upsilon_y$, $\upsilon_z$的分布互不相关,故联立事件$P(\upsilon_x>0, \upsilon_y>0, \upsilon_z>0) =\left(\frac{1}{2}\right)^3=\frac{1}{8}$}
\item[3]{取$\upsilon_x$方向为南北极轴,建立速度空间的球坐标系,则$\upsilon_x = \upsilon \cos\theta$。所以$|\upsilon_x|>\frac{1}{3}\upsilon$的概率即$|\cos\theta| > \frac{1}{3}$的概率。


球坐标的体积元为$\upsilon^2d\upsilon\sin\theta d\theta d\varphi = -\upsilon^2d\upsilon d(\cos\theta) d\varphi$,故对各向同性分布(只依赖于$\upsilon$的分布$f(\upsilon)$)而言,以$\mu = \cos\theta$为变量的概率密度
$$\frac{dP}{d\mu} = \int_0^\infty \upsilon^2d\upsilon f(\upsilon) \int d\varphi $$ 
是常数。$\cos\theta$的范围是$[-1,1]$,故$|\cos\theta|>\frac{1}{3}$的概率是$\frac{2}{3}$。}
\eitem

}

\ech
\end{frame}


\begin{frame}
\chtitle{思考题}
\bch
\addfig{1}{think2.jpg}
对各向同性分布,$|\upsilon_x|>\sqrt{3}|\upsilon_y|$的概率是多少?
\ech
\end{frame}



\begin{frame}
\chtitle{各向同性分布:泻流速率和平均速率的关系}
\bch
{\blue 当速率分布各向同性时,泻流速率(单位时间单位面积泄漏的分子数和箱内的分子数密度之比)总是平均速率的$\frac{1}{4}$}

\skipline

{\small
设三维概率密度函数为$f(\upsilon_x, \upsilon_y,\upsilon_z) = g(\upsilon)$ (因各向同性所以可以这样设),则速率$\upsilon$的概率密度函数为$4\pi\upsilon^2g(\upsilon)$。

不妨设小孔开在$z$方向,在球坐标里计算泻流速率,体积元为$\upsilon^2 d\upsilon \sin\theta d\theta d\varphi$,$\upsilon_z = \upsilon\cos\theta$


\bea
\upsilon_{n,\rm leak} &=& \int_0^\infty \upsilon^2 d\upsilon \int_0^{\frac{\pi}{2}} \sin\theta d\theta \int_0^{2\pi} d\varphi \,(\upsilon \cos\theta) g(\upsilon) \newl
&=& \pi \int_0^\infty \upsilon^2 d\upsilon\,  \upsilon g(\upsilon) \newl
&=& \frac{1}{4} \int_0^\infty \upsilon\left(4\pi\upsilon^2g(\upsilon) d\upsilon\right) \newl
&=& \frac{1}{4} \bar{\upsilon}
\eea

}

\ech
\end{frame}


\begin{frame}
\chtitle{各向同性分布:泻能速率}
\bch
{\blue \small 一个绝热箱上有个小孔,当箱内分子速率分布各向同性时,泻能速率(单位时间单位面积泄漏能量和箱内能量密度之比)为$\frac{\overline{\upsilon\varepsilon}}{4\overline\varepsilon}$。\bitem
\item{如果是处于热平衡的非相对论单原子理想气体,则根据麦克斯韦分布,泻能速率等于$\frac{1}{3}\overline{\upsilon}$。}
\item{在极端非相对论情形,泻能速率等于$\frac{1}{4}c$。}
\eitem}

\skipline

{\scriptsize
与前面一样在球坐标里计算泻能速率,
\be
\upsilon_{\varepsilon,\rm leak} = \int_0^\infty \upsilon^2 d\upsilon \int_0^{\frac{\pi}{2}} \sin\theta d\theta \int_0^{2\pi} d\varphi \,(\frac{\varepsilon}{\bar{\varepsilon}} \upsilon\cos\theta) g(\upsilon)=\frac{\overline{\varepsilon\upsilon}}{4\overline{\varepsilon}} 
\ee
对处于热平衡的非相对论单原子理想气体,能量仅为平动动能,即$\varepsilon\propto \upsilon^2$,所以
\be
\upsilon_{\varepsilon,\rm leak}  =\frac{\overline{\upsilon^3}}{4\overline{\upsilon^2}} = \frac{1}{3}\overline{\upsilon}
\ee
最后一步用到了类高斯积分的一些技巧,我们后面会更详细地介绍。



极端相对论情形$\bar{\upsilon} = c$,结论显然。
}

\ech
\end{frame}



\section{Maxwell Distribution}

\begin{frame}
\chtitle{麦克斯韦分布的计算技能}
\bch
\bmini{0.35}
\addfig{1}{newtech.jpg}
\emini
\bmini{0.61}
\tbox{\blue 无量纲速度:去掉讨厌的系数}
\tbox{无穷区间积分: 递归公式}
\tbox{小区间积分:当成常数}
\tbox{尾区间积分:小窍门}
\tbox{大区间积分:神奇的近似公式}
\emini
\ech
\end{frame}


\begin{frame}
\chtitle{无量纲速度的概率密度}
\bch

麦克斯韦分布的计算中总带着一堆讨厌的$k, T, m$,一不小心就写错。为此,我们定义特征速率$\upsilon_c = \sqrt{\frac{kT}{m}}$,并对理想气体分子定义无量纲速度$\mathbf{u} \equiv \frac{\vecv}{\upsilon_c}$。


然后我们考虑无量纲速度的概率密度函数:
\bitem
\item{以$u_x$为变量的概率密度$\tilde{f}_{\rm 1D}(u_x) $}
\item{以$u_x, u_y, u_z$为变量的三维概率密度$\tilde{f}_{\rm M}(u_x, u_y, u_z)$}
\item{以$u = |\mathbf{u}|$为变量的概率密度$\tilde{F}_{M}(u)$ ($u\ge 0$)}
\eitem

\ech
\end{frame}

\begin{frame}
\chtitle{无量纲速度的概率密度(续)}
\bch
\bmini{0.55}
{\scriptsize
因为是一一映射,换算概率密度就比较容易。由
$$\tilde{f}_{\rm 1D}(u_x) |d u_x| = f_{\rm 1D}(\upsilon_x) |d\upsilon_x|$$
得到
\bea
\tilde{f}_{\rm 1D}(u_x) &=& f_{\rm 1D}(\upsilon_x) \left\vert\frac{d\upsilon_x}{du_x}\right\vert \newl
&=& \frac{1}{\sqrt{2\pi}\upsilon_c} e^{-\frac{\upsilon_x^2}{2 \upsilon_c^2}} \upsilon_c \newl
&=& \frac{1}{\sqrt{2\pi}} e^{-\frac{u_x^2}{2}}
\eea
}
\emini
\bmini{0.4}
\addfig{1.5}{f1D.pdf}
\emini

{\scriptsize

概率密度$ \frac{1}{\sqrt{2\pi}} e^{-\frac{u_x^2}{2}}$称为标准正态分布。容易验证服从标准正态分布的变量的平方平均为$1$: $\overline{u_x^2} = 1$。

{\blue 无量纲速度的每个分量都服从标准正态分布。}
}

\ech
\end{frame}


\begin{frame}
\chtitle{无量纲速度的概率密度(续)}
\bch
{\small

因为$u_x$, $u_y$, $u_z$的分布相互独立,就有
$$\tilde{f}_M(u_x, u_y, u_z) =  \left(\frac{1}{2\pi}\right)^{\frac{3}{2}}e^{-\frac{u_x^2+u_y^2+u_z^2}{2}}$$ 
转换到球坐标即可求出{\blue 无量纲速率分布:

$$\tilde{F}_M(u) =  \sqrt{\frac{2}{\pi}}u^2 e^{-\frac{u^2}{2}} $$}
}
\ech
\end{frame}

\begin{frame}
\chtitle{无量纲速度总结}
\bch
\bitem
\item{用$\upsilon_c \equiv \sqrt{\frac{kT}{m}}$作为单位就得到了速度的无量纲表示。}
\item{无量纲速度的分布是固定的(不随$m$,$T$变化):它的每个分量独立地服从标准正态分布:
$$\tilde{f}_{1D}(u_x) = \frac{1}{\sqrt{2\pi}} e^{-\frac{u_x^2}{2}}$$
服从标准正态分布的变量的平方平均为$1$。
}
\item{无量纲速率的分布为
$$\tilde{F}_{M}(u) = \sqrt{\frac{2}{\pi}} u^2 e^{-\frac{u^2}{2}}, \ \ \ u\ge 0$$
}
\eitem
\ech
\end{frame}

\begin{frame}
\chtitle{麦克斯韦分布的计算技能}
\bch
\bmini{0.35}
\addfig{1}{newtech.jpg}
\emini
\bmini{0.61}
\tbox{\blue 无量纲速度:去掉讨厌的系数}
\tbox{\blue 无穷区间积分: 递归公式}
\tbox{小区间积分:当成常数}
\tbox{尾区间积分:小窍门}
\tbox{大区间积分:神奇的近似公式}
\emini
\ech
\end{frame}

\begin{frame}
\chtitle{无穷区间积分:递归公式}
\bch
{\scriptsize
我们常常需要对服从标准正态分布的变量$x$计算$|x|^n$,它可以写成
{\blue
$$\overline{|x|^n} = \frac{1}{\sqrt{2\pi}}\int_{-\infty}^\infty|x|^ne^{-x^2/2}dx = \sqrt{\frac{2}{\pi}}\int_0^\infty x^n e^{-x^2/2} dx$$}
显然
\bea
\overline{|x|^0} &=&  1 \newl 
\overline{|x|} &=& \left.\sqrt{\frac{2}{\pi}}e^{-x^2/2}\right\vert^\infty_0 = \sqrt{\frac{2}{\pi}} 
\eea
对$n>1$,则可分部积分得到
\bea
\overline{|x|^n} &=& -\sqrt{\frac{2}{\pi}}\int_0^\infty x^{n-1}d\left(e^{-x^2/2}\right) \newl
 &=& \left.-x^{n-1}e^{-x^2/2}\right\vert_0^\infty+\sqrt{\frac{2}{\pi}}\int_0^\infty (n-1)x^{n-2} e^{-x^2/2}dx \newl
&=&  (n-1)\sqrt{\frac{2}{\pi}}\int_0^\infty x^{n-2} e^{-x^2/2}dx \newl
&=& (n-1) \overline{|x|^{n-2}}
\eea
}
\ech
\end{frame}

\begin{frame}
\chtitle{无穷区间积分:递归公式}
\bch
{\small
总结起来就是,对服从标准正态分布的变量$x$,其绝对值的$n$次平均等价于下述两种积分表达式:
{\blue
$$\overline{|x|^n} = \frac{1}{\sqrt{2\pi}}\int_{-\infty}^\infty|x|^ne^{-x^2/2}dx = \sqrt{\frac{2}{\pi}}\int_0^\infty x^n e^{-x^2/2} dx$$}
它可以由下列递归关系求出:{\blue
\bea
\overline{|x|^0} &=&  1 \newl 
\overline{|x|} &=&  \sqrt{\frac{2}{\pi}}  \newl
\overline{|x|^n} &=& (n-1) \overline{|x|^{n-2}}, \ \ n\ge 2 
\eea
}
}
{\scriptsize
不难求出$\overline{|x|^2} = 1$, $\overline{|x|^3} = \sqrt{\frac{8}{\pi}}$, $\overline{|x|^4} = 3$, $\ldots$

很多高斯类的分布(并不需要是正态分布)的平均值问题都可以转化成上面的积分式,从而转化为标准正态分布的$\overline{|x|^n}$计算问题。
}
\ech
\end{frame}


\begin{frame}
\chtitle{无穷区间积分:递归公式}
\bch
{\small
例如,我们前面计算非相对论单原子理想气体的泻能速率时需要证明
$$\overline{u^3} = \frac{4}{3}\overline{u^2}\;\overline{u}$$
这只要直接计算即可,设$x$为满足标准正态分布的变量,
\bea
\overline{u^3} &=& \sqrt{\frac{2}{\pi}} \int_0^\infty  u^3 (u^2 e^{-u^2/2}) du =\overline{|x|^5}= (5-1)\times (3-1) \overline{|x|} = 8\sqrt{\frac{2}{\pi}} \newl
\overline{u^2} &=& \sqrt{\frac{2}{\pi}} \int_0^\infty  u^2 (u^2 e^{-u^2/2}) du =\overline{|x|^4}= (4-1)\times (2-1) \overline{|x|^0} = 3 \newl
\overline{u}   &=& \sqrt{\frac{2}{\pi}} \int_0^\infty  u   (u^2 e^{-u^2/2}) du =\overline{|x|^3}= (3-1)\overline{|x|} = 2\sqrt{\frac{2}{\pi}}
\eea
注意,我们只是把$u$的某种平均和$|x|$的若干次平均通过相同的积分式联系起来,$u$本身并不服从标准正态分布。
}
\ech
\end{frame}


\begin{frame}
\chtitle{麦克斯韦分布的计算技能}
\bch
\bmini{0.35}
\addfig{1}{newtech.jpg}
\emini
\bmini{0.61}
\tbox{\blue 无量纲速度:去掉讨厌的系数}
\tbox{\blue 无穷区间积分: 递归公式}
\tbox{\blue 小区间积分:当成常数}
\tbox{尾区间积分:小窍门}
\tbox{大区间积分:神奇的近似公式}
\emini
\ech
\end{frame}

\begin{frame}
\chtitle{小区间积分:当成常数}
\bch
{\small
如果积分范围比较小,我们可以假设概率密度函数在小区间内是常数,用乘法代替积分。

\skipline

例如分子速率在 $\bar{\upsilon}$附近$\pm 1\%$之内的概率为
$$ \tilde{F}_M(\sqrt{\frac{8}{\pi}}) (\sqrt{\frac{8}{\pi}} \times 0.02) =\sqrt{\frac{2}{\pi}} \frac{8}{\pi}e^{-4/\pi} \times\left(\sqrt{\frac{8}{\pi}}\times 0.02 \right) \approx 0.018  $$
注意我们已经开始使用无量纲速度来进行计算。
}
\ech
\end{frame}


\begin{frame}
\chtitle{麦克斯韦分布的计算技能}
\bch
\bmini{0.35}
\addfig{1}{newtech.jpg}
\emini
\bmini{0.61}
\tbox{\blue 无量纲速度:去掉讨厌的系数}
\tbox{\blue 无穷区间积分: 递归公式}
\tbox{\blue 小区间积分:当成常数}
\tbox{\blue 尾区间积分:小窍门}
\tbox{大区间积分:神奇的近似公式}
\emini
\ech
\end{frame}

\begin{frame}
\chtitle{尾区间积分:小窍门}
\bch
{\scriptsize
有时候我们要做风险评估,计算超出平均值很多倍的事件的概率。即对$a\gg 1$,要计算
$P(u_x >a)$, $P(|u_x|>a)$,$P(u>a)$等。下面我们介绍这种积分的小窍门:
一般地,我们考虑积分
$$P = \int_a^\infty x^n e^{-x^2/2}dx,  \ \ a\gg \sqrt{n} $$
对$x\gg \sqrt{n}$,有
$$\frac{d}{dx} \left(-\left(x+\frac{1}{x}\right)^{n-1}e^{-x^2/2}\right) = x^n e^{-x^2/2}\left(1+O\left(\left(\frac{\sqrt{n}}{x}\right)^4\right)\right)$$
忽略掉$O\left(\left(\frac{\sqrt{n}}{x}\right)^4\right)$并两边积分,即有
$$P \approx \left.\left(-\left(x+\frac{1}{x}\right)^{n-1}e^{-x^2/2}\right)\right\vert_{a}^\infty = \left(a+\frac{1}{a}\right)^{n-1}e^{-a^2/2}$$
}
\ech
\end{frame}

\begin{frame}
\chtitle{尾区间积分总结}
\bch
{\blue
$$ \int_a^\infty x^n e^{-x^2/2}dx \approx  \left(a+\frac{1}{a}\right)^{n-1}e^{-a^2/2}, \ \ a\gg \sqrt{n}$$
}
虽然我们是在$a\gg \sqrt{n}$的情况下推导的,实际上在{\blue $n$不大时该近似公式对$a>2$都是不错的近似}。
\ech
\end{frame}


\begin{frame}
\chtitle{麦克斯韦分布的计算技能}
\bch
\bmini{0.35}
\addfig{1}{newtech.jpg}
\emini
\bmini{0.61}
\tbox{\blue 无量纲速度:去掉讨厌的系数}
\tbox{\blue 无穷区间积分: 递归公式}
\tbox{\blue 小区间积分:当成常数}
\tbox{\blue 尾区间积分:小窍门}
\tbox{\blue 大区间积分:神奇的近似公式}
\emini
\ech
\end{frame}

\begin{frame}
\chtitle{大区间积分:神奇的近似公式}
\bch
{\small
如果既不是小区间,又不是尾区间的积分问题,我们就不得不祭出最后一招必杀技:
{\blue
对满足标准正态分布的变量$x$,
$$P(|x|<a) = \frac{1}{\sqrt{2\pi}} \int_{-a}^a e^{-\frac{x^2}{2}}d x \approx  \sqrt{1-e^{-\frac{a^2}{2}\left(\frac{\frac{4}{\pi}+ 0.074a^2}{1+0.074a^2}\right)}}$$}
最后一步只是一个好用的近似公式,在{\blue $a\lesssim 10$的范围内都很好用}。当$a$很大时我们要用前面介绍的尾区间积分的计算方法。
}
\ech
\end{frame}

\begin{frame}
\chtitle{置信区间和置信度}
\bch
{\scriptsize
对满足标准正态分布的变量$x$,易由前面介绍的方法算出
\bea
P(|x|<1) &=& 0.683 \newl
P(|x|<2) &=& 0.954 \newl
P(|x|<3) &=& 0.997 \newl
P(|x|<4) &=& 0.99994 \newl
P(|x|<5) &=& 0.9999994 
\eea
在科学研究中,常常使用置信区间和置信度。在论文中常可以看到"68.3\% confidence level" , "95.4\% confidence level" 等术语。粒子物理实验往往要求达到$5\sigma$精度,就是指可信度$99.99994\%$。
}
\ech
\end{frame}

\begin{frame}
\chtitle{高斯类积分技巧总结}
\bch

\bitem
\item{需要记住高斯积分$$\int_{-\infty}^\infty e^{-ax^2} dx= \sqrt{\frac{\pi}{a}}$$或者记住标准正态分布$$\frac{1}{\sqrt{2\pi}}e^{-x^2/2}$$(其归一化因子隐含了高斯积分的结果)。}
\item{如果你选择记住高斯积分,则需要掌握分部积分和交换积分和求导次序这两种常用技巧。如果你选择记住标准正态分布,则需要掌握$\overline{|x|^n}$的递归计算法。}
\item{尾区间积分和大区间积分属于“超纲”内容,如果觉得太难学不会,可在计算时保留积分表达式而不给出数值结果。}
\eitem

\ech
\end{frame}

\begin{frame}
\bch
{\Large 麦克斯韦分布应用举例}
\ech
\end{frame}


\begin{frame}
\chtitle{Maxwell分布例1 (一维Maxwell分布)}
\bch
{\blue 教材习题2-8}

{\small
\bitem
\item[(1)]{由$\overline{u_x^2} = 1$ 得到$\overline{\upsilon_x^2} = \upsilon_c^2$,即一维的$\upsilon_{\rm rms} = \upsilon_c = \sqrt{\frac{kT}{m}}$}
\item[(2)]{由$\overline{|u_x|} = \sqrt{\frac{2}{\pi}}$  得到一维的$\bar{\upsilon} =  \sqrt{\frac{2}{\pi}}\upsilon_c = \sqrt{\frac{2kT}{\pi m}}$}
\item[(3)]{显然一维情况$\upsilon_{\max} = 0$}
\eitem

}
\ech
\end{frame}


\begin{frame}
\chtitle{Maxwell分布例2 (二维Maxwell分布)}
\bch
{\blue 教材习题2-7}

{\scriptsize
\bitem
\item[(1)]{由$\overline{u_x^2} = \overline{u_y^2} = 1$ 得到二维的$\overline{u^2} = 2$,即二维的$u_{\rm rms} =  \sqrt{2}$,回到普通单位制即$\upsilon_{\rm rms} = \sqrt{\frac{2kT}{m}}$。}
\item[(2)]{二维的无量纲速率的概率密度为 $\tilde{F}_M(u) = 2\pi u \left(\frac{1}{2\pi} e^{-u^2/2}\right)= u e^{-u^2/2}$,所以平均无量纲速率:
$$\bar{u} = \int_0^\infty u^2 e^{-u^2/2} du = \sqrt{\frac{\pi}{2}}\left(\sqrt{\frac{2}{\pi}}\int_0^\infty u^2 e^{-u^2/2} du\right)$$ 
括号内的积分在数学上等于标准正态分布变量$x$的$|x|^2$的平均,故按递归公式等于$(2-1)\overline{|x|^0} = 1$。所以$\bar{u} = \sqrt{\pi/2}$,回到普通单位制,即
$$\bar{\upsilon} = \sqrt{\frac{\pi kT}{2m}}$$
}
\item[(3)]{对$\tilde{F}_M(u)$求导并令其为零,得到$u_{\max} = 1$。 回到普通单位制$\upsilon_{\max} =  \sqrt{\frac{kT}{m}}$}
\eitem

}
\ech
\end{frame}


\begin{frame}
\chtitle{Maxwell分布例3 (速度的函数的平均值)}
\bch
{\blue 教材习题2-3}

{\scriptsize

先用无量纲速率进行计算:
\bea
\overline{\left(\frac{1}{u}\right)} &=& \int_0^\infty \frac{1}{u}\tilde{F}_M(u) du \newl
&=& \sqrt{\frac{2}{\pi}} \int_0^\infty \frac{1}{u}  u^2e^{-u^2/2} du \newl
&=& \sqrt{\frac{2}{\pi}} \int_0^\infty   u e^{-u^2/2} du 
\eea
这在数学上等于标准正态分布变量$x$的 $|x|$的平均,故等于 $\sqrt{\frac{2}{\pi}}$。再换回到普通单位制:
$$\overline{\left(\frac{1}{\upsilon}\right)} = \sqrt{\frac{2 m}{\pi kT}}$$

又由 $\bar{\upsilon} = \sqrt{\frac{8kT}{\pi m}}$可以得到$\overline{\left(\frac{1}{\upsilon}\right)} >\frac{1}{\bar{\upsilon}} $。
 
当然,我们至少有四种方法可以证明这个不等式是普遍成立的。
}
\ech
\end{frame}



\begin{frame}
\chtitle{Maxwell分布例4 (小区间求概率的近似方法)}
\bch
{\blue 教材习题2-2}

\skipline

{\scriptsize
设$u_{\max}$使$\tilde{F}_M(u)$最大, 由$\frac{d \tilde{F}_M(u)}{du}\vert_{u=u_{\max}} = 0$解出$u_{\max} = \sqrt{2}$。回到普通单位制$\upsilon_{\max} = \sqrt{\frac{2kT}{m}}$ 

\bitem
\item[(1)]{由于速度变化范围小,可以近似认为概率密度在该范围内不变,计算出概率为
$$ \left.\tilde{F}_M(u) \delta u\right\vert_{u = \sqrt{2}} = \left.(0.02 u) \tilde{F}_M(u)\right\vert_{u = \sqrt{2}} =\left.0.02\sqrt{\frac{2}{\pi}}u^3 e^{-u^2/2}\right\vert_{u = \sqrt{2}} = 0.0166$$
}
\item[(2)]{$$\left.\tilde{f}_{\rm 1D}(u_x) \delta u_x\right\vert_{u_x = \sqrt{2}}= \left. (0.02 u_x) \tilde{f}_{\rm 1D}(u_x) \right\vert_{u_x = \sqrt{2}} =\left. \frac{0.02}{\sqrt{2\pi}} u_x e^{-u_x^2/2} \right\vert_{u_x = \sqrt{2}} = 0.00415$$
}
\item[(3)]{由于三个速度分量的分布互相独立,故概率为$(0.00415)^3 = 7.15\times 10^{-8}$.
}
\eitem
}

\ech
\end{frame}


\begin{frame}
\chtitle{Maxwell分布例5 (尾积分)}
\bch
{\blue 教材习题2-15}

{\scriptsize

$\upsilon>10\upsilon_{\max}$等价于$u>10 u_{\max} = 10\sqrt{2}$,所以概率为
\bea
P &=& \int_{10\sqrt{2}}^\infty \tilde{F}_M(u)du \newl
&=& \sqrt{\frac{2}{\pi}}\int_{10\sqrt{2}}^\infty u^2e^{-u^2/2} du \newl
&\approx & \sqrt{\frac{2}{\pi}} \left(10\sqrt{2}+\frac{1}{10\sqrt{2}}\right)e^{-(10\sqrt{2})^2/2} \newl
&=& 4.219 \times 10^{-43}
\eea
(计算机给出的精确数值解为$4.206\times 10^{-43}$,尾积分近似方法的相对误差仅为千分之三)
}
\ech
\end{frame}


\begin{frame}
\chtitle{Maxwell分布例6 (大区间求概率)}
\bch
{\blue 教材习题 2-14}

{\scriptsize
条件$\upsilon> \upsilon_{\max}$等同于 $u>\sqrt{2}$。其概率为
\bea
\int_{\sqrt{2}}^\infty \tilde{F}_M(u) du &=& \sqrt{\frac{2}{\pi}} \int_{\sqrt{2}}^\infty u^2e^{-\frac{u^2}{2}} du \newl
 &=& -\left.\sqrt{\frac{2}{\pi}} u e^{-\frac{u^2}{2}}\right\vert_{\sqrt{2}}^\infty +\sqrt{\frac{2}{\pi}} \int_{\sqrt{2}}^\infty e^{-\frac{u^2}{2}} du \newl
 &=& \frac{2}{e\sqrt{\pi}} + 1 - \sqrt{\frac{1}{2\pi}} \int_{-\sqrt{2}}^{\sqrt{2}} e^{-\frac{u^2}{2}} du \newl
&\approx & \frac{2}{e\sqrt{\pi}} + 1- \sqrt{1-e^{-\left(\frac{\frac{4}{\pi}+ 0.148}{1+0.148}\right)}} \newl
&=& 0.572\newl
\eea
同样的方法可以求出$\upsilon>2\upsilon_{\max}$的概率为 $0.046$。

用尾积分的方法求解则分别得到$0.62$以及$0.047$,可见当对结果精度要求不高时,尾积分近似甚至可以适用于$a\sim 1$的情形。
}
\ech
\end{frame}


\begin{frame}
\chtitle{Maxwell分布例7 (大区间求概率,迭代法求解)}
\bch
{\blue 教材思考题2-10}

{\scriptsize
设$\upsilon_0 = u_0 \sqrt{\frac{kT}{m}}$,则用和上例同样的方法可以算出

\bea
\frac{1}{2} &=& \int_{0}^{u_0} \tilde{F}_M(u) du \newl
&\approx & -\sqrt{\frac{2}{\pi}} u_0 e^{-u_0^2/2} +  \sqrt{1-e^{-\frac{u_0^2}{2}\left(\frac{\frac{4}{\pi}+ 0.074 u_0^2}{1+ 0.074 u_0^2}\right)}}  \newl
&\equiv& G(u_0)
\eea
把$u_0=1$作为零级近似,迭代计算出

一级近似$u_0 = 1 + \frac{0.5-G(1)}{\tilde{F}_M(1)} = 1.623 $

二级近似$u_0 = 1.623+\frac{0.5-G(1.623)}{\tilde{F}_M(1.623)} = 1.537$

三级近似$u_0 = 1.537+\frac{0.5-G(1.537)}{\tilde{F}_M(1.537)} = 1.538$


从而$\upsilon_0 = 1.538\sqrt{\frac{kT}{m}}$

显然,$\upsilon_0$和方均根速率$1.73\sqrt{\frac{kT}{m}}$,平均速率$1.60\sqrt{\frac{kT}{m}}$,泻流速率$0.40\sqrt{\frac{kT}{m}}$都不相同。
}
\ech
\end{frame}

\begin{frame}
\chtitle{方均根速率例1}
\bch
{\blue 教材习题2-24}

\skipline

{\small
温度$T = 273.15\SIK$。

如果认为灰尘的自由度为2(仅在水面运动),则方均根速率为
$$\sqrt{\frac{2kT}{m}} = 2.7\times 10^{-5} \SIm/\SIs$$  
}

\ech
\end{frame}


\begin{frame}
\chtitle{方均根速率例2}
\bch
{\blue 教材习题2-24}

\skipline

{\small
温度$T = 300.15\SIK$。

浮游微粒的的自由度为3,则方均根速率为
$$\sqrt{\frac{3kT}{m}} = 3.5\times 10^{-4} \SIm/\SIs$$  
}

\ech
\end{frame}


\begin{frame}
\chtitle{泻流法提纯}
\bch
{\blue 教材习题2-5}

\skipline

{\scriptsize
因泻流分子数正比于$n\overline{\upsilon^+_x} \propto \frac{n}{\sqrt{m}}$每次泻流之后$U^{235}$和$U^{238}$分子数密度之比提高了
$$\sqrt{\frac{m_{U^{238}F_6}}{m_{U^{238}F_6}}} = \sqrt{\frac{238+19\times 6}{235+19\times 6}} = 1.0042888$$
倍。
所以需要提纯
$$ n = \frac{\ln \frac{\frac{0.99}{0.01}}{\frac{0.007}{0.993}}}{\ln 1.0042888} = 2232$$
次  
}
\ech
\end{frame}


\begin{frame}
\chtitle{绝热箱泻压}
\bch
{\blue \small 很大的真空室内有一容积为$V=1\SIm^3$的绝热容器,内装有热平衡的单原子理想气体。在容器壁上开一个面积为$S=10^{-6}\SIm^2$的小孔,经过$\Delta t = 10\SIs$后把小孔堵上,发现容器内气体压强降低了$0.1\%$。因漏气的过程缓慢,可近似认为整个过程中容器内气体一直处于热平衡。试估算容器内气体的平均速率。}

\skipline

{\small

理想气体内能为$\nu C_V^{\rm mol} T = \frac{C_V^{\rm mol}}{R} pV$,故压强的降低百分比即内能的降低百分比。内能降低百分比可以用单原子理想气体的泻能速率进行计算:
$$\frac{1}{3}\overline{\upsilon} \frac{S\Delta t}{V} = 10^{-3}$$

从而解出$\overline{\upsilon} = 300\SIm/\SIs$。
}
\ech
\end{frame}


\begin{frame}
\chtitle{思考题}
\bch
\addfig{1}{think2.jpg}

很大的真空室内有一装有温度为$300\SIK$的单原子理想气体的绝热容器,现在容器上打一个很小的孔,经过一小时容器内气体温度变为$270\SIK$。问:再经过一小时容器内气体的温度为多少?
\ech
\end{frame}



\begin{frame}
\chtitle{思考题}
\bch
\addfig{1}{think3.jpg}

很大的真空室内有体积为$1\SIm^3$的绝热容器,容器内装有分子平均速率为$400\SIm/\SIs$的热平衡单原子理想气体,现在容器上打一个面积为$10^{-6}\SIm^2$的孔,问:经过一小时容器内气体分子平均速率变为多少?
\ech
\end{frame}


\section{MB Distribution}

\begin{frame}
\chtitle{压强随高度变化}
\bch
{\blue 教材习题2-18}

\skipline

{\scriptsize
由“万能法则”(或按2,3班的叫法:玻尔兹曼分布),分子数密度正比于$e^{-\frac{mgh}{kT}}$,又由理想气体状态方程,压强正比于分子数密度。故
$$ e^{-\frac{mgh}{kT}} = \frac{0.8\atm}{1\atm} = 0.8$$
空气分子平均质量
$$m = \frac{29\SIg}{6.02\times 10^{23}} = 4.82\times 10^{-26}\SIkg$$
即
$$h = -\frac{kT\ln 0.8}{mg} = 1.96\times 10^3 \SIm$$ 
}
\ech
\end{frame}


\begin{frame}
\chtitle{离心力和压强}
\bch
{\blue 半径为$1\SIm$,温度为$300\SIK$的圆柱形恒温容器里装有氮气。容器绕中心轴以每秒10圈的速度旋转,求中心轴附近氮气压强和容器壁附近氮气压强之比。}


\skipline

{\small
分子数密度正比于$e^{\frac{mr^2\omega^2}{2kT}}$,又由理想气体状态方程,压强正比于分子数密度。故中心轴和半径$r$处的压强比为
\be
e^{-\frac{mr^2\omega^2}{2kT}} = e^{-\frac{28\times 1.66\times 10^{-27}\times 1^2\times (20\pi)^2}{2\times 1.38\times 10^{-23}\times 300}} = 0.978
\ee

}
\ech
\end{frame}


\section{Heat Capacity}


\begin{frame}
\chtitle{气体的定压比热容}
\bch
{\blue 证明理想气体的摩尔定压比热容$C_p^{\rm mol}$和摩尔定体比热容$C_V^{\rm mol}$相差$R$}

\skipline
{\scriptsize
气体的内能$U(T)$ 满足
$$\frac{dU}{dT} = \nu C_V^{\rm mol}$$
在固定压强时,当气体温度变化$dT$,体积变化$dV = \frac{ \nu R dT}{p}$
故对气体做功为
$$-p dV = - \nu R dT$$
设气体吸收热量$dQ$,则由能量守恒,有
$$dQ - pdV = dU$$
(我们以后会叫它热力学第一定律),即
$$dQ = dU + pdV =  \nu C_V^{\rm mol}dT + \nu RdT$$
所以
$$C_p^{\rm mol} = \frac{1}{\nu} \frac{dQ}{dT} = C_V^{\rm mol} + R$$

}
\ech
\end{frame}

\begin{frame}
\chtitle{第六周习题(序号接第五周)}
\bch
{\small
\bitem
\item[16]{某$300\SIK$的恒温容器内$1\SImol$的氦气压强为$0.20000\atm$。把氦气抽掉$0.2\SImol$,压强变为$0.1584\atm$。再把氦气抽掉$0.2\SImol$,压强变为$0.1176\atm$。试估算容器体积。}
\item[17]{对温度为$T$,分子质量为$m$的热平衡理想气体
\bitem
\item[(1)]{分别计算分子速率四次方的平均$\overline{\upsilon^4}$和速率三次方的平均$\overline{\upsilon^3}$。}
\item[(2)]{证明两者之比不小于分子平均速率:
$$ \frac{\overline{\upsilon^4}}{\overline{\upsilon^3}} \ge \overline{\upsilon}$$}
\item[(3)]{证明上述不等式对任意速度分布成立。}
\eitem}
\item[18]{在温度为$300\SIK$,压强$p = 1\atm$的氧气中放一个纳米音乐盒。音乐盒有个表面积为$10^{-4}\SImm^2$的探头。当有速率超过$2792\SIm/\SIs$的氧气分子撞击探头表面时将触动音乐盒开关。问:音乐盒开关平均多久被触发一次?}
\eitem
}
\ech
\end{frame}

\end{document}

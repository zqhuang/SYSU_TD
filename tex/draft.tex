\documentclass[12pt,CJK]{article}
\usepackage{geometry}
\input{reduced_macros.tex}
\geometry{tmargin=0.3in, bmargin=0.5in, lmargin=1in, rmargin=1in, nohead, nofoot}
\def\mark#1{{\color{blue} (#1分)}}
\renewcommand{\thepage}{}
\begin{document}
\bch
%{\large 2017年中山大学物理与天文学院 热学 期中考试 }

%{\vskip 0.3in}

%姓名 ....................... {\hskip 0.5in}    学号 .......................

%{\vskip 0.1in}

\bitem


\item[(一)]{双原子分子气体的定体摩尔热容如何随温度变化?试简要解释原因。  (难度$\star\star$)}

\item[(二)]{叙述气体,液体,晶体,以及非晶体分子在短程和长程分别表现出的统计特性。非晶体和液体的区别在哪里? (难度$\star\star\star$)}

\item[(三)]{把$80\cdeg$的水和等质量的$0\cdeg$的冰混合,得到了$0\cdeg$的水。那么,把$90\cdeg$的水和等质量的$0\cdeg$的冰混合,得到的水温度为多少?  (难度$\star$)}

\item[(四)]{一个体积为$0.025\SIm^3$的恒温容器里的$1\SImol$气体压强为$10120 \SIPa$。抽去$\frac{1}{3}\SImol$气体后,容器内气体达到热平衡时压强变为$67200\SIPa$。再抽去$\frac{1}{3}\SImol$气体,容器内气体达到热平衡时压强变为$33467\SIPa$。试计算恒温容器的温度。  (难度$\star\star\star$)
}

\item[(五)]{ 考虑温度为$T$,分子质量为$m$的热平衡理想气体。记分子速度为$(\upsilon_x, \upsilon_y, \upsilon_z)$,速率为$\upsilon = \sqrt{\upsilon_x^2+\upsilon_y^2+\upsilon_z^2}$。对气体中随机选取的分子,求下述事件的概率:
    \bitem
	\item[(1)] { $\upsilon_x > \upsilon_y>\upsilon_z $  (难度$\star\star$)}
	\item[(2)] { $\upsilon_x > \frac{1}{2} \upsilon $  (难度$\star\star\star$)}
	\item[(3)] { $ 0.99\sqrt{\frac{kT}{m}}<\upsilon_x < 1.01 \sqrt{\frac{kT}{m}} $  (难度$\star\star$)}
	\item[(4)] { $ 0.99\sqrt{\frac{kT}{m}}<\upsilon < 1.01 \sqrt{\frac{kT}{m}} $  (难度$\star\star$)}
    \eitem
}

\item[(六)]{ 考虑温度为$T$,分子质量为$m$的热平衡理想气体。
    \bitem
	\item[(1)] {求其所有分子的方均根速率$\upsilon_{\rm rms}$。  (难度$\star$)}
	\item[(2)] {写出所有速率超过$3\upsilon_{\rm rms}$的分子的平均速率的积分表达式。  (难度$\star\star$)}
	\item[(3)] {由于分子的速率分布是指数衰减的,所以大多数速率大于$3\upsilon_{\rm rms}$的分子的速率都会集中在$3\upsilon_{\rm rms}$附近。我们近似认为速率大于$3\upsilon_{\rm rms}$的分子的平均速率为$3\upsilon_{\rm rms}$。由此估算分子速率大于$3\upsilon_{\rm rms}$的概率。  (难度$\star\star\star\star$)}
    \eitem
}

\item[(七)]{把$1\SImol$氢气和$1\SImol$氮气混合置于体积为$0.5\SIm^3$,温度为$T = 300\SIK$的恒温箱内达到热平衡。
\bitem
\item[(1)]{求混合气体的压强。(难度$\star\star$)}
\item[(2)]{求混合气体的分子方均根速率$\upsilon_{\rm rms}$。 (难度$\star\star\star$)}
\eitem
}

\item[(八)]{置于很大的真空室内的绝热容器里装有稀薄氦气。在容器壁上开一个小孔,经过一段时间后把小孔堵上,发现容器内氦气压强降低了$0.4\%$,问容器内氦气的分子数减少了百分之多少?漏气的过程很缓慢,可以近似认为整个过程中容器内氦气一直处于热平衡。(难度$\star\star\star\star\star$)}

\item[(九)]{假设内径为$5\SIm$的封闭球形飞船绕中子星做每$2$秒一周的匀速圆周运动。飞船的质心在球心,且自转和公转同步(即保持同一面对着中子星)。飞船内有温度为$285\SIK$的氧气。问:氧气的压强是均匀的吗?如果不均匀,最小压强和最大压强之比为多少? (难度$\star\star\star\star$)}

\item[(十)]{一个装有某双原子气体(分子式为$X_2$)的圆柱形恒温容器绕中心轴以每秒$30$转的速度旋转。已知容器的半径为$0.4\SIm$,温度为$300\SIK$。等容器内气体达到热平衡后,测得容器中心轴附近的气体压强为容器壁附近的气体压强的$96.4\%$。问$X$是什么元素? (难度$\star\star\star$)}
\eitem


\ech
\end{document}

\documentclass[12pt,CJK]{article}
\usepackage{geometry}
\input{reduced_macros.tex}
\geometry{tmargin=0.3in, bmargin=0.5in, lmargin=0.7in, rmargin=0.7in, nohead, nofoot}
\def\mark#1{{\color{blue} (#1分)}}
\renewcommand{\thepage}{}
\begin{document}
\bch
{\large 热学课堂练习 II A卷 (总分$\times 0.03$换算成平时分);}

%{\vskip 0.3in}

姓名 ....................... {\hskip 0.5in}    学号 .......................{\hskip 0.5in}  分数 ...................

%{\vskip 0.1in}

\bitem
\item[(一)]{判断题,请用$\checkmark$表示正确,$\times$表示错误,每题5分。

  \bitem
\item[(1)]{定义温标必须用到固定标准点,测温物质和测温属性,缺一不可。\bropt}
\item[(2)]{热力学第一定律对不可逆过程也适用。\bropt}
\item[(3)]{对任何平衡态物质均有$dU = C_VdT$,其中$U$为内能,$T$为温度,$C_V$为定体热容。\bropt}
\item[(4)]{液体的特点是分子排列短程无序,长程有序。\bropt}
\item[(3)]{热力学第二定律的微分表示$\dbar Q \le TdS$对偏离平衡态很远的不可逆过程也适用。\bropt}  

  \eitem
}

\item[(二)]{选择题,每题5分。

  \bitem

\item[(1)]{按照能均分定理,每一个 \bropt 贡献$\frac{1}{2}kT$的分子平均能量
  
  \optlist{自由度}{动量分量或坐标分量的二次型能量}{任意变量的二次型能量}

}    
\item[(2)]{当氮气温度和压强为下列那组数值时,氮气的节流效应是致冷的? \bropt
  
  \optlist{$T=3000\SIK, p = 1\atm $ }{$T=10\SIK, p = 100\atm$}{$T = 300\SIK, p = 0.5\atm $}

}


\item[(3)]{已知气体分子的速率分布函数为$F(\upsilon)$,下列哪个表达式是速率不超过$\upsilon_0$的所有分子的平均速率?
  \bropt

  \optlist{$\upsilon_0 \int_0^{\upsilon_0}  F(\upsilon)d\upsilon$}{$\int_0^{\upsilon_0}\upsilon F(\upsilon)d\upsilon$}{$\frac{\int_0^{\upsilon_0}\upsilon F(\upsilon)d\upsilon}{\int_0^{\upsilon_0} F(\upsilon)d\upsilon}$}
}
  

  \item[(4)]{
  在大气中进行的化学反应过程中的吸热量等于生成物和反应物的 \bropt

  \optlist{内能差}{焓差}{自由焓差}

}

  \item[(5)]{
  日常环境下的金属具有良好的抗压性,是因为金属

  \optlist{电子的简并压很大}{原子的排列非常整齐}{原子核之间的静电排斥力很强}
}
  \eitem
}

\item[(三)]{某干燥地区的地面大气的压强为$1\atm$,温度为$300\SIK$。在该地区的一处山顶,水的沸点为$90\cdeg$。已知水的汽化热为$4.1\times 10^4\SIJ/\SImol$。试估算山顶处的:(1) 大气压强; (10分) (2) 大气温度。(10分)
    
       \vspace{3.2in}
  }
  
\item[(四)]{某种气体的状态方程为$\left(p+\frac{a\nu^2}{V^3}\right)V = \nu R T$,其中$\nu$为摩尔数,$p$为压强,$V$为体积,$T$为热力学温度,$a>0$为固定常量。
    \bitem
  \item[(1)]{利用内能和状态方程的关系计算$\pfrac UVT$ (表示为$\nu$, $V$, $T$的函数); (5分)}
  \item[(2)]{$1\SImol$该气体经过等温过程体积增大了$10\%$。它的内能增大还是减少了?(5分) 熵变化了多少? (5分)}
    \eitem
    \vspace{3.5in}
  }
  

  \item[(五)]{某可逆理想气体热机按下述循环工作:
      \bitem
    \item{以多方指数$n=\frac{5}{3}$的准静态多方过程从$423\SIK$升温到$777\SIK$;}
    \item{准静态绝热膨胀,降温到$200\SIK$;}
    \item{以$200\SIK$的温度准静态等温压缩;}
    \item{准静态绝热压缩,回到初始状态。}      
      \eitem
      在循环过程的温度范围内,该理想气体的定体摩尔热容随温度变化规律为
      $$C_V^{\rm mol} = \left(\frac{3}{2} + \frac{T}{T_0}\right)R,$$
      其中$T_0$为某固定常量。选择你喜欢的变量为坐标画出该循环的大致示意图(5分), 并求该热机的效率 (10分)。
  }  

\eitem


\ech
\end{document}

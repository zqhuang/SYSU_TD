\documentclass[CJK]{beamer}
\usepackage{CJKutf8}
\usepackage{beamerthemesplit}
\usetheme{Malmoe}
\useoutertheme[footline=authortitle]{miniframes}
\usepackage{amsmath}
\usepackage{amssymb}
\usepackage{graphicx}
\usepackage{eufrak}
\usepackage{color}
\usepackage{slashed}
\usepackage{simplewick}
\usepackage{tikz}
\graphicspath{{../figures/}}
\def\addfig#1#2{\begin{center}\includegraphics[width=#1 in]{#2}\end{center}}
\def\blacktext#1{{\color{black}#1}}
\def\bluetext#1{{\color{blue}#1}}
\def\redtext#1{{\color{red}#1}}
\def\darkbluetext#1{{\color[rgb]{0,0.2,0.6}#1}}
\def\skybluetext#1{{\color[rgb]{0.2,0.7,1.}#1}}
\def\cyantext#1{{\color[rgb]{0.,0.5,0.5}#1}}
\def\greentext#1{{\color[rgb]{0,0.7,0.1}#1}}
\def\darkgray{\color[rgb]{0.2,0.2,0.2}}
\def\lightgray{\color[rgb]{0.6,0.6,0.6}}
\def\gray{\color[rgb]{0.4,0.4,0.4}}
\def\blue{\color{blue}}
\def\red{\color{red}}
\def\green{\color{green}}
\def\darkblue{\color[rgb]{0,0.2,0.6}}
\def\skyblue{\color[rgb]{0.2,0.7,1.}}
\def\fdeg{{^\circ \mathrm{F}}}
\def\cdeg{^\circ \mathrm{C}}
\def\be{\begin{equation}}
\def\ee{\nonumber\end{equation}}
\def\bea{\begin{eqnarray}}
\def\eea{\nonumber\end{eqnarray}}
\def\ii{{\dot{\imath}}}
\def\bch{\begin{CJK}{UTF8}{gbsn}}
\def\ech{\end{CJK}}
\def\bitem{\begin{itemize}}
\def\eitem{\end{itemize}}
\def\bcenter{\begin{center}}
\def\ecenter{\end{center}}
\def\bex{\begin{minipage}{0.3\textwidth}\includegraphics[width=1in]{jugelizi.png}\end{minipage}\begin{minipage}{0.6\textwidth}}
\def\eex{\end{minipage}}
\def\chtitle#1{\frametitle{\bch#1\ech}}
\def\skipline{{\vskip0.1in}}
\def\skiplines{{\vskip0.2in}}
\def\lagr{{\mathcal{L}}}
\def\hamil{{\mathcal{H}}}
\def\vecv{{\mathbf{v}}}
\def\vecx{{\mathbf{x}}}
\def\vecy{{\mathbf{y}}}
\def\veck{{\mathbf{k}}}
\def\vecp{{\mathbf{p}}}
\def\vecn{{\mathbf{n}}}
\def\vecA{{\mathbf{A}}}
\def\vecP{{\mathbf{P}}}
\def\vecsigma{{\mathbf{\sigma}}}
\def\hatJn{{\hat{J_\vecn}}}
\def\hatJx{{\hat{J_x}}}
\def\hatJy{{\hat{J_y}}}
\def\hatJz{{\hat{J_z}}}
\def\hatj#1{\hat{J_{#1}}}
\def\hatphi{{\hat{\phi}}}
\def\hatq{{\hat{q}}}
\def\hatpi{{\hat{\pi}}}
\def\vel{\upsilon}
\def\Dint{{\mathcal{D}}}
\def\adag{{\hat{a}^\dagger}}
\def\bdag{{\hat{b}^\dagger}}
\def\cdag{{\hat{c}^\dagger}}
\def\ddag{{\hat{d}^\dagger}}
\def\hata{{\hat{a}}}
\def\hatb{{\hat{b}}}
\def\hatc{{\hat{c}}}
\def\hatd{{\hat{d}}}
\def\hatN{{\hat{N}}}
\def\hatH{{\hat{H}}}
\def\hatp{{\hat{p}}}
\def\Fup{{F^{\mu\nu}}}
\def\Fdown{{F_{\mu\nu}}}
\def\newl{\nonumber \\}
\def\SIkm{\,\mathrm{km}}
\def\SIyr{\,\mathrm{yr}}
\def\SIGyr{\,\mathrm{Gyr}}
\def\SIeV{\,\mathrm{eV}}
\def\SIkeV{\,\mathrm{keV}}
\def\SIMeV{\,\mathrm{MeV}}
\def\SIGeV{\,\mathrm{GeV}}
\def\SIcal{\,\mathrm{cal}}
\def\SIkcal{\,\mathrm{kcal}}
\def\SImol{\,\mathrm{mol}}
\def\SIm{\,\mathrm{m}}
\def\SIcm{\,\mathrm{cm}}
\def\SIfm{\,\mathrm{fm}}
\def\SImm{\,\mathrm{mm}}
\def\SInm{\,\mathrm{nm}}
\def\SImum{\,\mathrm{\mu m}}
\def\SIJ{\,\mathrm{J}}
\def\SIkJ{\,\mathrm{kJ}}
\def\SIs{\,\mathrm{s}}
\def\SIkg{\,\mathrm{kg}}
\def\SIg{\,\mathrm{g}}
\def\SIK{\,\mathrm{K}}
\def\SImmHg{\,\mathrm{mmHg}}
\def\SIPa{\,\mathrm{Pa}}
\def\vece{\mathrm{e}}
\def\bmat#1{\left(\begin{array}{#1}}
\def\emat{\end{array}\right)}
\def\bcase#1{\left\{\begin{array}{#1}}
\def\ecase{\end{array}\right.}
\def\calM{{\mathcal{M}}}
\def\calT{{\mathcal{T}}}
\def\calR{{\mathcal{R}}}
\def\barpsi{\bar{\psi}}
\def\baru{\bar{u}}
\def\barv{\bar{\upsilon}}
\def\bmini#1{\begin{minipage}{#1\textwidth}}
\def\emini{\end{minipage}}
\def\qeq{\stackrel{?}{=}}
\def\torder#1{\mathcal{T}\left(#1\right)}
\def\rorder#1{\mathcal{R}\left(#1\right)}
\def\contr#1#2{\contraction{}{#1}{}{#2}#1#2}
\def\trof#1{\mathrm{Tr}\left(#1\right)}
\def\trace{\mathrm{Tr}}
\def\comm#1{\ \ \ \left(\mathrm{used}\ #1\right)}
\def\tcomm#1{\ \ \ (\text{#1})}
\def\slp{\slashed{p}}
\def\slk{\slashed{k}}
\def\wulian{\includegraphics[width=0.18in]{emoji_wulian.jpg}}
\def\bye{\includegraphics[width=0.18in]{emoji_bye.jpg}}
\def\calp{{\mathfrak{p}}}
\def\veccalp{\mathbf{\mathfrak{p}}}
\def\atm{\,\mathrm{atm}}
\def\angstrom{\,\text{\AA}}
\def\Tthree{T_{\tiny \textcircled{3}}}
\def\pthree{p_{\tiny \textcircled{3}}}

\def\courseurl{http://zhiqihuang.top}

\def\tpage#1#2{
\begin{frame}
\bch
\begin{center}
\begin{large}
热学 \\
第#1讲 #2

\end{large}

\skiplines

黄志琦


\end{center}

\skiplines

{\small 
教材:《热学》第二版,赵凯华,罗蔚茵,高等教育出版社


课件下载
}
\courseurl 
\ech
\end{frame}
}

\def\bfr#1{
\begin{frame}
\chtitle{#1} 
\bch
}

\def\efr{
\ech 
\end{frame}
}

\title{Lesson 07 - 1st Law of Thermaldynamics}
  \author{}
  \date{}
\begin{document}
\tpage{7}{热力学第一定律}


\section{Review}

\begin{frame}
\chtitle{上讲内容回顾}
\bch
\bitem
\item{我们根本没有学会第一二章}
\item{我们根本不知道期中考什么}
\eitem
\ech
\end{frame}

\begin{frame}
\chtitle{不信的话来检测一下}
\bch
{ 已知无量纲速度的概率密度函数为
$$f(u_x,u_y,u_z) = Cu_x^2 e^{-u}$$
其中$C$待定的归一化常数,$u = \sqrt{u_x^2+u_y^2+u_z^2}$为速率。
\bitem
\item{求$u$的概率密度函数}
\item{求平均速率$\bar{u}$}
\item{求方均根速率$u_{\rm rms}$}
\item{求最该然速率$u_{\max}$}
\item{求$x$方向的泻流速率$u_{n, x, \rm leak}$}
\item{求$z$方向的泻流速率$u_{n, z, \rm leak}$}
\eitem
}
\ech
\end{frame}


\begin{frame}
\chtitle{本讲内容}
\bch
\bitem
\item{内能}
\item{热力学第一定律}
\item{理想气体的各种准静态过程}
\item{传热的本质和热量的显式方程}
\eitem
\ech
\end{frame}


\section{Internal Energy}

\begin{frame}
\chtitle{内能是个态函数}
\bch
\bitem
\item{\bf 态函数里的“态”:指热平衡态。}
\item{\bf 态函数定义:态函数由系统的宏观状态参量确定,和如何达到这个状态的过程无关。}
\eitem

\bex
固定摩尔数的气体的内能$U$由体积$V$和温度$T$决定,
$$ U  = U(V, T)$$
当然,$U$也可以看成$p, T$的函数或者$p, V$的函数。
\eex

\ech
\end{frame}

\begin{frame}
\chtitle{50\%送分几率}
\bch
下面的量是不是态函数
\bitem
\item{温度$T$}
\item{热量$Q$}
\item{压强$p$}
\item{体积$V$}
\item{做功$A$}
\item{摩尔数$\nu$}
\item{平衡态的分子平均速率$\overline{\upsilon}$}
\eitem
\ech
\end{frame}

\begin{frame}
\chtitle{理想气体的内能只跟温度有关}
\bch

理想气体的定体热容$C_V$只和温度有关,又在经典图像下$U(T\rightarrow 0) = 0$,所以
$$ U(V, T) = U(V, 0) + \int_0^T \nu C_V(T') dT' =\nu  \int_0^T C_V(T') dT'  $$

可见,固定摩尔数的理想气体的内能只是温度的函数。

\skiplines

另一种理解方式:理想气体每个分子的平均能量(由能均分定理)被温度唯一决定,而分子之间相互作用的势能在理想气体模型中被认为是零,故内能只是温度的函数。

\ech
\end{frame}



\begin{frame}
\chtitle{思考题}
\bch
\addfig{1}{songfen.jpg}

考虑实际气体分子之间有微弱的吸引力,实际气体的内能$U(V,T)$对体积有微弱的依赖。问:$\left(\frac{\partial U}{\partial V}\right)_T$一般是正的还是负的?

\skiplines

注:偏微分$\pfrac ABC$表示保持$C$不变时,$A$和$B$的小变化量之比。显然,这样写的前提是默认了$A,B,C$受某个状态方程约束。
\ech
\end{frame}


\begin{frame}
\chtitle{思考题}
\bch
\addfig{1}{songfen2.jpg}

由于$p, V, T$中只有两个是独立的,我们也可以把内能写成$U(p, T)$。那么实际气体的$\pfrac UpT$的符号一般是什么呢?

\ech
\end{frame}


\section{1st Law}

\begin{frame}
\chtitle{热力学第一定律:先约定下符号}
\bch

\bmini{0.5}
\bitem
\item{环境给系统传热$Q$}
\item{系统给环境传热$Q'$}
\item{环境对系统做功$A$}
\item{系统对环境做功$A'$}
\eitem
\emini
\bmini{0.46}
\addfig{0.8}{njjzw.png}
从左边的例子里找规律,什么时候用带撇的符号?
\emini
\ech
\end{frame}

\begin{frame}
\chtitle{热力学第一定律:先约定下符号}
\bch

\bmini{0.5}
\bitem
\item{系统从环境吸热$Q$}
\item{环境从系统吸热$Q'$}
\item{环境对系统做功$A$}
\item{系统对环境做功$A'$}
\eitem
\emini
\bmini{0.46}

输入能量不带撇;输出能量就带撇。

\huaixiao 你答对了吗?


\emini
\ech
\end{frame}

\begin{frame}
\chtitle{热力学第一定律:先约定下符号}
\bch

显然$A'=-A$, $Q'=-Q$,要那么多符号干什么呢?

\skiplines

\huaixiao不把你绕晕怎么体现热学的难度呢。

\ech
\end{frame}

\begin{frame}
\chtitle{经济学第一定律}
\bch
\tbox{
存款增加 = 收入 - 支出
}
\ech
\end{frame}


\begin{frame}
\chtitle{热力学第一定律}
\bch
\tbox
{\small 系统内能增加$\Delta U$ = 系统从环境吸热$Q$ $-$  系统对环境做功$A'$}

\ech
\end{frame}


\begin{frame}
\chtitle{热力学第一定律}
\bch
\tbox
{\small 系统内能增加$\Delta U$ = 环境对系统做功$A$ $-$ 系统对环境放热$Q'$}

\ech
\end{frame}

\begin{frame}
\chtitle{热力学第一定律}
\bch
\tbox
{\small 系统内能增加$\Delta U$ = 系统从环境吸热$Q$ $+$ 环境对系统做功$A$}
\ech
\end{frame}

\begin{frame}
\chtitle{热力学第一定律}
\bch

\huaixiao 热学是不是比经济学难很多?
\ech
\end{frame}


\begin{frame}
\chtitle{热力学第一定律的数学表达式}
\bch
$$ \Delta U =  Q + A $$

\skiplines

这里功$A$是{\blue 广义功},可以是机械功$-\int pdV$,也可以是电流做功$\int UIdt$等。
\ech
\end{frame}

\begin{frame}
\chtitle{准静态过程}
\bch
准静态过程的定义:{\blue 进行得足够缓慢,以至于系统连续经过的每个中间态都可以近似看成平衡态。}

\skiplines

{\small 例如:缓慢加热的过程,缓慢压缩气体的过程,政府工作人员的办公过程等等}

\addfig{2}{slowmotion.jpg}

\ech
\end{frame}


\begin{frame}
\chtitle{准静态过程的热力学第一定律表述}
\bch
准静态过程中内能在过程中间都是有定义的,所以:
{\blue
$$dU = \dbar A + \dbar Q$$
}
热学里特有的符号$\dbar$代表这个微元和过程有关。

\ech
\end{frame}


\begin{frame}
\chtitle{$p$-$V$图和做的功}
\bch
准静态过程可以用在$p$-$V$图上的一条曲线描述。气体对外界做功
$$ A' = \int p dV $$
等于$p$-$V$曲线下的面积
\addfig{2}{pdVwork.png}

\ech
\end{frame}

\section{Application}

\begin{frame}
\chtitle{理想气体等温过程}
\bch
理想气体的等温过程比较简单,利用理想气体状态方程$pV = \nu RT$即可求出对理想气体做的功
$$ A = -\int p dV = -\nu R T \int \frac{dV}{V} =- \nu R T\ln\frac{V_{\rm fin}}{V_{\rm ini}}$$
如果没有额外的自由度被激发,理想气体的内能不变。则可推算出等温过程吸收的热量为
$$ Q = - A $$
\ech
\end{frame}


\begin{frame}
\chtitle{理想气体绝热过程(adiabatic process)}
\bch
{\small
理想气体的绝热过程则稍显复杂,由$\dbar Q=0$得到
$$ dU = -p dV$$
又$$ dU = \nu C_V^{\rm mol} dT = \frac{C_V^{\rm mol}}{R}(p dV + V dp)$$
两式相减得到
$$ C_V^{\rm mol} Vdp + C_p^{\rm mol} p dV = 0$$
其中$C_p^{\rm mol} = C_V^{\rm mol} + R$是摩尔定压热容(推导见题霸集最后一题)。
记$$\gamma = \frac{C_p^{\rm mol}}{C_V^{\rm mol}}$$
对理想气体$\gamma$是个常数(对很多实际气体也是近似常数),由上述方程可推出{\blue 理想气体绝热状态方程}(又称{\bf 泊松公式})
{\blue 
$$p V^\gamma = \const$$}
}
\ech
\end{frame}


\begin{frame}
\chtitle{各种气体的$\gamma$}
\bch
{\small
对单原子理想气体,$\gamma = \frac{5}{3}$。

\skipline

对室温下的双原子理想气体,$\gamma \approx \frac{7}{5}$。

\skipline

因空气大部分都是氮气和氧气(双原子分子),所以我们可以近似认为室温下空气的$\gamma = \frac{7}{5}$。
}

由$pV^\gamma = \const $以及理想气体状态方程可推出绝热状态方程的另外两个形式:

$$ TV^{\gamma-1} = \const$$

$$T p^{\frac{1}{\gamma}-1} = \const $$
\ech
\end{frame}

\begin{frame}
\chtitle{空气中声速}
\bch
\addfig{1.5}{soundwave.jpg}

{\scriptsize
力学里无敌公式$F = ma$:

$$ -dp dy dz = (\rho dx dy dz) \left(\frac{d\upsilon}{dx/\upsilon}\right) $$
即
$$ dp = - \rho \upsilon d\upsilon $$

然后根据物质流守恒:$ \rho \upsilon = (\rho + d\rho)(\upsilon+d\upsilon)$
忽略高阶小量即$ -\rho d\upsilon = \upsilon d\rho$
代入前面的$dp$表达式得到:
$$ dp = \upsilon ^2 d\rho$$
即
$$\upsilon = \sqrt{\frac{dp}{d\rho}}$$
}
\ech
\end{frame}

\begin{frame}
\chtitle{关于空气中声速的补充知识(续)}
\bch
{\small
空气是热的不良导体,故做绝热近似 
$$p\rho^{-\gamma} = \const$$
即 $$\frac{dp}{d\rho} = \gamma \frac{p}{\rho} =  \frac{\gamma R T}{M^{\rm mol}}$$
其中空气摩尔质量$M^{\rm mol} = 0.0289 \SIkg/\SImol$
算出空气中声速为
$$c_s = \sqrt{\frac{\gamma RT}{M^{\rm mol}}} = 347 \sqrt{\frac{T}{300\SIK}} \SIm/\SIs$$
用无量纲速率来表示就是
$$u_s = \sqrt{\gamma}$$
因此声速和空气分子方均根速率之比为(见课本习题3-21)
$$\sqrt{\frac{\gamma}{3}} = \sqrt{\frac{7}{15}} = 0.683$$
}
\ech
\end{frame}

\begin{frame}
\chtitle{思考题}
\bch

\addfig{1.}{songfen.jpg}

前面关于空气中声速和空气分子的方均根速率的比的计算是有点问题的,你能指出问题在哪里吗?

\ech
\end{frame}


\begin{frame}
\chtitle{高处不胜寒}
\bch
{\small 
我们以前计算大气压强梯度时把空气温度当成了常数,事实上我们都知道“高处不胜寒”。因为大气是热的不良导体,绝热近似是更好的描述。

由力学平衡有
$$dp = -\rho g dz$$
由$Tp^{\frac{1}{\gamma}-1}= \const$,可得$dT = \left(1-\frac{1}{\gamma}\right)\frac{T}{p} dp $,
故
$$\frac{dT}{dz} = -\frac{\gamma - 1}{\gamma} \frac{T}{p}\rho g= -\frac{\gamma - 1}{\gamma} \frac{V}{\nu R}\rho g =  -\frac{\gamma - 1}{\gamma} \frac{M^{\rm mol}}{ R} g$$
取$\gamma  = 7/5$, $M^{\rm mol} = 29\SIg/\SImol$,$g = 9.8 \mathrm{N}/\SIkg$得到
$$\frac{dT}{dz} \approx -10\SIK/\SIkm$$
这个结果的数量级是正确的,但实际温度梯度往往比它小。空气里的饱和水蒸气是一个重要的影响因素(见教材152页)。
}
\ech
\end{frame}


\begin{frame}
\chtitle{多方过程(polytropic process)}
\bch
满足状态方程$pV^n = C $({\bf 多方指数}$n$为常数, $C$为常数)的过程称为{\bf 多方过程}。

多方过程对外界做功为:
{\scriptsize
\bea
A' &=& \int p dV \newl
 &=& C \int \frac{dV}{V^n} \newl
 &=& \frac{C}{n-1}\left(V_{\rm ini}^{1-n} -  V_{\rm fin}^{1-n}\right) \newl
 &=& \frac{1}{n-1}\left(p_{\rm ini}V_{\rm ini} -  p_{\rm fin}V_{\rm fin}\right) \newl
 &=& -\frac{\nu R}{n-1}\Delta T 
\eea
}
即
{\blue
$$ A' = -\frac{\nu R}{n-1} \Delta T$$
}
\ech
\end{frame}

\begin{frame}
\chtitle{理想气体多方过程的热容量}
\bch
{\small
多方过程的热容量$C_n$满足
$$dU = C_VdT = C_n dT - pdV$$
即
$$pdV + (C_V - C_n) \frac{pdV+Vdp}{\nu R} = 0$$
又由多方过程的定义可推出
$$pdV + (n-1)(pdV + Vdp) = 0$$
对比两式即得
{\blue
$$C_n = C_V - \frac{\nu R}{n-1}$$
}
这结果很好理解,$C_V$用于描述内能的增加,$- \frac{\nu R}{n-1}$用于描述气体对外做功消耗的能量。

或者写成摩尔热容
$$C_n^{\rm mol} = C_V^{\rm mol} - \frac{R}{n-1}$$
}
\ech
\end{frame}


\begin{frame}
\chtitle{多方过程的例子:等压过程}
\bch
\bex
等压过程是$n=0$的多方过程。故
做功
$$A' = \nu R\Delta T $$
热容量
$$C_p^{\rm mol} = C_V^{\rm mol}+ R $$

\skipline

我们可以通过计算内能变化来检验上面的结果:
$$ \Delta U = -A' + Q = -\nu R\Delta T +(C_V + \nu R)\Delta T =  C_V\Delta T$$

\eex
\ech
\end{frame}

\begin{frame}
\chtitle{多方过程的例子:绝热过程}
\bch
\bex
绝热过程是$n=\gamma$的多方过程。故
做功
$$A' = -\frac{\nu R}{\gamma-1}\Delta T$$
热容量
$$C = 0 $$

我们可以通过计算内能变化来检验上面的结果:
$$ \Delta U = -A' + Q = \frac{\nu R}{\gamma-1} \Delta T  =  C_V\Delta T$$

\eex
\ech
\end{frame}

\begin{frame}
\chtitle{多方过程的例子:等体过程}
\bch
\bex
等体过程是$n=\infty$的多方过程。故
做功
$$A' = 0$$
热容量
$$C = C_V $$

我们可以通过计算内能变化来检验上面的结果:
$$ \Delta U = -A' + Q =  C_V\Delta T$$

\eex
\ech
\end{frame}


\begin{frame}
\chtitle{多方过程的例子:等温过程}
\bch
\bex
等温过程是$n=1$的多方过程。这时无法直接用多方过程的做功公式。我们考虑$n=1+\epsilon$的情形,再让$\epsilon \rightarrow 0$。

做功
$$\dbar A' = -\frac{\nu R}{\epsilon}dT = \frac{\nu R}{\epsilon}\frac{T\epsilon}{V}dV = \nu RT d\ln V $$
积分即得
$$ A' = -\frac{\nu R}{\epsilon}dT = \frac{\nu R}{\epsilon}\frac{T\epsilon}{V}dV = \nu RT \Delta(\ln V) $$
热容量
$$C = \infty $$

\eex
\ech
\end{frame}


\begin{frame}
\chtitle{消化下}
\bch
\addfig{1}{songfen.jpg}
$1\SImol$氧气经过$n=2$的多方过程从$T=300K$升温到$T=310K$。问这个过程中氧气对环境做功多少?从环境吸热多少?
\ech
\end{frame}


\begin{frame}
\chtitle{传热的本质}
\bch
初中物理复习完了,下面我们开始深邃的思考

\addfig{2}{think4.jpg}

热一律里$Q$和$A$的界限在哪里?“传热”难道不是很多微观的“做功”的和吗?

\ech
\end{frame}


\begin{frame}
\chtitle{吸热的本质(续)}
\bch
我们把热一律写成“收入=存款+开支”的形式:
$$ \dbar Q =  d U + \dbar A'$$
为了简化讨论,我们只考虑非相对论的粒子数守恒系统,并假设分子自由度只依赖于温度。先考虑准静态过程。

\skipline

先把内能分为两部分。我们知道每个分子单独具有的平均能量只是温度的函数,我们把这样的能量加起来,称之为{\blue 局部内能$U_{\rm local}$}。此外,分子之间还有相互作用的势能,我们把这部分能量称为{\blue 全局内能$U_{\rm global}$}。


再把物质对外做功也分为两部分。因为分子运动造成的{\blue 动理压强$p_k$对外做功记$ A_k'$},因为分子之间相互吸引造成的{\blue 内压强$p_U$对外做功记为$ A'_U$}。显然
{\blue
$$ d U_{\rm global} = - \dbar A'_U$$
}
\ech
\end{frame}


\begin{frame}
\chtitle{吸热的本质(续)}
\bch
因为$d U_{\rm global}$和$\dbar A'_U$互相抵消,热一律就可以写成:
$$ \dbar Q = d U_{\rm local}  +\dbar A'_k  $$
它揭示了准静态过程的吸热量$Q$的本质:

\tbox{\blue 准静态过程吸热量用来提高系统的局部内能和提供系统动理压强做功。}

更具体地说:吸热确实是很多次系统分子和环境分子的微观碰撞的总和。这些微观碰撞一方面改变了系统分子的平均能量(改变系统局部内能),另一方面又做了功(宏观上表现为动理压强做功)。


\ech
\end{frame}

\begin{frame}
\chtitle{热量的显式方程}
\bch
根据能均分定理,{\blue 局部内能只是温度的函数},
$$d U_{\rm local} = f(T) dT$$
其中$f(T)$是由能均分定理决定的函数。

\skipline

动理压强$p_k = \frac{2}{3}n_{\rm eff}\overline{\varepsilon}$,其中分子平均平动动能$\overline{\varepsilon}$和温度成正比,但这里的“有效分子数密度”$n_{\rm eff}$一般来讲不是真正的分子数密度,因为在大多数物质中分子并不能随心所欲地在整个体积内运动(只有理想气体分子可以)。我们可以一般地假设$n_{\rm eff}$是体积的函数,并把动理压强写成
{\blue
$$ p_k = g(V) T$$}
这样,动理压强做功为$g(V)T dV$。最后得到热量显式方程:$$\dbar Q = f(T)dT + Tg(V) dV$$
下面我们来求$f(T)$和$g(V)$。

\ech
\end{frame}



\begin{frame}
\chtitle{$f(T)$和$g(V)$的数学表达式}
\bch
显然,因为只和分子平均距离有关,{\blue 全局内能和内压强只是体积的函数}。所以固定体积时,

\bitem
\item{内能对温度的偏导数等于局部内能对温度的偏导数:
$$ \pfrac UTV = \pfrac {U_{\rm local}}TV = f(T) $$
也就是说$f(T)$其实就是$C_V$。}
\item{
总压强对温度的偏导数等于动理压强对温度的偏导数:
$$\pfrac pTV = \pfrac {p_k}TV = g(V)$$}
\eitem
\ech
\end{frame}

\begin{frame}
\chtitle{总结}
\bch
准静态过程热量的显式方程
\tbox
{$$\dbar Q = C_V(T)dT + T g(V) dV$$}
其中$C_V(T)$代表了分子自由度数目随温度变化的情况;$Tg(V)$是动理压强,其中
\tbox{
$$g(V) = \pfrac pTV $$}
代表了能够和环境发生碰撞的有效分子数密度随总体积变化的情况。

\ech
\end{frame}

\begin{frame}
\bch

下面我们用热量的显式方程来推导一系列超出初中水平的结果

\ech
\end{frame}


\begin{frame}
\chtitle{内能和物态方程的关系}
\bch
固定温度,变化体积时:
$$\dbar Q = T g(V)dV = T\pfrac pTV dV$$
$$\dbar A = -pdV$$
故
$$dU = \left(T \pfrac pTV - p\right)dV$$
即{\blue
$$\pfrac UVT = T \pfrac pTV - p$$
}
这是一个非常重要的方程,它把物质的内能和状态方程联系起来了。
\ech
\end{frame}

\begin{frame}
\chtitle{思考题}
\bch

\addfig{1}{think2.jpg}
$$\pfrac UVT = T \pfrac pTV - p$$
这个方程的右边是什么物理量?试直接给出该方程的物理诠释。

\ech
\end{frame}


\begin{frame}
\chtitle{相变温度和相变压强的关系}
\bch
{\small

在恒定压强$p$下,物质相变时保持相变温度$T$不变(不考虑非晶体等特殊情形),设每摩尔物质从$\alpha$态变到$\beta$态需要吸热量为$\Lambda^{\rm mol}$,$\alpha$态和$\beta$态的摩尔体积分别为$V_\alpha^{\rm mol}$和$V_\beta^{\rm mol}$。

\skipline

我们曾经定性地分析过,相变温度$T$对压强$p$有依赖性,相变过程体积变化($V_\beta^{\rm mol}-V_\alpha^{\rm mol}$)越大,相变温度对$p$的依赖性越强。

\skipline

下面我们来把这种依赖性定量化。
}
\ech
\end{frame}


\begin{frame}
\chtitle{图解}
\bch
先从图像上理解我们要计算的是什么量:

\lfig{1.8}{PVTdiagram.png}\hspace{0.1in}\lfig{1.8}{PTdiagram.png}

三相图上有三个两相共存的曲面(固液共存,气液共存,固气共存),这些曲面投影到三相图($PT$图)上后即为三条相变曲线。
这些曲线的斜率反映了相变温度随压强变化的情况。我们要求解的就是这些曲面上的$\pfrac pTV$。
\ech
\end{frame}


\begin{frame}
\chtitle{克拉珀龙(Clapeyron)方程}
\bch
假设有摩尔数为$d\nu$的$\alpha$态物质转化为$\beta$态物质,因为温度不变,吸热量等于$Tg(V)dV$
$$ \Lambda^{\rm mol} d\nu = Tg(V)dV = T\pfrac pTV dV $$
再根据$dV = \left(V_\beta^{\rm mol} - V_\alpha^{\rm mol}\right)d\nu$,即得到
{\blue 克拉珀龙(Clapeyron)方程
$$ \pfrac pTV = \frac{\Lambda^{\rm mol}}{T\left(V_\beta^{\rm mol} - V_\alpha^{\rm mol}\right)}$$}
\ech
\end{frame}


\begin{frame}
\chtitle{范德瓦尔斯气体的内能}
\bch
{\small
固定温度改变体积时,局部内能不变,全局内能因内压强做功而改变。

对范德瓦尔斯气体,我们假设了内压强$P_U = -\frac{a\nu^2}{V^2}$,故
$$\pfrac UVT = \frac{a\nu^2}{V^2}$$
由此可以积分得到
$$U(V, T) = \int_{T_0}^T C_V(T) dT -\frac{\nu^2a}{V} + U_0$$
}
\ech
\end{frame}

\begin{frame}
\chtitle{克劳修斯熵}
\bch
假设物质从状态$(T_1, V_1)$经过准静态过程变化到$(T_2, V_2)$,则可以定义{\blue(克劳修斯)熵变:
$$ \Delta S \equiv \int\frac{\dbar Q}{T} = \int_{T_1}^{T_2} \frac{C_V(T)}{T}dT + \int_{V_1}^{V_2} g(V)dV$$}
显然,{\blue $S$是和过程无关的态函数}。

准静态过程的吸热量可以写成
$$ \dbar Q  = T dS$$
热力学第一定律微分形式就可以写成:
{\blue $$ dU = TdS - pdV$$}
\ech
\end{frame}

\begin{frame}
\chtitle{偏导数的噩梦}
\bch
根据上面熵$S$的积分表达式,不难写出
\bea
\pfrac STV &=& \frac{C_V}{T} = \frac{1}{T} \pfrac UTV \newl
\pfrac SVT &=& g(V) = \pfrac pTV \newl
\eea
热学里可以写出几十个这样“莫名其妙”的偏导数等式,我们之后会系统地学习怎样推导它们(三四章的难点)。
\ech
\end{frame}

\begin{frame}
\chtitle{下周期中考试}
\bch
\addfig{2}{wdsws.jpg}
\ech
\end{frame}


\begin{frame}
\chtitle{第七-八周作业}
\bch
\bitem
\item[19]{ 已知无量纲速度的概率密度函数为
$$f(u_x,u_y,u_z) = C (u_x^2+u_y^2) e^{-u}$$
其中$C$待定的归一化常数,$u = \sqrt{u_x^2+u_y^2+u_z^2}$为速率。
\bitem
\item{求平均速率$\bar{u}$}
\item{求$z$方向的泻流速率$u_{n, z, \rm leak}$}
\eitem
}
\item[20]{在温度为$300K$,压强为$p=1\atm$的氧气中放一个纳米音乐盒,音乐盒有个表面积为$10^{-4}\SImm^2$的探头,当速率超过$2792\SIm/\SIs$的氧气分子撞击探头表面时将触动音乐盒开关。问:音乐盒开关平均多久触发一次?}
\eitem
\ech
\end{frame}


\end{document}

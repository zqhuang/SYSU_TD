\documentclass[CJK]{beamer}
\usepackage{CJKutf8}
\usepackage{beamerthemesplit}
\usetheme{Malmoe}
\useoutertheme[footline=authortitle]{miniframes}
\usepackage{amsmath}
\usepackage{amssymb}
\usepackage{graphicx}
\usepackage{eufrak}
\usepackage{color}
\usepackage{slashed}
\usepackage{simplewick}
\usepackage{tikz}
\graphicspath{{../figures/}}
\def\addfig#1#2{\begin{center}\includegraphics[width=#1 in]{#2}\end{center}}
\def\blacktext#1{{\color{black}#1}}
\def\bluetext#1{{\color{blue}#1}}
\def\redtext#1{{\color{red}#1}}
\def\darkbluetext#1{{\color[rgb]{0,0.2,0.6}#1}}
\def\skybluetext#1{{\color[rgb]{0.2,0.7,1.}#1}}
\def\cyantext#1{{\color[rgb]{0.,0.5,0.5}#1}}
\def\greentext#1{{\color[rgb]{0,0.7,0.1}#1}}
\def\darkgray{\color[rgb]{0.2,0.2,0.2}}
\def\lightgray{\color[rgb]{0.6,0.6,0.6}}
\def\gray{\color[rgb]{0.4,0.4,0.4}}
\def\blue{\color{blue}}
\def\red{\color{red}}
\def\green{\color{green}}
\def\darkblue{\color[rgb]{0,0.2,0.6}}
\def\skyblue{\color[rgb]{0.2,0.7,1.}}
\def\fdeg{{^\circ \mathrm{F}}}
\def\cdeg{^\circ \mathrm{C}}
\def\be{\begin{equation}}
\def\ee{\nonumber\end{equation}}
\def\bea{\begin{eqnarray}}
\def\eea{\nonumber\end{eqnarray}}
\def\ii{{\dot{\imath}}}
\def\bch{\begin{CJK}{UTF8}{gbsn}}
\def\ech{\end{CJK}}
\def\bitem{\begin{itemize}}
\def\eitem{\end{itemize}}
\def\bcenter{\begin{center}}
\def\ecenter{\end{center}}
\def\bex{\begin{minipage}{0.3\textwidth}\includegraphics[width=1in]{jugelizi.png}\end{minipage}\begin{minipage}{0.6\textwidth}}
\def\eex{\end{minipage}}
\def\chtitle#1{\frametitle{\bch#1\ech}}
\def\skipline{{\vskip0.1in}}
\def\skiplines{{\vskip0.2in}}
\def\lagr{{\mathcal{L}}}
\def\hamil{{\mathcal{H}}}
\def\vecv{{\mathbf{v}}}
\def\vecx{{\mathbf{x}}}
\def\vecy{{\mathbf{y}}}
\def\veck{{\mathbf{k}}}
\def\vecp{{\mathbf{p}}}
\def\vecn{{\mathbf{n}}}
\def\vecA{{\mathbf{A}}}
\def\vecP{{\mathbf{P}}}
\def\vecsigma{{\mathbf{\sigma}}}
\def\hatJn{{\hat{J_\vecn}}}
\def\hatJx{{\hat{J_x}}}
\def\hatJy{{\hat{J_y}}}
\def\hatJz{{\hat{J_z}}}
\def\hatj#1{\hat{J_{#1}}}
\def\hatphi{{\hat{\phi}}}
\def\hatq{{\hat{q}}}
\def\hatpi{{\hat{\pi}}}
\def\vel{\upsilon}
\def\Dint{{\mathcal{D}}}
\def\adag{{\hat{a}^\dagger}}
\def\bdag{{\hat{b}^\dagger}}
\def\cdag{{\hat{c}^\dagger}}
\def\ddag{{\hat{d}^\dagger}}
\def\hata{{\hat{a}}}
\def\hatb{{\hat{b}}}
\def\hatc{{\hat{c}}}
\def\hatd{{\hat{d}}}
\def\hatN{{\hat{N}}}
\def\hatH{{\hat{H}}}
\def\hatp{{\hat{p}}}
\def\Fup{{F^{\mu\nu}}}
\def\Fdown{{F_{\mu\nu}}}
\def\newl{\nonumber \\}
\def\SIkm{\,\mathrm{km}}
\def\SIyr{\,\mathrm{yr}}
\def\SIGyr{\,\mathrm{Gyr}}
\def\SIeV{\,\mathrm{eV}}
\def\SIkeV{\,\mathrm{keV}}
\def\SIMeV{\,\mathrm{MeV}}
\def\SIGeV{\,\mathrm{GeV}}
\def\SIcal{\,\mathrm{cal}}
\def\SIkcal{\,\mathrm{kcal}}
\def\SImol{\,\mathrm{mol}}
\def\SIm{\,\mathrm{m}}
\def\SIcm{\,\mathrm{cm}}
\def\SIfm{\,\mathrm{fm}}
\def\SImm{\,\mathrm{mm}}
\def\SInm{\,\mathrm{nm}}
\def\SImum{\,\mathrm{\mu m}}
\def\SIJ{\,\mathrm{J}}
\def\SIkJ{\,\mathrm{kJ}}
\def\SIs{\,\mathrm{s}}
\def\SIkg{\,\mathrm{kg}}
\def\SIg{\,\mathrm{g}}
\def\SIK{\,\mathrm{K}}
\def\SImmHg{\,\mathrm{mmHg}}
\def\SIPa{\,\mathrm{Pa}}
\def\vece{\mathrm{e}}
\def\bmat#1{\left(\begin{array}{#1}}
\def\emat{\end{array}\right)}
\def\bcase#1{\left\{\begin{array}{#1}}
\def\ecase{\end{array}\right.}
\def\calM{{\mathcal{M}}}
\def\calT{{\mathcal{T}}}
\def\calR{{\mathcal{R}}}
\def\barpsi{\bar{\psi}}
\def\baru{\bar{u}}
\def\barv{\bar{\upsilon}}
\def\bmini#1{\begin{minipage}{#1\textwidth}}
\def\emini{\end{minipage}}
\def\qeq{\stackrel{?}{=}}
\def\torder#1{\mathcal{T}\left(#1\right)}
\def\rorder#1{\mathcal{R}\left(#1\right)}
\def\contr#1#2{\contraction{}{#1}{}{#2}#1#2}
\def\trof#1{\mathrm{Tr}\left(#1\right)}
\def\trace{\mathrm{Tr}}
\def\comm#1{\ \ \ \left(\mathrm{used}\ #1\right)}
\def\tcomm#1{\ \ \ (\text{#1})}
\def\slp{\slashed{p}}
\def\slk{\slashed{k}}
\def\wulian{\includegraphics[width=0.18in]{emoji_wulian.jpg}}
\def\bye{\includegraphics[width=0.18in]{emoji_bye.jpg}}
\def\calp{{\mathfrak{p}}}
\def\veccalp{\mathbf{\mathfrak{p}}}
\def\atm{\,\mathrm{atm}}
\def\angstrom{\,\text{\AA}}
\def\Tthree{T_{\tiny \textcircled{3}}}
\def\pthree{p_{\tiny \textcircled{3}}}

\def\courseurl{http://zhiqihuang.top}

\def\tpage#1#2{
\begin{frame}
\bch
\begin{center}
\begin{large}
热学 \\
第#1讲 #2

\end{large}

\skiplines

黄志琦


\end{center}

\skiplines

{\small 
教材:《热学》第二版,赵凯华,罗蔚茵,高等教育出版社


课件下载
}
\courseurl 
\ech
\end{frame}
}

\def\bfr#1{
\begin{frame}
\chtitle{#1} 
\bch
}

\def\efr{
\ech 
\end{frame}
}

\title{Lesson 07 - 1st Law of Thermaldynamics}
  \author{}
  \date{}
\begin{document}
\tpage{7}{热力学第一定律}

\section{1st Law}

\begin{frame}
\chtitle{内能是个态函数}
\bch
\bitem
\item{\bf 态函数里的“态”:指热平衡态。}
\item{\bf 态函数定义:态函数由系统的宏观状态参量确定,和如何达到这个状态的过程无关。}
\eitem

\bex
气体的内能$U$由体积$V$和温度$T$决定,
$$ U  = U(V, T)$$
\eex

\ech
\end{frame}


\begin{frame}
\chtitle{热力学第一定律的数学表达式}
\bch
热力学第一定律:

{\blue
$$ \Delta U = A + Q$$
内能的增加等于对系统做的功和传给系统的热量之和。
}

\skiplines

这里的功是{\blue 广义功},可以是机械功$-\int pdV$,也可以是电流做功$\int UIdt$等。
\ech
\end{frame}


\begin{frame}
\chtitle{准静态过程}
\bch
准静态过程的定义:{\blue 进行得足够缓慢,以至于系统连续经过的每个中间态都可以近似看成平衡态。}

\skiplines

{\small 例如:缓慢加热的过程,缓慢压缩气体的过程,SYSU-SECURE数据传输过程,政府工作人员的办公过程等等}

\addfig{2}{slowmotion.jpg}

\ech
\end{frame}


\begin{frame}
\chtitle{准静态过程的热力学第一定律表述}
\bch
准静态过程下内能在过程中间都是有定义的,所以:
{\blue
$$dU = \dbar A + \dbar Q$$
}
热学里特有的符号$\dbar$代表这个微元和过程有关。

\skiplines

非准静态过程无法定义$dU$,所以不存在热力学第一定律的微分形式。
\ech
\end{frame}


\begin{frame}
\chtitle{$p$-$V$图和做的功}
\bch
准静态过程可以用在$p$-$V$图上的一条曲线描述。气体对外界做功(为了区别外界对气体做功,通常加一撇)
$$ A' = \int p dV $$
等于$p$-$V$曲线下的面积
\addfig{2}{pdVwork.png}

\ech
\end{frame}

\section{Application}

\begin{frame}
\chtitle{理想气体等温过程}
\bch
理想气体的等温过程比较简单,利用理想气体状态方程$pV = \nu RT$即可求出对理想气体做的功
$$ A = -\int p dV = -\nu R T \int \frac{dV}{V} =- \nu R T\ln\frac{V_{\rm fin}}{V_{\rm ini}}$$
如果没有额外的自由度被激发,理想气体的内能不变。则可推算出等温过程吸收的热量为
$$ Q = - A $$
\ech
\end{frame}


\begin{frame}
\chtitle{理想气体绝热过程(adiabatic process)}
\bch
{\small
理想气体的绝热过程则稍显复杂,由$\dbar Q=0$得到
$$ dU = -p dV$$
又$$ dU = \nu C_V^{\rm mol} dT = \frac{C_V^{\rm mol}}{R}(p dV + V dp)$$
两式相减得到
$$ C_V^{\rm mol} Vdp + C_p^{\rm mol} p dV = 0$$
其中$C_p^{\rm mol} = C_V^{\rm mol} + R$是摩尔定压热容(推导见题霸集最后一题)。
记$$\gamma = \frac{C_p^{\rm mol}}{C_V^{\rm mol}}$$
对理想气体$\gamma$是个常数(对很多实际气体也是近似常数),由上述方程可推出{\blue 理想气体绝热状态方程}(又称{\bf 泊松公式})
{\blue 
$$p V^\gamma = \const$$}
}
\ech
\end{frame}


\begin{frame}
\chtitle{各种气体的$\gamma$}
\bch
{\small
对单元子理想气体,$\gamma = \frac{5}{3}$。

\skipline

对室温下的双原子理想气体,$\gamma \approx \frac{7}{5}$。

\skipline

因空气大部分都是氮气和氧气(双原子分子),所以我们可以近似认为室温下空气的$\gamma = \frac{7}{5}$。
}

由$pV^\gamma = \const $以及理想气体状态方程可推出绝热状态方程的另外两个形式:

$$ TV^{\gamma-1} = \const$$

$$T p^{\frac{1}{\gamma}-1} = \const $$
\ech
\end{frame}


\begin{frame}
\chtitle{高处不胜寒}
\bch
{\small 
我们以前计算大气压强梯度时把空气温度当成了常数,事实上我们都知道“高处不胜寒”。因为大气是热的不良导体,绝热近似是更好的描述。

由力学平衡有
$$dp = -\rho g dz$$
由$Tp^{\frac{1}{\gamma}-1}= \const$,可得$dT = \left(1-\frac{1}{\gamma}\right)\frac{T}{p} dp $,
故
$$\frac{dT}{dz} = -\frac{\gamma - 1}{\gamma} \frac{T}{p}\rho g= -\frac{\gamma - 1}{\gamma} \frac{V}{\nu R}\rho g =  -\frac{\gamma - 1}{\gamma} \frac{M^{\rm mol}}{ R} g$$
取$\gamma  = 7/5$, $M^{\rm mol} = 29\SIg/\SImol$,$g = 9.8 \mathrm{N}/\SIkg$得到
$$\frac{dT}{dz} \approx -10\SIK/\SIkm$$
这个结果的数量级是正确的,但实际温度梯度往往比它小。空气里的饱和水蒸气是一个重要的影响因素(见教材152页)。
}
\ech
\end{frame}


\begin{frame}
\chtitle{多方过程(polytropic process)}
\bch
满足状态方程$pV^n = C $({\bf 多方指数}$n$为常数, $C$为常数)的过程称为{\bf 多方过程}。

多方过程对外界做功为:
{\scriptsize
\bea
A' &=& \int p dV \newl
 &=& C \int \frac{dV}{V^n} \newl
 &=& \frac{C}{n-1}\left(V_{\rm ini}^{1-n} -  V_{\rm fin}^{1-n}\right) \newl
 &=& \frac{1}{n-1}\left(p_{\rm ini}V_{\rm ini} -  p_{\rm fin}V_{\rm fin}\right) \newl
 &=& -\frac{\nu R}{n-1}\Delta T 
\eea
}
即
{\blue
$$ A' = -\frac{\nu R}{n-1} \Delta T$$
}
\ech
\end{frame}

\begin{frame}
\chtitle{理想气体多方过程的热容量}
\bch
{\small
多方过程的热容量$C_n$满足
$$dU = C_VdT = C_n dT - pdV$$
即
$$pdV + (C_V - C_n) \frac{pdV+Vdp}{\nu R} = 0$$
又由多方过程的定义可推出
$$pdV + (n-1)(pdV + Vdp) = 0$$
对比两式即得
{\blue
$$C_n = C_V - \frac{\nu R}{n-1}$$
}
这结果很好理解,$C_V$用于描述内能的增加,$- \frac{\nu R}{n-1}$用于描述气体对外做功消耗的能量。

或者写成摩尔热容
$$C_n^{\rm mol} = C_V^{\rm mol} - \frac{R}{n-1}$$
}
\ech
\end{frame}


\begin{frame}
\chtitle{多方过程的例子:等压过程}
\bch
\bex
等压过程是$n=0$的多方过程。故
做功
$$A' = \nu R\Delta T $$
热容量
$$C_p^{\rm mol} = C_V^{\rm mol}+ R $$

\skipline

我们可以通过计算内能变化来检验上面的结果:
$$ \Delta U = -A' + Q = -\nu R\Delta T +(C_V + \nu R)\Delta T =  C_V\Delta T$$

\eex
\ech
\end{frame}

\begin{frame}
\chtitle{多方过程的例子:绝热过程}
\bch
\bex
绝热过程是$n=\gamma$的多方过程。故
做功
$$A' = -\frac{\nu R}{\gamma-1}\Delta T$$
热容量
$$C = 0 $$

我们可以通过计算内能变化来检验上面的结果:
$$ \Delta U = -A' + Q = \frac{\nu R}{\gamma-1} \Delta T  =  C_V\Delta T$$

\eex
\ech
\end{frame}

\begin{frame}
\chtitle{多方过程的例子:等体过程}
\bch
\bex
等体过程是$n=\infty$的多方过程。故
做功
$$A' = 0$$
热容量
$$C = C_V $$

我们可以通过计算内能变化来检验上面的结果:
$$ \Delta U = -A' + Q =  C_V\Delta T$$

\eex
\ech
\end{frame}


\begin{frame}
\chtitle{多方过程的例子:等温过程}
\bch
\bex
等温过程是$n=1$的多方过程。这时无法直接用多方过程的做功公式。我们考虑$n=1+\epsilon$的情形,再让$\epsilon \rightarrow 0$。

做功
$$\dbar A' = -\frac{\nu R}{\epsilon}dT = \frac{\nu R}{\epsilon}\frac{T\epsilon}{V}dV = \nu RT d\ln V $$
积分即得
$$ A' = -\frac{\nu R}{\epsilon}dT = \frac{\nu R}{\epsilon}\frac{T\epsilon}{V}dV = \nu RT \Delta(\ln V) $$
热容量
$$C = \infty $$

\eex
\ech
\end{frame}


\begin{frame}
\chtitle{消化下栗子}
\bch
\addfig{1}{chibaolizi.jpg}
$1\SImol$氧气经过$n=2$的多方过程从$T=300K$升温到$T=310K$。问这个过程中对氧气做功多少?氧气吸热多少?
\ech
\end{frame}


\begin{frame}
\chtitle{什么?栗子太好消化了?}
\bch

下周期中考试


前面所有课的讲义(包括题霸集)有看不懂的地方(或者发现错误的地方)请举手提问。

\addfig{2}{wdsws.jpg}

\ech
\end{frame}


\begin{frame}
\chtitle{第七八周作业(序号接第六周)}
\bch
\bitem
\item[19]{教材思考题3-18 (送分题,简单叙述拿分走人,请勿长篇大论累死助教)}
\item[20]{教材习题3-5}
\item[21]{教材习题3-6}
\eitem
\ech
\end{frame}


\end{document}

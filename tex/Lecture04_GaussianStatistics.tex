\documentclass[CJK]{beamer}
\usepackage{CJKutf8}
\usepackage{beamerthemesplit}
\usetheme{Malmoe}
\useoutertheme[footline=authortitle]{miniframes}
\usepackage{amsmath}
\usepackage{amssymb}
\usepackage{graphicx}
\usepackage{eufrak}
\usepackage{color}
\usepackage{slashed}
\usepackage{simplewick}
\usepackage{tikz}
\graphicspath{{../figures/}}
\def\addfig#1#2{\begin{center}\includegraphics[width=#1 in]{#2}\end{center}}
\def\blacktext#1{{\color{black}#1}}
\def\bluetext#1{{\color{blue}#1}}
\def\redtext#1{{\color{red}#1}}
\def\darkbluetext#1{{\color[rgb]{0,0.2,0.6}#1}}
\def\skybluetext#1{{\color[rgb]{0.2,0.7,1.}#1}}
\def\cyantext#1{{\color[rgb]{0.,0.5,0.5}#1}}
\def\greentext#1{{\color[rgb]{0,0.7,0.1}#1}}
\def\darkgray{\color[rgb]{0.2,0.2,0.2}}
\def\lightgray{\color[rgb]{0.6,0.6,0.6}}
\def\gray{\color[rgb]{0.4,0.4,0.4}}
\def\blue{\color{blue}}
\def\red{\color{red}}
\def\green{\color{green}}
\def\darkblue{\color[rgb]{0,0.2,0.6}}
\def\skyblue{\color[rgb]{0.2,0.7,1.}}
\def\fdeg{{^\circ \mathrm{F}}}
\def\cdeg{^\circ \mathrm{C}}
\def\be{\begin{equation}}
\def\ee{\nonumber\end{equation}}
\def\bea{\begin{eqnarray}}
\def\eea{\nonumber\end{eqnarray}}
\def\ii{{\dot{\imath}}}
\def\bch{\begin{CJK}{UTF8}{gbsn}}
\def\ech{\end{CJK}}
\def\bitem{\begin{itemize}}
\def\eitem{\end{itemize}}
\def\bcenter{\begin{center}}
\def\ecenter{\end{center}}
\def\bex{\begin{minipage}{0.3\textwidth}\includegraphics[width=1in]{jugelizi.png}\end{minipage}\begin{minipage}{0.6\textwidth}}
\def\eex{\end{minipage}}
\def\chtitle#1{\frametitle{\bch#1\ech}}
\def\skipline{{\vskip0.1in}}
\def\skiplines{{\vskip0.2in}}
\def\lagr{{\mathcal{L}}}
\def\hamil{{\mathcal{H}}}
\def\vecv{{\mathbf{v}}}
\def\vecx{{\mathbf{x}}}
\def\vecy{{\mathbf{y}}}
\def\veck{{\mathbf{k}}}
\def\vecp{{\mathbf{p}}}
\def\vecn{{\mathbf{n}}}
\def\vecA{{\mathbf{A}}}
\def\vecP{{\mathbf{P}}}
\def\vecsigma{{\mathbf{\sigma}}}
\def\hatJn{{\hat{J_\vecn}}}
\def\hatJx{{\hat{J_x}}}
\def\hatJy{{\hat{J_y}}}
\def\hatJz{{\hat{J_z}}}
\def\hatj#1{\hat{J_{#1}}}
\def\hatphi{{\hat{\phi}}}
\def\hatq{{\hat{q}}}
\def\hatpi{{\hat{\pi}}}
\def\vel{\upsilon}
\def\Dint{{\mathcal{D}}}
\def\adag{{\hat{a}^\dagger}}
\def\bdag{{\hat{b}^\dagger}}
\def\cdag{{\hat{c}^\dagger}}
\def\ddag{{\hat{d}^\dagger}}
\def\hata{{\hat{a}}}
\def\hatb{{\hat{b}}}
\def\hatc{{\hat{c}}}
\def\hatd{{\hat{d}}}
\def\hatN{{\hat{N}}}
\def\hatH{{\hat{H}}}
\def\hatp{{\hat{p}}}
\def\Fup{{F^{\mu\nu}}}
\def\Fdown{{F_{\mu\nu}}}
\def\newl{\nonumber \\}
\def\SIkm{\,\mathrm{km}}
\def\SIyr{\,\mathrm{yr}}
\def\SIGyr{\,\mathrm{Gyr}}
\def\SIeV{\,\mathrm{eV}}
\def\SIkeV{\,\mathrm{keV}}
\def\SIMeV{\,\mathrm{MeV}}
\def\SIGeV{\,\mathrm{GeV}}
\def\SIcal{\,\mathrm{cal}}
\def\SIkcal{\,\mathrm{kcal}}
\def\SImol{\,\mathrm{mol}}
\def\SIm{\,\mathrm{m}}
\def\SIcm{\,\mathrm{cm}}
\def\SIfm{\,\mathrm{fm}}
\def\SImm{\,\mathrm{mm}}
\def\SInm{\,\mathrm{nm}}
\def\SImum{\,\mathrm{\mu m}}
\def\SIJ{\,\mathrm{J}}
\def\SIkJ{\,\mathrm{kJ}}
\def\SIs{\,\mathrm{s}}
\def\SIkg{\,\mathrm{kg}}
\def\SIg{\,\mathrm{g}}
\def\SIK{\,\mathrm{K}}
\def\SImmHg{\,\mathrm{mmHg}}
\def\SIPa{\,\mathrm{Pa}}
\def\vece{\mathrm{e}}
\def\bmat#1{\left(\begin{array}{#1}}
\def\emat{\end{array}\right)}
\def\bcase#1{\left\{\begin{array}{#1}}
\def\ecase{\end{array}\right.}
\def\calM{{\mathcal{M}}}
\def\calT{{\mathcal{T}}}
\def\calR{{\mathcal{R}}}
\def\barpsi{\bar{\psi}}
\def\baru{\bar{u}}
\def\barv{\bar{\upsilon}}
\def\bmini#1{\begin{minipage}{#1\textwidth}}
\def\emini{\end{minipage}}
\def\qeq{\stackrel{?}{=}}
\def\torder#1{\mathcal{T}\left(#1\right)}
\def\rorder#1{\mathcal{R}\left(#1\right)}
\def\contr#1#2{\contraction{}{#1}{}{#2}#1#2}
\def\trof#1{\mathrm{Tr}\left(#1\right)}
\def\trace{\mathrm{Tr}}
\def\comm#1{\ \ \ \left(\mathrm{used}\ #1\right)}
\def\tcomm#1{\ \ \ (\text{#1})}
\def\slp{\slashed{p}}
\def\slk{\slashed{k}}
\def\wulian{\includegraphics[width=0.18in]{emoji_wulian.jpg}}
\def\bye{\includegraphics[width=0.18in]{emoji_bye.jpg}}
\def\calp{{\mathfrak{p}}}
\def\veccalp{\mathbf{\mathfrak{p}}}
\def\atm{\,\mathrm{atm}}
\def\angstrom{\,\text{\AA}}
\def\Tthree{T_{\tiny \textcircled{3}}}
\def\pthree{p_{\tiny \textcircled{3}}}

\def\courseurl{http://zhiqihuang.top}

\def\tpage#1#2{
\begin{frame}
\bch
\begin{center}
\begin{large}
热学 \\
第#1讲 #2

\end{large}

\skiplines

黄志琦


\end{center}

\skiplines

{\small 
教材:《热学》第二版,赵凯华,罗蔚茵,高等教育出版社


课件下载
}
\courseurl 
\ech
\end{frame}
}

\def\bfr#1{
\begin{frame}
\chtitle{#1} 
\bch
}

\def\efr{
\ech 
\end{frame}
}

\title{Lesson 04 - Gaussian Statistics}
  \author{}
  \date{}
\begin{document}
\tpage{4}{高斯统计}

\section{Reivew}

\begin{frame}
\chtitle{上一讲内容回顾}
\bch
上一讲的内容基本上都不考
\ech
\end{frame}


\begin{frame}
\chtitle{本讲内容预告}
\bch
\bitem
\item{概率密度函数}
\item{高斯统计}
\eitem
\ech
\end{frame}

\section{Probability Density Function}

\secpage{概率密度函数(Probability Density Function)}{$$Q(y) |dy| = \sum_{x\rightarrow y} P(x) |dx|$$}

\begin{frame}
\chtitle{概率密度函数}
\bch
(一元随机变量的)概率密度函数是在小区间内出现的概率和该小区间的宽度之比(即概率密度)对小区间的位置的依赖关系。
\addfig{2.5}{pdf.jpg}

概率元可以写成$P(x) dx$,所有概率元之和应是归一化的:

$$\int_{-\infty}^\infty P(x) dx  = 1.$$
\ech
\end{frame}

\begin{frame}
\chtitle{概率密度函数的换元}
\bch
假设随机变量$x$的概率密度函数为$P(x)$,我们把$x$以一定的规则映射到某个新的随机变量$y$,并考察$y$的概率密度函数$Q(y)$。

\skipline

做变量替换后要注意不变量是概率元:
\tbox{$$Q(y) |dy| = \sum_{x\rightarrow y} P(x) |dx|$$}
也就是说,把所有能映射为$y$的$x$附近的概率加起来,然后除以$|dy|$,得到的就是$y$的概率密度函数。
\ech
\end{frame}


\begin{frame}
\chtitle{思考题}
\bch

\addfig{0.5}{think.jpg}

假设随机变量$x$的概率密度函数为$P(x)$,定义$y=x/2$,求$y$的概率密度函数。
\ech
\end{frame}

\begin{frame}
\chtitle{思考题}
\bch

\addfig{0.7}{think2.jpg}

假设随机变量$x$的概率密度函数为$P(x)$,定义$y=x^2$,求$y$的概率密度函数。
\ech
\end{frame}

\begin{frame}
\chtitle{二元随机变量的概率密度函数}
\bch

二元随机变量的概率密度函数定义为在一个小块内出现的概率和该小块面积的比值(即概率密度)对小块位置的依赖关系。

\addfig{2.}{pdf2d.jpg}

概率元可以写成$P(x,y) dx dy$,所有概率元之和应是归一化的:

$$\iint P(x,y) dx dy  = 1.$$

\ech
\end{frame}


\begin{frame}
\chtitle{二元随机变量的概率密度函数进行投影}
\bch

从二元随机变量的概率密度函数$P(x,y)$可以进行投影,分别得到$x$, $y$的概率密度函数。

\addfig{1.9}{pdf2dproj.jpg}

$$f(x) =  \int_{-\infty}^\infty P(x,y) dy;  $$
$$g(y) =  \int_{-\infty}^\infty P(x,y) dx.  $$

\ech
\end{frame}


\begin{frame}
\chtitle{多元随机变量的概率密度函数和投影}
\bch

$n$元随机变量的概率密度函数定义为$n$维空间的一小块内的概率和该小块的$n$维体积之比(即概率密度)对该小块的位置的依赖关系。

\skiplines

要投影得到其中某些分量的概率密度函数,只要把不需要的分量进行积分。例如从四元概率密度函数$P(x,y,z,w)$可以得到二元变量$(x,y)$的概率密度函数为:
$$ f(x,y) = \iint P(x,y,z,w) dz dw. $$


\ech
\end{frame}

\begin{frame}
\chtitle{投影的逆操作}
\bch
一般来说,投影的操作会遗失统计信息,所以投影的逆操作无法进行。

但对{\bf 没有统计关联}的变量而言,可以进行投影的逆操作:


{\blue 假设$x,y,z\ldots$是已知概率密度函数,且两两没有统计关联的随机变量,则多元变量$(x, y,z,\ldots)$的概率密度函数等于它们各自的概率密度函数$p(x)$, $q(y)$, $r(z)$ \ldots 的乘积:
  $$P(x,y,z,\ldots) = p(x)q(y)r(z)\ldots $$
  }
\ech
\end{frame}


\begin{frame}
\chtitle{速度分布函数和速率分布函数}
\bch
    {\bf 速度分布函数}定义为{\bf 速度(三元变量$(\upsilon_x, \upsilon_y,\upsilon_z)$)的概率密度函数.}

    \skiplines
    
{\bf 速率分布函数}定义为{\bf 速率(一元变量$\upsilon = \sqrt{\upsilon_x^2+\upsilon_y^2+\upsilon_z^2}$)的概率密度函数.} 显然,速率分布函数在$\upsilon<0$处为零。
    
\ech
\end{frame}

%====================================================

\section{Gaussian Statistics}

\secpage{高斯统计(Gaussian Statistics)}{$$G_0 = 1,\ G_1 = \sqrt{\frac{2}{\pi}},\ G_{n+2} = (n+1) G_n.$$}


\begin{frame}
\chtitle{很多个一样的东西的和总是满足高斯分布}
\bch
    {\blue 中心极限定理:满足任何给定分布的多个随机变量的和总是趋向于服从高斯分布(也叫正态分布):}

    (我认为统计学里只有两条非平凡的定律,中心极限定理是其中一条)

    \skiplines
    
   高斯分布的{\bf 概率密度函数}为:
  \tbox{ $$ P(x) = \frac{1}{\sqrt{2\pi}\sigma} e^{-\frac{(x-\mu)^2}{2\sigma^2}}.  $$}

   其中{\blue $\sigma$是标准差(standard deviation),$\mu$是平均值(mean)}。


\ech
\end{frame}




\begin{frame}
\chtitle{高斯分布}
\addfig{3.5}{normal.png}
\end{frame}

\begin{frame}
\chtitle{中心极限定理举例}
\bch
\bex
小张测量体重,得到的数值是一个随机变量。这个随机变量服从的分布可能非常复杂(取决于测量工具,测量方式等)。但多次测量的和(或者平均)总是趋向于服从正态分布。
多次测量求平均后的结果可以写成 $(175.0\pm0.5)$ (斤?公斤?吨?)的形式: 默认就是$\mu = 175$, $\sigma = 0.5$的正态分布。
\eex
\ech
\end{frame}


\begin{frame}
\chtitle{标准正态分布}
\bch
因为$\sigma$反映了$x$的大致变化尺度,$\mu$代表了中心值。我们总是可以重新定义变量(相当于重新取零点和单位)
$$y \equiv \frac{x-\mu}{\sigma}$$
那么$y$的概率密度函数就是
$$Q(y) =  \frac{1}{\sqrt{2\pi}} e^{-\frac{y^2}{2}}  $$
也就是说,通过重新选取零点和单位,我们得到$\mu=0$, $\sigma=1$的正态分布——即{\bf 标准正态分布}。
\ech
\end{frame}


\begin{frame}
\chtitle{标准正态分布的$n$次绝对平均}
\bch
如果$x$服从标准正态分布,则$|x|^n$的平均值(统计术语叫期待值)$G_n$为
$$G_n \equiv \overline{|x|^n} =  \frac{1}{\sqrt{2\pi}} \int_{-\infty}^{\infty} |x|^n e^{-\frac{x^2}{2}} dx.$$
虽然可以直接用二重积分或者$\Gamma$函数的知识硬算$G_n$,但更快捷的方法是用下述递推公式(证明见附录,不要求掌握):
\tbox{
$$G_0 = 1,\ G_1 = \sqrt{\frac{2}{\pi}},\ G_{n+2} = (n+1) G_n.$$
}
\ech
\end{frame}


\begin{frame}
\chtitle{思考题}
\bch
\addfig{0.6}{think1.jpg}

如果$x$服从标准正态分布,计算下列量的期待值:

\bitem
\item{$x^2$}
\item{$x^3$}
\item{$|x|^3$}
\item{$x^4$}  
\eitem
\ech
\end{frame}



\begin{frame}
\chtitle{思考题}
\bch
\addfig{0.6}{think2.jpg}

如果$x$服从均值为$0$,标准差为$\sigma$的正态分布,计算下列量的期待值:

\bitem
\item{$x^2$}
\item{$x^3$}
\item{$|x|^3$}
\item{$x^4$}  
\eitem
\ech
\end{frame}


\begin{frame}
\chtitle{思考题}
\bch
\addfig{0.6}{think3.jpg}

如果$x$服从均值为$\mu$,标准差为$\sigma$的正态分布,计算下列量的期待值:

\bitem
\item{$x$}
\item{$x^2$}
\item{$x^3$}
\eitem
\ech
\end{frame}


\begin{frame}
\chtitle{思考题}
\bch
\addfig{0.6}{think1.jpg}

如果$x$, $y$, $z$分别都服从标准正态分布,且$x,y,z$之间没有统计关联,定义新的随机变量$r=\sqrt{x^2+y^2+z^2}$,那么$r$的概率密度函数是怎样的?请计算$r$, $r^2$, $r^3$的期待值。

\ech
\end{frame}


\begin{frame}
\chtitle{理想气体分子速度的麦克斯韦分布}
\bch
根据我们掌握的$e^{-\frac{\varepsilon}{kT}}$法则,如果重力势能等可以忽略,处于平衡态的理想气体的分子速度的三个分量$\upsilon_x$, $\upsilon_y$, $\upsilon_z$都独立地(相互无关联地)服从均值为零,标准差为$\sqrt{\frac{kT}{m}}$的正态分布。这个定律称为{\bf 麦克斯韦速度分布律}。

\skipline

利用独立分布的反投影技术,我们可以得到完整的三维速度分布函数:
$$ f_M(\upsilon_x, \upsilon_y, \upsilon_z) = \left(\frac{m}{2\pi kT}\right)^{3/2}e^{-\frac{m\upsilon^2}{2kT}}, $$
其中$\upsilon^2 = \upsilon_x^2 + \upsilon_y^2 + \upsilon_z^2$.

\skipline

通过与前面思考题类似的操作,我们可以得到速率分布函数,并计算速度/速率的各种平均值。这些计算技巧是本课程在期中考之前的核心内容。
\ech
\end{frame}



\section{Appendix}

\begin{frame}
\chtitle{附录I:高斯积分($a=\frac{1}{2}$对应标准正态分布归一化)}
\bch
{\blue
$$\int_{-\infty}^\infty e^{-ax^2}dx =\sqrt{\frac{\pi}{a}} \, \ \ \ \ (a>0).$$}

{\scriptsize
证明:设$I = \int_{-\infty}^\infty e^{-ax^2}dx$, 则

\bmini{0.6}
\bea
I^2 &=& \left(\int_{-\infty}^\infty e^{-ax^2}dx\right) \left(\int_{-\infty}^\infty e^{-ay^2}dy\right) \newl 
&=& \int_{-\infty}^\infty dx \int_{-\infty}^\infty dy \, e^{-a(x^2+y^2)}  \newl
&=& \int_0^\infty dr \int_0^{2\pi} rd\theta \, e^{-ar^2}  \newl  
&=& 2\pi \int_0^{\infty} r e^{-ar^2}dr \newl
&=& -\left.\frac{\pi}{a} e^{-ar^2}\right\vert_0^{\infty} \newl
&=& \frac{\pi}{a}
\eea
\emini
\bmini{0.35}
\hspace{0.2in}

转换到极坐标$(r,\theta)$,面积元变为 $r dr d\theta$
\emini

又显然 $I>0$,故$I =\sqrt{\frac{\pi}{a}}$.
}
\ech
\end{frame}


\begin{frame}
\chtitle{附录II:标准正态分布$n$次绝对平均的递推公式证明}
\bch
证明:
{\scriptsize
因为$G_0$是$|x|^0 = 1$的期待值,当然$G_0 = 1$。
$$G_1 = \frac{1}{\sqrt{2\pi}}\int_{-\infty}^\infty x e^{-\frac{x^2}{2}}dx =\left. -\frac{1}{\sqrt{2\pi}}  e^{-\frac{x^2}{2}} \right\vert_{-\infty}^\infty = \sqrt{\frac{2}{\pi}}.$$


\bea
G_{n+2} &=& \frac{1}{\sqrt{2\pi}} \int_{-\infty}^\infty |x|^{n+2} e^{-x^2/2} dx \newl
&=& \sqrt{\frac{2}{\pi}} \int_{0}^\infty x^{n+2} e^{-x^2/2} dx  \newl
&=& -\sqrt{\frac{2}{\pi}} \int_{0}^\infty x^{n+1} d\left(e^{-x^2/2}\right)   \newl
&=& \left.-\sqrt{\frac{2}{\pi}}x^{n+1} e^{-x^2/2}\right\vert_0^\infty + \sqrt{\frac{2}{\pi}}\int_0^\infty  e^{-x^2/2} d(x^{n+1}) \newl
&=& (n+1) \sqrt{\frac{2}{\pi}}\int_0^\infty x^n e^{-x^2/2} dx \newl
&=& (n+1)\frac{1}{\sqrt{2\pi}} \int_{-\infty}^\infty |x|^n e^{-x^2/2} dx \newl
&=& (n+1) G_n
\eea
}

\ech
\end{frame}


\end{document}

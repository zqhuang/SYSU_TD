\documentclass[CJK]{beamer}
\usepackage{CJKutf8}
\usepackage{beamerthemesplit}
\usetheme{Malmoe}
\useoutertheme[footline=authortitle]{miniframes}
\usepackage{amsmath}
\usepackage{amssymb}
\usepackage{graphicx}
\usepackage{eufrak}
\usepackage{color}
\usepackage{slashed}
\usepackage{simplewick}
\usepackage{tikz}
\graphicspath{{../figures/}}
\def\addfig#1#2{\begin{center}\includegraphics[width=#1 in]{#2}\end{center}}
\def\blacktext#1{{\color{black}#1}}
\def\bluetext#1{{\color{blue}#1}}
\def\redtext#1{{\color{red}#1}}
\def\darkbluetext#1{{\color[rgb]{0,0.2,0.6}#1}}
\def\skybluetext#1{{\color[rgb]{0.2,0.7,1.}#1}}
\def\cyantext#1{{\color[rgb]{0.,0.5,0.5}#1}}
\def\greentext#1{{\color[rgb]{0,0.7,0.1}#1}}
\def\darkgray{\color[rgb]{0.2,0.2,0.2}}
\def\lightgray{\color[rgb]{0.6,0.6,0.6}}
\def\gray{\color[rgb]{0.4,0.4,0.4}}
\def\blue{\color{blue}}
\def\red{\color{red}}
\def\green{\color{green}}
\def\darkblue{\color[rgb]{0,0.2,0.6}}
\def\skyblue{\color[rgb]{0.2,0.7,1.}}
\def\fdeg{{^\circ \mathrm{F}}}
\def\cdeg{^\circ \mathrm{C}}
\def\be{\begin{equation}}
\def\ee{\nonumber\end{equation}}
\def\bea{\begin{eqnarray}}
\def\eea{\nonumber\end{eqnarray}}
\def\ii{{\dot{\imath}}}
\def\bch{\begin{CJK}{UTF8}{gbsn}}
\def\ech{\end{CJK}}
\def\bitem{\begin{itemize}}
\def\eitem{\end{itemize}}
\def\bcenter{\begin{center}}
\def\ecenter{\end{center}}
\def\bex{\begin{minipage}{0.3\textwidth}\includegraphics[width=1in]{jugelizi.png}\end{minipage}\begin{minipage}{0.6\textwidth}}
\def\eex{\end{minipage}}
\def\chtitle#1{\frametitle{\bch#1\ech}}
\def\skipline{{\vskip0.1in}}
\def\skiplines{{\vskip0.2in}}
\def\lagr{{\mathcal{L}}}
\def\hamil{{\mathcal{H}}}
\def\vecv{{\mathbf{v}}}
\def\vecx{{\mathbf{x}}}
\def\vecy{{\mathbf{y}}}
\def\veck{{\mathbf{k}}}
\def\vecp{{\mathbf{p}}}
\def\vecn{{\mathbf{n}}}
\def\vecA{{\mathbf{A}}}
\def\vecP{{\mathbf{P}}}
\def\vecsigma{{\mathbf{\sigma}}}
\def\hatJn{{\hat{J_\vecn}}}
\def\hatJx{{\hat{J_x}}}
\def\hatJy{{\hat{J_y}}}
\def\hatJz{{\hat{J_z}}}
\def\hatj#1{\hat{J_{#1}}}
\def\hatphi{{\hat{\phi}}}
\def\hatq{{\hat{q}}}
\def\hatpi{{\hat{\pi}}}
\def\vel{\upsilon}
\def\Dint{{\mathcal{D}}}
\def\adag{{\hat{a}^\dagger}}
\def\bdag{{\hat{b}^\dagger}}
\def\cdag{{\hat{c}^\dagger}}
\def\ddag{{\hat{d}^\dagger}}
\def\hata{{\hat{a}}}
\def\hatb{{\hat{b}}}
\def\hatc{{\hat{c}}}
\def\hatd{{\hat{d}}}
\def\hatN{{\hat{N}}}
\def\hatH{{\hat{H}}}
\def\hatp{{\hat{p}}}
\def\Fup{{F^{\mu\nu}}}
\def\Fdown{{F_{\mu\nu}}}
\def\newl{\nonumber \\}
\def\SIkm{\,\mathrm{km}}
\def\SIyr{\,\mathrm{yr}}
\def\SIGyr{\,\mathrm{Gyr}}
\def\SIeV{\,\mathrm{eV}}
\def\SIkeV{\,\mathrm{keV}}
\def\SIMeV{\,\mathrm{MeV}}
\def\SIGeV{\,\mathrm{GeV}}
\def\SIcal{\,\mathrm{cal}}
\def\SIkcal{\,\mathrm{kcal}}
\def\SImol{\,\mathrm{mol}}
\def\SIm{\,\mathrm{m}}
\def\SIcm{\,\mathrm{cm}}
\def\SIfm{\,\mathrm{fm}}
\def\SImm{\,\mathrm{mm}}
\def\SInm{\,\mathrm{nm}}
\def\SImum{\,\mathrm{\mu m}}
\def\SIJ{\,\mathrm{J}}
\def\SIkJ{\,\mathrm{kJ}}
\def\SIs{\,\mathrm{s}}
\def\SIkg{\,\mathrm{kg}}
\def\SIg{\,\mathrm{g}}
\def\SIK{\,\mathrm{K}}
\def\SImmHg{\,\mathrm{mmHg}}
\def\SIPa{\,\mathrm{Pa}}
\def\vece{\mathrm{e}}
\def\bmat#1{\left(\begin{array}{#1}}
\def\emat{\end{array}\right)}
\def\bcase#1{\left\{\begin{array}{#1}}
\def\ecase{\end{array}\right.}
\def\calM{{\mathcal{M}}}
\def\calT{{\mathcal{T}}}
\def\calR{{\mathcal{R}}}
\def\barpsi{\bar{\psi}}
\def\baru{\bar{u}}
\def\barv{\bar{\upsilon}}
\def\bmini#1{\begin{minipage}{#1\textwidth}}
\def\emini{\end{minipage}}
\def\qeq{\stackrel{?}{=}}
\def\torder#1{\mathcal{T}\left(#1\right)}
\def\rorder#1{\mathcal{R}\left(#1\right)}
\def\contr#1#2{\contraction{}{#1}{}{#2}#1#2}
\def\trof#1{\mathrm{Tr}\left(#1\right)}
\def\trace{\mathrm{Tr}}
\def\comm#1{\ \ \ \left(\mathrm{used}\ #1\right)}
\def\tcomm#1{\ \ \ (\text{#1})}
\def\slp{\slashed{p}}
\def\slk{\slashed{k}}
\def\wulian{\includegraphics[width=0.18in]{emoji_wulian.jpg}}
\def\bye{\includegraphics[width=0.18in]{emoji_bye.jpg}}
\def\calp{{\mathfrak{p}}}
\def\veccalp{\mathbf{\mathfrak{p}}}
\def\atm{\,\mathrm{atm}}
\def\angstrom{\,\text{\AA}}
\def\Tthree{T_{\tiny \textcircled{3}}}
\def\pthree{p_{\tiny \textcircled{3}}}

\def\courseurl{http://zhiqihuang.top}

\def\tpage#1#2{
\begin{frame}
\bch
\begin{center}
\begin{large}
热学 \\
第#1讲 #2

\end{large}

\skiplines

黄志琦


\end{center}

\skiplines

{\small 
教材:《热学》第二版,赵凯华,罗蔚茵,高等教育出版社


课件下载
}
\courseurl 
\ech
\end{frame}
}

\def\bfr#1{
\begin{frame}
\chtitle{#1} 
\bch
}

\def\efr{
\ech 
\end{frame}
}

\title{Lesson 14 Course Review II}
  \author{}
  \date{}
\begin{document}
\tpage{14}{课程复习II}

\newcounter{chap}
\newcounter{problem}[chap]
\def\proid{{Problem \thechap.\theproblem}}

\section{Thermodynamic Process}
\setcounter{chap}{4}
\setcounter{problem}{0}

\begin{frame}
  \chtitle{第四篇:热力学过程}
  \addfig{4}{process.png}
  \skipline
  
  \bch
  这。是。期。末。考。试。重。点。
  \ech
\end{frame}

\begin{frame}
  \chtitle{纲要:热力学第一定律}
  \bch
  \tbox{$$\Delta U = Q + A$$}

  符号约定:不带撇的为外界给系统能量(系统吸热量,外界对系统做功等),带撇的为系统给外界能量(系统放热量,系统对外界做功等)。

  对$pVT$系统,$\dbar A = -pdV$,即
  \tbox{$$dU = \dbar Q - pdV$$}

  对{\bf 准静态多过程}$\dbar Q = TdS$
  \tbox{$$dU = TdS - pdV$$}
  
  \ech
\end{frame}

\begin{frame}
  \chtitle{几个名词}
  \bch
  \bitem
\item{绝热:$Q  = 0$ ($\Rightarrow$准静态绝热$\Delta S = 0$)}
\item{自由:$A=0$}
\item{节流:$\Delta H = 0$}
\item{准静态过程 = 可逆过程; 非准静态过程=不可逆过程}
\item{正循环:对外做功的热机$A'>0$;$p$-$V$图和$T$-$S$图上都是顺时针。}
\item{逆循环:外界对系统做功的制冷机$A>0$;$p$-$V$图和$T$-$S$图上都是逆时针。}
\item{可逆循环:对外界影响可以通过相反的一个循环来完全消除。环境熵变为零。}
\item{不可逆循环:对外界影响不能通过任何方式完全消除。环境熵变大于零。}
\eitem
  \ech  
\end{frame}

\begin{frame}
  \chtitle{理想气体的准静态过程}
  \bch
  \bitem
  \item{理想气体内能和焓均只是温度的函数:$dU = C_V dT$, $dH = C_p dT$, $C_p = C_V+\nu R$}
  \item{等温过程对外做功$A'=\nu RT \Delta \ln V$}
  \item{
  多方过程($n\ne 1$)对外做功
  $A' = -\frac{\nu R}{n-1}\Delta T$
  
  热容相应地即
  $C_n = C_V - \frac{\nu R}{n-1}$}
  \item{理想气体的熵变$ dS = C_V d \ln T + \nu R d\ln V$}
  \item{常数$\gamma$近似下绝热方程$pV^\gamma = C$, $TV^{\gamma-1} =C$, $Tp^{\frac{1}{\gamma}-1} = C$}
  \item{可逆多方循环热机效率$\eta = 1- \frac{T_2}{T_1}$其中$T_2$, $T_1$分别是任意一条绝热线两端的低温值和高温值。可逆多方循环的制冷机效率则是$T_2/(T_1-T_2)$}
  \eitem
  \ech  
\end{frame}

\stepcounter{problem}
\begin{frame}
  \chtitle{\proid  (\sthree)}
  \bch

  教材166页思考题3-19:
  
  冬天用空调机或者电炉取暖,何者较省电?


  \ech
\end{frame}


\begin{frame}
  \chtitle{\proid 解答}
  \bch
  空调机比较省电。空调的制热原理是把室外当成需要制冷的低温热源。如果看成可逆循环,空调释放给室内(高温热源)的热量大于外界对空调做功(即空调耗能)。用电炉的话释放的热量不会大于消耗的电能。当然,这些讨论都是基于空调可以看成高效率的可逆制冷机的前提。
  \ech
\end{frame}

\stepcounter{problem}
\begin{frame}
  \chtitle{\proid   (\stwo)}
  \bch
  教材166页习题3-2:

  分别通过下列过程把标准状态下的$0.014\SIkg$氮气压缩为原体积的一半:(1)等温过程;(2)绝热过程;(3)等压过程。试分别求出这些过程中内能的改变,传递的热量和外界对气体做的功。设氮气可以看作理想气体,且$C_V^{\rm mol} = \frac{5}{2}R$。
  
  \ech
\end{frame}


\begin{frame}
  \chtitle{\proid 解答}
  \bch
  {\small
  氮气的摩尔数$\nu = \frac{14\SIg}{28\SIg/\SImol} = 0.5\SImol$
  \bitem
\item{等温过程$\Delta U = 0$, $A = -\nu R T\Delta \ln V = 787\SIJ$, $Q = \Delta U - A = - 787\SIJ$
}
\item{绝热过程$Q=0$, $\Delta U = C_V \Delta T = \frac{5}{2}\nu R (2^{\gamma-1}T -T) = 907\SIJ$, $A = \Delta U - Q = 907\SIJ$}
\item{等压过程$A = -p\Delta V = \frac{pV}{2} = \frac{1}{2}\nu R T = 567\SIJ$,$\Delta U = C_V \Delta T = \frac{5}{2}\nu R (\frac{T}{2}-T) = -1419\SIJ$,$Q = \Delta U - A = -1986 \SIJ$}
  \eitem
  }
  \ech
\end{frame}

\stepcounter{problem}
\begin{frame}
  \chtitle{\proid   (\stwo)}
  \bch
  教材166页习题3-3:

  
  在标准状态下$0.016\SIkg$的氧气,分别经过下列过程从外界吸收了$80\SIcal$的热量。(1)若为等温过程,求终态体积。(2)若为等体过程,求终态压强。(3)若为等压过程,求气体内能的变化。设氧气可看作理想气体,且$C_V^{\rm mol} = \frac{5}{2}R$。
  \ech
\end{frame}

\begin{frame}
  \chtitle{\proid 解答}
  \bch
  摩尔数$\nu = \frac{16\SIg}{32\SIg/\SImol} = 0.5\SImol$
  
  体积$V_0 = 22.4\SIL/\SImol \times 0.5\SImol = 11.2\SIL$

  $Q = 80\times 4.185 \SIJ = 335\SIJ$
  \bitem
\item{等温过程$Q = -A = \nu RT \ln\frac{V}{V_0}$,由此推出$V = V_0e^{\frac{Q}{\nu RT}} = 15.0\SIL$}
\item{等体过程$T = T_0+ \frac{Q}{C_V} = 273.15\SIK + \frac{335}{0.5\times 5/2\times 8.31}\SIK = 305\SIK$, $p = p_0 T/T_0 = 1\atm \times 305/273.15 = 1.12 \atm = 1.13\times 10^5\SIPa$}
\item{等压过程$\gamma = 7/5$, 内能变化$\Delta U = C_V\Delta T = C_p\Delta T /\gamma = Q/\gamma = 239\SIJ$}
  \eitem
  
  \ech
\end{frame}

\stepcounter{problem}
\begin{frame}
  \chtitle{\proid (\stwo)}
  \bch
  教材166页习题3-4:

  室温下一定理想气体氧的体积为$2.3\SIL$,压强为$1.0\atm$,经过以多方过程体积变为$4.1\SIL$,压强变为$0.5\atm$。试求:(1)多方指数$n$;(2)内能的变化;(3)吸收的热量;(4)氧膨胀对外界所作的功。设氧的$C_V^{\rm mol}=\frac{5R}{2}$。
  \ech
\end{frame}

\begin{frame}
  \chtitle{\proid 解答}
  \bch
      {\small
        \bitem
      \item{
        多方过程$pV^n = C$,取对数并考虑其变化量,即$\Delta \ln p + n \Delta \ln V = 0$,故
        $$n = - \frac{\Delta ln p}{\Delta \ln V} = -\frac{\ln\frac{0.5}{1}}{\ln\frac{4.1}{2.3}} = 1.20 $$}
      \item{内能的变化$\Delta U  = C_V\Delta T = \frac{C_V^{\rm mol}}{R}\Delta (\nu RT) = \frac{5}{2}\Delta(pV) = -63.3\SIJ$}
      \item{吸收热量$Q = C_n \Delta T = \frac{\gamma-n}{1-n}C_V\Delta T = -C_V\Delta T = -\Delta U = 63.3\SIJ$}
      \item{对外做功$A'=-A = Q-\Delta U = 127\SIJ$}
        \eitem
        \skipline
        
注:本题也可以利用$A' = -\frac{\nu R \Delta T}{n-1} =-\frac{\Delta(pV)}{n-1}$计算对外做功。然后根据$\Delta U = Q-A'$来计算内能改变。        
  }
  \ech
\end{frame}

\stepcounter{problem}
\begin{frame}
  \chtitle{\proid (\sone)}
  \bch
  教材166页习题3-5:

  $1\SImol$的理想气体氦,原来的体积为$8.0\SIL$,温度为$27\cdeg$,设经过准静态绝热过程后体积倍压缩到$1.0\SIL$,求在压缩过程中外界对系统所作的功。设氦的$C_V^{\rm mol} = \frac{3}{2}R$。
  \ech
\end{frame}

\begin{frame}
  \chtitle{\proid 解答}
  \bch
  $$T=T_0\left(\frac{V_0}{V}\right)^{\gamma-1} = 300\SIK\times 8^{2/3} = 1200\SIK $$
  绝热过程$Q=0$,
  $$A = \Delta U =C_V\Delta T  = 1.12\times 10^4\SIJ$$
  \ech
\end{frame}

\stepcounter{problem}
\begin{frame}
  \chtitle{\proid (\sthree)}
  \bch
  教材168页习题3-15:

  试证明:按绝热大气模型,高度$h$与压强$p$的关系为
  $$ h = \frac{C_p^{\rm mol}T_0}{M^{\rm mol}g}\left[1-\left(\frac{p}{p_0}\right)^{1-\frac{1}{\gamma}}\right],$$
  式中$p_0$和$T_0$为地面$h=0$处的压强和温度。
  \ech
\end{frame}

\begin{frame}
  \chtitle{\proid 解答}
  \bch
  {\small
    大气密度为$ \rho = \frac{M^{\rm mol}}{V^{\rm mol}} = \frac{M^{\rm mol}p}{RT} $,   按绝热大气模型,$T = \left(\frac{p}{p_0}\right)^{1-\frac{1}{\gamma}} T_0$。
    
  由力学平衡,考虑一个面积为$S$的水平空气薄层的力学平衡$-S dp = (\rho S dz)g $, 即
  $$ dz =-\frac{dp}{\rho g} = -\frac{RT }{M^{\rm mol}gp}dp= -\frac{RT_0}{M^{\rm mol}gp_0^{1-\frac{1}{\gamma}}} p^{-\frac{1}{\gamma}} dp$$
  从$0$到$h$积分,得到
  $$ h = \frac{\gamma}{\gamma-1}\frac{RT_0}{M^{\rm mol}g}\left[1-\left(\frac{p}{p_0}\right)^{1-\frac{1}{\gamma}}\right]$$
  再利用$\frac{\gamma}{\gamma-1} = \frac{C_p^{\rm mol}}{C_p^{\rm mol}-C_V^{\rm mol}} = \frac{C_p^{\rm mol}}{R}$
    即得证。
  }
  \ech
\end{frame}

\stepcounter{problem}
\begin{frame}
  \chtitle{\proid (\stwo)}
  \bch
  教材167页习题3-10:

  证明:当$\gamma$为常数时,若理想气体在某一过程中的热容量也是常量,则这个过程一定是多方过程。
  \ech
\end{frame}


\begin{frame}
  \chtitle{\proid 解答}
  \bch
  {\small
  设过程的热容为$C_X$。由$\gamma$为常数知$C_V$为常数。

  由热一律$\dbar Q = dU + pdV$以及$\dbar Q = C_X dT$, $dU = C_V dT$,得到
  $$ C_X dT = C_V dT + p dV$$
  若$C_X = C_V$,则$dV = 0$,即已经是等体过程(多方过程$n=\infty$的特殊情况)。
    
    否则,两边同乘以$\nu R$, 再由$\nu R dT =  (pdV + Vdp) $,
  $$ (C_V-C_X + \nu R) pdV + (C_V-C_X) Vdp = 0$$
  令$n =\frac{C_V-C_X + \nu R}{C_V-C_X }$,则
  $$ \frac{dp}{p} + n\frac{dV}{V} = 0$$
  或更明确地写成
  $$ d \ln (pV^n) = 0$$
  即$pV^n$为常数。
  }
  \ech
\end{frame}

\stepcounter{problem}
\begin{frame}
  \chtitle{\proid (\stwo)}
  \bch
  教材168页习题3-17:

  (题图请参考教材)
  \ech
\end{frame}

\begin{frame}
  \chtitle{\proid 解答}
  \bch
  {\scriptsize
    右边是绝热过程,右侧气体末态温度为
    $T_R = T_0\left(\frac{p_0}{p}\right)^{\frac{1}{\gamma}-1} = \frac{3}{2}T_0$
    
    终压强$p_L=p_R=\frac{27}{8}p_0$,右侧终体积
    $$V_R = V_0\left(\frac{p_0}{p_R}\right)^{\frac{1}{\gamma}} = \frac{4}{9}V_0$$
    $$C_V = \frac{\nu R}{\gamma - 1} = 2 \nu R = \frac{2p_0V_0}{T_0}$$
  
    \bitem
  \item[1]{对右边气体做功等于右边气体内能的增加
    $$A = C_V(T_R-T_0) = \frac{2p_0V_0}{T_0}\left(\frac{3}{2}T_0-T_0\right) = p_0V_0$$}
  \item[2]{上面已经算出右侧气体的终温为$\frac{3}{2}T_0$}
  \item[3]{左侧气体终体积为$V_L=2V_0-V_R = \frac{14}{9}V_0$,温度
    $$T_L = \frac{p_LV_L}{\nu R} = \frac{21}{4}T_0$$}
  \item[4]{左侧气体吸热用于做功和增加左侧气体内能:
    $$ Q = \Delta U_L + A = C_V(T_L-T_0) + p_0V_0 = \frac{19}{2}p_0V_0$$}
    \eitem
  }
  \ech
\end{frame}

\stepcounter{problem}
\begin{frame}
  \chtitle{\proid (\sthree)}
  \bch
  教材习题3-20:

  $1\SImol$氩气从初始温度$300\SIK$和初始体积$10^{-3}\SIm^3$分别经过下列三过程膨胀到$2\times 10^{-3}\SIm^3$的体积,计算温度的降低。
  \bitem
\item[(1)]{自由膨胀}
\item[(2)]{可逆绝热膨胀}
\item[(3)]{绝热节流膨胀}
  \eitem
  设$C_V^{\rm mol} = 12.6\SIJ/(\SImol\cdot\SIK)$, $C_p^{\rm mol} = 20.9\SIJ/(\SImol\cdot\SIK)$,在(3)中设氩气满足范德瓦尔斯方程,其中$a = 0.136\SIm^6\cdot Pa/\SImol^2$, $b =3.22\times 10^{-5}\SIm^3/\SImol$。
  \ech
\end{frame}


\begin{frame}
  \chtitle{\proid 解答}
  \bch
  {\scriptsize
    (1)因未指明自由膨胀中是否绝热,无法计算温度变化。
    
  (2)可逆绝热膨胀用理想气体近似
  $$T_2 = T_1 \left(\frac{V_1}{V_2}\right)^{\gamma-1} = 190 \SIK $$
    即温度降低了$110\SIK$。
    
  (3) 对范德瓦尔斯气体,由状态方程
$$\left(p + \frac{\nu^2 a}{V^2}\right) \left(V - \nu b\right) = \nu RT$$
以及焓和状态方程的关系,得到
$$\pfrac HpT = V \left(\delta_V - \frac{2\delta_p}{1+2\delta_p}\right) \frac{1+2\delta_p}{1-2\delta_p\left(1-\delta_V\right)}$$
其中$\delta_p \equiv \frac{\frac{\nu^2a}{V^2}}{p+\frac{\nu^2 a}{V^2}}$为压强修正的大小,$\delta_V\equiv \frac{\nu b}{V}$为体积修正的大小。
  }
  \ech
\end{frame}

\begin{frame}
  \chtitle{\proid 解答(续)}
  \bch
      {\scriptsize
        易估算出初态和末态的$\delta_p\approx 0.05, \delta_V \approx 0.03$,作为粗略的估算,我们用线性近似,并取$\delta_p\approx 0.05,\delta_V=0.03$对线性项对系数做二级修正:
        $$\pfrac HpT \approx 1.2 V\left(\frac{\nu b}{V} - 1.7\frac{a\nu^2}{pV^2}\right) = 1.2 \nu b - 2.0\frac{a\nu}{RT}$$
        于是由三变量的新链式法则易得焦耳-汤姆孙系数
        $$\pfrac TpH = -\frac{\pfrac HpT}{C_p} \approx \frac{2.0\frac{a}{RT}-1.2b}{C_p^{\rm mol}} =  3.37\times 10^{-6}\SIK/\SIPa$$
        由状态方程得到
        $$\Delta p = \nu RT\left(\frac{1}{V_2-\nu b}-\frac{1}{V_1-\nu b}\right) -a\nu^2\left(\frac{1}{V_2^2}-\frac{1}{V_1^2}\right)= -1.21\times 10^6\SIPa$$
        故
        $$\Delta T \approx \pfrac TpH \Delta p = -4.07\SIK$$
        (跟书上较为精确的答案差了9\%)
  }
  \ech
\end{frame}

\stepcounter{problem}
\begin{frame}
  \chtitle{\proid (\sthree)}
  \bch
  (摘自你们的微信群讨论)

  火箭通过高速喷射燃气产生推力,设温度为$T_1$,压强$p_1$的炽热高压气体在燃烧室不断生成,并通过管道由狭窄的喷气口排入气压$p_2$的环境,假设燃气可视为理想气体,其摩尔质量为$\mu$,每摩尔燃气内能为$C_VT$($C_V$为常量,$T$为燃气温度),在快速流动过程中,对管道内任意处两个非常靠近对横截面间对气体,可以认为它与周围没有热交换,但其内部则达到平衡状态且满足绝热方程。求喷气口处气体的温度与相对火箭的喷射速率。
  \ech
\end{frame}


\begin{frame}
  \chtitle{\proid 解答}
  \bch
  绝热过程$T\propto p^{\frac{R}{C_V+R}}$,故
  $$T_2 = T_1\left(\frac{p_2}{p_1}\right)^{\frac{R}{C_V+R}}$$
  由于是连续不断的过程,内能损失率和动能产生率平衡:
  $$  C_V(T_1-T_2) =  \frac{1}{2}\mu\upsilon^2$$
  即
  $$\upsilon = \sqrt{\frac{2C_VT_1\left[1-\left(\frac{p_2}{p_1}\right)^{\frac{R}{C_V+R}}\right]}{\mu}}$$
  
  \ech
\end{frame}

\stepcounter{problem}
\begin{frame}
  \chtitle{\proid (\stwo)}
  \bch
  \addfig{1.5}{nncycle.png}

  如图,某定体热容$C_V$为常数的理想气体的可逆循环由两个绝热过程和$n=2$的多方过程组成。其中的绝热压缩过程(即CD线)中温度升高一倍。求该循环的效率$\eta$。
  \ech
\end{frame}

\begin{frame}
  \chtitle{\proid 解答}
  \bch
  按照第11讲讨论,可逆多方循环的效率为$1-T_2/T_1$,其中$T_1$, $T_2$为任意一条绝热线上的高温值和低温值。故本题答案为$\eta = 1 - \frac{1}{2} = \frac{1}{2}$。
  \ech
\end{frame}

\stepcounter{problem}
\begin{frame}
  \chtitle{\proid (\sthree)}
  \bch
  \addfig{1.5}{Tpcycle.png}

  如图,某单原子理想气体的可逆循环ABCD。已知等温压缩过程CD的温度为$T$,等压膨胀过程中温度从$T_A=\frac{3}{2}T$升高到$T_B=2T$。求该循环的效率。
  
  \ech
\end{frame}

\begin{frame}
  \chtitle{\proid 解答}
  \bch
  {\scriptsize
  单原子理想气体的$\gamma = \frac{5}{3}$, $C_p = \frac{5}{2}\nu R$。易得等压膨胀过程中吸热
  $$Q_1 = C_p(T_B-T_A) = \frac{5}{4}\nu R T$$

  绝热过程中$p \propto T^{5/2}$.
  所以
  $$\frac{p_c}{p_B} = \left(\frac{T_c}{T_B}\right)^{5/2} = \left(\frac{1}{2}\right)^{5/2}$$
  $$\frac{p_D}{p_A} =\left(\frac{T_D}{T_A}\right)^{5/2} = \left(\frac{2}{3}\right)^{5/2}$$
  再由$p_A=p_B$,得到
  $$\frac{p_D}{p_C} =   \left(\frac{4}{3}\right)^{5/2}$$
  由此可算出等温压缩过程放热
  $$ Q_2' = \nu R T \ln \frac{V_C}{V_D} = \nu R T \ln \frac{p_D}{p_C} = \frac{5\nu RT}{2} \ln\frac{4}{3} $$
  从而得出循环效率
  $$\eta = 1 - \frac{Q_2'}{Q_1} = 1 - \frac{\frac{5}{2}\ln\frac{4}{3}}{\frac{5}{4}} = 1-2\ln\frac{4}{3} = 0.425 $$
  }
  \ech
\end{frame}

\stepcounter{problem}
\begin{frame}
  \chtitle{\proid (\sfour)}
  \bch
  光子气体的状态方程为
  $$ U = a T^4 V$$
  其中$a$为常量。考虑光子气体的一个准静态循环:先由体积$V_2$等体加热升温,然后绝热膨胀直至体积为$V_1$,然后等体放热降温,最后绝热压缩至初始状态。求这个循环的效率。
  \ech
\end{frame}


\begin{frame}
  \chtitle{\proid 解答}
  \bch
  {\scriptsize

\begin{equation}
    \pfrac STV = \frac{1}{T}\pfrac UTV = 4aT^2 V \label{eq1}
\end{equation}
极端相对论气体压强为能量密度的$1/3$,即
$$p = \frac{1}{3}aT^4$$
又根据$ dF = -SdT - pdV$为全微分,得到
\begin{equation}
  \pfrac SVT = \pfrac pTV = \frac{4}{3}aT^3 \label{eq2}
\end{equation}
  结合\eqref{eq1}和\eqref{eq2}得到
  $$ S = S_0 + \frac{4}{3} aT^3 V $$
  
  故两个等体过程在$T$-$S$图上是成正比的,比例为$\left(\frac{V_2}{V_1}\right)^{1/3}$。
  效率为$1$减去两条等体线下面积之比,即
  $$\eta = 1 - \left(\frac{V_2}{V_1}\right)^{1/3}$$
  }
  \ech
\end{frame}


\section{Clapeyron Eq.}
\setcounter{chap}{5}
\setcounter{problem}{0}

\begin{frame}
  \chtitle{第五篇 克拉珀龙方程的应用}
  \bch
  \bmini{0.6}
  \lfig{2.5}{PVTdiagram.png}
  \emini
  \bmini{0.36}
  克拉珀龙方程是准静态过程吸热公式(准静态过程吸热量等于按照$C_V$计算的吸热量,加上热压强做功消耗的能量)在相变时$dT=0$的特殊情形。
  \emini
  
  \tbox{  $$\pfrac pTV = \frac{\Lambda^{\rm mol}}{T\left(V^{\rm mol}_\beta - V^{\rm mol}_\alpha\right)} $$
  }
  \ech
\end{frame}

\begin{frame}
  \chtitle{汽化过程的克拉珀龙方程的近似}
  \bch
  在汽化过程中,液态的摩尔体积可以忽略。气态的摩尔体积按理想气体近似为$RT/p$。故
  $$ \frac{dp}{dT} = \frac{p\Lambda^{\rm mol}}{RT^2}$$
  或者写成
{\blue  $$ \Lambda^{\rm mol} = RT \frac{d \ln p}{d \ln T} $$}
  \ech
\end{frame}


\begin{frame}
  \chtitle{蒸气压方程}
  \bch
  在一定温度范围内可以对摩尔汽化热$\Lambda^{\rm mol}$作线性近似即可积分得到
  $$\ln\left(\frac{p}{p_0}\right) = A\left(\frac{1}{T_0} - \frac{1}{T}\right) + B \ln\left(\frac{T}{T_0}\right)$$
  上式称为蒸气压方程。
  \ech
\end{frame}

\stepcounter{problem}
\begin{frame}
  \chtitle{\proid (\sone)}
  \bch
  为什么在远低于$100\cdeg$的室温下,空气中仍含有水蒸气?空气中含有的水蒸气比例有上限吗?平时所说的湿度70\%是什么意思?
  \ech
\end{frame}


\begin{frame}
  \chtitle{\proid 解答}
  \bch
  $100\cdeg$是一个大气压下的水的沸点。按照克拉珀龙方程,低压低沸点。空气中水蒸气的分压很小,远低于一个大气压。故沸点可以低于室温。

  \skipline

  如果持续增大空气湿度(即增大水蒸气分压),使得沸点超过室温,那么水滴会自动从空气中凝结出来(降雨,露珠等自然现象)。这个临界的水蒸气分压,称为饱和蒸气压(水的蒸发曲线上对应于室温的压强值)。稳定情况下,空气中的水蒸气比例不能超过饱和蒸气压所限定的水蒸气分压上限。

  \skipline
  
  平时所说的湿度70\%,是指水蒸气分压是饱和蒸气压的70\%。当湿度达到100\%,就会出现水滴自动凝结的现象。
  \ech
\end{frame}

\stepcounter{problem}
\begin{frame}
  \chtitle{\proid (\sone)}
  \bch
  教材225页习题4-8:

  水从温度$99\cdeg$升高到$101\cdeg$时,饱和蒸气压从$733.7\SImmHg$增大到$788.0\SImmHg$。假定这时水蒸气可看作理想气体,求$100\cdeg$时水的汽化热。
  \ech
\end{frame}


\begin{frame}
  \chtitle{\proid 解答}
  \bch
  {\small
  饱和蒸气压即气液共存态时的压强。
  
  按照克拉珀龙方程的汽化近似,
  
  $$\Lambda^{\rm mol} = RT \frac{\Delta ln p}{\Delta \ln T} = 8.314\times 373.15 \times \frac{\ln\frac{788}{733.7}}{\ln\frac{374.15}{372.15}} \SIJ/\SImol = 4.13\times 10^4 \SIJ/\SImol$$
  又$1\SImol$水为$0.018\SIkg$,故按习惯把汽化热写成
  $$ \Lambda = \frac{4.13\times 10^4}{0.018} \SIJ/\SIkg = 2.29\times 10^6 \SIJ/\SIkg$$
  }
  \ech
\end{frame}

\stepcounter{problem}
\begin{frame}
  \chtitle{\proid (\stwo)}
  \bch
  教材226页习题4-15:

  固态氨的蒸气压方程和液态氨的蒸气压方程分别为
  
  固态$\ln(p/\SImmHg) = 23.3 - \frac{3754\SIK}{T}$

  液态$\ln(p/\SImmHg) = 19.49 -\frac{3063\SIK}{T}$

  求:

  (1)三相点的压强和温度。

  (2)三相点的汽化热,熔化热和升华热。
  \ech
\end{frame}


\begin{frame}
  \chtitle{\proid 解答}
  \bch
      {\scriptsize
        三相点既满足固态蒸气压方程,又满足液态蒸气压方程。
        $$\ln\left(\frac{\pthree}{\SImmHg}\right) = 23.3-\frac{3754\SIK}{\Tthree} = 19.49 - \frac{3063 \SIK}{\Tthree}$$
        解出
        $$\Tthree = 181.36\SIK;\ \pthree = 13.48\SImmHg$$
        忽略固体和液体的体积,计算出升华热:
        $$\Lambda^{\rm mol}_{\rm sublimate} = R\Tthree \frac{d\ln p}{d\ln T} = R\Tthree \frac{ \frac{3754\SIK }{\Tthree^2} dT }{\frac{1}{\Tthree} dT} = 3.121\times 10^4 \SIJ/\SImol $$
        汽化热
        $$\Lambda^{\rm mol}_{\rm evaporate} = R\Tthree \frac{d\ln p}{d\ln T} = R\Tthree \frac{ \frac{3063\SIK }{\Tthree^2} dT }{\frac{1}{\Tthree} dT} = 2.547\times 10^4 \SIJ/\SImol $$
        在三相点有固定的压强,所以各种潜热即是各种物态的焓差,由此得熔化热为
        $$\Lambda^{\rm mol}_{\rm melt} = \Lambda^{\rm mol}_{\rm sublimate} - \Lambda^{\rm mol}_{\rm evaporate} = 5.74 \times 10^3 \SIJ/\SImol$$
        
      }
  \ech
\end{frame}

\stepcounter{problem}
\begin{frame}
  \chtitle{\proid (\stwo)}
  \bch
  教材226页习题4-13:

  证明相变时内能的变化为:
  $$U_2^{\rm mol}-U^{\rm mol}_1 = \Lambda^{\rm mol}\left(1-\frac{d\ln T}{d\ln p}\right).$$
  \ech
\end{frame}


\begin{frame}
  \chtitle{\proid 解答}
  \bch
      {\small
        考虑$1\SImol$物质,根据热一律
        $$ \Delta U = Q + A = \Lambda^{\rm mol} - p \Delta V$$
        又温度不变时,吸热量等于热压强做功
        $$\Lambda^{\rm mol} = \pfrac p{\ln T}V \Delta V = p\Delta V \pfrac {\ln p}{\ln T}V $$
        结合上面两式即得证。
      }
  \ech
\end{frame}

\section{Homework}

\begin{frame}
  \chtitle{第14周作业(序号接第13周)}
  \bch
  {\small 
    \bitem
\item[36]{教材习题3-6}
\item[37]{教材习题3-14}
\item[38]{教材习题4-9}
  \eitem
  }
  \ech
\end{frame}

\end{document}

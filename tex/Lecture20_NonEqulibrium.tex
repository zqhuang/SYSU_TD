\documentclass[CJK]{beamer}
\usepackage{CJKutf8}
\usepackage{beamerthemesplit}
\usetheme{Malmoe}
\useoutertheme[footline=authortitle]{miniframes}
\usepackage{amsmath}
\usepackage{amssymb}
\usepackage{graphicx}
\usepackage{eufrak}
\usepackage{color}
\usepackage{slashed}
\usepackage{simplewick}
\usepackage{tikz}
\graphicspath{{../figures/}}
\def\addfig#1#2{\begin{center}\includegraphics[width=#1 in]{#2}\end{center}}
\def\blacktext#1{{\color{black}#1}}
\def\bluetext#1{{\color{blue}#1}}
\def\redtext#1{{\color{red}#1}}
\def\darkbluetext#1{{\color[rgb]{0,0.2,0.6}#1}}
\def\skybluetext#1{{\color[rgb]{0.2,0.7,1.}#1}}
\def\cyantext#1{{\color[rgb]{0.,0.5,0.5}#1}}
\def\greentext#1{{\color[rgb]{0,0.7,0.1}#1}}
\def\darkgray{\color[rgb]{0.2,0.2,0.2}}
\def\lightgray{\color[rgb]{0.6,0.6,0.6}}
\def\gray{\color[rgb]{0.4,0.4,0.4}}
\def\blue{\color{blue}}
\def\red{\color{red}}
\def\green{\color{green}}
\def\darkblue{\color[rgb]{0,0.2,0.6}}
\def\skyblue{\color[rgb]{0.2,0.7,1.}}
\def\fdeg{{^\circ \mathrm{F}}}
\def\cdeg{^\circ \mathrm{C}}
\def\be{\begin{equation}}
\def\ee{\nonumber\end{equation}}
\def\bea{\begin{eqnarray}}
\def\eea{\nonumber\end{eqnarray}}
\def\ii{{\dot{\imath}}}
\def\bch{\begin{CJK}{UTF8}{gbsn}}
\def\ech{\end{CJK}}
\def\bitem{\begin{itemize}}
\def\eitem{\end{itemize}}
\def\bcenter{\begin{center}}
\def\ecenter{\end{center}}
\def\bex{\begin{minipage}{0.3\textwidth}\includegraphics[width=1in]{jugelizi.png}\end{minipage}\begin{minipage}{0.6\textwidth}}
\def\eex{\end{minipage}}
\def\chtitle#1{\frametitle{\bch#1\ech}}
\def\skipline{{\vskip0.1in}}
\def\skiplines{{\vskip0.2in}}
\def\lagr{{\mathcal{L}}}
\def\hamil{{\mathcal{H}}}
\def\vecv{{\mathbf{v}}}
\def\vecx{{\mathbf{x}}}
\def\vecy{{\mathbf{y}}}
\def\veck{{\mathbf{k}}}
\def\vecp{{\mathbf{p}}}
\def\vecn{{\mathbf{n}}}
\def\vecA{{\mathbf{A}}}
\def\vecP{{\mathbf{P}}}
\def\vecsigma{{\mathbf{\sigma}}}
\def\hatJn{{\hat{J_\vecn}}}
\def\hatJx{{\hat{J_x}}}
\def\hatJy{{\hat{J_y}}}
\def\hatJz{{\hat{J_z}}}
\def\hatj#1{\hat{J_{#1}}}
\def\hatphi{{\hat{\phi}}}
\def\hatq{{\hat{q}}}
\def\hatpi{{\hat{\pi}}}
\def\vel{\upsilon}
\def\Dint{{\mathcal{D}}}
\def\adag{{\hat{a}^\dagger}}
\def\bdag{{\hat{b}^\dagger}}
\def\cdag{{\hat{c}^\dagger}}
\def\ddag{{\hat{d}^\dagger}}
\def\hata{{\hat{a}}}
\def\hatb{{\hat{b}}}
\def\hatc{{\hat{c}}}
\def\hatd{{\hat{d}}}
\def\hatN{{\hat{N}}}
\def\hatH{{\hat{H}}}
\def\hatp{{\hat{p}}}
\def\Fup{{F^{\mu\nu}}}
\def\Fdown{{F_{\mu\nu}}}
\def\newl{\nonumber \\}
\def\SIkm{\,\mathrm{km}}
\def\SIyr{\,\mathrm{yr}}
\def\SIGyr{\,\mathrm{Gyr}}
\def\SIeV{\,\mathrm{eV}}
\def\SIkeV{\,\mathrm{keV}}
\def\SIMeV{\,\mathrm{MeV}}
\def\SIGeV{\,\mathrm{GeV}}
\def\SIcal{\,\mathrm{cal}}
\def\SIkcal{\,\mathrm{kcal}}
\def\SImol{\,\mathrm{mol}}
\def\SIm{\,\mathrm{m}}
\def\SIcm{\,\mathrm{cm}}
\def\SIfm{\,\mathrm{fm}}
\def\SImm{\,\mathrm{mm}}
\def\SInm{\,\mathrm{nm}}
\def\SImum{\,\mathrm{\mu m}}
\def\SIJ{\,\mathrm{J}}
\def\SIkJ{\,\mathrm{kJ}}
\def\SIs{\,\mathrm{s}}
\def\SIkg{\,\mathrm{kg}}
\def\SIg{\,\mathrm{g}}
\def\SIK{\,\mathrm{K}}
\def\SImmHg{\,\mathrm{mmHg}}
\def\SIPa{\,\mathrm{Pa}}
\def\vece{\mathrm{e}}
\def\bmat#1{\left(\begin{array}{#1}}
\def\emat{\end{array}\right)}
\def\bcase#1{\left\{\begin{array}{#1}}
\def\ecase{\end{array}\right.}
\def\calM{{\mathcal{M}}}
\def\calT{{\mathcal{T}}}
\def\calR{{\mathcal{R}}}
\def\barpsi{\bar{\psi}}
\def\baru{\bar{u}}
\def\barv{\bar{\upsilon}}
\def\bmini#1{\begin{minipage}{#1\textwidth}}
\def\emini{\end{minipage}}
\def\qeq{\stackrel{?}{=}}
\def\torder#1{\mathcal{T}\left(#1\right)}
\def\rorder#1{\mathcal{R}\left(#1\right)}
\def\contr#1#2{\contraction{}{#1}{}{#2}#1#2}
\def\trof#1{\mathrm{Tr}\left(#1\right)}
\def\trace{\mathrm{Tr}}
\def\comm#1{\ \ \ \left(\mathrm{used}\ #1\right)}
\def\tcomm#1{\ \ \ (\text{#1})}
\def\slp{\slashed{p}}
\def\slk{\slashed{k}}
\def\wulian{\includegraphics[width=0.18in]{emoji_wulian.jpg}}
\def\bye{\includegraphics[width=0.18in]{emoji_bye.jpg}}
\def\calp{{\mathfrak{p}}}
\def\veccalp{\mathbf{\mathfrak{p}}}
\def\atm{\,\mathrm{atm}}
\def\angstrom{\,\text{\AA}}
\def\Tthree{T_{\tiny \textcircled{3}}}
\def\pthree{p_{\tiny \textcircled{3}}}

\def\courseurl{http://zhiqihuang.top}

\def\tpage#1#2{
\begin{frame}
\bch
\begin{center}
\begin{large}
热学 \\
第#1讲 #2

\end{large}

\skiplines

黄志琦


\end{center}

\skiplines

{\small 
教材:《热学》第二版,赵凯华,罗蔚茵,高等教育出版社


课件下载
}
\courseurl 
\ech
\end{frame}
}

\def\bfr#1{
\begin{frame}
\chtitle{#1} 
\bch
}

\def\efr{
\ech 
\end{frame}
}

\title{Lesson 20 - Non-Equilibrium Process}
  \author{}
  \date{}
\begin{document}
\tpage{20}{非平衡过程}

\begin{frame}
\chtitle{本讲内容}
\bch
\bitem
\item{平衡判据}
\item{化学势}
\item{热传导方程}  
\eitem
\ech
\end{frame}



\section{Equilibrium Conditions}


\begin{frame}
\chtitle{眼花撩乱的平衡判据}
\bch
教材从198页开始叨念各种(完全看不懂的)“平衡判据”:

\bitem
\item{熵判据:在内能和体积不变的情况下,对于一切可能的变动来说,平衡态的熵最大。}
\item{自由能判据:在温度和体积不变的情况下,对于一切可能的变动来说,平衡态的自由能最小。}
\item{自由焓判据:在温度和压强不变的情况下,对于一切可能的变动来说,平衡态的自由焓最小。}
\eitem

\addfig{1.5}{bzdsh.jpg}
\ech
\end{frame}

\begin{frame}
\chtitle{背书宝的福音}
\bch
\addfig{1}{beishubao.jpg}

要写出“$X$判据”,

\bitem
\item{第一步:把$X$写成全微分:$dX + \ldots dY + \ldots dZ = 0$}
\item{第二步:背书。在$Y$和$Z$不变的情况下,对于一切可能的变动来说,平衡态的$X$最$\ldots$。}
\item{第三步:判定是最大还是最小。$dX$, $dY$, $dZ$中必然有一个能量型的量(指$dU, dH, dF, dG$中的一个),它前面系数为正则为最小,为负则为最大。}
\eitem

{\scriptsize
似乎有点理解了:

$Y$和$Z$不变,则$dX = 0$,代表$X$是某种极值?

能量型的量当然有下限?
}
\ech
\end{frame}


\begin{frame}
\chtitle{例子1:熵判据}
\bch
写出
$$dS - \frac{1}{T} dU - \frac{p}{T} dV = 0 $$
$dU$前系数为负,取最大。

在内能$U$和体积$V$不变的情况下,对于一切可能的变动来说,平衡态的熵$S$最大。

\skiplines

熵判据也常常说成,孤立系的平衡态熵最大。内能和体积不变,是孤立系的一种。
\ech
\end{frame}


\begin{frame}
\chtitle{例子2:自由能判据}
\bch
写出
$$dF  + S dT + p dV = 0 $$
$dF$前系数为正,取最小。


在温度$T$和体积$V$不变的情况下,对于一切可能的变动来说,平衡态的自由能$F$最小。

\ech
\end{frame}


\begin{frame}
\chtitle{例子3:自由焓判据}
\bch
写出
$$dG + S dT - V dp = 0$$
$dG$前系数为正,取最小。

在温度$T$和压强$p$不变的情况下,对于一切可能的变动来说,平衡态的自由焓$G$最小。

\ech
\end{frame}


\begin{frame}
\chtitle{例子4: 压强判据之一}
\bch
写出
$$dp - \frac{1}{V} dH + \frac{T}{V} dS = 0$$
$dH$前系数为负,取最大。

在焓$H$和熵$S$不变的情况下,对于一切可能的变动来说,平衡态的压强$p$最大。

\ech
\end{frame}


\begin{frame}
  \chtitle{练习一下}
  \bch
  \addfig{1}{songfen.jpg}
  补充完整下列平衡判据:
  \bitem
\item[(1)]{在\_\_和\_\_不变的情况下,平衡态的焓$H$最\_\_。}
\item[(2)]{在自由能$F$和\_\_不变的情况下,平衡态的温度$T$最\_\_。}
\item[(3)]{在焓$H$和\_\_不变的情况下,平衡态的熵$S$最\_\_。}
\item[(4)]{在自由焓$G$和\_\_不变的情况下,平衡态的压强$p$最\_\_。}
  \eitem
  \ech
\end{frame}

\begin{frame}
\chtitle{真懂了吗}
\bchL
“一切可能的变动”是什么意思?

\addfig{1.5}{zhendongma.jpg}
\echL
\end{frame}

\begin{frame}
\chtitle{理解熵判据}
\bchL
“一切可能的变动”是指平衡态和一切非平衡态。


\addfig{1}{meimaobing.jpg}

熵判据的证明:根据孤立系的熵增大原理,任何非平衡态的熵会持续增大直至到达平衡态。显然平衡态的熵是最大的。

\echL
\end{frame}


\begin{frame}
\chtitle{理解自由能判据}
\bchL
自由能判据的条件是定温定体。可把“定温”理解为固定环境温度$T_e$。当偏离平衡态不远时,也可以理解为系统的平均温度。

\skiplines

证明:从非平衡态到平衡态的过程为不可逆过程。定体过程不对外做功,又由于恒温条件下对外做功小于自由能的减少(对不可逆过程)。故自由能的减少大于零。即非平衡态的自由能大于平衡态的自由能。

\echL
\end{frame}

\begin{frame}
\chtitle{理解自由焓判据}
\bchL
自由焓判据的条件是定温定压。可把“定温”理解为固定环境温度$T_e$。当偏离平衡态不远时,也可以理解为系统的平均温度。


\skiplines

证明:从非平衡态到平衡态的过程为不可逆过程。定温定压条件下,对不可逆过程,其他形式的做功小于自由焓的减少,又这里不考虑其他形式的做功(即没有额外安装发电机什么的),故自由焓的减少大于零。即非平衡态的自由焓大于平衡态的自由焓。

\echL
\end{frame}

\begin{frame}
\chtitle{三大平衡条件}
\bchL
\bitem
\item[1]{\blue 热平衡条件:系统内部温度均匀}
\item[2]{\blue 力学平衡条件:系统内部压强均匀}
\item[3]{\blue 相平衡条件:系统内各相化学势相等}
\eitem

{\scriptsize
注:第三条仅考虑了物态变化(如液态水和水蒸气之间的平衡),其实它可以推广到涉及化学反应的一般的化学平衡条件。化学反应比相变更为复杂,各种反应物或者产物有可能混合在一起产生额外的混合熵(请参考教材208页)。
}
\echL
\end{frame}


\begin{frame}
\chtitle{热平衡条件}
\bch
热平衡条件可以从孤立系的熵判据推出:如果有相邻的温度不相等的子系统,则由高温到低温的热量传输必然增大熵。这说明系统还未达到平衡态(熵最大的状态)。
\addfig{1.5}{eqc1.jpg}
$$\Delta S = \frac{-Q}{T_1} + \frac{Q}{T_2} > 0$$
\ech
\end{frame}

\begin{frame}
\chtitle{力学平衡条件}
\bch
力学平衡条件可以从定温定体系统的自由能判据推出:如果有相邻的压强不相等的子系统,则压强大的子系统等温膨胀$\Delta V$,压强小的子系统等温收缩$\Delta V$,可使系统温度和体积不变的情况下总自由能减小。这说明系统还未达到平衡态(自由能最低的态)。
\addfig{1.5}{eqc2.jpg}
$$\Delta F = -p_1 \Delta V + p_2 \Delta V < 0 $$
\ech
\end{frame}


\begin{frame}
\chtitle{相平衡条件}
\bch
设有种可以相互转化的物态。每种物态的粒子数分别为$N_1$, $N_2$, $\ldots$, $N_n$。每种物态成分的化学势即该成分的平均每个粒子的自由焓,即
$$ \mu_i \equiv \pfrac G{N_i}{T,p,N_1, N_2, \ldots, N_{i-1}, N_{i+1}, \ldots, N_n} $$

\addfig{1.5}{eqc3.jpg}

相平衡条件可以从定温定压系统的自由焓判据推出:假设已经达到热平衡和力学平衡,若定温定压条件下有$ \mu_i > \mu_j$,则可以通过$i\rightarrow j$的相变降低系统的总自由焓,这说明系统还未达到平衡态。
\ech
\end{frame}

\section{Chemical Potential}

\begin{frame}
\chtitle{经典理想气体的化学势}
\bch
{\small
考虑经典的理想气体,在$i$态上的分子数为
$$ n_i = e^{-\frac{\varepsilon_i-\mu}{kT}}$$
下面我们证明,这里的$\mu$和用自由焓定义的化学势相同}

{\scriptsize
  我们来计算温度为$T$的处于平衡态的单一成分理想气体的自由焓。
  $$G = U + pV - TS$$
  设第i个态上有$n_i$个分子,总分子数$N=\sum_i n_i$。因为要用到熵的绝对数值,我们就不能像以前那样直接忽略粒子的全同性了(混合气体的熵不同于单一成分气体的熵)。因为置换$N$个粒子共有$N!$种可能性。按照熵是状态数的对数的理解,按可区分粒子计算出来的熵则要减去$\ln (N!) \approx N \ln N - N $。最后,按照热力学的单位制,熵以$k$为单位,于是
  $$\frac{S}{k} = N\sum_i\left(-\frac{n_i}{N}\ln\frac{n_i}{N}\right)  - (N\ln N - N) = N-\sum_i (n_i\ln n_i) = N - \sum_in_i\frac{\mu - \varepsilon_i}{kT} =N+\frac{U-\mu N}{kT}  $$
  即
  $$TS = kNT + U - \mu N = pV + U - \mu N$$
  移项即得结论$$ G = \mu N$$
}
\ech
\end{frame}

\section{Heat Equation}

\begin{frame}
  \chtitle{一维热传导方程}
  \bch
  考虑一根很长的不良导体棒(这里指热的传导)的温度$T(x, t)$,其中$x$是在导体棒上的位置,$t$是时间。
  在温度梯度不大的情况下,可以用线性近似:{\blue 热流正比于温度梯度}。
  显然,这个比例系数是负的,
  $$ j =  -\lambda \frac{\partial T}{\partial x} $$
  其中$\lambda >0$为常量,称为{\blue 导热系数}。

  \addfig{2.2}{heatflux.png}

  {\small 注:热流是指单位时间通过单位面积的热量。}
  \ech
\end{frame}


\begin{frame}
  \chtitle{一维热传导方程}
  \bch

  \addfig{2.2}{heateq.png}
{\small
  考虑长为$dx$的一小段不良导体棒,进入和出去的热流差为
  $$j(x)-j(x+dx) = -\lambda \left[\left.\frac{\partial T}{\partial x}\right\vert_{x}-\left.\frac{\partial T}{\partial x}\right\vert_{x+dx}\right]\approx \lambda \frac{\partial^2 T}{\partial x^2} dx. $$
  设材料单位质量的比热为$c$,质量密度为$\rho$,横截面积为$S$,则
  $$ (c \rho S dx) dT = \dbar Q = [j(x)-j(x+dx)] S dt =  \lambda \frac{\partial^2 T}{\partial x^2} dx  (S dt) $$
  这里的$dT$和$dt$都是对固定$x$而言,两者之比为$\frac{\partial T}{\partial t}$。令$a = \frac{\lambda}{\rho c}$,则:\tbox{
  $$ \frac{\partial T}{\partial t} - a \frac{\partial^2T}{\partial x^2} = 0$$
  }

  }
  \ech
\end{frame}


\begin{frame}
  \chtitle{例题}
  \bch
  \addfig{2.5}{heatflux2.png}
  
  在一根长为$2L$的不良导体棒在$t=0$时刻温度为$T_0$。在$t>0$时刻,不良导体棒两端均有强度为$j$的热流进入。设材料的导热系数$\lambda$,质量密度$\rho$,单位质量的比热$c$均已知,试计算$t\ge 0$时刻不良导体棒各处的温度$T(x,t)$。
  \ech
\end{frame}

\begin{frame}
  \chtitle{ 解答}
  \bch
  根据对称性,在棒中间处热流和温度梯度均为零。写出如下的方程和边界条件:
  \bea
  \frac{\partial T}{\partial t} - a\frac{\partial^2 T}{\partial x^2} &=& 0 \newl
  \left.\frac{\partial T}{\partial x}\right\vert_{x=0} &=& 0 \newl
  \left.\frac{\partial T}{\partial x}\right\vert_{x=L} &=& \frac{j}{\lambda}  \newl
  \left.T\right\vert_{t=0} &=&  T_0 
  \eea
  其中$a = \frac{\lambda}{\rho c} $。
  
  \ech
\end{frame}


\begin{frame}
  \chtitle{解答}
  \bch
  先分析主要图像。

  \skiplines

  在$t$时刻,累计流入的热量为$Q =  2 j St$ (其中$S$为横截面积)。棒子热容为$C =  c \rho (2SL)$。所以$t$时刻棒子的平均温度为
  $$ \bar{T} =   T_0  + \frac{Q}{C} = T_0 + \frac{j}{\rho cL}t $$
  
  \ech
\end{frame}


\begin{frame}
  \chtitle{解答 (续)}
  \bch
  把平均温度去掉,研究各处温度起伏:$\Delta T(x, t) = T(x, t) - \left(T_0+\frac{j}{\rho cL} t\right)$。显然$\Delta T$满足方程:
  \bea
  \frac{\partial \Delta T }{\partial t} - a \frac{\partial^2 \Delta T}{\partial x^2} &=&  -\frac{j}{\rho cL} \newl
  \left.\frac{\partial \Delta T}{\partial x}\right\vert_{x=0} &=& 0 \newl
  \left.\frac{\partial \Delta  T}{\partial x}\right\vert_{x=L} &=& \frac{j}{\lambda}  \newl
  \left.\Delta T\right\vert_{t=0} = 0
  \eea
  因为$\Delta T$描述的是温度起伏,还有一个额外条件:
  $$\int_0^L \Delta T(x, t) dx = 0 $$
  \ech
\end{frame}

\begin{frame}
  \chtitle{解答 (续)}
  \bch
  当$t$很大时,棒上的温度梯度趋于稳定,即$\Delta T$仅仅依赖于$x$,满足
  \bea
  - a \frac{\partial^2 \Delta T}{\partial x^2} &=&  -\frac{j}{\rho cL} \newl
  \left.\frac{\partial \Delta T}{\partial x}\right\vert_{x=0} &=& 0 \newl
  \left.\frac{\partial \Delta  T}{\partial x}\right\vert_{x=L} &=& \frac{j}{\lambda}  \newl
  \int_0^L \Delta T(x, t) dx &=& 0  
  \eea
  由此不难解出
  $$\Delta T = \frac{j}{2\lambda} \left(\frac{x^2}{L} - \frac{L}{3}\right) $$
  \ech
\end{frame}

\begin{frame}
  \chtitle{课后思考题}
  \bch
  如果我们急于喝一杯奶茶,以下哪种冷却方法好?
  \bitem
\item[A]{先把热茶冷5分钟,加一匙冷牛奶。}
\item[B]{先把热茶冷2.5分钟,再加一匙冷牛奶,再冷2.5分钟。}
\item[C]{先将一匙冷牛奶加入热茶中,再冷却5分钟。}
\item[D]{以上效果一样}
  \eitem
  \ech
\end{frame}


\end{document}

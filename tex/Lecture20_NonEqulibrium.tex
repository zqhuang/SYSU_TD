\documentclass[CJK,14pt]{beamer}
\usepackage{CJKutf8}
\usepackage{beamerthemesplit}
\usetheme{Malmoe}
\useoutertheme[footline=authortitle]{miniframes}
\usepackage{amsmath}
\usepackage{amssymb}
\usepackage{graphicx}
\usepackage{eufrak}
\usepackage{color}
\usepackage{slashed}
\usepackage{simplewick}
\usepackage{tikz}
\graphicspath{{../figures/}}
\def\addfig#1#2{\begin{center}\includegraphics[width=#1 in]{#2}\end{center}}
\def\blacktext#1{{\color{black}#1}}
\def\bluetext#1{{\color{blue}#1}}
\def\redtext#1{{\color{red}#1}}
\def\darkbluetext#1{{\color[rgb]{0,0.2,0.6}#1}}
\def\skybluetext#1{{\color[rgb]{0.2,0.7,1.}#1}}
\def\cyantext#1{{\color[rgb]{0.,0.5,0.5}#1}}
\def\greentext#1{{\color[rgb]{0,0.7,0.1}#1}}
\def\darkgray{\color[rgb]{0.2,0.2,0.2}}
\def\lightgray{\color[rgb]{0.6,0.6,0.6}}
\def\gray{\color[rgb]{0.4,0.4,0.4}}
\def\blue{\color{blue}}
\def\red{\color{red}}
\def\green{\color{green}}
\def\darkblue{\color[rgb]{0,0.2,0.6}}
\def\skyblue{\color[rgb]{0.2,0.7,1.}}
\def\fdeg{{^\circ \mathrm{F}}}
\def\cdeg{^\circ \mathrm{C}}
\def\be{\begin{equation}}
\def\ee{\nonumber\end{equation}}
\def\bea{\begin{eqnarray}}
\def\eea{\nonumber\end{eqnarray}}
\def\ii{{\dot{\imath}}}
\def\bch{\begin{CJK}{UTF8}{gbsn}}
\def\ech{\end{CJK}}
\def\bitem{\begin{itemize}}
\def\eitem{\end{itemize}}
\def\bcenter{\begin{center}}
\def\ecenter{\end{center}}
\def\bex{\begin{minipage}{0.3\textwidth}\includegraphics[width=1in]{jugelizi.png}\end{minipage}\begin{minipage}{0.6\textwidth}}
\def\eex{\end{minipage}}
\def\chtitle#1{\frametitle{\bch#1\ech}}
\def\skipline{{\vskip0.1in}}
\def\skiplines{{\vskip0.2in}}
\def\lagr{{\mathcal{L}}}
\def\hamil{{\mathcal{H}}}
\def\vecv{{\mathbf{v}}}
\def\vecx{{\mathbf{x}}}
\def\vecy{{\mathbf{y}}}
\def\veck{{\mathbf{k}}}
\def\vecp{{\mathbf{p}}}
\def\vecn{{\mathbf{n}}}
\def\vecA{{\mathbf{A}}}
\def\vecP{{\mathbf{P}}}
\def\vecsigma{{\mathbf{\sigma}}}
\def\hatJn{{\hat{J_\vecn}}}
\def\hatJx{{\hat{J_x}}}
\def\hatJy{{\hat{J_y}}}
\def\hatJz{{\hat{J_z}}}
\def\hatj#1{\hat{J_{#1}}}
\def\hatphi{{\hat{\phi}}}
\def\hatq{{\hat{q}}}
\def\hatpi{{\hat{\pi}}}
\def\vel{\upsilon}
\def\Dint{{\mathcal{D}}}
\def\adag{{\hat{a}^\dagger}}
\def\bdag{{\hat{b}^\dagger}}
\def\cdag{{\hat{c}^\dagger}}
\def\ddag{{\hat{d}^\dagger}}
\def\hata{{\hat{a}}}
\def\hatb{{\hat{b}}}
\def\hatc{{\hat{c}}}
\def\hatd{{\hat{d}}}
\def\hatN{{\hat{N}}}
\def\hatH{{\hat{H}}}
\def\hatp{{\hat{p}}}
\def\Fup{{F^{\mu\nu}}}
\def\Fdown{{F_{\mu\nu}}}
\def\newl{\nonumber \\}
\def\SIkm{\,\mathrm{km}}
\def\SIyr{\,\mathrm{yr}}
\def\SIGyr{\,\mathrm{Gyr}}
\def\SIeV{\,\mathrm{eV}}
\def\SIkeV{\,\mathrm{keV}}
\def\SIMeV{\,\mathrm{MeV}}
\def\SIGeV{\,\mathrm{GeV}}
\def\SIcal{\,\mathrm{cal}}
\def\SIkcal{\,\mathrm{kcal}}
\def\SImol{\,\mathrm{mol}}
\def\SIm{\,\mathrm{m}}
\def\SIcm{\,\mathrm{cm}}
\def\SIfm{\,\mathrm{fm}}
\def\SImm{\,\mathrm{mm}}
\def\SInm{\,\mathrm{nm}}
\def\SImum{\,\mathrm{\mu m}}
\def\SIJ{\,\mathrm{J}}
\def\SIkJ{\,\mathrm{kJ}}
\def\SIs{\,\mathrm{s}}
\def\SIkg{\,\mathrm{kg}}
\def\SIg{\,\mathrm{g}}
\def\SIK{\,\mathrm{K}}
\def\SImmHg{\,\mathrm{mmHg}}
\def\SIPa{\,\mathrm{Pa}}
\def\vece{\mathrm{e}}
\def\bmat#1{\left(\begin{array}{#1}}
\def\emat{\end{array}\right)}
\def\bcase#1{\left\{\begin{array}{#1}}
\def\ecase{\end{array}\right.}
\def\calM{{\mathcal{M}}}
\def\calT{{\mathcal{T}}}
\def\calR{{\mathcal{R}}}
\def\barpsi{\bar{\psi}}
\def\baru{\bar{u}}
\def\barv{\bar{\upsilon}}
\def\bmini#1{\begin{minipage}{#1\textwidth}}
\def\emini{\end{minipage}}
\def\qeq{\stackrel{?}{=}}
\def\torder#1{\mathcal{T}\left(#1\right)}
\def\rorder#1{\mathcal{R}\left(#1\right)}
\def\contr#1#2{\contraction{}{#1}{}{#2}#1#2}
\def\trof#1{\mathrm{Tr}\left(#1\right)}
\def\trace{\mathrm{Tr}}
\def\comm#1{\ \ \ \left(\mathrm{used}\ #1\right)}
\def\tcomm#1{\ \ \ (\text{#1})}
\def\slp{\slashed{p}}
\def\slk{\slashed{k}}
\def\wulian{\includegraphics[width=0.18in]{emoji_wulian.jpg}}
\def\bye{\includegraphics[width=0.18in]{emoji_bye.jpg}}
\def\calp{{\mathfrak{p}}}
\def\veccalp{\mathbf{\mathfrak{p}}}
\def\atm{\,\mathrm{atm}}
\def\angstrom{\,\text{\AA}}
\def\Tthree{T_{\tiny \textcircled{3}}}
\def\pthree{p_{\tiny \textcircled{3}}}

\def\courseurl{http://zhiqihuang.top}

\def\tpage#1#2{
\begin{frame}
\bch
\begin{center}
\begin{large}
热学 \\
第#1讲 #2

\end{large}

\skiplines

黄志琦


\end{center}

\skiplines

{\small 
教材:《热学》第二版,赵凯华,罗蔚茵,高等教育出版社


课件下载
}
\courseurl 
\ech
\end{frame}
}

\def\bfr#1{
\begin{frame}
\chtitle{#1} 
\bch
}

\def\efr{
\ech 
\end{frame}
}

\title{Lesson 20 - Non-Equilibrium Process}
  \author{}
  \date{}
\begin{document}
\bch
  
\tpage{20}{不可逆过程}

\begin{frame}
\frametitle{本讲内容}

\bitem
\item{可逆过程的量化理解}
\item{自由焓和自由能的名称由来}  
\item{平衡判据}
\eitem

\end{frame}


\section{Reversibility}

\secpage{可逆过程的量化理解}{可逆过程是$\Delta S\rightarrow 0^+$的理想极限}


\begin{frame}
  \frametitle{可逆过程是理想极限}

  原则上来讲,只要高温物体向低温物体传递了热量,熵就增大($\Delta S>0$),过程就不可逆。我们一直讨论的“可逆过程”只是$\Delta S\rightarrow 0^+$的一种理想极限。

  \addfig{1}{juhenduolizi.jpg}
  
  下面我们通过一些例子来加深理解。
  

\end{frame}


\begin{frame}
  \frametitle{思考题}

  \addfig{0.5}{think1.jpg}
  
  设温度为$T$的物体从温度为$T+\Delta T$的物体处吸收了热量$\Delta Q$,设$\Delta T$和$\Delta Q$都很小,估算两个物体的总熵变(取最低阶非零近似)。
\end{frame}


\begin{frame}
\frametitle{思考题}
\bch

\addfig{0.5}{think2.jpg}

试说明:如果把一个宏观的热传递过程划分为$N$步,每一步中,热传递物体之间的温差$\Delta T$和传递的热量$\Delta Q$满足$N\Delta T\Delta Q \rightarrow 0$,则该宏观过程可以看成是可逆的。

\ech
\end{frame}


\begin{frame}
\frametitle{思考题}


\addfig{0.5}{think2.jpg}

试说明:可逆的卡诺循环原则上可行。


\end{frame}


\begin{frame}
  \frametitle{思考题}

  \addfig{1}{think3.jpg}

试说明:对于教材191页的$N$个热库逐级加热的例子,当$N\rightarrow \infty$时可以理想为可逆过程。

\end{frame}


\begin{frame}
  \frametitle{思考题}

\addfig{3.1}{dTdx0.png}

如图,两个不等温热库之间通过很长的一段导热棒进行缓慢传热,导热棒上有稳恒的热流。这样的热传导过程可逆吗?

\end{frame}


\section{GF}
\secpage{自由焓和自由能的名称由来}{有了 G F,还能有自由?}


\begin{frame}
  \frametitle{定温过程中,对外做功不大于自由能的减少}

  设环境温度恒为$T$。和环境保持接触的系统初态是温度为$T$的平衡态,在某过程(不一定可逆)中做功$A'$,末态仍然是温度为$T$的平衡态。设过程中自由能改变了$\Delta F$。


  证明: $A'\le -\Delta F$,等号当且仅当过程可逆时成立。

  (也就是说:{\blue 自由能是定温条件下能自由用来做功的能量。})

\end{frame}


\begin{frame}
  \frametitle{定温过程中,对外做功不大于自由能的减少}
证明:设系统从环境吸热$Q$,则由热一律有$Q = A' + \Delta U$。
由$F$的定义,有
$$\Delta F = \Delta U - \Delta (TS) = \Delta U - T\Delta S $$
又根据热二律的微分形式,有$T\Delta S \ge Q = A' +\Delta U$。故
$$A' \le - \Delta F$$
等号当且仅当过程可逆时成立。
\end{frame}


\begin{frame}
\frametitle{定温定体过程}

在定温定体过程中,因为对环境不做功,那么$A'$就代表对除了环境之外的纯机械系统做的功。{\blue 自由能是定温定体条件下能自由用来(对除了环境之外的纯机械系统)做额外功的能量。}

  \skipline

  如果在定温定体过程中系统仅仅和环境接触,没有外接任何纯机械系统,那么$A'=0$
  $$\Delta F \le - A' = 0. $$
  即自由能只能减少。

\end{frame}


\begin{frame}
\frametitle{定温定压过程中,对环境之外的额外做功不大于自由焓的减少}

  设环境温度恒为$T$,压强恒为$p$。和环境保持接触的系统初态是温度为$T$,压强为$p$的平衡态,在某过程(不一定可逆)中对除了环境之外的纯机械系统{\blue 额外做功$A'_{\rm else}$}(例如,系统额外接了个发电机),末态仍然是温度为$T$,压强为$p$的平衡态。设过程中自由焓改变了$\Delta G$。

  证明: $A'_{\rm else}  \le -\Delta G$,等号当且仅当过程可逆时成立。


   也就是说:{\blue 自由焓是定温定压条件下能自由(对除了环境之外的纯机械系统)做额外功的能量。}
  
\end{frame}


\begin{frame}
\frametitle{定温定压过程中,对环境之外的额外做功不大于自由焓的减少}

证明: 显然系统对外做的总功 $$ A' = A'_{\rm else} + p \Delta V.$$

由于是定温条件下的过程,对外做功不大于自由能的减少:
$$ A' \le - \Delta F = -\Delta (G- pV) = - \Delta G + p \Delta V$$
两边减去$p\Delta V$即得
$$A'_{\rm else}  \le -\Delta G.$$

\end{frame}




\begin{frame}
\frametitle{定温定压过程中的自由焓单调减小性}
如果系统仅仅和定温定压的环境接触,没有外接任何机械系统,则$A'_{\rm else}=0$。
$$\Delta G \le - A'_{\rm else} = 0 ,$$
即自由焓只能单调减少。

\end{frame}


\section{Equilibrium Conditions}

\secpage{平衡判据}{平衡判据,研究的其实是非平衡态。}

\begin{frame}
\frametitle{平衡判据}

对$pVT$系统,我们经常假想一些虚变动(微扰),并使用下列平衡判据:
\bitem
\item{熵判据:在内能和体积不变的情况下,对于一切可能的虚变动来说,平衡态的熵最大。}
\item{自由能判据:在温度和体积不变的情况下,对于一切可能的虚变动来说,平衡态的自由能最小。}
\item{自由焓判据:在温度和压强不变的情况下,对于一切可能的虚变动来说,平衡态的自由焓最小。}
\eitem
\end{frame}


\begin{frame}

  看了这些判据后的感受是:

  \skipline

\addfig{1.2}{bzdsh.jpg}

\end{frame}


\begin{frame}
\frametitle{写平衡判据的套路}

对$pVT$系统,要写出“$X$判据”,

\bitem
\item{第一步:把$X$写成全微分:$dX + \ldots dY + \ldots dZ = 0$}
\item{第二步:背书。在$Y$和$Z$不变的情况下,对于一切可能的虚变动来说,平衡态的$X$最$\ldots$。}
\item{第三步:判定是“最大”还是“最小”。$dX$, $dY$, $dZ$中必然有一个能量型的量(指$dU, dH, dF, dG$中的一个),它前面系数为正则为“最小”,为负则为“最大”。}
\eitem

{\small

似懂非懂:$Y$和$Z$不变,则$dX = 0$,代表$X$是某种极值?能量型的量当然有下限?
}

\end{frame}


\begin{frame}
\frametitle{例子1:熵判据}

写出
$$dS - \frac{1}{T} dU - \frac{p}{T} dV = 0 $$
$dU$前系数为负,取“最大”。

\skipline

{\blue 在内能$U$和体积$V$不变的情况下,对于一切可能的变动来说,平衡态的熵$S$最大。}

\end{frame}



\begin{frame}
\frametitle{例子2:自由能判据}

写出
$$dF  + S dT + p dV = 0 $$
$dF$前系数为正,取“最小”。


\skipline

{\blue 在温度$T$和体积$V$不变的情况下,对于一切可能的变动来说,平衡态的自由能$F$最小。}


\end{frame}



\begin{frame}
\frametitle{例子3:自由焓判据}

写出
$$dG + S dT - V dp = 0$$
$dG$前系数为正,取“最小”。

\skipline

{\blue 在温度$T$和压强$p$不变的情况下,对于一切可能的变动来说,平衡态的自由焓$G$最小。}


\end{frame}




\begin{frame}
  \frametitle{练习一下}
  
  \addfig{1}{songfen.jpg}
  补充完整下列平衡判据:
  \bitem
\item{在\_\_和\_\_不变的情况下,平衡态的焓最\_\_。}
\item{在焓和\_\_不变的情况下,平衡态的熵最\_\_。}
  \eitem
  
\end{frame}


\begin{frame}
\frametitle{热,力,相平衡条件}

\bitem
\item[1]{\blue 热平衡条件:系统内部温度均匀}
\item[2]{\blue 力学平衡条件:系统内部压强均匀}
\item[3]{\blue 相平衡条件:系统内各相化学势相等}
\eitem

{\small
注:第三条仅考虑了物态变化(如液态水和水蒸气之间的平衡),其实它可以推广到涉及化学反应的一般的化学平衡条件。化学反应比相变更为复杂,各种反应物或者产物有可能混合在一起产生额外的混合熵(请参考教材208页)。
}
\end{frame}


\begin{frame}
\frametitle{热平衡条件}

热平衡条件可以从孤立系的熵判据推出:如果有相邻的温度不相等的子系统,则由高温到低温的热量传输必然增大熵。这说明系统还未达到平衡态(熵最大的状态)。
\addfig{1.4}{eqc1.jpg}
$$\Delta S = \frac{-Q}{T_1} + \frac{Q}{T_2} > 0$$
\end{frame}

\begin{frame}
\frametitle{力学平衡条件}

力学平衡条件可以从定温定体系统的自由能判据推出:如果有相邻的压强不相等的子系统,则压强大的子系统等温膨胀$\Delta V$,压强小的子系统等温收缩$\Delta V$,可使系统温度和体积不变的情况下总自由能减小。这说明系统还未达到平衡态(自由能最低的态)。
\addfig{1.4}{eqc2.jpg}
$$\Delta F = -p_1 \Delta V + p_2 \Delta V < 0 $$

\end{frame}


\begin{frame}
  \frametitle{化学势}
  设有$n$种可以相互转化的物态。每种物态的粒子数分别为$N_1$, $N_2$, $\ldots$, $N_n$。每种物态成分的{\blue 化学势即该成分增加一个粒子带来的自由焓变化},即
$$ \mu_i \equiv \pfrac G{N_i}{T,p,N_1, N_2, \ldots, N_{i-1}, N_{i+1}, \ldots, N_n} $$
\end{frame}
  

\begin{frame}
\frametitle{相平衡条件}

\addfig{1.4}{eqc3.jpg}

相平衡条件可以从定温定压系统的自由焓判据推出:假设已经达到热平衡和力学平衡,若定温定压条件下有$ \mu_i > \mu_j$,则可以通过$i\rightarrow j$的相变降低系统的总自由焓,这说明系统还未达到平衡态。

\end{frame}


\begin{frame}
\frametitle{化学势的物理意义}
对达到相平衡(化学平衡)的系统,如果人为地增加其中$i$相(化学成分)的粒子数,则$i$相(化学成分)的化学势$\mu_i$会升高,从而使$i$相(化学成分)更快地转变为其他相,直到$i$粒子数下降,其他粒子数增多,达到新的相平衡。

\skiplines

所以,可以把化学势粗略地理解为反映该种类粒子多寡的“势”。(嗯,人多势众。)

\end{frame}


\begin{frame}
\frametitle{附录1:经典理想气体的化学势}

考虑经典的平衡态理想气体,在第$i$($i=1,2,\ldots$)个微观态上的平均分子数
$$ N_i = c e^{-\frac{\varepsilon_i}{kT}} $$
我们把和$i$无关(但是可能依赖于宏观态函数)的常数$c$写成$e^{\frac{\mu}{kT}}$。
$$ N_i = e^{-\frac{\varepsilon_i-\mu}{kT}},$$

下面我们证明,这里的$\mu$和用自由焓定义的化学势相同。
\end{frame}


\begin{frame}
\frametitle{附录1:经典理想气体的化学势}

  温度为$T$的处于平衡态的单一成分理想气体的自由焓。
  $$G = U + pV - TS$$
  设第$i$个态上平均有$N_i=Np_i$个分子,其中$N$为总分子数,$p_i$为出现在第$i$个微观态的概率。假如按照概率熵的定义:
  \bea
  S = N(-\sum_i p_i\ln p_i) 
  &=&  -\sum_i N_i\ln\frac{N_i}{N} \newl
  &=& N\ln N -\sum_i N_i\ln N_i
  \eea
\end{frame}
\begin{frame}
  \frametitle{附录1:经典理想气体的化学势}
  {\small
  这个表达式其实并不正确,因为置换$N$个粒子共有$N!$种可能性。按照熵是状态数的对数的理解,按可区分粒子计算出来的熵则要减去$\ln (N!) \approx N \ln N - N $(以前没有考虑这个是因为以前研究的都是熵的变化,不在意熵的零点)。 除此之外,按照热力学的单位制,热力学的熵都要多乘个$k$,
  \bea
  S = k\left( N-\sum_i N_i\ln N_i\right) &=& k\left(N + \sum_iN_i\frac{\varepsilon_i-\mu}{kT}\right) \newl
  &=& k N+\frac{U-\mu N}{T}.
  \eea
  整理并利用$pV = nkTV =NkT$即可得到 $G = U + pV - TS = \mu N$.
}
\end{frame}


\begin{frame}
\frametitle{熵判据的物理解释}
虽然平衡判据里的“一切可能的变动”是指平衡态附近很小的虚变动。但在熵判据里,也可以理解为任何平衡态或非平衡态。

\skipline

孤立系的平衡态熵最大,是孤立系的熵增大原理的结果。
\end{frame}


\begin{frame}
\frametitle{自由能判据的物理解释}
在自由能判据里,一切可能的变动可以理解为

\skipline


\end{frame}


\ech
\end{document}

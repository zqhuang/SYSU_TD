\documentclass[CJK]{beamer}
\usepackage{CJKutf8}
\usepackage{beamerthemesplit}
\usetheme{Malmoe}
\useoutertheme[footline=authortitle]{miniframes}
\usepackage{amsmath}
\usepackage{amssymb}
\usepackage{graphicx}
\usepackage{eufrak}
\usepackage{color}
\usepackage{slashed}
\usepackage{simplewick}
\usepackage{tikz}
\graphicspath{{../figures/}}
\def\addfig#1#2{\begin{center}\includegraphics[width=#1 in]{#2}\end{center}}
\def\blacktext#1{{\color{black}#1}}
\def\bluetext#1{{\color{blue}#1}}
\def\redtext#1{{\color{red}#1}}
\def\darkbluetext#1{{\color[rgb]{0,0.2,0.6}#1}}
\def\skybluetext#1{{\color[rgb]{0.2,0.7,1.}#1}}
\def\cyantext#1{{\color[rgb]{0.,0.5,0.5}#1}}
\def\greentext#1{{\color[rgb]{0,0.7,0.1}#1}}
\def\darkgray{\color[rgb]{0.2,0.2,0.2}}
\def\lightgray{\color[rgb]{0.6,0.6,0.6}}
\def\gray{\color[rgb]{0.4,0.4,0.4}}
\def\blue{\color{blue}}
\def\red{\color{red}}
\def\green{\color{green}}
\def\darkblue{\color[rgb]{0,0.2,0.6}}
\def\skyblue{\color[rgb]{0.2,0.7,1.}}
\def\fdeg{{^\circ \mathrm{F}}}
\def\cdeg{^\circ \mathrm{C}}
\def\be{\begin{equation}}
\def\ee{\nonumber\end{equation}}
\def\bea{\begin{eqnarray}}
\def\eea{\nonumber\end{eqnarray}}
\def\ii{{\dot{\imath}}}
\def\bch{\begin{CJK}{UTF8}{gbsn}}
\def\ech{\end{CJK}}
\def\bitem{\begin{itemize}}
\def\eitem{\end{itemize}}
\def\bcenter{\begin{center}}
\def\ecenter{\end{center}}
\def\bex{\begin{minipage}{0.3\textwidth}\includegraphics[width=1in]{jugelizi.png}\end{minipage}\begin{minipage}{0.6\textwidth}}
\def\eex{\end{minipage}}
\def\chtitle#1{\frametitle{\bch#1\ech}}
\def\skipline{{\vskip0.1in}}
\def\skiplines{{\vskip0.2in}}
\def\lagr{{\mathcal{L}}}
\def\hamil{{\mathcal{H}}}
\def\vecv{{\mathbf{v}}}
\def\vecx{{\mathbf{x}}}
\def\vecy{{\mathbf{y}}}
\def\veck{{\mathbf{k}}}
\def\vecp{{\mathbf{p}}}
\def\vecn{{\mathbf{n}}}
\def\vecA{{\mathbf{A}}}
\def\vecP{{\mathbf{P}}}
\def\vecsigma{{\mathbf{\sigma}}}
\def\hatJn{{\hat{J_\vecn}}}
\def\hatJx{{\hat{J_x}}}
\def\hatJy{{\hat{J_y}}}
\def\hatJz{{\hat{J_z}}}
\def\hatj#1{\hat{J_{#1}}}
\def\hatphi{{\hat{\phi}}}
\def\hatq{{\hat{q}}}
\def\hatpi{{\hat{\pi}}}
\def\vel{\upsilon}
\def\Dint{{\mathcal{D}}}
\def\adag{{\hat{a}^\dagger}}
\def\bdag{{\hat{b}^\dagger}}
\def\cdag{{\hat{c}^\dagger}}
\def\ddag{{\hat{d}^\dagger}}
\def\hata{{\hat{a}}}
\def\hatb{{\hat{b}}}
\def\hatc{{\hat{c}}}
\def\hatd{{\hat{d}}}
\def\hatN{{\hat{N}}}
\def\hatH{{\hat{H}}}
\def\hatp{{\hat{p}}}
\def\Fup{{F^{\mu\nu}}}
\def\Fdown{{F_{\mu\nu}}}
\def\newl{\nonumber \\}
\def\SIkm{\,\mathrm{km}}
\def\SIyr{\,\mathrm{yr}}
\def\SIGyr{\,\mathrm{Gyr}}
\def\SIeV{\,\mathrm{eV}}
\def\SIkeV{\,\mathrm{keV}}
\def\SIMeV{\,\mathrm{MeV}}
\def\SIGeV{\,\mathrm{GeV}}
\def\SIcal{\,\mathrm{cal}}
\def\SIkcal{\,\mathrm{kcal}}
\def\SImol{\,\mathrm{mol}}
\def\SIm{\,\mathrm{m}}
\def\SIcm{\,\mathrm{cm}}
\def\SIfm{\,\mathrm{fm}}
\def\SImm{\,\mathrm{mm}}
\def\SInm{\,\mathrm{nm}}
\def\SImum{\,\mathrm{\mu m}}
\def\SIJ{\,\mathrm{J}}
\def\SIkJ{\,\mathrm{kJ}}
\def\SIs{\,\mathrm{s}}
\def\SIkg{\,\mathrm{kg}}
\def\SIg{\,\mathrm{g}}
\def\SIK{\,\mathrm{K}}
\def\SImmHg{\,\mathrm{mmHg}}
\def\SIPa{\,\mathrm{Pa}}
\def\vece{\mathrm{e}}
\def\bmat#1{\left(\begin{array}{#1}}
\def\emat{\end{array}\right)}
\def\bcase#1{\left\{\begin{array}{#1}}
\def\ecase{\end{array}\right.}
\def\calM{{\mathcal{M}}}
\def\calT{{\mathcal{T}}}
\def\calR{{\mathcal{R}}}
\def\barpsi{\bar{\psi}}
\def\baru{\bar{u}}
\def\barv{\bar{\upsilon}}
\def\bmini#1{\begin{minipage}{#1\textwidth}}
\def\emini{\end{minipage}}
\def\qeq{\stackrel{?}{=}}
\def\torder#1{\mathcal{T}\left(#1\right)}
\def\rorder#1{\mathcal{R}\left(#1\right)}
\def\contr#1#2{\contraction{}{#1}{}{#2}#1#2}
\def\trof#1{\mathrm{Tr}\left(#1\right)}
\def\trace{\mathrm{Tr}}
\def\comm#1{\ \ \ \left(\mathrm{used}\ #1\right)}
\def\tcomm#1{\ \ \ (\text{#1})}
\def\slp{\slashed{p}}
\def\slk{\slashed{k}}
\def\wulian{\includegraphics[width=0.18in]{emoji_wulian.jpg}}
\def\bye{\includegraphics[width=0.18in]{emoji_bye.jpg}}
\def\calp{{\mathfrak{p}}}
\def\veccalp{\mathbf{\mathfrak{p}}}
\def\atm{\,\mathrm{atm}}
\def\angstrom{\,\text{\AA}}
\def\Tthree{T_{\tiny \textcircled{3}}}
\def\pthree{p_{\tiny \textcircled{3}}}

\title{Lesson 01 - 1st Law of Thermal Dynamics}
  \author{}
  \date{}


\begin{document}

\begin{frame}
\begin{center}
\begin{Large}
\bch
热学 \\
第1讲 温度和热量的本质

{\vskip 0.3in}

黄志琦

\ech
\end{Large}
\end{center}

\vskip 0.2in

\bch
教材:《热学》第二版,赵凯华,罗蔚茵,高等教育出版社
\ech

\bch
课件下载
\ech
https://github.com/zqhuang/SYSU\_TD
\end{frame}


\begin{frame}
\chtitle{温度(Temperature): 热的强度}
\bch
{\large 

\skiplines

\bmini{0.5}
\includegraphics[width = 2in]{ice.jpeg}

温度低
\emini
\bmini{0.4}
\includegraphics[width = 2in]{fire.jpg}

温度高
\emini
}
\ech
\end{frame}


\begin{frame}
\chtitle{温标(Temperature Scale): 温度的数值表示法}
\bch

日常用温标:
\skiplines

\bmini{0.35}
华氏温标($\fdeg$)

(Fahrenheit scale)
\emini
\bmini{0.25}
\includegraphics[width=0.6in]{thermometer.jpg}
\emini
\bmini{0.35}
摄氏温标($\cdeg$)

(Celsius scale)
\emini

\ech
\end{frame}


\begin{frame}
\chtitle{摄氏度和华氏度之间的转换}
\bch
\skipline
转换口诀:{\bf 三十二冰,九华五摄。}
\begin{itemize}
\item{$32\fdeg$和$0\cdeg$度等价,是(1 atm下)水的冰点;}
\item{超过或者不足冰点部分,按照$9\fdeg$等价于$5\cdeg$换算。}
\end{itemize}

\skiplines

请举手抢答:一个标准大气压下水的沸点是多少华氏度?
\ech
\end{frame}



\begin{frame}
\chtitle{温标的定义方法}
\bch
以水银温度计校准的摄(华)氏温标为例:
\bmini{0.8}
\begin{itemize}
\item{{\bf 固定标准点}:规定一个标准大气压下,水的冰点为$0\cdeg$ ($32\fdeg$),水的沸点为$100\cdeg$ ($212\fdeg$)。}
\item{{\bf 测温物质}: 水银。}
\item{{\bf 测温属性}: 热胀冷缩,规定温度与水银体积成线性关系。} 
\end{itemize}
\emini
\bmini{0.15}
\includegraphics[width=0.6in]{thermometer.jpg}
\emini

\skipline

一般地,定义一个温标需要上述三要素:固定标准点,测温物质,测温属性。
\ech
\end{frame}

\begin{frame}
\chtitle{幻想实验:标度一个水银温度计}
\bch
\bmini{0.3}
\includegraphics[width=0.8in]{scale0deg.png}

{\scriptsize 1)放入1 atm下的冰水混合物,达到热平衡时在水银柱最高处标上$0\cdeg$。}
\emini
\hspace{0.1in}
\bmini{0.3}
\includegraphics[width=0.8in]{scale100deg.png}

{\vskip 0.3in}

{\scriptsize 2)放入1 atm下的沸水,达到热平衡时在水银柱最高处标上$100\cdeg$。}
\emini
\hspace{0.1in}
\bmini{0.3}
\includegraphics[width=0.8in]{scalefull.png}

{\vskip 0.3in}

{\scriptsize 3)均匀地标上其他刻度

\vspace{0.36in} 

}
\emini
\ech
\end{frame}

\begin{frame}
\bch
刚刚复习了一些小学知识,下面我们要进行深邃的思考……

\begin{center}\includegraphics[width=1in]{think.jpg}\end{center}
\ech
\end{frame}


\begin{frame}
\bch
\bmini{0.2}
\includegraphics[width=0.8in]{scale0deg.png}
\emini
\bmini{0.75}
\begin{itemize}
\item[1]{ “热平衡”是什么?}
\item[2]{ 为什么我们认为达到热平衡时温度计与冰水混合物的温度相同?}
\end{itemize}
\emini
\ech
\end{frame}


\begin{frame}
\chtitle{热平衡}
\bch
在外界条件影响隔绝的条件下,当两个宏观物体发生接触时,可能发生热传导,力的相互作用,化学反应,电磁相互作用等。

\skipline

物体所有宏观性质不发生变化 = 热平衡 + 力学平衡 + 化学平衡 + 电磁平衡 + ... 

\skiplines

在我们的幻想实验里,我们用真空玻璃管隔离了水银和冰水混合物,使得它们之间只能发生热传导,在这个设定下:

\skipline

物体所有宏观性质不发生变化 = 热平衡

\ech
\end{frame}

\begin{frame}
\chtitle{热力学第零定律}
\bch
{\large \bluetext{在外界影响隔绝的条件下,如果物体A、B分别与{\bf 处于确定状态}的物体C达到热平衡,则物体A和B也是相互热平衡的。}}

\skiplines

comments:
\begin{itemize}
\item{“在外界影响隔绝的条件下”是一句很正确的废话。{\small(我们既然讨论热平衡,就已经假定了这个条件。)}}
\item{要求C“处于确定状态”非常重要。{\small (如果允许C的状态发生变化,由于A和C,B和C都总能分别达到热平衡,我们就得到任何两个物体A和B是热平衡的这样的荒谬结论。)}}
\item{热力学第零定律是一条实验定律。{\small (无须推导。)}}
\end{itemize}
\ech
\end{frame}

\begin{frame}
\bch
解决了水银温度计的标度问题,我们再进一步思考更深邃的问题……
\begin{center}
\includegraphics[width=2in]{think2.jpg}
\end{center}
\ech
\end{frame}

\begin{frame}
\chtitle{更深邃的问题1}
\bch
一定要用水银作为测温物质吗?

\skiplines

\begin{itemize}
\item{如果用酒精,固定压强的气体等代替水银,因为我们人为地“规定”了温度随体积线性地变化,那么水银温标和酒精温标、气体温标并不一定会完全相同。}
\item{日常所需测量的温度往往不需要非常精确,不同温标之间的微小差异(一般不大于$0.1\cdeg$的数量级)可以忽略。}
\end{itemize}

\ech
\end{frame}


\begin{frame}
\chtitle{更深邃的问题2}
\bch
一定要用热胀冷缩的测温属性吗?

\skipline

\bitem
\item{除了热胀冷缩,温变电阻金属的电阻随温度变化,热电偶效应等也都可以用来定义温标。}
\item{同样,不同测温属性定义的温标不会完全相同,但日常测温需求往往可以忽略这些差别。}
\eitem

\bcenter
\includegraphics[width=1.4in]{resistance_thermometer.png}\hspace{0.1in}
\includegraphics[width=2.2in]{thermocouple2.jpg}
\ecenter
\ech
\end{frame}


\begin{frame}
\chtitle{更深邃的问题3}
\bch
除了水的冰点和沸点,还有其他固定标准点吗?

\skipline

\bitem
\item{有。例如{\bf 绝对零度}和{\bf 水的三相点}等是更好的固定标准点。(\wulian 这些都是什么鬼)}
\eitem

\ech
\end{frame}

\begin{frame}
\bch
学习了这么多互有差异的温标,我们真是太开心了\wulian

\skipline

大学套路深,我要回小学\bye

\ech
\end{frame}

\begin{frame}
\chtitle{热力学温标}
\bch
\bitem
\item{在热力学里,我们将使用“热力学温标”。(是的,前面讲的温标我们都不用\bye)}
\item{热力学温标的定义不涉及测温物质和测温属性。(是的,所谓的“三要素”是逗你玩的\bye)}
\item{热力学温标选取的固定标准点为绝对零度(规定为$0K$)和水的三相点(规定为$273.16K$)。(是不是除了数字你什么也没看懂\bye)}
\eitem

取$273.16K$这样一个看起来很奇怪的数字是为了使$1K$的温度变化和$1\cdeg$的温度变化相同。在摄氏温标中,水的三相点为$0.01\cdeg$。所以摄氏温标转换到热力学温标只需数值上平移
$273.15$。例如: 
$$0\cdeg = 273.15 K$$
$$100\cdeg = 373.15K$$

\ech
\end{frame}


\begin{frame}
\chtitle{热力学温标}
\bch
等等,还有很多疑问:
\bitem
\item{热力学温标是如何做到不依赖测温物质和测温属性进行标度的?}
\item{什么是绝对零度?}
\item{什么是水的三相点?}
\item{摄氏温标中$1\cdeg$的变化量和热力学温标$1K$的变化量相同,是指水银温度计标度的$1\cdeg$还是酒精温度计标度的$1\cdeg$,还是其他?}
\eitem
\ech
\end{frame}



\begin{frame}
\chtitle{理想气体状态方程}
\bch
{\large 达到平衡状态的理想气体的摩尔数$\nu$, 压强$p$, 体积$V$,热力学温度$T$满足
{\color{blue}
$$p V = \nu RT$$
}
其中$R=8.31451 \mathrm{J/(mol\cdot K)}$为{\bf 普适气体常量}。
}

{\small
\begin{itemize}
\item{实际稀薄的气体的属性接近理想气体,固定$p$, $V$, $T$中的一个即分别得到历史上的气体实验定律:盖吕萨克定律,查理定律,波意耳定律。}
\item{在我们学习气体的统计描述方法之后,理想气体状态方程可以从微观上进行推导。这里我们先把它当作是对稀薄气体的实验定律进行外推得到的。}
\end{itemize}
}
\ech
\end{frame}


\begin{frame}
\chtitle{温度的比值}
\bch
如果固定理想气体的体积,则温度的比值等于压强的比值:
$$T_1/T_2 = p_1/p_2$$

\skipline
同样,如果固定理想气体的压强,则温度的比值等于体积的比值:
$$T_1/T_2 = V_1/V_2$$


\ech
\end{frame}




\begin{frame}
\chtitle{常见温度计}
\includegraphics[width=0.5in]{thermometer.jpg}
\hspace{0.2in}
\includegraphics[width=2.5in]{gas_thermometer.png}

\includegraphics[width=1.3in]{resistance_thermometer.png}
\hspace{0.2in}
\includegraphics[width=1.1in]{thermocouple.jpg}
\end{frame}



\end{document}

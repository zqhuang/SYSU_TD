\documentclass[CJK]{beamer}
\usepackage{CJKutf8}
\usepackage{beamerthemesplit}
\usetheme{Malmoe}
\useoutertheme[footline=authortitle]{miniframes}
\usepackage{amsmath}
\usepackage{amssymb}
\usepackage{graphicx}
\usepackage{eufrak}
\usepackage{color}
\usepackage{slashed}
\usepackage{simplewick}
\usepackage{tikz}
\graphicspath{{../figures/}}
\def\addfig#1#2{\begin{center}\includegraphics[width=#1 in]{#2}\end{center}}
\def\blacktext#1{{\color{black}#1}}
\def\bluetext#1{{\color{blue}#1}}
\def\redtext#1{{\color{red}#1}}
\def\darkbluetext#1{{\color[rgb]{0,0.2,0.6}#1}}
\def\skybluetext#1{{\color[rgb]{0.2,0.7,1.}#1}}
\def\cyantext#1{{\color[rgb]{0.,0.5,0.5}#1}}
\def\greentext#1{{\color[rgb]{0,0.7,0.1}#1}}
\def\darkgray{\color[rgb]{0.2,0.2,0.2}}
\def\lightgray{\color[rgb]{0.6,0.6,0.6}}
\def\gray{\color[rgb]{0.4,0.4,0.4}}
\def\blue{\color{blue}}
\def\red{\color{red}}
\def\green{\color{green}}
\def\darkblue{\color[rgb]{0,0.2,0.6}}
\def\skyblue{\color[rgb]{0.2,0.7,1.}}
\def\fdeg{{^\circ \mathrm{F}}}
\def\cdeg{^\circ \mathrm{C}}
\def\be{\begin{equation}}
\def\ee{\nonumber\end{equation}}
\def\bea{\begin{eqnarray}}
\def\eea{\nonumber\end{eqnarray}}
\def\ii{{\dot{\imath}}}
\def\bch{\begin{CJK}{UTF8}{gbsn}}
\def\ech{\end{CJK}}
\def\bitem{\begin{itemize}}
\def\eitem{\end{itemize}}
\def\bcenter{\begin{center}}
\def\ecenter{\end{center}}
\def\bex{\begin{minipage}{0.3\textwidth}\includegraphics[width=1in]{jugelizi.png}\end{minipage}\begin{minipage}{0.6\textwidth}}
\def\eex{\end{minipage}}
\def\chtitle#1{\frametitle{\bch#1\ech}}
\def\skipline{{\vskip0.1in}}
\def\skiplines{{\vskip0.2in}}
\def\lagr{{\mathcal{L}}}
\def\hamil{{\mathcal{H}}}
\def\vecv{{\mathbf{v}}}
\def\vecx{{\mathbf{x}}}
\def\vecy{{\mathbf{y}}}
\def\veck{{\mathbf{k}}}
\def\vecp{{\mathbf{p}}}
\def\vecn{{\mathbf{n}}}
\def\vecA{{\mathbf{A}}}
\def\vecP{{\mathbf{P}}}
\def\vecsigma{{\mathbf{\sigma}}}
\def\hatJn{{\hat{J_\vecn}}}
\def\hatJx{{\hat{J_x}}}
\def\hatJy{{\hat{J_y}}}
\def\hatJz{{\hat{J_z}}}
\def\hatj#1{\hat{J_{#1}}}
\def\hatphi{{\hat{\phi}}}
\def\hatq{{\hat{q}}}
\def\hatpi{{\hat{\pi}}}
\def\vel{\upsilon}
\def\Dint{{\mathcal{D}}}
\def\adag{{\hat{a}^\dagger}}
\def\bdag{{\hat{b}^\dagger}}
\def\cdag{{\hat{c}^\dagger}}
\def\ddag{{\hat{d}^\dagger}}
\def\hata{{\hat{a}}}
\def\hatb{{\hat{b}}}
\def\hatc{{\hat{c}}}
\def\hatd{{\hat{d}}}
\def\hatN{{\hat{N}}}
\def\hatH{{\hat{H}}}
\def\hatp{{\hat{p}}}
\def\Fup{{F^{\mu\nu}}}
\def\Fdown{{F_{\mu\nu}}}
\def\newl{\nonumber \\}
\def\SIkm{\,\mathrm{km}}
\def\SIyr{\,\mathrm{yr}}
\def\SIGyr{\,\mathrm{Gyr}}
\def\SIeV{\,\mathrm{eV}}
\def\SIkeV{\,\mathrm{keV}}
\def\SIMeV{\,\mathrm{MeV}}
\def\SIGeV{\,\mathrm{GeV}}
\def\SIcal{\,\mathrm{cal}}
\def\SIkcal{\,\mathrm{kcal}}
\def\SImol{\,\mathrm{mol}}
\def\SIm{\,\mathrm{m}}
\def\SIcm{\,\mathrm{cm}}
\def\SIfm{\,\mathrm{fm}}
\def\SImm{\,\mathrm{mm}}
\def\SInm{\,\mathrm{nm}}
\def\SImum{\,\mathrm{\mu m}}
\def\SIJ{\,\mathrm{J}}
\def\SIkJ{\,\mathrm{kJ}}
\def\SIs{\,\mathrm{s}}
\def\SIkg{\,\mathrm{kg}}
\def\SIg{\,\mathrm{g}}
\def\SIK{\,\mathrm{K}}
\def\SImmHg{\,\mathrm{mmHg}}
\def\SIPa{\,\mathrm{Pa}}
\def\vece{\mathrm{e}}
\def\bmat#1{\left(\begin{array}{#1}}
\def\emat{\end{array}\right)}
\def\bcase#1{\left\{\begin{array}{#1}}
\def\ecase{\end{array}\right.}
\def\calM{{\mathcal{M}}}
\def\calT{{\mathcal{T}}}
\def\calR{{\mathcal{R}}}
\def\barpsi{\bar{\psi}}
\def\baru{\bar{u}}
\def\barv{\bar{\upsilon}}
\def\bmini#1{\begin{minipage}{#1\textwidth}}
\def\emini{\end{minipage}}
\def\qeq{\stackrel{?}{=}}
\def\torder#1{\mathcal{T}\left(#1\right)}
\def\rorder#1{\mathcal{R}\left(#1\right)}
\def\contr#1#2{\contraction{}{#1}{}{#2}#1#2}
\def\trof#1{\mathrm{Tr}\left(#1\right)}
\def\trace{\mathrm{Tr}}
\def\comm#1{\ \ \ \left(\mathrm{used}\ #1\right)}
\def\tcomm#1{\ \ \ (\text{#1})}
\def\slp{\slashed{p}}
\def\slk{\slashed{k}}
\def\wulian{\includegraphics[width=0.18in]{emoji_wulian.jpg}}
\def\bye{\includegraphics[width=0.18in]{emoji_bye.jpg}}
\def\calp{{\mathfrak{p}}}
\def\veccalp{\mathbf{\mathfrak{p}}}
\def\atm{\,\mathrm{atm}}
\def\angstrom{\,\text{\AA}}
\def\Tthree{T_{\tiny \textcircled{3}}}
\def\pthree{p_{\tiny \textcircled{3}}}

\def\courseurl{http://zhiqihuang.top}

\def\tpage#1#2{
\begin{frame}
\bch
\begin{center}
\begin{large}
热学 \\
第#1讲 #2

\end{large}

\skiplines

黄志琦


\end{center}

\skiplines

{\small 
教材:《热学》第二版,赵凯华,罗蔚茵,高等教育出版社


课件下载
}
\courseurl 
\ech
\end{frame}
}

\def\bfr#1{
\begin{frame}
\chtitle{#1} 
\bch
}

\def\efr{
\ech 
\end{frame}
}

\title{Lesson 02 - Heat and State}
  \author{}
  \date{}
\begin{document}
\tpage{2}{热量与物态}

\begin{frame}
\chtitle{上节课内容回顾}
\bch
\bitem
\item{温度是热的强度,它代表了物体内部微观粒子的运动剧烈程度。}
\item{传统温标由固定标准点,测温物质和测温属性来确定。但最后我们将使用与测温物质和测温属性无关的热力学温标,热力学理论用热力学温标来表述是最简洁的。}
\item{理想气体状态方程$pV = \nu RT$可以由微观粒子在各个态上的分布规律($P_i\propto e^{-\frac{\varepsilon_i}{kT}}$)推出。}
\eitem

\ech
\end{frame}


\begin{frame}
\chtitle{理想气体分子平均动能正比于温度}
\bch
{\small
参看教材26页的例子,质量为$m$速度为$\vecv$的分子和质量为$M$速度为$\mathbf{V}$的分子碰撞。仅需考虑在动量交换方向上(设为$x$轴)的速度的变化。由能动量守恒得出两者速度的变化为
$$\Delta \upsilon_x = \frac{2M}{m+M}\left(V_x-\upsilon_x\right);\ \Delta V_x= \frac{2m}{m+M}\left(\upsilon_x-V_x\right)$$
质量为$m$的粒子获得的能量为
$$\Delta\varepsilon = \frac{1}{2}m\Delta\upsilon_x (2\upsilon_x + \Delta\upsilon_x) = \frac{2mM}{(m+M)^2}\left[MV^2-m\upsilon^2 +(m-M)\upsilon_x V_x\right]$$


\skipline

设$\upsilon_x V_x$的统计平均为零,则统计上两个分子碰撞时能量从动能大的分子流向动能小的分子。可见分子平均动能具有温度的性质。上节课我们已经算出了分子平均动能为$\bar{\varepsilon} = \frac{3}{2}kT$。
}
\ech
\end{frame}


\begin{frame}
\bch
今天的内容比较乱,请做好风中凌乱的准备…
\ech
\end{frame}

\begin{frame}
\chtitle{热量是能量的一种形式}
\bch
\bitem
\item{既然组成物体的微观粒子(为了叙述简单,以下我们统称分子)可以带有能量,那么物体就有一个“内能”。{\bf 内能不仅包括分子的动能,还包括由分子之间的分子力而产生的势能。}}
\item{\bf 加热是传递能量给该物体,改变它存储的内能。}
\item{历史上习惯用的热量单位是“卡”(cal),最初定义是在$1\atm$下使$1\SIg$纯水从$14.5\cdeg$升高到$15.5\cdeg$时需要的热量。日常环境下这个量对压强和温度的依赖并不敏感,我们往往{\bf 粗略地说1卡就是日常条件下$1\SIg$水升温$1\cdeg$需要的热量。}}
\item{{\bf 热量是能量的一种形式},经实验测定 $1\SIcal \approx 4.2\SIJ$。后来$\SIcal$重新按下式进行了{\bf 定义}:
$$1\SIcal = 4.184 \SIJ$$
普适气体常量$R$的值用卡来写比较简洁,大约为$2\SIcal/(\SImol\cdot\SIK)$。}
\eitem

\ech
\end{frame}

\begin{frame}
\chtitle{思考题}
\bch
一个$60\SIkg$的成年人日常需要消耗大约1500大卡(就是千卡,$\SIkcal$)的热量来维持身体机能,试计算这些能量能够把他举高到多少米?
\ech
\end{frame}



\begin{frame}
\chtitle{热容量(Heat Capacity)和比热容(Specific Heat Capacity)}
\bch
\bitem
\item{物质温度升高(降低)$1\SIK$时所吸收(放出)的热量,代表了该物质存储热量的能力,我们称之为称为{\bf 热容量},通常用大写$C$来表示。}
\item{{\bf 单位质量物质的热容量称为 比热容},或简称比热,通常用小写$c$来表示。撇开对环境的弱依赖性不谈,我们可粗略地说水的比热容是$4190\SIJ/(\SIkg\cdot\SIK)$,水银的比热容是$139\SIJ/(\SIkg\cdot\SIK)$,等等。}
\item{在常见物质中,水的比热几乎是最大的。}
\item{我们也可以定义每$\SImol$物质的热容量为摩尔热容(记为$C^{\SImol}$),了解即可,在本课就不多做介绍了。}
\eitem
\ech
\end{frame}


\begin{frame}
\chtitle{思考题}
\bch
\bcenter
\includegraphics[width=1.in]{zhuhai.jpg}
\ecenter

为什么靠近海边的地方(例如珠海)温度变化范围小?
\ech
\end{frame}


\begin{frame}
\chtitle{思考题}
\bch
加热一定使物体的温度升高吗?
\ech
\end{frame}


\begin{frame}
\chtitle{加了热不一定变更热(\wulian我怀疑我加了假热)}
\bch
\bitem
\item{虽然加热的本质是传递能量给物体使物体内能增大,但物体的内能包含了分子的动能和势能,原则上来讲,传递的能量可以都用来改变势能,分子平均动能未必一定会增大,即温度未必会升高。}
\eitem

\skipline

\wulian只是“原则上来讲”而已,真有这样的事情吗?

\skipline

真有!例如大部分物质的熔化或者汽化过程吸收热量却保持温度不变。

\ech
\end{frame}


\begin{frame}
\chtitle{大部分物质的物态变化规律}
\bch
\bitem
\item{大部分物质有固态,液态,气态三种形态。物态发生变化时吸收或放出的热量称为{\bf 潜热}。}
\item{如果加热固体,一开始它的温度升高。到达某一温度(称为该物质的{\bf 熔点})后,固体开始{\bf 熔化}为液体,在这个过程中物体继续吸收热量,温度却不变。{\bf单位质量物质熔化(凝固)吸收(放出)的潜热称为熔化热}。}
\item{当熔化完成,物质完全转化为液体,继续加热会使液体温度升高,到达另一临界温度(称为该物质的{\bf 沸点})时,液体开始沸腾{\bf 汽化},在这个过程中物体继续吸收热量,温度却不变。{\bf单位质量物质汽化(液化)吸收(放出)的潜热称为汽化热}。}
\item{ 熔点和熔化热对外界压强有弱依赖性,沸点和汽化热对外界压强有很显著的依赖性。大多数物质的熔点,沸点和潜热都是随着压强增大而增大。}
\eitem
\ech
\end{frame}

\begin{frame}
\chtitle{思考题}
\bch
$1\atm$下冰的熔化热为$333 \SIkJ/\SIkg$,请估算大约什么温度的热水恰好可以熔化等质量的$0\cdeg$的冰。
\ech
\end{frame}


\begin{frame}
\bch

复习完小学知识,我们又要进行深邃的思考了

\bcenter
\includegraphics[width=1.in]{think.jpg}
\ecenter
上述物态变化规律能用微观模型来解释吗?
\ech
\end{frame}

\begin{frame}
\chtitle{分子力和势能的经典模型}
\bch

\bmini{0.47}
{\small
右图是一个典型的分子势能随分子间距离$r$变化的经典模型

在某“平衡距离” $r_0$ (大都为一两个\AA) 处势能最低,分子力为零。

{\bf 束缚能是处于平衡点的分子逃逸到无穷远需要消耗的能量}。
$$E_B = U(\infty) - U(r_0) $$
$E_B$的强弱取决于化学键形式,一般在$\mathrm{eV}$附近几个数量级内。}
\emini
\hspace{0.2in}
\bmini{0.46}

\includegraphics[width=1.8in]{atom_potential.png}

\vspace{0.05in}

\includegraphics[width=1.8in]{atom_force.png}
\emini
\ech
\end{frame}

\begin{frame}
\chtitle{物态(State of Matter)}
\bch
物体的形态取决于分子的平均动能和束缚能的强弱比较:
\bitem
\item{{\bf 气态:平均动能 $\gg$ 束缚能}

{\small 大部分分子完全逃离势能的束缚成为近自由粒子,分子之间的平均距离远远大于$r_0$}
}
\item{{\bf 液态:平均动能 $\sim$ 束缚能}

{\small 大部分分子能显著偏离平衡点,但无法脱离势能束缚,分子之间的平均距离比$r_0$略大。}
}
\item{{\bf 固态:平均动能 $\ll$ 束缚能}

{\small 分子被束缚在势能平衡点附近很小范围内,距离几乎就是固定在$r_0$}
}
\eitem

{\small 注意:
\bitem
\item{分子的动能包括平动,转动,以及多原子分子自己内部的振动等。显然用来逃离束缚的主要是分子的平动动能。我们在上面的讨论中没有特别指定是平动动能是因为一般其他形式动能和平动动能是同一个数量级的。}
\eitem
}
\ech
\end{frame}

\begin{frame}
\chtitle{课堂练习}
\bch
\bitem
\item[1]{试估算标准状态($0\cdeg$,$1\atm$)下气体分子之间的平均距离,并跟典型的$r_0$比较。}
\item[3]{如果物质分子量为$\mathcal{A}$,密度为$\rho$,证明分子之间的平均距离约为
$$\left(\frac{1.66 \mathcal{A} }{\rho / (\mathrm{g}\cdot \mathrm{cm}^{-3})}\right)^{1/3}\angstrom$$
}
\item[3]{试估算通常状态下液态水(原子量18,密度$1\SIg/\SIcm^3$)分子之间的平均距离和铁块(原子量56,密度$7.9\SIg/\SIcm^3$)分子之间的平均距离,分别跟典型的$r_0$比较。}
\eitem
\ech
\end{frame}


\begin{frame}
\bch
听上去好像很厉害的样子,but ...
\bcenter
\includegraphics[width=1in]{think2.jpg}
\ecenter
既然分子势能曲线是连续光滑的,为何物态变化在宏观上有着明显的非连续性?
\ech
\end{frame}

\begin{frame}
\chtitle{整体效应}
\bch
要更完整地理解物态变化,光考虑局域的两个分子之间的相互作用是不行的,还必须考虑整体效应。
\ech
\end{frame}

\begin{frame}
\chtitle{完美的周期性结构更稳定}
\bch
哪组积木更容易推动使之发生结构变化?
\includegraphics[width=3in]{stackcubes.png}

(A)胡乱堆砌 \ \ \ \ \  (B)不是很整齐 \ \ (C)非常整齐

\ech
\end{frame}

\begin{frame}
\chtitle{晶体 (Crystalline Solid)}
\bch
\bitem
\item{自然界大多数固体都是分子排列具有周期性的完美结构的{\bf 晶体}。完美的分子排列使得固体占据体积更小,(多分子相互作用的)总势能更低。}
\item{当我们加热固体,固体的温度逐渐升高(分子平均动能随之增大),到达某一临界值(即熔点)时,完美的分子排列结构被破坏,就发生了固态往液态的相变。}
\item{当温度处在熔点时,虽然分子动能保持不变,破坏完美分子结构需要额外吸收能量(因为胡乱排列的分子总势能更高)。所以{\bf 晶体熔化,是一个温度不变但需要吸收能量的过程。}}
\eitem
\ech
\end{frame}

\begin{frame}
\chtitle{固液转化宏观非连续性的微观解释}
\bch
所以,对晶体而言,虽然两分子之间的势能曲线是连续的。但是多分子相互作用的总势能却会随着晶体完美的分子排列结构瓦解而有个跃变。这就是为什么宏观上固态到液态的转变有明显的非连续性。
\ech
\end{frame}

\begin{frame}
\chtitle{多晶和单晶}
\bch
\bitem
\item{单晶是肉眼可见的大块晶体,有着明显的规则外形。例如水晶,金刚石等。

\includegraphics[width=1in]{diamond.jpg}
}

\item{多晶由肉眼不可见,微米量级的小块晶粒组成。很多金属都是多晶。多晶有可能被加工成单晶。

\includegraphics[width=1in]{gold_irregular.jpg}
}
\item{纯冰一般是单晶,我们日常见到冰块很不规则,是冰含了很多杂质的缘故。}
\eitem
\ech
\end{frame}


\begin{frame}
\chtitle{晶体可能有多种对称性}
\bch
除了{\bf 空间平移对称性}以外,晶体分子排列而成的{\bf 晶格}还可能有{\bf 中心反演对称性},{\bf 镜像反演对称性}, {\bf 绕轴的$n$重旋转对称性($n=2,3,4,6$)}等。

一般来说,对称性越高,晶格越稳定,破坏晶格时需要的能量越大。

\bcenter
\includegraphics[width=3in]{graphite.jpg}
\ecenter

左图晶格比右图更稳定
\ech
\end{frame}


\begin{frame}
\chtitle{实际晶体可能有各种缺陷}
\bch
实际的晶体可能会有晶界(二维缺陷),位错(一维缺陷),空位或杂质(点缺陷)等。

\bcenter
\includegraphics[width=2.5in]{solid_dislocate.jpg}
\ecenter

产生缺陷的晶体的牢固程度可能会大幅度下降。
\ech
\end{frame}


\begin{frame}
\bch
固体一定是排得整整齐齐的晶体吗?

\skiplines

未必 \bye 玻璃、松香、沥青、蜂蜡等都是{\bf 非晶体}。
\ech
\end{frame}

\begin{frame}
\chtitle{非晶体 (Amorphous Solid)}
\bch
在很小范围内非晶体的分子排列是有规则的,但“不是很整齐”。从大范围看,非晶体的分子排列方向是无序的。总结起来就是:

非晶体分子排列{\bf 短程有序而长程无序}

\skipline
\includegraphics[width=4.3in]{amorphous_solid.jpg}

非晶体由固态向液态转变时则是各个小范围内的分子排列结构被逐渐破坏,不存在一个从完美排列到混乱排列的突变,所以非晶体熔融时温度还是在继续上升。{\bf 非晶体不存在熔点}。

\ech
\end{frame}

\begin{frame}
\bch
{\bf 液体分子排列也是短程有序长程无序},非晶体熔融为液体时又没有体积的突变,那非晶体和液体的区别在哪里呢?
\bcenter
\includegraphics[width=1.5in]{think3.jpg}
\ecenter
\ech
\end{frame}

\begin{frame}
\chtitle{非晶体和液体的区别(它们的名字不一样\bye)}
\bch
\bitem
\item{非晶体虽然“堆得不大整齐”,但既然名为“固”体,在不发生宏观形变的情况下,其内部分子的位置是固定的。(危楼也是楼啊\wulian)}
\item{液体内分子并无固定位置,会发生游动,一般定居时间只有$10^{-10}s$左右。}
\eitem
\ech
\end{frame}


\begin{frame}
\chtitle{思考题}
\bch
\bcenter
\includegraphics[width=1.5in]{think3.jpg}
\ecenter

为什么自然界中大多数物质都是晶体?
\ech

\end{frame}


\begin{frame}
\chtitle{快速淬火制作非晶体}
\bch
因为自然界中物态变化的过程都相对缓慢,物质中的分子有足够的时间“排好队”,就会形成全局能量最低态——晶体。

\skipline

如果把熔点附近的液体以极快的速度降温,使它的温度快速跨过熔点(即凝固点),并迅速降到某个比熔点更低的温度“玻璃化点”,物质中的分子没有足够的时间“排好队”,就得到“排得不那么整齐”的非晶体(局域能量最低态)。这种技术叫做“快速淬火”。几乎所有物质都可以通过这种方法人工制成非晶体。

\ech

\end{frame}


\begin{frame}
\bch
解决了固体熔化的问题,下面我们考虑液体汽化。
\ech
\end{frame}

\begin{frame}
\chtitle{汽化的宏观非连续性的微观解释}
\bch
\bitem
\item{液体内的分子虽然有较大的活动空间,但还是被周围的分子势能所束缚,做着无规则的振荡(一般在同一位置可以振荡10-100次左右)。}
\item{当我们加热液体,液体分子的平均动能(即液体的温度)逐渐增大,到达某一临界值时,分子就挣脱了势能的束缚变为在容器内到处飞行的近自由粒子(除短暂的碰撞,几乎不受力)。}
\item{虽然从两分子之间的势能曲线来看这是一个连续的过程(从几乎逃逸到逃逸),但多分子相互作用的整体效应仍不可忽略。分子大都成为自由粒子之后,原先液体的聚团特性立刻被破坏,又一次造成总体势能的跃变。所以{\bf 液体汽化也是一个温度不变但需要吸收能量的过程}。}
\eitem
\ech
\end{frame}

\begin{frame}
\chtitle{外界压强的影响}
\bch
\bitem
\item{前面的讨论忽略了外界环境的影响,实际上无论是熔化还是汽化,大部分物体的体积都会突然增大,在固定压强下体积增大就会对外界做功,需要消耗额外的能量。外界压强越大,做的功越大。所以大部分物质的熔点,沸点和潜热都随着压强增加而增加。

\scriptsize 注: 冰的熔化过程比较特殊,它的体积在这个过程中减小(微观原因和特殊的化学键有关,我们之后再讨论),所以熔点和熔化热反而随着压强增大而减小。}
\item{由于固液态的体积变化不大,所以压强对熔点和熔化热的影响不显著。}
\item{汽化过程的体积增大是非常显著的(一般体积要增大三个数量级左右),在标准大气压下做的功往往跟总体分子势能的变化的数量级接近,属于不可忽略的效应,所以沸点和汽化热对压强的依赖性很显著。}
\eitem
\ech
\end{frame}

\begin{frame}
\chtitle{思考题}
\bch
\includegraphics[width=1.5in]{gaoyaguo.jpg}
\includegraphics[width=1.5in]{dunzhuti.jpg}

为什么高压锅炖猪蹄炖得更软?
\ech
\end{frame}


\begin{frame}
\chtitle{思考题}
\bch
\includegraphics[width=2in]{eatraw.jpg}

在线等,急!为什么在西藏煮东西煮不熟?(\wulian你确定这煮过了?)
\ech
\end{frame}

\end{document}

\documentclass[CJK]{beamer}
\usepackage{CJKutf8}
\usepackage{beamerthemesplit}
\usetheme{Malmoe}
\useoutertheme[footline=authortitle]{miniframes}
\usepackage{amsmath}
\usepackage{amssymb}
\usepackage{graphicx}
\usepackage{eufrak}
\usepackage{color}
\usepackage{slashed}
\usepackage{simplewick}
\usepackage{tikz}
\graphicspath{{../figures/}}
\def\addfig#1#2{\begin{center}\includegraphics[width=#1 in]{#2}\end{center}}
\def\blacktext#1{{\color{black}#1}}
\def\bluetext#1{{\color{blue}#1}}
\def\redtext#1{{\color{red}#1}}
\def\darkbluetext#1{{\color[rgb]{0,0.2,0.6}#1}}
\def\skybluetext#1{{\color[rgb]{0.2,0.7,1.}#1}}
\def\cyantext#1{{\color[rgb]{0.,0.5,0.5}#1}}
\def\greentext#1{{\color[rgb]{0,0.7,0.1}#1}}
\def\darkgray{\color[rgb]{0.2,0.2,0.2}}
\def\lightgray{\color[rgb]{0.6,0.6,0.6}}
\def\gray{\color[rgb]{0.4,0.4,0.4}}
\def\blue{\color{blue}}
\def\red{\color{red}}
\def\green{\color{green}}
\def\darkblue{\color[rgb]{0,0.2,0.6}}
\def\skyblue{\color[rgb]{0.2,0.7,1.}}
\def\fdeg{{^\circ \mathrm{F}}}
\def\cdeg{^\circ \mathrm{C}}
\def\be{\begin{equation}}
\def\ee{\nonumber\end{equation}}
\def\bea{\begin{eqnarray}}
\def\eea{\nonumber\end{eqnarray}}
\def\ii{{\dot{\imath}}}
\def\bch{\begin{CJK}{UTF8}{gbsn}}
\def\ech{\end{CJK}}
\def\bitem{\begin{itemize}}
\def\eitem{\end{itemize}}
\def\bcenter{\begin{center}}
\def\ecenter{\end{center}}
\def\bex{\begin{minipage}{0.3\textwidth}\includegraphics[width=1in]{jugelizi.png}\end{minipage}\begin{minipage}{0.6\textwidth}}
\def\eex{\end{minipage}}
\def\chtitle#1{\frametitle{\bch#1\ech}}
\def\skipline{{\vskip0.1in}}
\def\skiplines{{\vskip0.2in}}
\def\lagr{{\mathcal{L}}}
\def\hamil{{\mathcal{H}}}
\def\vecv{{\mathbf{v}}}
\def\vecx{{\mathbf{x}}}
\def\vecy{{\mathbf{y}}}
\def\veck{{\mathbf{k}}}
\def\vecp{{\mathbf{p}}}
\def\vecn{{\mathbf{n}}}
\def\vecA{{\mathbf{A}}}
\def\vecP{{\mathbf{P}}}
\def\vecsigma{{\mathbf{\sigma}}}
\def\hatJn{{\hat{J_\vecn}}}
\def\hatJx{{\hat{J_x}}}
\def\hatJy{{\hat{J_y}}}
\def\hatJz{{\hat{J_z}}}
\def\hatj#1{\hat{J_{#1}}}
\def\hatphi{{\hat{\phi}}}
\def\hatq{{\hat{q}}}
\def\hatpi{{\hat{\pi}}}
\def\vel{\upsilon}
\def\Dint{{\mathcal{D}}}
\def\adag{{\hat{a}^\dagger}}
\def\bdag{{\hat{b}^\dagger}}
\def\cdag{{\hat{c}^\dagger}}
\def\ddag{{\hat{d}^\dagger}}
\def\hata{{\hat{a}}}
\def\hatb{{\hat{b}}}
\def\hatc{{\hat{c}}}
\def\hatd{{\hat{d}}}
\def\hatN{{\hat{N}}}
\def\hatH{{\hat{H}}}
\def\hatp{{\hat{p}}}
\def\Fup{{F^{\mu\nu}}}
\def\Fdown{{F_{\mu\nu}}}
\def\newl{\nonumber \\}
\def\SIkm{\,\mathrm{km}}
\def\SIyr{\,\mathrm{yr}}
\def\SIGyr{\,\mathrm{Gyr}}
\def\SIeV{\,\mathrm{eV}}
\def\SIkeV{\,\mathrm{keV}}
\def\SIMeV{\,\mathrm{MeV}}
\def\SIGeV{\,\mathrm{GeV}}
\def\SIcal{\,\mathrm{cal}}
\def\SIkcal{\,\mathrm{kcal}}
\def\SImol{\,\mathrm{mol}}
\def\SIm{\,\mathrm{m}}
\def\SIcm{\,\mathrm{cm}}
\def\SIfm{\,\mathrm{fm}}
\def\SImm{\,\mathrm{mm}}
\def\SInm{\,\mathrm{nm}}
\def\SImum{\,\mathrm{\mu m}}
\def\SIJ{\,\mathrm{J}}
\def\SIkJ{\,\mathrm{kJ}}
\def\SIs{\,\mathrm{s}}
\def\SIkg{\,\mathrm{kg}}
\def\SIg{\,\mathrm{g}}
\def\SIK{\,\mathrm{K}}
\def\SImmHg{\,\mathrm{mmHg}}
\def\SIPa{\,\mathrm{Pa}}
\def\vece{\mathrm{e}}
\def\bmat#1{\left(\begin{array}{#1}}
\def\emat{\end{array}\right)}
\def\bcase#1{\left\{\begin{array}{#1}}
\def\ecase{\end{array}\right.}
\def\calM{{\mathcal{M}}}
\def\calT{{\mathcal{T}}}
\def\calR{{\mathcal{R}}}
\def\barpsi{\bar{\psi}}
\def\baru{\bar{u}}
\def\barv{\bar{\upsilon}}
\def\bmini#1{\begin{minipage}{#1\textwidth}}
\def\emini{\end{minipage}}
\def\qeq{\stackrel{?}{=}}
\def\torder#1{\mathcal{T}\left(#1\right)}
\def\rorder#1{\mathcal{R}\left(#1\right)}
\def\contr#1#2{\contraction{}{#1}{}{#2}#1#2}
\def\trof#1{\mathrm{Tr}\left(#1\right)}
\def\trace{\mathrm{Tr}}
\def\comm#1{\ \ \ \left(\mathrm{used}\ #1\right)}
\def\tcomm#1{\ \ \ (\text{#1})}
\def\slp{\slashed{p}}
\def\slk{\slashed{k}}
\def\wulian{\includegraphics[width=0.18in]{emoji_wulian.jpg}}
\def\bye{\includegraphics[width=0.18in]{emoji_bye.jpg}}
\def\calp{{\mathfrak{p}}}
\def\veccalp{\mathbf{\mathfrak{p}}}
\def\atm{\,\mathrm{atm}}
\def\angstrom{\,\text{\AA}}
\def\Tthree{T_{\tiny \textcircled{3}}}
\def\pthree{p_{\tiny \textcircled{3}}}

\def\courseurl{http://zhiqihuang.top}

\def\tpage#1#2{
\begin{frame}
\bch
\begin{center}
\begin{large}
热学 \\
第#1讲 #2

\end{large}

\skiplines

黄志琦


\end{center}

\skiplines

{\small 
教材:《热学》第二版,赵凯华,罗蔚茵,高等教育出版社


课件下载
}
\courseurl 
\ech
\end{frame}
}

\def\bfr#1{
\begin{frame}
\chtitle{#1} 
\bch
}

\def\efr{
\ech 
\end{frame}
}

\title{Lesson 11 Cyclic Process}
  \author{}
  \date{}
\begin{document}
\tpage{11}{循环过程}

\section{Review}

\begin{frame}
\chtitle{上讲内容回顾}
\bch 
\bitem
\item{焓和状态方程的关系$\pfrac HpT = V - \pfrac V{\ln T}p$}
\item{定压比热$C_p = \pfrac HTp = T\pfrac STp$}
\item{节流是等焓的不可逆过程。焦耳-汤姆孙系数$\alpha \equiv \pfrac TpH$}
\item{焦耳-汤姆孙效应:正节流效应($\alpha >0$)降温和负节流效应$\alpha<0$升温}
\item{节流过程液化气体:预冷降温到正节流区$\alpha$较大的区域,节流液化气体。}
\eitem
\ech
\end{frame}

\begin{frame}
\chtitle{本讲内容}
\bch
\bitem
\item{孤立系统的熵增大原理}
\item{循环过程}
\item{卡诺循环}
\item{为计算做准备:理想气体的熵}
\item{计算理想气体可逆循环的效率}
\eitem
\ech
\end{frame}

\section{Increasing Entropy}

\begin{frame}
\chtitle{孤立系统的熵增大原理}
\bch
假想一个孤立的非热平衡系统$\Sigma$,把它划分成很多子系统$\Sigma_1$, $\Sigma_2$, $\ldots$, $\Sigma_N$。
每个子系统$\Sigma_i$在短时间内可以近似看成处于平衡态,有确定的温度$T_i$,压强$p_i$,体积$V_i$和熵$S_i$。各个子系统之间保持接触(非孤立)。

\skiplines


假设子系统边界处的分子相互作用可以忽略,系统总熵等于各个子系统熵之和:
$$S = \sum_i S_i$$
\ech
\end{frame}

\begin{frame}
\chtitle{孤立系统的熵增大原理(续)}
\bch
因系统处于非热平衡状态,总有两个互相接触的子系统温度不同,不妨设$T_i>T_j$。
我们假设{\bf 在宏观尺度上,热量会且只会自发地从高温物体传递到低温物体},$\Sigma_i$对$\Sigma_j$传热$Q_{i\rightarrow j}$ ($Q_{i\rightarrow j}>0$)。那么系统的总熵变化为
$$\Delta S = \Delta S_i + \Delta S_j = - \frac{Q_{i\rightarrow j}}{T_i}+ \frac{Q_{i\rightarrow j}}{T_j}  > 0$$
这样的熵增大过程会持续进行,直到系统达到平衡态(这时所有的$T_i = T_j$)后熵不再变化。

{\bf 孤立系统熵增大是一个不可逆的过程(因为无法使熵减小),熵不变的过程则原则上都可逆。}

\ech
\end{frame}

\begin{frame}
\chtitle{孤立系统的熵增大原理(续)}
\bch
总结一下:

\bitem
\item{\bf 孤立的非热平衡系统的熵总是自发地增大,直到达到热平衡态(熵最大的状态)。热平衡态的熵不再发生变化。}
\item{\bf 孤立系统的可逆过程熵不变,不可逆过程熵增大。}
\eitem

\addfig{2}{dengshan.png}

\ech
\end{frame}


\section{Cyclic Process}

\begin{frame}
\bch
{\large 
下面我们来谈热机和循环过程}
\ech
\end{frame}


\begin{frame}
\chtitle{蒸汽机}
\bch
\addfig{3}{steam_engine.jpg}
\ech
\end{frame}

\begin{frame}
\chtitle{循环过程}
\bch
{\bf 循环过程:一系统由某个状态出发,经过一系列过程,最后回到原状态。}

\skiplines
\bitem
\item{因循环过程的内能不变,按照热一律即有$Q+A = 0$。}
\item{把系统和环境看成一个总的孤立系统。系统回到原状态,熵不变。根据孤立系统的熵增大原理,{\bf 循环过程后环境的熵必须增大(如果循环过程对环境造成的影响是不可逆的)或者不变(如果循环过程对环境造成的影响是可逆的)。}}
\eitem

如果{\bf 循环过程对环境造成的影响是可逆}的,我们称之为{\bf 可逆循环}。否则称之为不可逆循环。

\ech
\end{frame}

\begin{frame}
\chtitle{正循环热机}
\bch

系统对外界做功($A'>0$)的循环为正循环。

\addfig{3}{heat_engine.jpg}
\ech
\end{frame}

\begin{frame}
\chtitle{逆循环制冷机}
\bch

外界对系统做功($A>0$)的循环为逆循环。逆循环热机也叫制冷机。

\addfig{3}{cold_engine.jpg}
\ech
\end{frame}

\begin{frame}
\chtitle{思考题}
\bch
下列过程是正循环还是逆循环?
\bitem
\item{蒸汽机的一个循环}
\item{制冷机的一个循环}
\item{净吸热量$Q>0$的循环}
\item{净放热量$Q'>0$的循环}
\item{$p$-$V$图上顺时针的闭合曲线}
\item{$p$-$V$图上逆时针的闭合曲线}
\item{$T$-$S$图上顺时针的闭合曲线}
\item{$T$-$S$图上逆时针的闭合曲线}
\eitem
\ech
\end{frame}


\begin{frame}
\chtitle{骨灰级难度思考题}
\bch

逆循环可不一定是可逆循环;

逆循环也许可以是可逆循环;

逆循环也可以不是可逆循环;

逆循环也可以是不可逆循环;

逆循环也可不是不可逆循环;

可逆循环也许可以是逆循环;

可逆循环也可以不是逆循环;

可逆循环可不一定是逆循环;

可逆循环必不是不可逆循环;

不可逆循环必不是可逆循环;

不可逆循环可以不是逆循环;

不可逆循环未必不是逆循环;

不可逆循环也可以是逆循环。

那么问题来了——

{\bf 如果逆循环是可逆循环,那么逆循环的逆循环是逆循环吗}?
\ech
\end{frame}


\begin{frame}
\chtitle{正循环热机的效率}
\bch

\addfig{2}{heat_engine.jpg}


{\bf 正循环把热量转化为机械功。}设正循环热机从高温热源1吸热$Q_1$,对外做功$A'$,并对低温热源2放热$Q_2'$。{\bf 热量转化为机械能的百分比称为正循环热机的效率,记作$\eta$。}

$$\eta \equiv \frac{A'}{Q_1}=\frac{Q_1-Q_2'}{Q_1} $$


\ech
\end{frame}


\begin{frame}
\chtitle{恒温热源间的正循环的效率}
\bch
\addfig{2}{heat_engine.jpg}
$$\eta = \frac{Q_1-Q_2'}{Q_1} = 1-\frac{T_2|\Delta S_2|}{T_1|\Delta S_1|}$$
{\small 其中$|\Delta S_1|$为高温热源的熵减少量,$|\Delta S_2|$为低温热源的熵增加量。

如果整个循环过程{\bf 可逆},则环境的总熵不变:$|\Delta S_1| = |\Delta S_2|$。

$$ \eta = 1 - \frac{T_2}{T_1}$$

如果整个循环过程{\bf 不可逆},则环境总熵增大:$|\Delta S_2| > |\Delta S_1|$,即
$$\eta < 1 - \frac{T_2}{T_1} $$
}
\ech
\end{frame}


\begin{frame}
\chtitle{卡诺定理}
\bch
\addfig{2}{heat_engine.jpg}

在温度为$T_1$的高温热源和温度为$T_2$的低温热源之间工作的热机:{\bf 可逆循环$\eta = 1 -\frac{T_2}{T_1}$,不可逆循环$\eta < 1 - \frac{T_2}{T_1}$,这称为卡诺(Carnot)定理}。
\ech
\end{frame}

\begin{frame}
\chtitle{逆循环热机的效率}
\bch

\addfig{2}{cold_engine.jpg}

{\bf 逆循环利用外界提供的机械功从低温热源吸热制冷。}设外界对逆循环热机做功$A$,使它从低温热源2吸热$Q_2$,并对高温热源放热$Q_1'$。{\bf 制冷量$Q_2$与外功$A$之比称为逆循环热机的制冷系数,记作$\varepsilon$。}

$$\varepsilon \equiv \frac{Q_2}{A}=\frac{Q_2}{Q_1'-Q_2} $$

\ech
\end{frame}


\begin{frame}
\chtitle{恒温热源间的逆循环的制冷效率}
\bch
{\small
设外界对热机做功$A$,使热机从温度恒为$T_2$的低温热源2吸热$Q_2$,并对温度恒为$T_1$的高温热源1放热$Q_1'$。则
$$\varepsilon = \frac{1}{Q_1'/Q_2 - 1} = \frac{1}{\frac{T_1|\Delta S_1|}{T_2|\Delta S_2|}-1}$$
其中$|\Delta S_1|$为高温热源的熵增加量,$|\Delta S_2|$为低温热源的熵减少量。

如果整个循环过程{\bf 可逆},则环境的总熵不变:$|\Delta S_1| = |\Delta S_2|$。

$$ \varepsilon = \frac{T_2}{T_1-T_2}$$

如果整个循环过程{\bf 不可逆},则环境总熵增大:$|\Delta S_2| < |\Delta S_1|$,即
$$\varepsilon <  \frac{T_2}{T_1-T_2} $$
}
\ech
\end{frame}

\begin{frame}
\chtitle{环保常识: 空调温度别开太低}
\bch

\addfig{1.5}{aclife2.png}

设夏天室外温度为$30\cdeg$。把空调近似看成可逆热机。试估算制冷温度分别为$28\cdeg$和$20\cdeg$时空调制冷效率之比。

\ech
\end{frame}

\begin{frame}
\bch
下面我们以理想气体为例讨论一些准静态循环(即可逆循环)。

\skipline

这些循环的中间过程都有明确的态函数$p$, $V$, $T$, $S$。
\ech
\end{frame}

\begin{frame}
\chtitle{正循环$p$-$V$图}
\bch
\addfig{3.2}{pVdiagram_cycle1.png}
\ech
\end{frame}

\begin{frame}
\chtitle{逆循环$p$-$V$图}
\bch
\addfig{3.2}{pVdiagram_cycle2.png}
\ech
\end{frame}


\begin{frame}
\chtitle{正循环$T$-$S$图}
\bch
\addfig{3.2}{TSdiagram_cycle2.png}
\ech
\end{frame}

\begin{frame}
\chtitle{逆循环$T$-$S$图}
\bch
\addfig{3.2}{TSdiagram_cycle1.png}
\ech
\end{frame}

\begin{frame}
\chtitle{dalao也有长得帅的}
\bch

\bmini{0.45}
\lfig{1.3}{njg_ndmz.png}
\emini
\bmini{0.5}
\addfig{1.4}{carnot.jpg}
\bcenter
Nicolas Léonard Sadi Carnot
\ecenter
\emini
\ech
\end{frame}


\begin{frame}
\chtitle{卡诺循环(Carnot Cycle)}
\bch

\addfig{2}{carnotcycle.jpg}

\ech
\end{frame}

\begin{frame}
\chtitle{理想气体的可逆卡诺循环$p$-$V$图}
\bch
\addfig{2.3}{Carnot_cycle.png}

思考题:对理想气体计算可逆卡诺循环的效率(设$C_V$仅是温度的函数,等温过程的$T_1$, $T_2$已知)。
\ech
\end{frame}


\begin{frame}
\chtitle{猜一猜}
\bch
刚才算出理想气体可逆卡诺循环的效率为$\eta = 1-\frac{T_2}{T_1}$,这个结果对非理想气体成立吗?
\ech
\end{frame}

\begin{frame}
\chtitle{可逆卡诺循环的$T$-$S$图}
\bch
\addfig{2.5}{Carnot_cycle_TS.png}

在这个图里计算热机效率特别容易(和工作物质无关):
$$\eta = \frac{A'}{Q_1} = \frac{(T_1-T_2)(S_2-S_1)}{T_1(S_2-S_1)} = 1- \frac{T_2}{T_1}$$

\ech
\end{frame}

\section{Ideal gas cycles}

\begin{frame}
\chtitle{下面的学习计划}
\bch

理想气体的熵 $\rightarrow$ 循环的效率

\addfig{2.2}{areyouready.jpg}

\ech
\end{frame}

\begin{frame}
\chtitle{理想气体的熵}
\bch
{\blue
$$ dS = \frac{\dbar Q}{T} = \frac{C_VdT +  p dV}{T} = C_Vd\ln T+ \nu R d\ln V $$}
因理想气体定体热容只是温度的函数,积分即得
$$\Delta S = \int \frac{C_V}{T} dT + \nu R \Delta \ln V$$
如果热容$C_V$是常数,则$\nu R = (\gamma-1)C_V$,易从上式得到
$$ \Delta S =  C_V \Delta \left[\ln\left(TV^{\gamma-1}\right) \right] =C_V \Delta \left[\ln\left(pV^\gamma\right) \right] = C_p \Delta \left[\ln\left(Tp^{\frac{1}{\gamma}-1}\right) \right]$$
由此易得理想气体绝热方程。
\ech
\end{frame}

\begin{frame}
\chtitle{思考题}
\bch
\addfig{1.}{songfen.jpg}
教材习题3-19

\ech
\end{frame}


\begin{frame}
\chtitle{思考题}
\bch
\addfig{1.}{songfen2.jpg}
某理想气体定体热容在一定范围内($200\SIK<T<500\SIK$)可以写成
$$ C_V = \left[\frac{3}{2} + \frac{T}{T_0}\right]\nu R $$
其中$T_0 = 300\SIK$。该气体从温度为$T_1= 450\SIK$时准静态绝热膨胀,体积变为原来4倍。求末态温度$T_2$。
\ech
\end{frame}

\begin{frame}
\chtitle{参考解答}
\bch
{\scriptsize
由$ dS = \frac{C_V}{T}dT + \frac{\nu R}{V}dV = 0$积分得到
$$\frac{3}{2} \nu R \ln \frac{T_2}{T_1} +\frac{\nu R}{T_0}\left(T_2-T_1\right) = -\nu R \ln 4 $$
令$x = \ln \frac{T_2}{T_1}$,则
$$  x + e^x = 1-\frac{2}{3} \ln 4 = 0.075804 $$
定义$f(x) \equiv x+e^x$,则$f'(x) = 1+e^x$,
先做近似$e^x = 1+x$得到零级近似解$x_0 =-0.4621 $,然后用牛顿迭代法:

一级近似$x_1 = x_0 + \frac{0.075804 - f(x_0)}{f'(x_0)} = -0.5186 $

二级近似$x_2 = x_1 + \frac{0.075804- f(x_1)}{f'(x_1)} = -0.5192 $

末态温度$T_2 = e^x T_1 = 267.7 \SIK$。
 }
\ech
\end{frame}

\begin{frame}
\chtitle{参考解答:方法2}
\bch
{\scriptsize
在$T=450\SIK$时,定体热容$C_V = 3 \nu R$,$C_p = C_V+\nu R = 4\nu R$,$\gamma = C_p/C_V = \frac{4}{3}$
{\bf 取$\gamma$不变的近似,则$T \propto V^{1-\gamma}$},即零级近似
$$T_2 \approx 4^{1-4/3} T_1 = 283.5 \SIK$$
然后考虑修正,在$T=283.5\SIK$时,$\gamma = 1.4090$,在整个过程中取平均$\gamma \approx \frac{1.4090+1.3333}{2} = 1.3712$,即得到一级近似
$$ T_2 \approx 4^{1-1.3712} T_1 = 269.0 \SIK$$

\skiplines

可见,即使在温度范围比较大的时候,常数$\gamma$近似往往很便捷且误差不大。但这个解法的缺点是只能求解到一级近似,无法逼近精确解。
 }
\ech
\end{frame}

\begin{frame}
\chtitle{思考题}
\bch
\addfig{1.}{songfen.jpg}
设理想气体定体热容$C_V$为常数。证明多方指数为$n$的准静态多方过程的热容为
$$C_n = \frac{n-\gamma}{n-1} C_V$$
其中$\gamma$为定压热容与定体热容之比。
\ech
\end{frame}



\begin{frame}
\chtitle{思考题}
\bch
\addfig{1.}{songfen2.jpg}
设理想气体定体热容$C_V$为常数。在多方指数为$n$ ($n\ne \gamma$)的准静态多方过程中,证明温度$T$正比于熵$S$的指数函数:
{\blue $$T\propto e^{\frac{S}{C_n}}$$}
其中$C_n$为多方热容。
\ech
\end{frame}


\begin{frame}
\bch
有了理想气体的熵的知识

\addfig{2.5}{takeabreath.jpg}

我们来计算理想气体更一般的可逆循环的效率
\ech
\end{frame}

\begin{frame}
\chtitle{定体热容固定的理想气体的(可逆)多方循环}
\bch
\bmini{0.5}
对定体热容固定的理想气体,把卡诺循环中的两个等温过程换成多方指数为$n$的多方过程,如图所示
\emini
\bmini{0.46}
\addfig{1.6}{polytropic_cycle_n.jpg}
\emini
{\small
由于多方过程中$T\propto e^{S/C_n}$,两个多方过程的温度成正比关系,也就是说曲线$CD$下的面积($Q_2'$)和曲线$AB$下的面积($Q_1$)之比为$Q_2'/Q_1=T_D/T_A = T_C/T_B$。热机效率为
$$ \eta = 1 - \frac{T_D}{T_A} = 1 - \frac{T_C}{T_B}$$
也就是说卡诺循环的效率计算公式仍然成立。利用AD过程(或BC过程)的绝热方程很容易把上式中的温度比转化成体积比或者压强比。
}
\ech
\end{frame}

\begin{frame}
\chtitle{定体热容固定的理想气体的的(可逆)广义多方循环}
\bch
\bmini{0.5}
广义多方循环由两个绝热过程和两个多方指数不同的多方过程组成,设多方膨胀的多方指数为$n_1$,多方压缩的多方指数为$n_2$:
\emini
\bmini{0.46}
\addfig{1.6}{polytropic_cycle.jpg}
\emini

{\scriptsize 
不妨设$S_1=0$,$S_2=S$。多方膨胀过程中$T = T_A e^{S/C_{n_1}}$,多方压缩过程中$T= T_D e^{S/C_{n_2}}$。积分求出
$$Q_2' = T_D \int_0^S e^{S/C_{n_2}} dS = T_DC_{n_2}\left(e^{S/C_{n_2}}-1\right)$$
$$Q_1  = T_A \int_0^S e^{S/C_{n_1}} dS = T_AC_{n_1}\left(e^{S/C_{n_1}}-1\right)$$
$$\eta = 1-\frac{Q_2'}{Q_1} = 1 - \frac{T_D}{T_A} \frac{C_{n_2}}{C_{n_1}} \frac{e^{S/C_{n_2}}-1}{e^{S/C_{n_1}}-1}$$
}
\ech
\end{frame}

\begin{frame}
\chtitle{定体热容固定的理想气体的的广义多方循环(续)}
\bch
{\scriptsize 
若$n_1=n_2$则回到前面讨论的结果。若$n_1\ne n_2$,则由$T_B = T_A e^{S/C_{n_1}}$以及$T_C = T_D e^{S/C_{n_2}}$得到
$$\frac{T_B T_D}{T_AT_C} = e^{S\left(\frac{1}{C_{n_1}} - \frac{1}{C_{n_2}}\right)}$$
即
$$ e^S = \left(\frac{T_B T_D}{T_AT_C}\right)^{\frac{C_{n_1}C_{n_2}}{C_{n_2}-C_{n_1}}}$$
记绝热压缩温度比$r_c = \frac{T_D}{T_A}$,绝热膨胀温度比$r_e = \frac{T_C}{T_B}$,则
$$ e^S = \left(\frac{r_c}{r_e}\right)^{\frac{C_{n_1}C_{n_2}}{C_{n_2}-C_{n_1}}}$$
记{\blue $\lambda = \frac{C_{n_2}}{C_{n_2}-C_{n_1}}$},代入前面的结果得到
{\blue $$\eta =1- \frac{\lambda}{\lambda-1} \frac{r_c^\lambda r_e - r_e^\lambda r_c}{r_c^\lambda - r_e^\lambda}$$}
}
\ech
\end{frame}

\section{Homework}

\begin{frame}
\chtitle{第十一周作业(序号接第十周)}
\bch
\bitem
\item[27]{教材习题3-8}
\item[28]{教材习题4-1}
\item[29]{教材习题3-18}
\eitem

\ech
\end{frame}

\end{document}

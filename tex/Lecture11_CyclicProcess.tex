\documentclass[CJK]{beamer}
\usepackage{CJKutf8}
\usepackage{beamerthemesplit}
\usetheme{Malmoe}
\useoutertheme[footline=authortitle]{miniframes}
\usepackage{amsmath}
\usepackage{amssymb}
\usepackage{graphicx}
\usepackage{eufrak}
\usepackage{color}
\usepackage{slashed}
\usepackage{simplewick}
\usepackage{tikz}
\graphicspath{{../figures/}}
\def\addfig#1#2{\begin{center}\includegraphics[width=#1 in]{#2}\end{center}}
\def\blacktext#1{{\color{black}#1}}
\def\bluetext#1{{\color{blue}#1}}
\def\redtext#1{{\color{red}#1}}
\def\darkbluetext#1{{\color[rgb]{0,0.2,0.6}#1}}
\def\skybluetext#1{{\color[rgb]{0.2,0.7,1.}#1}}
\def\cyantext#1{{\color[rgb]{0.,0.5,0.5}#1}}
\def\greentext#1{{\color[rgb]{0,0.7,0.1}#1}}
\def\darkgray{\color[rgb]{0.2,0.2,0.2}}
\def\lightgray{\color[rgb]{0.6,0.6,0.6}}
\def\gray{\color[rgb]{0.4,0.4,0.4}}
\def\blue{\color{blue}}
\def\red{\color{red}}
\def\green{\color{green}}
\def\darkblue{\color[rgb]{0,0.2,0.6}}
\def\skyblue{\color[rgb]{0.2,0.7,1.}}
\def\fdeg{{^\circ \mathrm{F}}}
\def\cdeg{^\circ \mathrm{C}}
\def\be{\begin{equation}}
\def\ee{\nonumber\end{equation}}
\def\bea{\begin{eqnarray}}
\def\eea{\nonumber\end{eqnarray}}
\def\ii{{\dot{\imath}}}
\def\bch{\begin{CJK}{UTF8}{gbsn}}
\def\ech{\end{CJK}}
\def\bitem{\begin{itemize}}
\def\eitem{\end{itemize}}
\def\bcenter{\begin{center}}
\def\ecenter{\end{center}}
\def\bex{\begin{minipage}{0.3\textwidth}\includegraphics[width=1in]{jugelizi.png}\end{minipage}\begin{minipage}{0.6\textwidth}}
\def\eex{\end{minipage}}
\def\chtitle#1{\frametitle{\bch#1\ech}}
\def\skipline{{\vskip0.1in}}
\def\skiplines{{\vskip0.2in}}
\def\lagr{{\mathcal{L}}}
\def\hamil{{\mathcal{H}}}
\def\vecv{{\mathbf{v}}}
\def\vecx{{\mathbf{x}}}
\def\vecy{{\mathbf{y}}}
\def\veck{{\mathbf{k}}}
\def\vecp{{\mathbf{p}}}
\def\vecn{{\mathbf{n}}}
\def\vecA{{\mathbf{A}}}
\def\vecP{{\mathbf{P}}}
\def\vecsigma{{\mathbf{\sigma}}}
\def\hatJn{{\hat{J_\vecn}}}
\def\hatJx{{\hat{J_x}}}
\def\hatJy{{\hat{J_y}}}
\def\hatJz{{\hat{J_z}}}
\def\hatj#1{\hat{J_{#1}}}
\def\hatphi{{\hat{\phi}}}
\def\hatq{{\hat{q}}}
\def\hatpi{{\hat{\pi}}}
\def\vel{\upsilon}
\def\Dint{{\mathcal{D}}}
\def\adag{{\hat{a}^\dagger}}
\def\bdag{{\hat{b}^\dagger}}
\def\cdag{{\hat{c}^\dagger}}
\def\ddag{{\hat{d}^\dagger}}
\def\hata{{\hat{a}}}
\def\hatb{{\hat{b}}}
\def\hatc{{\hat{c}}}
\def\hatd{{\hat{d}}}
\def\hatN{{\hat{N}}}
\def\hatH{{\hat{H}}}
\def\hatp{{\hat{p}}}
\def\Fup{{F^{\mu\nu}}}
\def\Fdown{{F_{\mu\nu}}}
\def\newl{\nonumber \\}
\def\SIkm{\,\mathrm{km}}
\def\SIyr{\,\mathrm{yr}}
\def\SIGyr{\,\mathrm{Gyr}}
\def\SIeV{\,\mathrm{eV}}
\def\SIkeV{\,\mathrm{keV}}
\def\SIMeV{\,\mathrm{MeV}}
\def\SIGeV{\,\mathrm{GeV}}
\def\SIcal{\,\mathrm{cal}}
\def\SIkcal{\,\mathrm{kcal}}
\def\SImol{\,\mathrm{mol}}
\def\SIm{\,\mathrm{m}}
\def\SIcm{\,\mathrm{cm}}
\def\SIfm{\,\mathrm{fm}}
\def\SImm{\,\mathrm{mm}}
\def\SInm{\,\mathrm{nm}}
\def\SImum{\,\mathrm{\mu m}}
\def\SIJ{\,\mathrm{J}}
\def\SIkJ{\,\mathrm{kJ}}
\def\SIs{\,\mathrm{s}}
\def\SIkg{\,\mathrm{kg}}
\def\SIg{\,\mathrm{g}}
\def\SIK{\,\mathrm{K}}
\def\SImmHg{\,\mathrm{mmHg}}
\def\SIPa{\,\mathrm{Pa}}
\def\vece{\mathrm{e}}
\def\bmat#1{\left(\begin{array}{#1}}
\def\emat{\end{array}\right)}
\def\bcase#1{\left\{\begin{array}{#1}}
\def\ecase{\end{array}\right.}
\def\calM{{\mathcal{M}}}
\def\calT{{\mathcal{T}}}
\def\calR{{\mathcal{R}}}
\def\barpsi{\bar{\psi}}
\def\baru{\bar{u}}
\def\barv{\bar{\upsilon}}
\def\bmini#1{\begin{minipage}{#1\textwidth}}
\def\emini{\end{minipage}}
\def\qeq{\stackrel{?}{=}}
\def\torder#1{\mathcal{T}\left(#1\right)}
\def\rorder#1{\mathcal{R}\left(#1\right)}
\def\contr#1#2{\contraction{}{#1}{}{#2}#1#2}
\def\trof#1{\mathrm{Tr}\left(#1\right)}
\def\trace{\mathrm{Tr}}
\def\comm#1{\ \ \ \left(\mathrm{used}\ #1\right)}
\def\tcomm#1{\ \ \ (\text{#1})}
\def\slp{\slashed{p}}
\def\slk{\slashed{k}}
\def\wulian{\includegraphics[width=0.18in]{emoji_wulian.jpg}}
\def\bye{\includegraphics[width=0.18in]{emoji_bye.jpg}}
\def\calp{{\mathfrak{p}}}
\def\veccalp{\mathbf{\mathfrak{p}}}
\def\atm{\,\mathrm{atm}}
\def\angstrom{\,\text{\AA}}
\def\Tthree{T_{\tiny \textcircled{3}}}
\def\pthree{p_{\tiny \textcircled{3}}}

\def\courseurl{http://zhiqihuang.top}

\def\tpage#1#2{
\begin{frame}
\bch
\begin{center}
\begin{large}
热学 \\
第#1讲 #2

\end{large}

\skiplines

黄志琦


\end{center}

\skiplines

{\small 
教材:《热学》第二版,赵凯华,罗蔚茵,高等教育出版社


课件下载
}
\courseurl 
\ech
\end{frame}
}

\def\bfr#1{
\begin{frame}
\chtitle{#1} 
\bch
}

\def\efr{
\ech 
\end{frame}
}

\title{Lesson 11 Cyclic Process}
  \author{}
  \date{}
\begin{document}
\tpage{11}{循环过程}

\section{Review}

\begin{frame}
\chtitle{上讲内容回顾}
\bch 
\bitem
\item{焓和状态方程的关系$\pfrac HpT = V - \pfrac V{\ln T}p$}
\item{定压比热$C_p = \pfrac HTp$}
\item{节流是等焓的不可逆过程。焦耳-汤姆孙系数$\alpha \equiv \pfrac TpH$}
\item{焦耳-汤姆孙效应:正节流效应($\alpha >0$)降温和负节流效应$\alpha<0$升温}
\item{节流过程液化气体:预冷降温到正节流区,反复节流液化气体。}
\eitem
\ech
\end{frame}

\begin{frame}
\chtitle{本讲内容}
\bch
\bitem
\item{定体热容固定的理想气体的熵}
\item{孤立系统的熵增大原理}
\item{循环过程}
\item{卡诺循环}
\item{理想气体循环举例}
\eitem
\ech
\end{frame}

\section{2nd Law}

\begin{frame}
\chtitle{热容固定的理想气体的熵}
\bch
如果理想气体的$C_V$为常数(单原子理想气体,或者多原子理想气体在一定温度范围内),则
$$ dS = \frac{\dbar Q}{T} = \frac{C_VdT +  p dV}{T} = \frac{C_V}{T}dT + \nu R\frac{dV}{V} $$
利用理想气体状态方程,有$\frac{dT}{T} = \frac{dp}{p}+\frac{dV}{V}$,代入上式,并利用$C_p = C_V+\nu R$以及$\gamma = \frac{C_p}{C_V}$,得到
$$ dS = C_V \left(\frac{dp}{p} + \gamma \frac{dV}{V}\right)$$
所以{\blue 理想气体 
$$ \Delta S = C_V \Delta \left(pV^\gamma\right) = C_V\Delta\left(TV^{\gamma-1}\right) = C_p\Delta\left(p^{\frac{1}{\gamma}-1}T\right)$$
}
利用熵的表达式立刻可以得到理想气体绝热方程。

\ech
\end{frame}

\begin{frame}
\chtitle{课堂练习}
\bch
教材习题3-19
\ech
\end{frame}


\begin{frame}
\chtitle{孤立系统的熵增大原理}
\bch
假想一个孤立的非热平衡系统$\Sigma$,把它划分成很多子系统$\Sigma_1$, $\Sigma_2$, $\ldots$, $\Sigma_N$。
每个子系统$\Sigma_i$在短时间内可以近似看成处于热平衡,有确定的温度$T_i$,压强$p_i$,体积$V_i$和熵$S_i$。各个子系统之间保持接触(非孤立)。

\skiplines


假设子系统边界处的分子相互作用可以忽略,系统总熵等于各个子系统熵之和:
$$S = \sum_i S_i$$
\ech
\end{frame}

\begin{frame}
\chtitle{孤立系统的熵增大原理(续)}
\bch
因系统处于非热平衡状态,总有两个互相接触的子系统温度不同,不妨设$T_i>T_j$。
我们假设{\bf 在宏观尺度上,热量会且只会自发地从高温物体传递到低温物体},在很短时间内$\Sigma_i$对$\Sigma_j$传热$Q_{i\rightarrow j}$ ($Q_{i\rightarrow j}>0$)。那么系统的总熵变化为
$$\Delta S = \Delta S_i + \Delta S_j = - \frac{Q_{i\rightarrow j}}{T_i}+ \frac{Q_{i\rightarrow j}}{T_j}  > 0$$
这样的熵增大过程会持续进行,直到系统达到热平衡(所有的$T_i = T_j$)后熵不再变化。

{\bf 孤立系统熵增大是一个不可逆的过程(因为无法使熵减小),熵不变的过程则原则上都可逆。}

\ech
\end{frame}

\begin{frame}
\chtitle{孤立系统的熵增大原理(续)}
\bch
总结一下:

\bitem
\item{\bf 孤立的非热平衡系统的熵总是自发地增大,直到达到热平衡态(熵最大的状态)。热平衡态的熵不再发生变化。}
\item{\bf 孤立系统的可逆过程熵不变,不可逆过程熵增大。}
\eitem

孤立系统的熵增大原理实际上是{\bf 热力学第二定律}的一种表述方式。

\ech
\end{frame}


\section{Cyclic Process}

\begin{frame}
\chtitle{蒸汽机}
\bch
\addfig{3}{steam_engine.jpg}
\ech
\end{frame}

\begin{frame}
\chtitle{循环过程}
\bch
{\bf 循环过程:一系统由某个状态出发,经过一系列过程,最后回到原状态。}

\skiplines
\bitem
\item{因循环过程的内能不变,按照热一律即有$Q+A = 0$。}
\item{把系统和环境看成一个总的孤立系统。系统回到原状态,熵不变。根据孤立系统的熵增大原理,{\bf 循环过程后环境的熵必须增大(如果循环过程对环境造成的影响是不可逆的)或者不变(如果循环过程对环境造成的影响是可逆的)。}}
\eitem

如果{\bf 循环过程对环境造成的影响是可逆}的,我们称之为{\bf 可逆循环}。否则称之为不可逆循环。

\ech
\end{frame}

\begin{frame}
\chtitle{正循环热机}
\bch

系统对外界做功($A'>0$)的循环为正循环。

\addfig{3}{heat_engine.jpg}
\ech
\end{frame}

\begin{frame}
\chtitle{逆循环制冷机}
\bch

外界对系统做功($A>0$)的循环为逆循环。逆循环热机也叫制冷机。

\addfig{3}{cold_engine.jpg}
\ech
\end{frame}

\begin{frame}
\chtitle{思考题}
\bch
下列过程是正循环还是逆循环?
\bitem
\item{蒸汽机的一个循环}
\item{制冷机的一个循环}
\item{净吸热量$Q>0$的循环}
\item{净放热量$Q'>0$的循环}
\item{$p$-$V$图上顺时针的闭合曲线}
\item{$p$-$V$图上逆时针的闭合曲线}
\item{$T$-$S$图上顺时针的闭合曲线}
\item{$T$-$S$图上逆时针的闭合曲线}
\eitem
\ech
\end{frame}


\begin{frame}
\chtitle{骨灰级难度思考题}
\bch

逆循环也许可以是可逆循环;

逆循环可不一定是可逆循环;

逆循环也可以不是可逆循环;

逆循环也可以是不可逆循环;

逆循环也可不是不可逆循环;

可逆循环也许可以是逆循环;

可逆循环也可以不是逆循环;

可逆循环可不一定是逆循环;

可逆循环必不是不可逆循环;

不可逆循环必不是可逆循环;

不可逆循环可以不是逆循环;

不可逆循环未必不是逆循环;

不可逆循环也可以是逆循环。

那么问题来了——

如果逆循环是可逆循环,那么逆循环的逆循环是逆循环吗?




\ech
\end{frame}


\begin{frame}
\chtitle{正循环热机的效率}
\bch

\addfig{2}{heat_engine.jpg}


{\bf 正循环把热量转化为机械功。}设正循环热机从高温热源1吸热$Q_1$,对外做功$A'$,并对低温热源2放热$Q_2'$。{\bf 热量转化为机械能的百分比称为正循环热机的效率,记作$\eta$。}

$$\eta \equiv \frac{A'}{Q_1}=\frac{Q_1-Q_2'}{Q_1} $$


\ech
\end{frame}


\begin{frame}
\chtitle{恒温热源间的正循环的效率}
\bch
\addfig{2}{heat_engine.jpg}
$$\eta = \frac{Q_1-Q_2'}{Q_1} = 1-\frac{T_2|\Delta S_2|}{T_1|\Delta S_1|}$$
{\small 其中$|\Delta S_1|$为高温热源的熵减少量,$|\Delta S_2|$为低温热源的熵增加量。

如果整个循环过程{\bf 可逆},则环境的总熵不变:$|\Delta S_1| = |\Delta S_2|$。

$$ \eta = 1 - \frac{T_2}{T_1}$$

如果整个循环过程{\bf 不可逆},则环境总熵增大:$|\Delta S_2| > |\Delta S_1|$,即
$$\eta < 1 - \frac{T_2}{T_1} $$
}
\ech
\end{frame}


\begin{frame}
\chtitle{卡诺定理}
\bch
\addfig{2}{heat_engine.jpg}

在温度为$T_1$的高温热源和温度为$T_2$的低温热源之间工作的热机:{\bf 可逆循环$\eta = 1 -\frac{T_2}{T_1}$,不可逆循环$\eta < 1 - \frac{T_2}{T_1}$,这称为卡诺(Carnot)定理}。
\ech
\end{frame}

\begin{frame}
\chtitle{逆循环热机的效率}
\bch

\addfig{2}{cold_engine.jpg}

{\bf 逆循环利用外界提供的机械功从低温热源吸热制冷。}设外界对逆循环热机做功$A$,使它从低温热源2吸热$Q_2$,并对高温热源放热$Q_1'$。{\bf 制冷量$Q_2$与外功$A$之比称为逆循环热机的制冷系数,记作$\varepsilon$。}

$$\varepsilon \equiv \frac{Q_2}{A}=\frac{Q_2}{Q_1'-Q_2} $$

\ech
\end{frame}


\begin{frame}
\chtitle{恒温热源间的逆循环的制冷效率}
\bch
{\small
设外界对热机做功$A$,使热机从温度恒为$T_2$的低温热源2吸热$Q_2$,并对温度恒为$T_1$的高温热源1放热$Q_1'$。则
$$\varepsilon = \frac{1}{Q_1'/Q_2 - 1} = \frac{1}{\frac{T_1|\Delta S_1|}{T_2|\Delta S_2|}-1}$$
其中$|\Delta S_1|$为高温热源的熵增加量,$|\Delta S_2|$为低温热源的熵减少量。

如果整个循环过程{\bf 可逆},则环境的总熵不变:$|\Delta S_1| = |\Delta S_2|$。

$$ \varepsilon = \frac{T_2}{T_1-T_2}$$

如果整个循环过程{\bf 不可逆},则环境总熵增大:$|\Delta S_2| < |\Delta S_1|$,即
$$\varepsilon <  \frac{T_2}{T_1-T_2} $$
}
\ech
\end{frame}

\begin{frame}
\chtitle{环保常识: 空调温度别开太低}
\bch
设夏天室外温度为$30\cdeg$。把空调近似看成可逆热机。试估算制冷温度分别为$28\cdeg$和$20\cdeg$时空调制冷效率之比。
\ech
\end{frame}

\begin{frame}
\bch
下面我们以理想气体为例讨论一些准静态循环(即可逆循环)。

\skipline

这些循环的中间过程都有明确的态函数$p$, $V$, $T$, $S$。
\ech
\end{frame}

\begin{frame}
\chtitle{正循环$p$-$V$图}
\bch
\addfig{3.2}{pVdiagram_cycle1.png}
\ech
\end{frame}

\begin{frame}
\chtitle{逆循环$p$-$V$图}
\bch
\addfig{3.2}{pVdiagram_cycle2.png}
\ech
\end{frame}


\begin{frame}
\chtitle{正循环$T$-$S$图}
\bch
\addfig{3.2}{TSdiagram_cycle2.png}
\ech
\end{frame}

\begin{frame}
\chtitle{逆循环$T$-$S$图}
\bch
\addfig{3.2}{TSdiagram_cycle1.png}
\ech
\end{frame}


\begin{frame}
\chtitle{卡诺循环(Carnot Cycle)}
\bch
卡诺循环 = 等温膨胀+ 绝热膨胀 + 等温压缩 + 绝热压缩
\ech
\end{frame}

\begin{frame}
\chtitle{卡诺循环的$p$-$V$图}
\bch
\addfig{2.3}{Carnot_cycle.png}

思考题:对理想气体计算可逆卡诺循环的效率(设等温过程的$T_1$, $T_2$已知)。
\ech
\end{frame}


\begin{frame}
\chtitle{猜一猜}
\bch
刚才算出理想气体可逆卡诺循环的效率为$\eta = 1-\frac{T_2}{T_1}$,这个结果对非理想气体成立吗?
\ech
\end{frame}

\begin{frame}
\chtitle{卡诺循环的$T$-$S$图}
\bch
\addfig{2.5}{Carnot_cycle_TS.png}

在这个图里计算热机效率特别容易(和工作物质无关):
$$\eta = \frac{A'}{Q_1} = \frac{(T_1-T_2)(S_2-S_1)}{T_1(S_2-S_1)} = 1- \frac{T_2}{T_1}$$

\ech
\end{frame}




\begin{frame}
\chtitle{奥托循环(Otto Cycle)}
\bch

阅读教材160页例题7,并试用$T$-$S$图计算。

\skiplines

{\small
提示:利用熵的表达式得出
$$ T \propto V^{1-\gamma} e^\frac{S}{C_V}$$
从而得出$T$-$S$图上两条等体线的比值为常数。
}
\ech
\end{frame}

\begin{frame}
\chtitle{多方循环}
\bch
设一个循环由两个绝热过程和两个多方过程组成,如图所示:
\bmini{0.5}
\addfig{1.6}{polytropic_cycle.jpg}
\emini
\bmini{0.46}
{\small
已知参数:
\bitem
\item{多方指数$n_1, n_2$}
\item{绝热压缩比$r=\frac{V_D}{V_A}>1$}
\item{体积交叉因子$\rho =\frac{V_BV_D}{V_AV_C}$}
\eitem
}
\emini

{\small

证明该循环的效率为
$$\eta = 1 - \frac{C_{n_2}}{C_{n_1}}\frac{1}{r^{\gamma-1}}\frac{\rho^{\frac{(1-n_2)(\gamma-n_1)}{n_1-n_2}}-1}{\rho^{\frac{(1-n_1)(\gamma-n_2)}{n_1-n_2}}-1} $$
当$n_1\rightarrow n_2$,上式成为
$$\eta = 1- \frac{1}{r^{\gamma-1}}$$
}

\ech
\end{frame}



\begin{frame}
\chtitle{第十一周作业(序号接第十周)}
\bch
\bitem
\item[27]{教材习题4-1}
\item[28]{教材习题3-8}
\item[29]{教材习题3-18}
\eitem

\ech
\end{frame}

\end{document}

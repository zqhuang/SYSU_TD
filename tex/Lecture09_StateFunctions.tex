\documentclass[CJK]{beamer}
\usepackage{CJKutf8}
\usepackage{beamerthemesplit}
\usetheme{Malmoe}
\useoutertheme[footline=authortitle]{miniframes}
\usepackage{amsmath}
\usepackage{amssymb}
\usepackage{graphicx}
\usepackage{eufrak}
\usepackage{color}
\usepackage{slashed}
\usepackage{simplewick}
\usepackage{tikz}
\graphicspath{{../figures/}}
\def\addfig#1#2{\begin{center}\includegraphics[width=#1 in]{#2}\end{center}}
\def\blacktext#1{{\color{black}#1}}
\def\bluetext#1{{\color{blue}#1}}
\def\redtext#1{{\color{red}#1}}
\def\darkbluetext#1{{\color[rgb]{0,0.2,0.6}#1}}
\def\skybluetext#1{{\color[rgb]{0.2,0.7,1.}#1}}
\def\cyantext#1{{\color[rgb]{0.,0.5,0.5}#1}}
\def\greentext#1{{\color[rgb]{0,0.7,0.1}#1}}
\def\darkgray{\color[rgb]{0.2,0.2,0.2}}
\def\lightgray{\color[rgb]{0.6,0.6,0.6}}
\def\gray{\color[rgb]{0.4,0.4,0.4}}
\def\blue{\color{blue}}
\def\red{\color{red}}
\def\green{\color{green}}
\def\darkblue{\color[rgb]{0,0.2,0.6}}
\def\skyblue{\color[rgb]{0.2,0.7,1.}}
\def\fdeg{{^\circ \mathrm{F}}}
\def\cdeg{^\circ \mathrm{C}}
\def\be{\begin{equation}}
\def\ee{\nonumber\end{equation}}
\def\bea{\begin{eqnarray}}
\def\eea{\nonumber\end{eqnarray}}
\def\ii{{\dot{\imath}}}
\def\bch{\begin{CJK}{UTF8}{gbsn}}
\def\ech{\end{CJK}}
\def\bitem{\begin{itemize}}
\def\eitem{\end{itemize}}
\def\bcenter{\begin{center}}
\def\ecenter{\end{center}}
\def\bex{\begin{minipage}{0.3\textwidth}\includegraphics[width=1in]{jugelizi.png}\end{minipage}\begin{minipage}{0.6\textwidth}}
\def\eex{\end{minipage}}
\def\chtitle#1{\frametitle{\bch#1\ech}}
\def\skipline{{\vskip0.1in}}
\def\skiplines{{\vskip0.2in}}
\def\lagr{{\mathcal{L}}}
\def\hamil{{\mathcal{H}}}
\def\vecv{{\mathbf{v}}}
\def\vecx{{\mathbf{x}}}
\def\vecy{{\mathbf{y}}}
\def\veck{{\mathbf{k}}}
\def\vecp{{\mathbf{p}}}
\def\vecn{{\mathbf{n}}}
\def\vecA{{\mathbf{A}}}
\def\vecP{{\mathbf{P}}}
\def\vecsigma{{\mathbf{\sigma}}}
\def\hatJn{{\hat{J_\vecn}}}
\def\hatJx{{\hat{J_x}}}
\def\hatJy{{\hat{J_y}}}
\def\hatJz{{\hat{J_z}}}
\def\hatj#1{\hat{J_{#1}}}
\def\hatphi{{\hat{\phi}}}
\def\hatq{{\hat{q}}}
\def\hatpi{{\hat{\pi}}}
\def\vel{\upsilon}
\def\Dint{{\mathcal{D}}}
\def\adag{{\hat{a}^\dagger}}
\def\bdag{{\hat{b}^\dagger}}
\def\cdag{{\hat{c}^\dagger}}
\def\ddag{{\hat{d}^\dagger}}
\def\hata{{\hat{a}}}
\def\hatb{{\hat{b}}}
\def\hatc{{\hat{c}}}
\def\hatd{{\hat{d}}}
\def\hatN{{\hat{N}}}
\def\hatH{{\hat{H}}}
\def\hatp{{\hat{p}}}
\def\Fup{{F^{\mu\nu}}}
\def\Fdown{{F_{\mu\nu}}}
\def\newl{\nonumber \\}
\def\SIkm{\,\mathrm{km}}
\def\SIyr{\,\mathrm{yr}}
\def\SIGyr{\,\mathrm{Gyr}}
\def\SIeV{\,\mathrm{eV}}
\def\SIkeV{\,\mathrm{keV}}
\def\SIMeV{\,\mathrm{MeV}}
\def\SIGeV{\,\mathrm{GeV}}
\def\SIcal{\,\mathrm{cal}}
\def\SIkcal{\,\mathrm{kcal}}
\def\SImol{\,\mathrm{mol}}
\def\SIm{\,\mathrm{m}}
\def\SIcm{\,\mathrm{cm}}
\def\SIfm{\,\mathrm{fm}}
\def\SImm{\,\mathrm{mm}}
\def\SInm{\,\mathrm{nm}}
\def\SImum{\,\mathrm{\mu m}}
\def\SIJ{\,\mathrm{J}}
\def\SIkJ{\,\mathrm{kJ}}
\def\SIs{\,\mathrm{s}}
\def\SIkg{\,\mathrm{kg}}
\def\SIg{\,\mathrm{g}}
\def\SIK{\,\mathrm{K}}
\def\SImmHg{\,\mathrm{mmHg}}
\def\SIPa{\,\mathrm{Pa}}
\def\vece{\mathrm{e}}
\def\bmat#1{\left(\begin{array}{#1}}
\def\emat{\end{array}\right)}
\def\bcase#1{\left\{\begin{array}{#1}}
\def\ecase{\end{array}\right.}
\def\calM{{\mathcal{M}}}
\def\calT{{\mathcal{T}}}
\def\calR{{\mathcal{R}}}
\def\barpsi{\bar{\psi}}
\def\baru{\bar{u}}
\def\barv{\bar{\upsilon}}
\def\bmini#1{\begin{minipage}{#1\textwidth}}
\def\emini{\end{minipage}}
\def\qeq{\stackrel{?}{=}}
\def\torder#1{\mathcal{T}\left(#1\right)}
\def\rorder#1{\mathcal{R}\left(#1\right)}
\def\contr#1#2{\contraction{}{#1}{}{#2}#1#2}
\def\trof#1{\mathrm{Tr}\left(#1\right)}
\def\trace{\mathrm{Tr}}
\def\comm#1{\ \ \ \left(\mathrm{used}\ #1\right)}
\def\tcomm#1{\ \ \ (\text{#1})}
\def\slp{\slashed{p}}
\def\slk{\slashed{k}}
\def\wulian{\includegraphics[width=0.18in]{emoji_wulian.jpg}}
\def\bye{\includegraphics[width=0.18in]{emoji_bye.jpg}}
\def\calp{{\mathfrak{p}}}
\def\veccalp{\mathbf{\mathfrak{p}}}
\def\atm{\,\mathrm{atm}}
\def\angstrom{\,\text{\AA}}
\def\Tthree{T_{\tiny \textcircled{3}}}
\def\pthree{p_{\tiny \textcircled{3}}}

\def\courseurl{http://zhiqihuang.top}

\def\tpage#1#2{
\begin{frame}
\bch
\begin{center}
\begin{large}
热学 \\
第#1讲 #2

\end{large}

\skiplines

黄志琦


\end{center}

\skiplines

{\small 
教材:《热学》第二版,赵凯华,罗蔚茵,高等教育出版社


课件下载
}
\courseurl 
\ech
\end{frame}
}

\def\bfr#1{
\begin{frame}
\chtitle{#1} 
\bch
}

\def\efr{
\ech 
\end{frame}
}

\title{Lesson 09 State Functions}
  \author{}
  \date{}
\begin{document}
\tpage{9}{$pVT$系统的态函数}


\begin{frame}
\chtitle{本讲内容}
\bch
\bitem
\item{数学准备知识}
\item{强度量和广延量}
\item{温度,熵,压强,体积}
\item{内能,焓,自由能和自由焓}
\item{内能的物理内涵}
\eitem
\ech
\end{frame}

\section{Math}

\begin{frame}
\chtitle{数学准备知识I:循环偏微分乘积定理}
\bch
设三个变量$X$, $Y$, $Z$满足某状态方程,则
{\blue 
$$\pfrac XYZ  \pfrac YZX \pfrac ZXY = -1$$
}
当然,我们这是讨论物理问题,可以假设状态方程足够光滑,且不出现偏导数无穷大的情况。

\skipline

\bye不会证的请默默捧起数学书。
\ech
\end{frame}


\begin{frame}
\chtitle{什么,数学书早卖了?}
\bch
{\scriptsize
证明:固定$Z$时,$dZ = 0$可以写成
$$\pfrac ZXY dX + \pfrac ZYX dY = 0$$
又固定$Z$时的$dX/dY$就是$\pfrac XYZ$,故
$$\pfrac XYZ \pfrac ZXY + \pfrac ZYX = 0$$
上式乘以$\pfrac YZX$即得
$$\pfrac XYZ \pfrac YZX \pfrac ZXY + 1 = 0$$
}
\ech
\end{frame}

\begin{frame}
\chtitle{新的链式法则}
\bch
我们熟悉的微分链式法则是:
$$\pfrac XYS \pfrac YZS = \pfrac XZS$$
{\small \bf (固定量不变时,普通微分的所有法则都适用于偏微分)}
 
\skipline

而循环偏微分乘积定理给出另一种链式法则:
$$\pfrac XYZ \pfrac YZX = -\pfrac XZY $$

\skiplines

请仔细体会两种链式法则的不同之处。
\ech
\end{frame}



\begin{frame}
\chtitle{思考题}
\bch
\addfig{1}{songfen.jpg}

证明物质的定体比热容可以写成:

$$
C_V =  - \pfrac UVT \pfrac VTU 
$$

\ech
\end{frame}

\begin{frame}
\chtitle{数学准备知识II:四变量链式法则}
\bch
设$W, X, Y, Z$中仅有两个独立变量,则有
$$ \pfrac WZX \pfrac ZXY = \pfrac WXY - \pfrac WXZ  $$

{\scriptsize
证明:固定$Y$,变化$X$时
\bea
dW &=& \pfrac WXZ dX + \pfrac WZX dZ \newl
&=& \pfrac WXZ dX + \pfrac WZX \pfrac ZXY dX \newl
&=& \left(\pfrac WXZ + \pfrac WZX \pfrac ZXY\right) dX
\eea
故
$$\pfrac WXY = \pfrac WXZ + \pfrac WZX \pfrac ZXY$$
}
\ech
\end{frame}

\begin{frame}
\chtitle{思考题}
\bch
\addfig{1}{songfen.jpg}
试用四变量链式法则证明循环偏微分乘积定理。
\ech
\end{frame}


\begin{frame}
\chtitle{思考题}
\bch
\addfig{1}{songfen2.jpg}

压强$p$, 体积$V$,温度$T$满足一定的物态方程,内能$U$是$p, T, V$的函数。
证明
$$\pfrac UVT \pfrac VTp + \pfrac UpT \pfrac pTV = 0$$

\ech
\end{frame}

\begin{frame}
\chtitle{数学准备知识III:全微分条件}
\bch
设系统可由两个独立参量$x$和$y$完整地描述。微分表达式
$$ f(x, y) dx + g(x, y) dy$$
是全微分(恒等于某函数$h(x,y)$的微分)的充要条件是
$$\pfrac fyx = \pfrac gxy$$
处处成立。

\ech
\end{frame}

\begin{frame}
\chtitle{数学准备知识III:全微分条件}
\bch
{\small 
证明:若存在$h(x,y)$使$dh = f dx + g dy$

则$ f = \pfrac hxy; g = \pfrac hyx $, 故
$$ \pfrac fyx = \frac{\partial^2 h}{\partial x \partial y} = \frac{\partial^2 h}{\partial y \partial x}= \pfrac gxy $$
(什么?二次偏导未必存在?偏导次序不能随便交换?\bye 数学老师再见)

\skipline
反之,若 $\pfrac fyx = \pfrac gxy $,则由Stokes定理,沿任意闭合回路(设为区域$C$的边界$\partial C$)的积分
$$\oint_{\partial C} fdx + gdy = \iint_C \left(\frac{\partial g}{\partial x} -\frac{\partial f}{\partial y} \right)dxdy = 0 $$
从而可取$h(x, y)$为从原点到$(x,y)$的积分$\int fdx + gdy$ (显然结果和积分路径无关)。

}
\ech
\end{frame}


\begin{frame}
\chtitle{数学准备知识IV:积分因子}
\bch
设系统可由两个独立参量$x$和$y$完整地描述。对任意微分表达式
$$ f(x, y) dx + g(x, y) dy$$
总能找到恒正的函数$\lambda(x, y)$使 
$$ \lambda\left[ f(x, y) dx + g(x, y) dy \right]$$
为全微分。$\lambda(x,y)$称为$f dx + g dy$的积分因子。
\ech
\end{frame}

\begin{frame}
\chtitle{数学准备知识IV:积分因子}
\bch
{\small
证明:
取$dx  = g(x,y) ds$, $dy = -f(x, y) ds$的曲线簇,每条曲线上的$\lambda$由下述微分式积分得到
$$d\ln \lambda = \left(\frac{\partial f}{\partial y} - \frac{\partial g}{\partial x}\right) ds$$
初始值可以在和曲线簇垂直的任意一条曲线上取任意光滑函数来确定。
}
{\scriptsize

\bea
 \frac{\partial(\lambda f)}{\partial y} - \frac{\partial(\lambda g)}{\partial x} &=& \lambda\left(\frac{\partial f}{\partial y} - \frac{\partial g}{\partial x} + f\frac{\partial \ln \lambda}{\partial y} -g\frac{\partial \ln \lambda}{\partial x}  \right) \newl 
&=& \frac{\lambda}{ds} \left[ d\ln \lambda  - dy \frac{\partial \ln \lambda}{\partial y} - dx \frac{\partial \ln \lambda}{\partial x}  \right] \newl 
&=& 0
\eea
}
{\small
故$\lambda f dx + \lambda g dy $为全微分。
}
\ech
\end{frame}

\begin{frame}
\bch
本讲我们讨论满足一定状态方程的$p$, $V$, $T$系统(也就是只有两个自由度的系统),我们称之为$pVT$系统。
\ech
\end{frame}


\section{Extensive/Intensive Quantities}


\begin{frame}
\chtitle{广延量(extensive quantity)和强度量(intensive quantity)}
\bch
\bitem
\item{广延量:和物质之量成正比,是可加的
$$U(\clubsuit + \clubsuit) = 2 U(\clubsuit)$$
$$U(\clubsuit + \spadesuit) = U(\clubsuit) +  U(\spadesuit)$$

}
\item{强度量:和物质之量无关,非相加的
$$ T(\clubsuit + \clubsuit) = T(\clubsuit)$$
$$ T(\clubsuit + \spadesuit) = \mathrm{undefined}$$
}
\eitem
\ech
\end{frame}


\begin{frame}
\chtitle{二猜一游戏又来了}
\bch
\addfig{0.7}{songfen.jpg}

下列量是广延量还是强度量?

\bitem
\item{体积}
\item{质量}
\item{温度}
\item{摩尔数}
\item{压强}
\item{内能}
\eitem

\ech
\end{frame}

\begin{frame}
\chtitle{思考题}
\bch
\addfig{1}{songfen2.jpg}

广延量和强度量的乘积是广延量还是强度量?

\ech
\end{frame}

\begin{frame}
\chtitle{强度量在非平衡态一般没有定义}
\bch

下列强度量在非平衡态没有定义:
\bitem
\item{温度}
\item{压强}
\eitem

\ech
\end{frame}


\begin{frame}
\chtitle{广延量的定义往往可以推广到非平衡态}
\bch

下列广延量在非平衡态也可以定义:
\bitem
\item{体积}
\item{质量}
\item{摩尔数}
\item{内能}
\eitem

定义非平衡态广延量的办法是把非平衡态系统划分为很多个可以近似看作平衡态的子系统,然后根据广延量可以相加的特性把子系统的广延量相加。


\ech
\end{frame}

\begin{frame}
\chtitle{准静态过程和不可逆过程}
\bch

\bitem
\item{准静态过程中,每个中间态可以看成平衡态,故强度量和广延量都有定义。
\bex
气体缓慢地绝热膨胀,任何时刻的温度和压强均有定义。
\eex
}
\item{不可逆过程中,中间态未必是平衡态,强度量一般没有定义。
\bex
突然增大容器体积,气体自由膨胀,中间过程的温度和压强没有定义。
\eex
}
\eitem

\ech
\end{frame}


\begin{frame}
\chtitle{$dU + pdV$一般不是全微分}
\bch

对一个准静态微过程,考虑下列微元
$$ dU + pdV$$
其中$U$为内能,$p$为压强,$V$为体积。

这对应于全微分判定法则中的$f = 1$, $g = p$的情形,而
$\pfrac fVU = 0$, $\pfrac gUV=\pfrac pUV$一般不为零,故$dU + pdV$一般不是全微分。

\ech
\end{frame}


\begin{frame}
\chtitle{加横的微分号的确切含义}
\bch
我们对准静态过程定义
$$\dbar Q \equiv dU + pdV$$
因为$dU + pdV$一般不是全微分,为了防止引起混淆对$Q$的微元使用了加横的微分号,以表示它不是一个态函数的微分。
\ech
\end{frame}

\section{T, S, p, V}

\begin{frame}
\chtitle{$dU + pdV$的积分因子}
\bch
设$dU + pdV$的积分因子为$\beta$,我们定义$T=1/\beta$为热力学温度,并把全微分
$$ dS = \frac{1}{T} \left(dU + p dV \right)$$
定义为熵的微分。如果{\bf 要求$T$为强度量,$S$为广延量},则积分因子的初始值不确定性可以消除。

\tbox{
$$ dU = T dS - pdV$$}
可以看作热力学温度和熵的定义式。

对准静态过程,根据定义即有
$$\dbar Q = TdS$$

\ech
\end{frame}


\begin{frame}
\chtitle{$dU + pdV$的积分因子——单粒子的情况}
\bch
{\small
对单粒子的系统,$V=0$,设粒子在态$i$上出现概率为$p_i$
$$dU = \sum  \varepsilon_i dp_i $$
对平衡态$p_i \propto e^{-\frac{\varepsilon_i}{kT}}$,故
$$dU + pdV = dU =  \sum_i \varepsilon_i d p_i = -kT \sum_i \ln p_i d p_i = -kT d(\sum_i p_i \ln p_i)$$
在推导上式的过程中我们用了总概率守恒$\sum_i p_i d(\ln p_i) = \sum_i d p_i = 0$

按我们第4讲定义的$ S = -k \sum_i p_i \ln p_i$,即有
$$dU+pdV = T dS$$

可见,{\bf 用积分因子定义的热力学温度和熵是对单粒子系统的温度和熵的概念的一个推广}。
}
\ech
\end{frame}


\begin{frame}
\chtitle{熵是广延量}
\bch
在
$$ dU =  T dS - pdV$$
中,$U$, $V$, $S$均为广延量,$T$, $p$均为强度量。

\skiplines

因熵为广延量,非平衡态也可以有确定的熵。
\ech
\end{frame}


\begin{frame}
\chtitle{基本态函数$T$, $S$, $p$, $V$}
\bch
到现在为止,我们讨论了四种基本的态函数$T$,$S$, $p$, $V$。

内能$U$可以由
$$ dU = T dS -  p dV$$
导出。

这些态函数中
\bitem
\item{$p$, $T$是强度量,因此只对平衡态有意义。不可逆过程的中间状态一般没有确定的$p$, $T$。}
\item{$S$, $V$, $U$是广延量,对非平衡态也有意义。我们可以安全地书写$dU$, $dS$, $dV$而无须担心过程是否可逆。}
\eitem

\ech
\end{frame}

\section{U, H, F, G}

\begin{frame}
\chtitle{焓,自由能和自由焓}
\bch
在本课程中,我们还要讨论另外三个导出的态函数,它们都是广延量。对非平衡态也有意义。
\bea
H &\equiv& U + pV \newl
F &\equiv& U - TS \newl
G &\equiv& H - TS \newl
\eea
$H, F, G$分别称为焓,自由能,自由焓。它们和$U$一起组成了四个重要的导出态函数。
\ech
\end{frame}


\begin{frame}
\chtitle{总结}
\bch
\bitem
\item{四个基本态函数,温度$T$, 熵$S$, 压强$p$, 体积$V$}
\item{四个导出态函数, 内能$U$, 焓$H=U+pV$, 自由能$F=U-TS$, 自由焓$G=H-TS$}
\eitem
{\blue
\bea
dU &=& TdS - pdV \newl
dH &=& TdS + Vdp \newl
dF &=& -SdT - pdV \newl
dG &=& -SdT + Vdp \newl
\eea
}
技能提示:{\blue 在等压情况往往考虑焓和自由焓比较方便,在等体情形一般考虑内能和自由能比较方便。}
\ech
\end{frame}


\section{U and H}

\begin{frame}
\bch
下面我们来发掘一下内能的物理内涵。
\ech
\end{frame}

\begin{frame}
\chtitle{内能和状态方程的关系}
\bch

因为$d F = -SdT - pdV$是全微分,所以
{\blue
$$\pfrac SVT = \pfrac pTV $$
}
{\scriptsize (根据$U$, $H$, $G$, $F$的全微分表达式,这样的式子一共可以写出四个)}

因此固定温度,变化体积时,
$$ dU = TdS-pdV = \left(T\pfrac SVT - p\right)dV = \left( T\pfrac pTV - p\right) dV$$
即
{\blue
$$\pfrac UVT = \pfrac p{\ln T}V - p$$
}
\ech
\end{frame}

\begin{frame}
\chtitle{热量和热压强}
\bch
\addfig{1}{think3.jpg}
证明在准静态过程中
{\blue
$$\dbar Q = C_V dT + \pfrac p{\ln T}V dV$$}
我们把$\pfrac p{\ln T}V$称为{\bf 热压强}。即:

{\blue 一般过程的吸热量等于按定体比热容计算的吸热量,再加上热压强做功消耗的能量。}

\ech
\end{frame}

\begin{frame}
\bch
{\small
对单一物态的均匀系统,假设分子自由度的激发只和温度有关,则动理压强可以写为$p_k=n_{\rm eff} kT$,其中$n_{\rm eff}$是能和环境发生碰撞的有效分子数密度,可以认为$n_{\rm eff}$只和体积有关。
\bitem
\item[(1)]{证明热压强等于动理压强$p_k = \pfrac p{\ln T}V$,并写出内压强表达式。}
\item[(2)]{因为分子自由度的激发只和温度有关,所以单个分子的平均能量只是温度的函数。当固定温度改变体积时,分子平均能量不变,改变的只是分子间的势能。分子势能的消耗完全由分子间吸引力做功所导致(即内压强做功导致)。由此说明$\pfrac UVT = \pfrac p{\ln T}V - p$的物理意义。}
\item[(3)]{因为内压强做功完全由分子间势能来提供,计算吸热量时就仅需计算动理压强做功和分子平均能量的改变。由此说明$\dbar Q = C_V dT + \pfrac p{\ln T}V dV$的物理意义。}
\item[(4)]{证明该系统的$C_V$只是温度的函数。}
\item[(4)]{克劳修斯定义两个状态的熵差为连接它们的准静态过程的积分$\Delta S = \int \frac{\dbar Q}{T}$。
试对该系统证明积分的结果与路径无关。}
\eitem
}
\ech
\end{frame}


\begin{frame}
\chtitle{范德瓦尔斯气体的内能}
\bch
{\small
对范德瓦尔斯气体,我们假设了内压强$P_U = -\frac{a\nu^2}{V^2}$,按上面思考题的讨论结果,有
$$\pfrac UVT = -p_U = \frac{a\nu^2}{V^2}$$
由此可以积分得到
$$U(V, T) = \int_{T_0}^T C_V(T) dT -\frac{\nu^2a}{V} + U_0$$
}
\ech
\end{frame}


\begin{frame}
\chtitle{沸点和压强的关系}
\bch
第二讲我们提到过的沸点和压强的关系(还记得西藏的生肉吗)。反映在$pVT$图上,这就是要求一个两相共存面的斜率$\pfrac pTV$,或者投影到三相图上这就是相变曲线的导数$dp/dT$。
\lfig{2.3}{PVTdiagram.png} \lfig{2.3}{PTdiagram.png}
\ech
\end{frame}

\begin{frame}
\chtitle{克拉珀龙方程}
\bch
当物质从$\alpha$态相变到$\beta$态,温度固定在相变温度$T$,则可以计算相变潜热为
$$ \Lambda^{\rm mol} = \pfrac p{\ln T}V \Delta V = T \pfrac pTV \left(V_{\beta}^{\rm mol}-V_{\alpha}^{\rm mol}\right)$$
由此即得到{\blue 克拉珀龙方程
$$\pfrac pTV = \frac{\Lambda^{\rm mol}}{T \left(V_{\beta}^{\rm mol}-V_{\alpha}^{\rm mol}\right)} $$}
它描述了相变温度,相变压强和相变潜热之间的关系。
\ech
\end{frame}





\begin{frame}
\chtitle{第九周作业(序号接第八周)}
\bch

\bitem
\item[21]{ 教材习题3-1}
\item[22]{ 教材习题3-11}
\item[24]{ 对$pVT$系统证明$\pfrac SpT = - \pfrac VTp$}
\eitem
\ech
\end{frame}


\end{document}

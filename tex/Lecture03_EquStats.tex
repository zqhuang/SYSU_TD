\documentclass[CJK]{beamer}
\usepackage{CJKutf8}
\usepackage{beamerthemesplit}
\usetheme{Malmoe}
\useoutertheme[footline=authortitle]{miniframes}
\usepackage{amsmath}
\usepackage{amssymb}
\usepackage{graphicx}
\usepackage{eufrak}
\usepackage{color}
\usepackage{slashed}
\usepackage{simplewick}
\usepackage{tikz}
\graphicspath{{../figures/}}
\def\addfig#1#2{\begin{center}\includegraphics[width=#1 in]{#2}\end{center}}
\def\blacktext#1{{\color{black}#1}}
\def\bluetext#1{{\color{blue}#1}}
\def\redtext#1{{\color{red}#1}}
\def\darkbluetext#1{{\color[rgb]{0,0.2,0.6}#1}}
\def\skybluetext#1{{\color[rgb]{0.2,0.7,1.}#1}}
\def\cyantext#1{{\color[rgb]{0.,0.5,0.5}#1}}
\def\greentext#1{{\color[rgb]{0,0.7,0.1}#1}}
\def\darkgray{\color[rgb]{0.2,0.2,0.2}}
\def\lightgray{\color[rgb]{0.6,0.6,0.6}}
\def\gray{\color[rgb]{0.4,0.4,0.4}}
\def\blue{\color{blue}}
\def\red{\color{red}}
\def\green{\color{green}}
\def\darkblue{\color[rgb]{0,0.2,0.6}}
\def\skyblue{\color[rgb]{0.2,0.7,1.}}
\def\fdeg{{^\circ \mathrm{F}}}
\def\cdeg{^\circ \mathrm{C}}
\def\be{\begin{equation}}
\def\ee{\nonumber\end{equation}}
\def\bea{\begin{eqnarray}}
\def\eea{\nonumber\end{eqnarray}}
\def\ii{{\dot{\imath}}}
\def\bch{\begin{CJK}{UTF8}{gbsn}}
\def\ech{\end{CJK}}
\def\bitem{\begin{itemize}}
\def\eitem{\end{itemize}}
\def\bcenter{\begin{center}}
\def\ecenter{\end{center}}
\def\bex{\begin{minipage}{0.3\textwidth}\includegraphics[width=1in]{jugelizi.png}\end{minipage}\begin{minipage}{0.6\textwidth}}
\def\eex{\end{minipage}}
\def\chtitle#1{\frametitle{\bch#1\ech}}
\def\skipline{{\vskip0.1in}}
\def\skiplines{{\vskip0.2in}}
\def\lagr{{\mathcal{L}}}
\def\hamil{{\mathcal{H}}}
\def\vecv{{\mathbf{v}}}
\def\vecx{{\mathbf{x}}}
\def\vecy{{\mathbf{y}}}
\def\veck{{\mathbf{k}}}
\def\vecp{{\mathbf{p}}}
\def\vecn{{\mathbf{n}}}
\def\vecA{{\mathbf{A}}}
\def\vecP{{\mathbf{P}}}
\def\vecsigma{{\mathbf{\sigma}}}
\def\hatJn{{\hat{J_\vecn}}}
\def\hatJx{{\hat{J_x}}}
\def\hatJy{{\hat{J_y}}}
\def\hatJz{{\hat{J_z}}}
\def\hatj#1{\hat{J_{#1}}}
\def\hatphi{{\hat{\phi}}}
\def\hatq{{\hat{q}}}
\def\hatpi{{\hat{\pi}}}
\def\vel{\upsilon}
\def\Dint{{\mathcal{D}}}
\def\adag{{\hat{a}^\dagger}}
\def\bdag{{\hat{b}^\dagger}}
\def\cdag{{\hat{c}^\dagger}}
\def\ddag{{\hat{d}^\dagger}}
\def\hata{{\hat{a}}}
\def\hatb{{\hat{b}}}
\def\hatc{{\hat{c}}}
\def\hatd{{\hat{d}}}
\def\hatN{{\hat{N}}}
\def\hatH{{\hat{H}}}
\def\hatp{{\hat{p}}}
\def\Fup{{F^{\mu\nu}}}
\def\Fdown{{F_{\mu\nu}}}
\def\newl{\nonumber \\}
\def\SIkm{\,\mathrm{km}}
\def\SIyr{\,\mathrm{yr}}
\def\SIGyr{\,\mathrm{Gyr}}
\def\SIeV{\,\mathrm{eV}}
\def\SIkeV{\,\mathrm{keV}}
\def\SIMeV{\,\mathrm{MeV}}
\def\SIGeV{\,\mathrm{GeV}}
\def\SIcal{\,\mathrm{cal}}
\def\SIkcal{\,\mathrm{kcal}}
\def\SImol{\,\mathrm{mol}}
\def\SIm{\,\mathrm{m}}
\def\SIcm{\,\mathrm{cm}}
\def\SIfm{\,\mathrm{fm}}
\def\SImm{\,\mathrm{mm}}
\def\SInm{\,\mathrm{nm}}
\def\SImum{\,\mathrm{\mu m}}
\def\SIJ{\,\mathrm{J}}
\def\SIkJ{\,\mathrm{kJ}}
\def\SIs{\,\mathrm{s}}
\def\SIkg{\,\mathrm{kg}}
\def\SIg{\,\mathrm{g}}
\def\SIK{\,\mathrm{K}}
\def\SImmHg{\,\mathrm{mmHg}}
\def\SIPa{\,\mathrm{Pa}}
\def\vece{\mathrm{e}}
\def\bmat#1{\left(\begin{array}{#1}}
\def\emat{\end{array}\right)}
\def\bcase#1{\left\{\begin{array}{#1}}
\def\ecase{\end{array}\right.}
\def\calM{{\mathcal{M}}}
\def\calT{{\mathcal{T}}}
\def\calR{{\mathcal{R}}}
\def\barpsi{\bar{\psi}}
\def\baru{\bar{u}}
\def\barv{\bar{\upsilon}}
\def\bmini#1{\begin{minipage}{#1\textwidth}}
\def\emini{\end{minipage}}
\def\qeq{\stackrel{?}{=}}
\def\torder#1{\mathcal{T}\left(#1\right)}
\def\rorder#1{\mathcal{R}\left(#1\right)}
\def\contr#1#2{\contraction{}{#1}{}{#2}#1#2}
\def\trof#1{\mathrm{Tr}\left(#1\right)}
\def\trace{\mathrm{Tr}}
\def\comm#1{\ \ \ \left(\mathrm{used}\ #1\right)}
\def\tcomm#1{\ \ \ (\text{#1})}
\def\slp{\slashed{p}}
\def\slk{\slashed{k}}
\def\wulian{\includegraphics[width=0.18in]{emoji_wulian.jpg}}
\def\bye{\includegraphics[width=0.18in]{emoji_bye.jpg}}
\def\calp{{\mathfrak{p}}}
\def\veccalp{\mathbf{\mathfrak{p}}}
\def\atm{\,\mathrm{atm}}
\def\angstrom{\,\text{\AA}}
\def\Tthree{T_{\tiny \textcircled{3}}}
\def\pthree{p_{\tiny \textcircled{3}}}

\def\courseurl{http://zhiqihuang.top}

\def\tpage#1#2{
\begin{frame}
\bch
\begin{center}
\begin{large}
热学 \\
第#1讲 #2

\end{large}

\skiplines

黄志琦


\end{center}

\skiplines

{\small 
教材:《热学》第二版,赵凯华,罗蔚茵,高等教育出版社


课件下载
}
\courseurl 
\ech
\end{frame}
}

\def\bfr#1{
\begin{frame}
\chtitle{#1} 
\bch
}

\def\efr{
\ech 
\end{frame}
}

\title{Lesson 03 - Statistics of Equilibrium States}
  \author{}
  \date{}
\begin{document}
\tpage{3}{热平衡态的统计分布律}

\section{Reivew of Last Lecture}

\begin{frame}
\chtitle{上一讲课内容回顾}
\bch
上一讲的内容基本上都不考
\ech
\end{frame}

\begin{frame}
\chtitle{好吧,还是有些内容需要掌握的}
\bch
\bitem
\item{热量是能量的一种形式: $1 \SIcal \approx 4.2 \SIJ$。}
\item{物质升温(降温)吸收(放出)的热量为热容量。单位质量物质的热容量称为比热。水的比热较大,约为$1\SIcal/\SIg$.}
\item{熔化(凝固),汽化(液化)都是吸热(放热)且温度不变的过程。单位质量吸的热分别称为熔化热和汽化热。}
\item{晶体长程有序。非晶体和液体都是短程有序,长程无序(区别是非晶体的分子位置是固定的)。}
\eitem
\ech
\end{frame}


\begin{frame}
\chtitle{本讲内容预告}
\bch
\bitem
\item{微观粒子的态}
\item{热平衡的微观描述:细致平衡}
\item{H定理和熵}
\item{热平衡态下微观态的分布律}
\eitem
\ech
\end{frame}

\section{Microstate}




\begin{frame}
\chtitle{态的例子}
\bch
\bitem
\item{硬币有两种态:正面朝上,反面朝上

\addfig{0.7}{coins.jpg}
}
\item{骰子有六种态:1,2,3,4,5,6

\addfig{0.3}{touzi.png}
}
\item{同学们有很多种态:上课,吃饭,睡觉,和床引力战斗……

\addfig{1.5}{xbxstates.jpg}
}

\eitem
\ech
\end{frame}

\begin{frame}
\chtitle{有的人反对了: 骰子的六种态不一定是1,2,3,4,5,6
}
\bch

\addfig{1.5}{touzi_work.jpg}

\skipline

\wulian好吧,这不是我们今天要讲的重点…
\ech
\end{frame}

\begin{frame}
\chtitle{相空间}
\bch
分子的“态”由位置$(x, y, z)$和速度$(\upsilon_x, \upsilon_y, \upsilon_z)$描述:
$$(x, y, z, \upsilon_x, \upsilon_y, \upsilon_z)$$
或者更专业的说法是由位置和动量描述:
$$(x, y, z, \calp_x, \calp_y, \calp_z)$$

由$x, y, z, \calp_x, \calp_y, \calp_z$这六个变量(坐标轴)张成的空间称为{\blue \bf 相空间}。

\bitem
\item{经典图像(错误):分子的态由相空间中的一个点来描述。}
\item{量子图像(正确):{\bf \blue 分子的态由相空间中的一个“小块”来描述。}}
\eitem

\ech
\end{frame}


\begin{frame}
\chtitle{相空间里的“小块”是什么鬼?}
\bch
以一维空间为例,相空间是$(x, \calp)$构成的二维空间,相空间里的“小块”如下图所示:
\bmini{0.35}
\addfig{0.8}{phasespaceblock.png}
\emini
\bmini{0.6}
\bitem
\item{普朗克常数$h=6.63\times 10^{-34}\SIJ\cdot\SIs$}
\item{只要面积为$h$,即是合法的“小块”。如何取$\Delta x$或$\Delta \calp$视测量需要而定。}
\item{量子力学的测不准原理保证了“小块”内的点是无法通过测量区分的。}
\eitem
\emini

\skipline

在三维空间的情况,相空间为六维空间,“小块”体积为$h^3$ (六条边长满足$\Delta \calp_x\Delta x =\Delta \calp_y\Delta y =\Delta \calp_z\Delta z = h$)
\ech
\end{frame}


\begin{frame}
\chtitle{态是离散的相空间小块}
\bch
\bitem
\item{\bf $n$维空间$(x_1, x_2, \ldots, x_n)$对应的相空间为$2n$维相空间$(x_1, x_2, \ldots, x_n, \calp_1, \calp_2, \ldots, \calp_n)$。}
\item{\bf 相空间内“小块”是一个假想的$2n$维体。在每个轴方向上的边长依次为$(\Delta x_1, \Delta x_2, \ldots, \Delta x_n, \Delta \calp_1, \Delta \calp_2, \ldots, \Delta \calp_n)$。边长两两成对满足$\Delta x_i \cdot \Delta \calp_i = h$ ($i=1, 2, \ldots, n$)。显然,“小块”的体积为$h^n$。}
\item{\bf 在每个维度上如何取$\Delta x_i$则视测量需求而定,有时甚至可以取$\Delta x_i$为宏观尺度。}
\item{\bf 相空间的每个“小块”对应一个微观粒子的态。虽然数学上“小块”仍然是无限可分的,在物理上却由于测不准原理而无法再划分更细的状态。}
\eitem
\ech
\end{frame}

\begin{frame}
\chtitle{思考题}
\bch
一个边长为$0.663\SIm$的正方体容器里,质量为$10^{-26}\SIkg$,速度不超过$10^3\SIm\SIs^{-1}$的粒子共有多少种可能的态?
\ech
\end{frame}

\begin{frame}
\chtitle{思考题}
\bch
一个边长为$0.663\SIm$的正方体容器里,质量为$10^{-26}\SIkg$,速度不超过$0.8$倍光速的粒子共有多少种可能的态?
\ech
\end{frame}


\begin{frame}
\chtitle{测量空气分子的动量}
\bch
考虑装在一个边长$L= 1\SIm$的正方体容器里的空气。取空气分子的平均质量,$\bar{m} = \frac{29\SIg\cdot\SImol^{-1}}{6.02\times 10^{23}\SImol^{-1}}\approx 5 \times 10^{-26}\SIkg$

取室温$T=300K$,则空气分子动量大小的数量级为
$$\calp \sim \sqrt{m^2\overline{\upsilon^2}} \sim \sqrt{3mkT} \approx 2\times 10^{-23} \SIkg \SIm\SIs^{-1} $$

我们准备测量空气分子的速度的分布规律,希望动量空间的分辨率尽可能地高,所以在位置空间尽可能地降低分辨率:取$\Delta x=\Delta y=\Delta z = L$(也就是说我们只要求分子在容器内,而不去测量它的具体位置)。那么动量空间的分辨率(即小块沿动量轴方向的边长)为
$$\Delta \calp_x = \Delta \calp_y = \Delta \calp_z= \frac{h}{L} = 6.63\times 10^{-34} \SIkg\SIm\SIs^{-1} \sim 10^{-11} \calp$$

\ech
\end{frame}


\begin{frame}
\chtitle{测量空气分子的动量(续)}
\bch
现在我们更贪心,希望把分子的位置确定到分子平均距离的数量级($d\sim 3\times 10^{-9}\SIm$),这时动量分辨率约为
$$\Delta \calp \sim \frac{h}{d} \sim 10^{-25} \SIkg\SIm\SIs^{-1} \sim 10^{-2} \calp$$

也就是说,课本上第二章讨论玻尔兹曼分布时的经典图像(同时知道分子的位置和动量),大致上可以实现。

\ech
\end{frame}


\begin{frame}
\chtitle{麦克斯韦分布(Maxwell distribution)}
\bch
现在我们考虑非相对论气体分子的速度分布规律。

\skipline

{\blue \bf 分子的“态” = 分子在相空间的哪个小块里 }

\skipline

我们仍取$\Delta x$, $\Delta y$, $\Delta z$为整个容器的长,宽,高。那么在位置空间就不需要进行划分了。分子的态由动量空间的坐标$(\calp_x, \calp_y, \calp_z)$唯一决定。当然,$\calp_x$, $\calp_y$, $\calp_z$的取值并不是连续的,它们分别是$\Delta \calp_x=\frac{h}{\Delta x}$, $\Delta \calp_y=\frac{h}{\Delta y}$, $\Delta \calp_z=\frac{h}{\Delta z}$的整数倍。

\skipline
根据我们第一讲给的“万能法则”,达到热平衡时,分子处在一个能量为$\varepsilon$的态的概率正比于$e^{-\frac{\varepsilon}{kT}}$。
所以,质量为$m$的分子处在任何一个态$(\calp_x, \calp_y, \calp_z)$的概率正比于
$$e^{-\frac{m\upsilon^2}{2kT}}$$
其中
$\upsilon = \frac{\sqrt{\calp_x^2 +\calp_y^2+ \calp_z^2}}{m}$
是分子速率。
\ech
\end{frame}

\begin{frame}
\chtitle{麦克斯韦分布(续)}
\bch
分子速率在$\upsilon$和$\upsilon+d\upsilon$之间的态(“小块”)有多少个呢?很简单,用相空间的体积除以小块的体积$h^3$即可。
$$N(\upsilon)d\upsilon = \frac{(\Delta x \Delta y \Delta z )4\pi (m\upsilon)^2d(m\upsilon)}{h^3}\propto \upsilon^2d\upsilon$$
\ech
\end{frame}

\begin{frame}
\chtitle{}
\bch
\ech
\end{frame}

\end{document}

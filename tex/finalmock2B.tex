\documentclass[12pt,CJK]{article}
\usepackage{geometry}
\input{reduced_macros.tex}
\geometry{tmargin=0.3in, bmargin=0.5in, lmargin=0.7in, rmargin=0.7in, nohead, nofoot}
\def\mark#1{{\color{blue} (#1分)}}
\renewcommand{\thepage}{}
\begin{document}
\bch
{\large 热学课堂练习 II 初入江湖版 (总分$\times 0.02$换算成平时分);}

%{\vskip 0.3in}

姓名 ....................... {\hskip 0.5in}    学号 .......................{\hskip 0.5in}  分数 ...................

%{\vskip 0.1in}

\bitem
\item[(一)]{判断题,请用$\checkmark$表示正确,$\times$表示错误,每题5分。

  \bitem
\item[(1)]{水的三相点的热力学温度为$273.16\SIK$。\bropt}
\item[(2)]{在两个恒温热源之间工作的可逆卡诺热机的效率跟工作物质无关。\bropt}
\item[(3)]{实际气体绝热自由膨胀后温度一般会降低。 \bropt}  
\item[(4)]{$p$-$V$图上顺时针的闭合曲线代表的是可逆的正循环。\bropt}
\item[(5)]{物体吸收热量后温度一定升高。\bropt}
  \eitem
}

\item[(二)]{选择题,每题5分。

  \bitem
  \item[(1)]{
    一碗米饭的热量最接近下列哪一个数量级? \bropt

  \optlist{$1\SIJ$}{$10^3\SIJ$}{$10^6\SIJ$}

}

\item[(2)]{分子质量为$m$,温度为$T$的热平衡理想气体,记分子速度的$x$分量的绝对值为$u = |\upsilon_x|$。则$u$的分布函数为 \bropt
  
  \optlist{$\sqrt{\frac{m}{2\pi  kT}} e^{-\frac{mu^2}{2kT}}$}{ $\sqrt{\frac{2m}{\pi kT }} e^{-\frac{mu^2}{2kT}}$ }{ $\left(\frac{m}{2\pi kT}\right)^{3/2} e^{-\frac{mu^2}{2kT}}$ }

}

\item[(3)]{理想气体的压强的微观机制是: \bropt

  \optlist{由分子运动造成的动理压强}{分子间吸引力造成的内压强}{既有动理压强,也有内压强}
}

\item[(4)]{实际气体经过节流过程,温度 \bropt

  \optlist{一定会降低}{一定会升高}{因已知条件不足,无法确定怎么变化}
  
}

\item[(5)]{
  把标准状态下的$1\SImol$氮气和$1\SImol$氧气保持压强和温度不变地混合,熵变化多少? \bropt

  \optlist{$R\ln 2$}{$2R\ln 2$}{$0$}
}


  \eitem
  }
\item[(三)]{$1\SImol$的温度为$300\SIK$的理想气体在准静态等体过程中吸收了$249.4\SIJ$的热量,温度变为$320\SIK$,然后经过准静态绝热膨胀温度又降回到$300\SIK$。设该气体的定体摩尔热容为常量。求过程中 (1) 气体对外做的功;(10分) (2) 末态体积和初始体积之比。(10分) 

    \vspace{3.8in}


}

  \item[(四)]{某干燥地区的地面大气的压强为$1\atm$,温度为$300\SIK$。在该地区的一处山顶,水的沸点为$90\cdeg$。已知水的汽化热为$4.1\times 10^4\SIJ/\SImol$。试估算山顶处的:(1) 大气压强; (10分) (2) 大气温度。(5分)
    
       \vspace{3.2in}
  }
    
  


\item[(五)]{某可逆理想气体热机的循环过程如下:
      \bitem
    \item{从温度为$500\SIK$开始,准静态等压膨胀至温度为$600\SIK$}
    \item{准静态绝热膨胀,温度降到$200\SIK$。}
    \item{以$200\SIK$的温度准静态等温压缩。}
    \item{准静态绝热压缩回到初始状态。}            
      \eitem
      选择你喜欢的变量为坐标画出循环的大致示意图(5分)。假设该理想气体的定体摩尔热容为常量,求该热机的效率(10分)。
    }
  

\eitem



\ech
\end{document}

\documentclass[CJK]{beamer}
\usepackage{CJKutf8}
\usepackage{beamerthemesplit}
\usetheme{Malmoe}
\useoutertheme[footline=authortitle]{miniframes}
\usepackage{amsmath}
\usepackage{amssymb}
\usepackage{graphicx}
\usepackage{eufrak}
\usepackage{color}
\usepackage{slashed}
\usepackage{simplewick}
\usepackage{tikz}
\graphicspath{{../figures/}}
\def\addfig#1#2{\begin{center}\includegraphics[width=#1 in]{#2}\end{center}}
\def\blacktext#1{{\color{black}#1}}
\def\bluetext#1{{\color{blue}#1}}
\def\redtext#1{{\color{red}#1}}
\def\darkbluetext#1{{\color[rgb]{0,0.2,0.6}#1}}
\def\skybluetext#1{{\color[rgb]{0.2,0.7,1.}#1}}
\def\cyantext#1{{\color[rgb]{0.,0.5,0.5}#1}}
\def\greentext#1{{\color[rgb]{0,0.7,0.1}#1}}
\def\darkgray{\color[rgb]{0.2,0.2,0.2}}
\def\lightgray{\color[rgb]{0.6,0.6,0.6}}
\def\gray{\color[rgb]{0.4,0.4,0.4}}
\def\blue{\color{blue}}
\def\red{\color{red}}
\def\green{\color{green}}
\def\darkblue{\color[rgb]{0,0.2,0.6}}
\def\skyblue{\color[rgb]{0.2,0.7,1.}}
\def\fdeg{{^\circ \mathrm{F}}}
\def\cdeg{^\circ \mathrm{C}}
\def\be{\begin{equation}}
\def\ee{\nonumber\end{equation}}
\def\bea{\begin{eqnarray}}
\def\eea{\nonumber\end{eqnarray}}
\def\ii{{\dot{\imath}}}
\def\bch{\begin{CJK}{UTF8}{gbsn}}
\def\ech{\end{CJK}}
\def\bitem{\begin{itemize}}
\def\eitem{\end{itemize}}
\def\bcenter{\begin{center}}
\def\ecenter{\end{center}}
\def\bex{\begin{minipage}{0.3\textwidth}\includegraphics[width=1in]{jugelizi.png}\end{minipage}\begin{minipage}{0.6\textwidth}}
\def\eex{\end{minipage}}
\def\chtitle#1{\frametitle{\bch#1\ech}}
\def\skipline{{\vskip0.1in}}
\def\skiplines{{\vskip0.2in}}
\def\lagr{{\mathcal{L}}}
\def\hamil{{\mathcal{H}}}
\def\vecv{{\mathbf{v}}}
\def\vecx{{\mathbf{x}}}
\def\vecy{{\mathbf{y}}}
\def\veck{{\mathbf{k}}}
\def\vecp{{\mathbf{p}}}
\def\vecn{{\mathbf{n}}}
\def\vecA{{\mathbf{A}}}
\def\vecP{{\mathbf{P}}}
\def\vecsigma{{\mathbf{\sigma}}}
\def\hatJn{{\hat{J_\vecn}}}
\def\hatJx{{\hat{J_x}}}
\def\hatJy{{\hat{J_y}}}
\def\hatJz{{\hat{J_z}}}
\def\hatj#1{\hat{J_{#1}}}
\def\hatphi{{\hat{\phi}}}
\def\hatq{{\hat{q}}}
\def\hatpi{{\hat{\pi}}}
\def\vel{\upsilon}
\def\Dint{{\mathcal{D}}}
\def\adag{{\hat{a}^\dagger}}
\def\bdag{{\hat{b}^\dagger}}
\def\cdag{{\hat{c}^\dagger}}
\def\ddag{{\hat{d}^\dagger}}
\def\hata{{\hat{a}}}
\def\hatb{{\hat{b}}}
\def\hatc{{\hat{c}}}
\def\hatd{{\hat{d}}}
\def\hatN{{\hat{N}}}
\def\hatH{{\hat{H}}}
\def\hatp{{\hat{p}}}
\def\Fup{{F^{\mu\nu}}}
\def\Fdown{{F_{\mu\nu}}}
\def\newl{\nonumber \\}
\def\SIkm{\,\mathrm{km}}
\def\SIyr{\,\mathrm{yr}}
\def\SIGyr{\,\mathrm{Gyr}}
\def\SIeV{\,\mathrm{eV}}
\def\SIkeV{\,\mathrm{keV}}
\def\SIMeV{\,\mathrm{MeV}}
\def\SIGeV{\,\mathrm{GeV}}
\def\SIcal{\,\mathrm{cal}}
\def\SIkcal{\,\mathrm{kcal}}
\def\SImol{\,\mathrm{mol}}
\def\SIm{\,\mathrm{m}}
\def\SIcm{\,\mathrm{cm}}
\def\SIfm{\,\mathrm{fm}}
\def\SImm{\,\mathrm{mm}}
\def\SInm{\,\mathrm{nm}}
\def\SImum{\,\mathrm{\mu m}}
\def\SIJ{\,\mathrm{J}}
\def\SIkJ{\,\mathrm{kJ}}
\def\SIs{\,\mathrm{s}}
\def\SIkg{\,\mathrm{kg}}
\def\SIg{\,\mathrm{g}}
\def\SIK{\,\mathrm{K}}
\def\SImmHg{\,\mathrm{mmHg}}
\def\SIPa{\,\mathrm{Pa}}
\def\vece{\mathrm{e}}
\def\bmat#1{\left(\begin{array}{#1}}
\def\emat{\end{array}\right)}
\def\bcase#1{\left\{\begin{array}{#1}}
\def\ecase{\end{array}\right.}
\def\calM{{\mathcal{M}}}
\def\calT{{\mathcal{T}}}
\def\calR{{\mathcal{R}}}
\def\barpsi{\bar{\psi}}
\def\baru{\bar{u}}
\def\barv{\bar{\upsilon}}
\def\bmini#1{\begin{minipage}{#1\textwidth}}
\def\emini{\end{minipage}}
\def\qeq{\stackrel{?}{=}}
\def\torder#1{\mathcal{T}\left(#1\right)}
\def\rorder#1{\mathcal{R}\left(#1\right)}
\def\contr#1#2{\contraction{}{#1}{}{#2}#1#2}
\def\trof#1{\mathrm{Tr}\left(#1\right)}
\def\trace{\mathrm{Tr}}
\def\comm#1{\ \ \ \left(\mathrm{used}\ #1\right)}
\def\tcomm#1{\ \ \ (\text{#1})}
\def\slp{\slashed{p}}
\def\slk{\slashed{k}}
\def\wulian{\includegraphics[width=0.18in]{emoji_wulian.jpg}}
\def\bye{\includegraphics[width=0.18in]{emoji_bye.jpg}}
\def\calp{{\mathfrak{p}}}
\def\veccalp{\mathbf{\mathfrak{p}}}
\def\atm{\,\mathrm{atm}}
\def\angstrom{\,\text{\AA}}
\def\Tthree{T_{\tiny \textcircled{3}}}
\def\pthree{p_{\tiny \textcircled{3}}}

\title{Lesson 00 Introduction}
  \author{}
  \date{}


\begin{document}

\begin{frame}
\begin{center}
\begin{Large}
\bch
热学 \\
第0讲 课程简介

{\vskip 0.3in}

黄志琦

\ech
\end{Large}
\end{center}

\vskip 0.2in

\bch
教材:《热学》第二版,赵凯华,罗蔚茵,高等教育出版社
\ech

\bch
课件下载
\ech
https://github.com/zqhuang/SYSU\_TD
\end{frame}




\begin{frame}
\chtitle{评分方案}
\bch

\begin{itemize}
\item{作业20\% (三个班不同)}
\item{期中考试20\% (三个班相同)}
\item{期末考试60\% (三个班相同)}
\end{itemize}

\skipline

期中考试时间:4月13日(第8周);覆盖内容:第一章和第二章

期末考试时间:待定(学校统一安排);覆盖内容:全部
\ech
\end{frame}


\begin{frame}
\chtitle{作业计分方式(仅限1班)}
\bch
\begin{itemize}
\item{{\bf 如发现抄袭作业,全部作业分记为零分。}}
\item{每周留若干作业题。第3周课间交第1,2周留的作业;第5周交3,4周留的作业,以此类推直至第17周,总计八次作业。}
\item{如不能按时完成请和助教协商补交,但不得迟于发作业或者公布答案时间。}
\item{每道作业题评分范围为0-5星。}
\item{举手抢答课堂问题也可以计入评分(0-5星)。}
\item{期末统计总星数$N_\star$,换算为作业分
$$ \min\left( \frac{1}{2}{\rm ceil}\left(\frac{N_\star}{5}\right), 20\right) \, . $$}
\end{itemize}

\ech
\end{frame}

\begin{frame}
\chtitle{作业计分方式(仅限1班)}
\bch
上述作业计分公式
$$ \min\left( \frac{1}{2}{\rm ceil}\left(\frac{N_\star}{5}\right), 20\right) \, , $$
其中$\mathrm{ceil}(x)$指不小于$x$的最小整数, $\min(x, y)$指$x, y$中的较小者。

\skiplines

请举手抢答:要不低于20分至少要答多少道题?

\ech
\end{frame}

\begin{frame}
\chtitle{知识细则符号约定}
\bch

\begin{itemize}
\item{黑色:要求定量地掌握,理解概念,会定量计算}
\item{\bluetext{蓝色:要求定性地掌握,理解概念,会数量级估计}}
\item{\cyantext{青色:因没有学习相应的基础知识(如量子力学),只要求大致了解课本中的唯象描述或结论(这类知识点会在括号里更细致地说明需了解的内容)}}
\item{\greentext{绿色:课外阅读内容,旨在拓宽视野,不要求掌握}}
\end{itemize}

\ech
\end{frame}


\begin{frame}
\chtitle{\blacktext{第一章 热学基本概念和物质聚集态}}
\bch
\S{1}  温度
\begin{itemize}
\item[1.1]{\bluetext{温度计和温标}}
\item[1.2]{\bluetext{热力学第零定律}}
\item[\blacktext{1.3}]{理想气体状态方程和理想气体温标}
\item[\greentext{1.4}]{\greentext{温度大观}}
\end{itemize}
\bluetext{\S{2} 热量及其本质}
\begin{itemize}
\item[\blacktext{2.1}]{量热学 \greentext{热质说与热动说}}
\item[2.2]{\bluetext{原子论}}
\item[2.3]{\bluetext{分子力与分子运动}}
\end{itemize}

\bluetext{\S{3}  物质聚集态随状态参量的转化与共存}
\begin{itemize}
\item[3.1]{\bluetext{闭合系的$p$-$V$-$T$曲面}}
\item[3.2]{\bluetext{等温线 多相共存}}
\item[3.3]{\bluetext{$p$-$T$三相图}}
\end{itemize}

\ech
\end{frame}

\begin{frame}
\chtitle{\blacktext{第一章 热学基本概念和物质聚集态}}
\bch

\S{4} 气体
\begin{itemize}
\item[4.1]{\bluetext{气体的微观模型和温度的微观意义}}
\item[\blacktext{4.2}]{理想气体压强公式}
\item[\blacktext{4.3}]{理想气体定律的推导}
\item[4.4]{\bluetext{实际气体}}
\end{itemize}

\bluetext{\S{5} 固体}
\begin{itemize}
\item[5.1]{\bluetext{晶体结构}}
\item[\cyantext{5.2}]{\cyantext{非晶态(短程有序长程无序)}\greentext{与准晶态}}
\item[\greentext{5.3}]{\greentext{固体中分子的热运动}}
\end{itemize}
\ech
\end{frame}


\begin{frame}
\chtitle{\blacktext{第一章 热学基本概念和物质聚集态}}
\bch
\bluetext{\S{6} 化学键}
\begin{itemize}
\item[6.1]{\bluetext{离子健}}
\item[\cyantext{6.2}]{\cyantext{共价键(两个原子共享价电子)}}
\item[\cyantext{6.3}]{\cyantext{金属键(外层电子被所有原子共有化并可在晶体内自由移动)}}
\item[\greentext{6.4}]{\greentext{范德瓦尔斯键}}
\item[\greentext{6.5}]{\greentext{氢键}}
\end{itemize}
\bluetext{\S{7} 液体}
\begin{itemize}
\item[\cyantext{7.1}]{\cyantext{液体——稠密的实际气体(分子力和分子运动势均力敌)}}
\item[\cyantext{7.2}]{\cyantext{液体——濒临瓦解的晶格(短程有序长程无序)}}
\item[7.3]{\bluetext{表面张力的由来}}
\end{itemize}
\ech
\end{frame}

\begin{frame}
\chtitle{\blacktext{第二章 热平衡态的统计分布律}}
\bch
\S{1} 麦克斯韦速度分布律
\begin{itemize}
\item[\blacktext{1.1}]{麦克斯韦速度分布律}
\item[\blacktext{1.2}]{速度空间与速度分布函数}
\item[1.3]{\bluetext{麦克斯韦分布律的导出}}
\item[\blacktext{1.4}]{方均根速率}
\item[\blacktext{1.5}]{平均速率}
\item[\blacktext{1.6}]{泄流速率}
\end{itemize}

\S{2} 玻尔兹曼密度分布
\begin{itemize}
\item[\blacktext{2.1}]{等温气压公式}
\item[\blacktext{2.2}]{玻尔兹曼密度分布律}
\item[\blacktext{2.3}]{麦克斯韦-玻尔兹曼密度分布律}
\end{itemize}
\ech
\end{frame}

\begin{frame}
\chtitle{\blacktext{第二章 热平衡态的统计分布律}}
\bch
\S{3} 能均分定理与热容量
\begin{itemize}
\item[\blacktext{3.1}]{自由度}
\item[\blacktext{3.2}]{能量按自由度均分定理}
\item[\blacktext{3.3}]{理想气体的热容量}
\item[3.4]{\bluetext{固体的热容量}}
\end{itemize}

\cyantext{\S{4} 量子气体中粒子按能级的分布}

\begin{itemize}
\item[\cyantext{4.1}]{\cyantext{能级与量子态(能量只能取一系列离散值)}}
\item[\cyantext{4.2}]{\cyantext{麦克斯韦-玻尔兹曼分布(细致平衡条件2.49式和MB分布2.52式。推导过程不要求掌握)}}
\item[\cyantext{4.3}]{\cyantext{H定理(只讲经典粒子版本,H函数非平衡态时单调下降,达到细致平衡不再变化;推导过程不要求掌握)}}
\item[\cyantext{4.4}]{\cyantext{能级的离散性对热容量的影响(能级离散$\Rightarrow$低温下某些自由度不被激发)}}
\item[\greentext{4.5}]{\greentext{玻色-爱因斯坦分布和费米-狄拉克分布}}
\end{itemize}

\ech
\end{frame}

\begin{frame}
\chtitle{\blacktext{第二章 热平衡态的统计分布律}}
\bch
\greentext{\S{5} 费米气体}

\greentext{\S{6} 玻色气体}

\cyantext{\S{7} 宏观的概率和熵}
\begin{itemize}
\item[\cyantext{7.1}]{\cyantext{宏观态的概率(热平衡态是包含微观态数目最多,即出现概率最大的宏观态)}}
\item[\cyantext{7.2}]{\cyantext{玻尔兹曼熵关系式(了解$S=\ln \Omega=-kH$,不要求掌握推导)}}
\item[\greentext{7.3}]{\greentext{信息熵与遗传密码}}
\end{itemize}
\ech
\end{frame}


\begin{frame}
\chtitle{\blacktext{第三章 热力学第一定律}}
\bch
\S{1} 从能量守恒到热力学第一定律
\begin{itemize}
\item[\greentext{1.1}]{\greentext{能量守恒定律的建立}}
\item[1.2]{\bluetext{广义功}}
\item[\blacktext{1.3}]{内能是个态函数}
\item[\blacktext{1.4}]{热力学第一定律的数学描述}
\item[\blacktext{1.5}]{准静态过程}
\end{itemize}

\S{2} 气体的热容量 内能和焓
\begin{itemize}
\item[\blacktext{2.1}]{热容量 焓}
\item[2.2]{\bluetext{焦耳实验及其改进}}
\item[2.3]{\bluetext{焦耳-汤姆孙效应}}
\item[2.4]{\bluetext{节流膨胀液化气体}}
\item[\greentext{2.5}]{\greentext{化学反应热和生成焓}}
\end{itemize}

\ech
\end{frame}


\begin{frame}
\chtitle{\blacktext{第三章 热力学第一定律}}
\bch
\S{3} 热力学第一定律对理想气体的应用
\begin{itemize}
\item[\blacktext{3.1}]{等温过程}
\item[\blacktext{3.2}]{绝热过程}
\item[\blacktext{3.3}]{大气的垂直温度梯度}
\item[\blacktext{3.4}]{多方过程}
\end{itemize}

\S{4} 循环过程和卡诺循环
\begin{itemize}
\item[4.1]{\bluetext{循环过程}}
\item[\blacktext{4.2}]{理想气体卡诺循环及其效率}
\end{itemize}

\ech
\end{frame}

\begin{frame}
\chtitle{\blacktext{第四章 热力学第二定律}}
\bch
\S{1} 热力学第二定律的表述和卡诺定理
\begin{itemize}
\item[1.1]{\bluetext{自然现象的不可逆性}}
\item[1.2]{\bluetext{热力学第二定律的语言表述}}
\item[1.3]{\bluetext{卡诺定理}}
\item[\blacktext{1.4}]{热力学温标}
\item[\greentext{1.5}]{\greentext{历史性回顾}}
\end{itemize}

\S{2} 卡诺定理的应用
\begin{itemize}
\item[\blacktext{2.1}]{内能和状态方程的关系}
\item[\blacktext{2.2}]{克拉珀龙方程及其在相变问题上的应用}
\end{itemize}
\ech
\end{frame}

\begin{frame}
\chtitle{\blacktext{第四章 热力学第二定律}}
\bch
\S{3} 克劳修斯不等式与熵定理
\begin{itemize}
\item[\blacktext{3.1}]{热力学第二定律的数学表述——克劳修斯不等式}
\item[\blacktext{3.2}]{熵是态函数}
\item[\blacktext{3.3}]{熵的计算}
\item[3.4]{\bluetext{熵增加原理}}
\item[\cyantext{3.5}]{\cyantext{热力学熵与玻尔兹曼熵的统一(克劳修斯熵和玻尔兹曼熵都正比于宏观状态概率的对数)}}
\end{itemize}

\greentext{\S{4} 关于热力学第二定律的若干诘难和佯谬}


\ech
\end{frame}

\begin{frame}
\chtitle{\blacktext{第四章 热力学第二定律}}
\bch

\bluetext{\S{5} 热平衡与自由能}
\begin{itemize}
\item[5.1]{\bluetext{孤立系的热平衡判据}}
\item[5.2]{\bluetext{定温定体条件下的热平衡判据 亥姆霍兹自由能}}
\item[5.3]{\bluetext{定温定压条件下的热平衡判据 吉布斯自由能}}
\item[\greentext{5.4}]{\greentext{物体系内各部分之间的平衡条件}}
\item[\greentext{5.5}]{\greentext{范德瓦尔斯气液相平衡}}
\item[\greentext{5.6}]{\greentext{混合气体的化学平衡}}
\end{itemize}

\greentext{\S{6} 连续相变 超流}

\ech
\end{frame}


\begin{frame}
\chtitle{ \greentext{第五章 非平衡过程} }
\bch

\greentext{\S{1} 近平衡态驰豫和输运过程}

\greentext{\S{2} 涨落 关联 布朗运动}

\greentext{\S{3} 分形}
 
\greentext{\S{4} 线性不可逆过程热力学}

\greentext{\S{5} 耗散结构}

\greentext{\S{6} 生命与生态环境}

\greentext{\S{7} 热宇宙模型}

\ech
\end{frame}


\end{document}







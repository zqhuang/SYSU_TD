\documentclass[CJK]{beamer}
\usepackage{CJKutf8}
\usepackage{beamerthemesplit}
\usetheme{Malmoe}
\useoutertheme[footline=authortitle]{miniframes}
\usepackage{amsmath}
\usepackage{amssymb}
\usepackage{graphicx}
\usepackage{eufrak}
\usepackage{color}
\usepackage{slashed}
\usepackage{simplewick}
\usepackage{tikz}
\graphicspath{{../figures/}}
\def\addfig#1#2{\begin{center}\includegraphics[width=#1 in]{#2}\end{center}}
\def\blacktext#1{{\color{black}#1}}
\def\bluetext#1{{\color{blue}#1}}
\def\redtext#1{{\color{red}#1}}
\def\darkbluetext#1{{\color[rgb]{0,0.2,0.6}#1}}
\def\skybluetext#1{{\color[rgb]{0.2,0.7,1.}#1}}
\def\cyantext#1{{\color[rgb]{0.,0.5,0.5}#1}}
\def\greentext#1{{\color[rgb]{0,0.7,0.1}#1}}
\def\darkgray{\color[rgb]{0.2,0.2,0.2}}
\def\lightgray{\color[rgb]{0.6,0.6,0.6}}
\def\gray{\color[rgb]{0.4,0.4,0.4}}
\def\blue{\color{blue}}
\def\red{\color{red}}
\def\green{\color{green}}
\def\darkblue{\color[rgb]{0,0.2,0.6}}
\def\skyblue{\color[rgb]{0.2,0.7,1.}}
\def\fdeg{{^\circ \mathrm{F}}}
\def\cdeg{^\circ \mathrm{C}}
\def\be{\begin{equation}}
\def\ee{\nonumber\end{equation}}
\def\bea{\begin{eqnarray}}
\def\eea{\nonumber\end{eqnarray}}
\def\ii{{\dot{\imath}}}
\def\bch{\begin{CJK}{UTF8}{gbsn}}
\def\ech{\end{CJK}}
\def\bitem{\begin{itemize}}
\def\eitem{\end{itemize}}
\def\bcenter{\begin{center}}
\def\ecenter{\end{center}}
\def\bex{\begin{minipage}{0.3\textwidth}\includegraphics[width=1in]{jugelizi.png}\end{minipage}\begin{minipage}{0.6\textwidth}}
\def\eex{\end{minipage}}
\def\chtitle#1{\frametitle{\bch#1\ech}}
\def\skipline{{\vskip0.1in}}
\def\skiplines{{\vskip0.2in}}
\def\lagr{{\mathcal{L}}}
\def\hamil{{\mathcal{H}}}
\def\vecv{{\mathbf{v}}}
\def\vecx{{\mathbf{x}}}
\def\vecy{{\mathbf{y}}}
\def\veck{{\mathbf{k}}}
\def\vecp{{\mathbf{p}}}
\def\vecn{{\mathbf{n}}}
\def\vecA{{\mathbf{A}}}
\def\vecP{{\mathbf{P}}}
\def\vecsigma{{\mathbf{\sigma}}}
\def\hatJn{{\hat{J_\vecn}}}
\def\hatJx{{\hat{J_x}}}
\def\hatJy{{\hat{J_y}}}
\def\hatJz{{\hat{J_z}}}
\def\hatj#1{\hat{J_{#1}}}
\def\hatphi{{\hat{\phi}}}
\def\hatq{{\hat{q}}}
\def\hatpi{{\hat{\pi}}}
\def\vel{\upsilon}
\def\Dint{{\mathcal{D}}}
\def\adag{{\hat{a}^\dagger}}
\def\bdag{{\hat{b}^\dagger}}
\def\cdag{{\hat{c}^\dagger}}
\def\ddag{{\hat{d}^\dagger}}
\def\hata{{\hat{a}}}
\def\hatb{{\hat{b}}}
\def\hatc{{\hat{c}}}
\def\hatd{{\hat{d}}}
\def\hatN{{\hat{N}}}
\def\hatH{{\hat{H}}}
\def\hatp{{\hat{p}}}
\def\Fup{{F^{\mu\nu}}}
\def\Fdown{{F_{\mu\nu}}}
\def\newl{\nonumber \\}
\def\SIkm{\,\mathrm{km}}
\def\SIyr{\,\mathrm{yr}}
\def\SIGyr{\,\mathrm{Gyr}}
\def\SIeV{\,\mathrm{eV}}
\def\SIkeV{\,\mathrm{keV}}
\def\SIMeV{\,\mathrm{MeV}}
\def\SIGeV{\,\mathrm{GeV}}
\def\SIcal{\,\mathrm{cal}}
\def\SIkcal{\,\mathrm{kcal}}
\def\SImol{\,\mathrm{mol}}
\def\SIm{\,\mathrm{m}}
\def\SIcm{\,\mathrm{cm}}
\def\SIfm{\,\mathrm{fm}}
\def\SImm{\,\mathrm{mm}}
\def\SInm{\,\mathrm{nm}}
\def\SImum{\,\mathrm{\mu m}}
\def\SIJ{\,\mathrm{J}}
\def\SIkJ{\,\mathrm{kJ}}
\def\SIs{\,\mathrm{s}}
\def\SIkg{\,\mathrm{kg}}
\def\SIg{\,\mathrm{g}}
\def\SIK{\,\mathrm{K}}
\def\SImmHg{\,\mathrm{mmHg}}
\def\SIPa{\,\mathrm{Pa}}
\def\vece{\mathrm{e}}
\def\bmat#1{\left(\begin{array}{#1}}
\def\emat{\end{array}\right)}
\def\bcase#1{\left\{\begin{array}{#1}}
\def\ecase{\end{array}\right.}
\def\calM{{\mathcal{M}}}
\def\calT{{\mathcal{T}}}
\def\calR{{\mathcal{R}}}
\def\barpsi{\bar{\psi}}
\def\baru{\bar{u}}
\def\barv{\bar{\upsilon}}
\def\bmini#1{\begin{minipage}{#1\textwidth}}
\def\emini{\end{minipage}}
\def\qeq{\stackrel{?}{=}}
\def\torder#1{\mathcal{T}\left(#1\right)}
\def\rorder#1{\mathcal{R}\left(#1\right)}
\def\contr#1#2{\contraction{}{#1}{}{#2}#1#2}
\def\trof#1{\mathrm{Tr}\left(#1\right)}
\def\trace{\mathrm{Tr}}
\def\comm#1{\ \ \ \left(\mathrm{used}\ #1\right)}
\def\tcomm#1{\ \ \ (\text{#1})}
\def\slp{\slashed{p}}
\def\slk{\slashed{k}}
\def\wulian{\includegraphics[width=0.18in]{emoji_wulian.jpg}}
\def\bye{\includegraphics[width=0.18in]{emoji_bye.jpg}}
\def\calp{{\mathfrak{p}}}
\def\veccalp{\mathbf{\mathfrak{p}}}
\def\atm{\,\mathrm{atm}}
\def\angstrom{\,\text{\AA}}
\def\Tthree{T_{\tiny \textcircled{3}}}
\def\pthree{p_{\tiny \textcircled{3}}}

\def\courseurl{http://zhiqihuang.top}

\def\tpage#1#2{
\begin{frame}
\bch
\begin{center}
\begin{large}
热学 \\
第#1讲 #2

\end{large}

\skiplines

黄志琦


\end{center}

\skiplines

{\small 
教材:《热学》第二版,赵凯华,罗蔚茵,高等教育出版社


课件下载
}
\courseurl 
\ech
\end{frame}
}

\def\bfr#1{
\begin{frame}
\chtitle{#1} 
\bch
}

\def\efr{
\ech 
\end{frame}
}

\title{Lesson 15 Problem Set 3}
  \author{}
  \date{}
\begin{document}
\tpage{15}{第三四章知识的应用II}
\newcounter{problem}

\stepcounter{problem}

\begin{frame}
  \chtitle{本讲内容}
  \bch
  \bitem
\item{克拉珀龙方程的应用}
\item{熵的计算}
\item{热力学第二定律的应用}
  \eitem
  \ech
\end{frame}

\section{Clapeyron Eq.}

\begin{frame}
  \chtitle{第三篇 克拉珀龙方程的应用}
  \bch
  \addfig{4}{clapeyron.png}
  \ech
\end{frame}

\begin{frame}
  \chtitle{克拉珀龙方程回顾}
  \bch
  \bmini{0.6}
  \lfig{2.5}{PVTdiagram.png}
  \emini
  \bmini{0.36}
  克拉珀龙方程是准静态过程吸热公式(准静态过程吸热量等于按照$C_V$计算的吸热量,加上热压强做功消耗的能量)在相变时$dT=0$的特殊情形。
  \emini
  
  \tbox{  $$\pfrac pTV = \frac{\Lambda^{\rm mol}}{T\left(V^{\rm mol}_\beta - V^{\rm mol}_\alpha\right)} $$
  }
  \ech
\end{frame}

\begin{frame}
  \chtitle{汽化过程的克拉珀龙方程的近似}
  \bch
  在汽化过程中,液态的摩尔体积可以忽略。气态的摩尔体积按理想气体近似为$RT/p$。故
  $$ \frac{dp}{dT} = \frac{p\Lambda^{\rm mol}}{RT^2}$$
  或者写成
{\blue  $$ \Lambda^{\rm mol} = RT \frac{d \ln p}{d \ln T} $$}
  \ech
\end{frame}


\begin{frame}
  \chtitle{蒸气压方程}
  \bch
  在一定温度范围内可以对摩尔汽化热$\Lambda^{\rm mol}$作线性近似即可积分得到
  $$\ln\left(\frac{p}{p_0}\right) = A\left(\frac{1}{T_0} - \frac{1}{T}\right) + B \ln\left(\frac{T}{T_0}\right)$$
  上式称为蒸气压方程。
  \ech
\end{frame}


\begin{frame}
  \chtitle{例题\theproblem (难度\sone)}
  \bch
  为什么在远低于$100\cdeg$的室温下,空气中仍含有水蒸气?空气中含有的水蒸气比例有上限吗?平时所说的湿度70\%是什么意思?
  \ech
\end{frame}


\begin{frame}
  \chtitle{例题\theproblem 解答}
  \bch
  $100\cdeg$是一个大气压下的水的沸点。按照克拉珀龙方程,低压低沸点。空气中水蒸气的分压很小,远低于一个大气压。故沸点可以低于室温。

  \skipline

  如果持续增大空气湿度(即增大水蒸气分压),使得沸点超过室温,那么水滴会自动从空气中凝结出来(降雨,露珠等自然现象)。这个临界的水蒸气分压,称为饱和蒸气压(水的蒸发曲线上对应于室温的压强值)。稳定情况下,空气中的水蒸气比例不能超过饱和蒸气压所限定的水蒸气分压上限。

  \skipline
  
  平时所说的湿度70\%,是指水蒸气分压是饱和蒸气压的70\%。当湿度达到100\%,就会出现水滴自动凝结的现象。
  \ech
\end{frame}

\stepcounter{problem}
\begin{frame}
  \chtitle{例题\theproblem (难度\sone)}
  \bch
  教材225页习题4-8
  \ech
\end{frame}


\begin{frame}
  \chtitle{例题\theproblem 解答}
  \bch
  {\small
  饱和蒸气压即气液共存态时的压强。
  
  按照克拉珀龙方程的汽化近似,
  
  $$\Lambda^{\rm mol} = RT \frac{\Delta ln p}{\Delta \ln T} = 8.314\times 373.15 \times \frac{\ln\frac{788}{733.7}}{\ln\frac{374.15}{372.15}} \SIJ/\SImol = 4.13\times 10^4 \SIJ/\SImol$$
  又$1\SImol$水为$0.018\SIkg$,故按习惯把汽化热写成
  $$ \Lambda = \frac{4.13\times 10^4}{0.018} \SIJ/\SIkg = 2.29\times 10^6 \SIJ/\SIkg$$
  }
  \ech
\end{frame}

\stepcounter{problem}
\begin{frame}
  \chtitle{例题\theproblem (难度\stwo)}
  \bch
  教材226页习题4-15
  \ech
\end{frame}


\begin{frame}
  \chtitle{例题\theproblem 解答}
  \bch
      {\scriptsize
        三相点既满足固态蒸气压方程,又满足液态蒸气压方程。
        $$\ln\left(\frac{\pthree}{\SImmHg}\right) = 23.3-\frac{3754\SIK}{\Tthree} = 19.49 - \frac{3063 \SIK}{\Tthree}$$
        解出
        $$\Tthree = 181.36\SIK;\ \pthree = 13.48\SImmHg$$
        忽略固体和液体的体积,计算出升华热:
        $$\Lambda^{\rm mol}_{\rm sublimate} = R\Tthree \frac{d\ln p}{d\ln T} = R\Tthree \frac{ \frac{3754\SIK }{\Tthree^2} dT }{\frac{1}{\Tthree} dT} = 3.121\times 10^4 \SIJ/\SImol $$
        汽化热
        $$\Lambda^{\rm mol}_{\rm evaporate} = R\Tthree \frac{d\ln p}{d\ln T} = R\Tthree \frac{ \frac{3063\SIK }{\Tthree^2} dT }{\frac{1}{\Tthree} dT} = 2.547\times 10^4 \SIJ/\SImol $$
        在三相点有固定的压强,所以各种潜热即是各种物态的焓差,由此得熔化热为
        $$\Lambda^{\rm mol}_{\rm melt} = \Lambda^{\rm mol}_{\rm sublimate} - \Lambda^{\rm mol}_{\rm evaporate} = 5.74 \times 10^3 \SIJ/\SImol$$
        
      }
  \ech
\end{frame}

\stepcounter{problem}
\begin{frame}
  \chtitle{例题\theproblem (难度\stwo)}
  \bch
  教材226页习题4-13
  \ech
\end{frame}


\begin{frame}
  \chtitle{例题\theproblem 解答}
  \bch
      {\small
        考虑$1\SImol$物质,根据热一律
        $$ \Delta U = Q + A = \Lambda^{\rm mol} - p \Delta V$$
        又吸热量等于热压强做功
        $$\Lambda^{\rm mol} = \pfrac p{\ln T}V \Delta V = p\Delta V \pfrac {\ln p}{\ln T}V $$
        结合上面两式即得证。
      }
  \ech
\end{frame}

\section{Entropy}

\begin{frame}
  \chtitle{第四篇:熵的计算}

  \addfig{4}{entropy.png}
\end{frame}

\stepcounter{problem}
\begin{frame}
  \chtitle{例题\theproblem (难度\sone)}
  \bch
  计算$1\SImol$冰在标准状态下熔解后的熵的变化,已知标准状态下冰的摩尔熔化热是$333\SIkJ$。
  \ech
\end{frame}

\begin{frame}
  \chtitle{例题\theproblem 解答}
  \bch
$$\Delta S = \frac{Q}{T} = \frac{333 \SIkJ}{273.15\SIK} = 1.22 \SIkJ/\SIK $$
  \ech
\end{frame}


\stepcounter{problem}
\begin{frame}
  \chtitle{例题\theproblem (难度\sone)}
  \bch
  教材226页习题4-21
  \ech
\end{frame}


\begin{frame}
  \chtitle{例题\theproblem 解答}
  \bch
  $$\Delta S =\Delta S_1 + \Delta S_2 =  \frac{-Q}{T_1} +\frac{Q}{T_2} = 3.49\times 10^{-3} \SIJ/\SIK$$
  \ech
\end{frame}


\stepcounter{problem}

\begin{frame}
  \chtitle{例题\theproblem (难度\sone)}
  \bch
  教材226页习题4-22
  \ech
\end{frame}


\begin{frame}
  \chtitle{例题\theproblem 解答}
  \bch
  转化为国际单位制
  $$T_1 = 294.15\SIK, T_2 = 268.15\SIK, \frac{dQ}{dt} = 2.91\times 10^4 \SIJ/\SIs$$
  
  $$\frac{d S}{dt} = \frac{d S_1}{dt} + \frac{d S_2}{dt} = \frac{-\frac{dQ}{dt}}{T_1} +\frac{\frac{dQ}{dt}}{T_2} = 9.58 \SIJ/(\SIK\cdot\SIs)$$
  \ech
\end{frame}


\stepcounter{problem}


\stepcounter{problem}
\begin{frame}
  \chtitle{例题\theproblem (难度\stwo)}
  \bch
  教材227页习题4-31
  \ech
\end{frame}


\begin{frame}
  \chtitle{例题\theproblem 解答}
  \bch
  {\small
    \bitem
  \item[1]{$A = Q_1'-Q_2  = 400\SIcal = 1674\SIJ$}
  \item[2]{$\Delta S = \left(\frac{Q_1'}{T_1}+\frac{-Q_2}{T_2} = \frac{600}{400}-\frac{200}{200}\right)\SIcal/\SIK = 0.5\SIcal/\SIK = 2.09\SIJ/\SIK$}
  \item[3]{卡诺可逆热机的吸放量绝对值与温度成正比。故在高温热源放热为$400\SIcal$。需要外界对制冷机做功$200\SIcal = 837\SIJ$。可逆热机造成的熵变为零。}
  \eitem
  }
  \ech
\end{frame}

\stepcounter{problem}
\begin{frame}
  \chtitle{例题\theproblem (难度\sthree)}
  \bch
  教材227页习题4-30
  \ech
\end{frame}


\begin{frame}
  \chtitle{例题\theproblem 解法1}
  \bch
  {\small
  闸瓦一共吸收(自机械能转化而来的)热量
  $$Q  = \frac{1}{2}m\upsilon^2 = 5.1852\times 10^5 \SIJ$$
  设闸瓦有固定的比热容并忽略刹车过程中的散热,则比热为
  $$ c = \frac{Q}{\Delta T} = \frac{5.1852\times 10^5}{60-20}\SIJ/\SIK = 1.30\times 10^4 \SIJ/\SIK$$
  
  增加的熵为
  $$ \Delta S_{\rm br} = \int_{T_0}^{T_1} \frac{c dT}{T} = c\ln\frac{T_1}{T_0} = \left(1.30\times 10^4 \times \ln\frac{333.15}{293.15}\right) \SIJ/\SIK = 1658 \SIJ/\SIK$$

  闸瓦散热时增加的熵为
  $$\int_{T_1}^{T_0} \left(\frac{cdT}{T} -\frac{cdT}{T_0} \right) = -\Delta S_{\rm br} + \frac{Q}{T_0} = \left(-1658 + \frac{5.185\times 10^5}{293.15}\right) = 111 \SIJ/\SIK$$
  
  }
  \ech
\end{frame}


\begin{frame}
  \chtitle{例题\theproblem 解法2(不会积分的小白解法)}
  \bch
      {\small
        $$Q  = \frac{1}{2}m\upsilon^2 = 5.1852\times 10^5 \SIJ$$
        闸瓦的吸热过程和散热过程都不是等温过程。不会积分的高数小白决定采用近似的办法,取中间温度$\overline{T} = 313.15\SIK$来计算熵增。
        
        刹车时产生的熵
        $$ \Delta S_{\rm br} = \frac{Q}{\overline{T}} = 1656 \SIJ/\SIK$$
        散热时产生的熵
        $$ \frac{-Q}{\overline{T}} + \frac{Q}{T_0} =   \left(-1656 + \frac{5.1852\times 10^5}{293.15}\right) = 113 \SIJ/\SIK $$
        
  }
  \ech
\end{frame}

\stepcounter{problem}
\begin{frame}
  \chtitle{例题\theproblem (难度\stwo)}
  \bch
  教材223页思考题4-20
  \ech
\end{frame}


\begin{frame}
  \chtitle{例题\theproblem 解答}
  \bch
  \bitem
\item[1]{理想气体绝热自由膨胀温度不变,$\Delta S = \nu R \ln\frac{V_2}{V_1} = \nu R \ln 4$}
\item[2]{与(1)相同}
\item[3]{可逆绝热过程熵变为零}
\item[4]{理想气体没有节流效应。节流过程后温度不变。所以结果也和(1)相同。}
  \eitem
  \ech
\end{frame}

\stepcounter{problem}
\begin{frame}
\chtitle{例题\theproblem (难度\stwo)}
\bch
光子气体的内能密度为$aT^4$,其中$a=\frac{\pi^2 k^4}{15\hbar^3c^3}$为常数,$T$为温度。已知$T=0$时光子气体的熵为零。试求温度为$T$的光子气体的熵密度。
\ech
\end{frame}

\begin{frame}
  \chtitle{例题\theproblem 解答}
  \bch
内能$U=aVT^4$,定体热容$\pfrac UTV = 4aVT^3$。

利用$dU = TdS - p dV$得到$\pfrac USV = T$,故
$$\pfrac STV = \frac{\pfrac UTV}{\pfrac USV} = \frac{4aVT^3}{T} = 4aVT^2$$
固定体积,对温度从$0$到$T$积分,得到
$$ S(T, V) = \int_0^T \pfrac STV dT = \frac{4}{3}aVT^3 $$
即熵密度为
$$ \frac{S}{V}= \frac{4}{3}aT^3$$
\ech
\end{frame}

\stepcounter{problem}
\begin{frame}
\chtitle{例题\theproblem (难度\stwo)}
\bch
把温度和压强都相同的,总摩尔数为$\nu$的$n$种不同的气体保持温度和压强不变地混合在一起。每种气体的摩尔分数(单种气体摩尔数/总摩尔数)分别为$c_1$, $c_2$, $\ldots$, $c_n$ ($\sum_i c_i = 1$)。试计算气体混合后相对于混合前的熵变。
\ech
\end{frame}

\begin{frame}
  \chtitle{例题\theproblem 解答}
  \bch
{\small
每种气体温度不变,体积由混合前的$Vc_i$变为$V$,熵增为$(\nu c_i) R\ln \frac{V}{c_iV} = - \nu R c_i\ln c_i$,对$n$种气体求和即得
$$\Delta S =   -\nu R \sum_{i=1}^n c_i \ln c_i$$}

\skipline

{\scriptsize
另解:

在混合前,任取一个分子,我们知道它是属于那一类气体。混合后,我们只知道它是第$i$种气体的概率为$c_i$,故单个分子熵变为
$-\sum_{i=1}^{n} kc_i\ln c_i $。 总共有$ \nu N_A $个分子,故总熵变为
$$ \Delta S =  -\nu N_A k\sum_{i=1}^n c_i \ln c_i = -\nu R \sum_{i=1}^n c_i \ln c_i$$  
}
\ech
\end{frame}

\stepcounter{problem}
\begin{frame}
\chtitle{例题\theproblem (难度\sthree)}
\bch
 某气体状态方程为$pV + f(V) = \nu RT$,其中$f$为某函数。气体经过准静态的等温加热膨胀体积从$V_1$变为$V_2$,这个过程的熵变。
   \ech
\end{frame}


\begin{frame}
\chtitle{例题\theproblem 解答}
\bch
准静态过程吸热量为按定体热容计算的吸热量加上热压强做功消耗的能量,在等温过程中仅需计算后者。
\bea
dS &=& \frac{\dbar Q}{T} \newl
&=&   \frac{T\pfrac pTV dV}{T} \newl
&=& \pfrac pTV dV  \newl
&=& \frac{\nu R}{V} dV
\eea
所以等温膨胀后熵变为$\nu R \ln\frac{V_2}{V_1}$。
\ech
\end{frame}


\stepcounter{problem}
\begin{frame}
  \chtitle{例题\theproblem (难度\sthree)}
  \bch
 某物质在$T_0 = 300\SIK$附近的物态方程可以写成
  $$ pV = \nu RT \left(1+\ln\frac{T}{T_0}\right) $$
 现有$1\SImol$的该物质在温度为$T_0$时等温膨胀体积变大一倍,求它的熵变。
  \ech
\end{frame}

\begin{frame}
  \chtitle{例题\theproblem 解答}
  \bch
  $$d S = \frac{\dbar Q}{T} = \frac{\pfrac p{\ln T}V dV}{T} = \pfrac pTV dV = \nu R\left(2+\ln\frac{T}{T_0}\right)\frac{dV}{V}$$
  在$T=T_0$时对上式积分即得
  $$\Delta S = 2\nu R \Delta\ln V = 11.5\SIJ/\SIK$$
  
  \ech
\end{frame}


\stepcounter{problem}
\begin{frame}
  \chtitle{例题\theproblem (难度\sfour)}
  \bch
  {\small
  某气体的焦耳-汤姆孙系数
  $$\alpha = \pfrac TpH =-\frac{a}{T^2}$$
  $a>0$为常量。

  当$p\rightarrow 0$时,该气体的定压比热容趋向于常量
  $$\lim_{p\rightarrow 0}C_p = c$$
  当温度趋向于零时,该气体的熵趋向于零。
  $$ \lim_{T\rightarrow 0} S = 0$$
  在标准状态下的该气体,经过准静态绝热压缩,压强增强为$8\atm$。求末态气体温度。
  }
  \ech
\end{frame}

\begin{frame}
  \chtitle{例题\theproblem 解答}
  \bch
  {\small
    先计算$H(T, p)$的函数形式
    $$H(T, p=0) = H(0, 0)+\int_0^T c dT = H_0+cT$$
    由于能量的基点可以任意选取,不妨取$H_0=0$。
  
    考察等焓线$p(T)$,对$\pfrac pTH = -\frac{T^2}{a}$积分得到
    $$ p = -\frac{T^3}{3a} +f(H) $$
    取$p=0$可以获得积分常数
    $$ p = \frac{-T^3 + \frac{H^3}{c^3}}{3a} $$
    即
    $$H = c\left(T^3 + 3ap\right)^{\frac{1}{3}} $$
  }
  \ech
\end{frame}

\begin{frame}
  \chtitle{例题\theproblem 解答(续)}
  \bch
  {\small
    $$ \pfrac STp = \frac{C_p}{T}= \frac{\pfrac HTp}{T}  = \frac{cT}{\left(T^3 +3ap \right)^{\frac{2}{3}}} $$
    再把上式从$0$到$T$积分
    $$ S(T, p) = c \int_0^T  \frac{x}{\left(x^3 +3ap \right)^{\frac{2}{3}}} dx$$
    令$x = (3ap)^{1/3}t$还可以把上式化简
    $$S(T, p) = c \int_0^{\frac{T}{(3ap)^{1/3}}} \frac{tdt}{\left(t^3+1\right)^{2/3}} $$
    可见熵只依赖于$Tp^{-1/3}$,绝热方程为$T\propto p^{1/3}$。即末态温度为$8^{1/3}\times 273.15\SIK = 546.3\SIK$。
}
  \ech
\end{frame}



\section{2nd Law}

\begin{frame}
  \chtitle{第五篇:热力学第二定律的应用}
  \addfig{4}{2ndlaw.png}
\end{frame}

\stepcounter{problem}
\begin{frame}
  \chtitle{例题\theproblem (难度\stwo)}
  \bch
  教材223页思考题4-14
  \ech
\end{frame}


\begin{frame}
  \chtitle{例题\theproblem 解答}
  \bch
  不违反。因为这个过程造成了汗水的相变。不满足热力学第二定律的“不引起其他变化”这个要求。
  \ech
\end{frame}


\stepcounter{problem}
\begin{frame}
  \chtitle{例题\theproblem (难度\stwo)}
  \bch
  教材223页思考题4-15
  \ech
\end{frame}


\begin{frame}
  \chtitle{例题\theproblem 解答}
  \bch
  有上限。物体温度不会超过太阳表面温度(约$6000\SIK$)。
  \ech
\end{frame}

\stepcounter{problem}
\begin{frame}
  \chtitle{例题\theproblem (难度\sthree)}
  \bch
  教材223页思考题4-24
  \ech
\end{frame}


\begin{frame}
  \chtitle{例题\theproblem 解答}
  \bch
  地球不是一个热平衡系统,因此是不可逆过程。地球熵减少了。但是环境熵增加得更多。不违反熵增加原理。
  \ech
\end{frame}

\stepcounter{problem}
\begin{frame}
  \chtitle{例题\theproblem (难度\stwo)}
  \bch
  教材223页思考题4-22
  \ech
\end{frame}


\begin{frame}
  \chtitle{例题\theproblem 解答}
  \bch
  实验中制造的水蒸气温度一般高于水温,过程不可逆。此过程中水蒸气的熵减少了,但水的熵增加得更多,不违反熵增加原理。
  \ech
\end{frame}

\stepcounter{problem}
\begin{frame}
  \chtitle{例题\theproblem (难度\stwo)}
  \bch
  教材223页思考题4-10
  \ech
\end{frame}


\begin{frame}
  \chtitle{例题\theproblem 解答}
  \bch
  \bitem
\item[1]{不可逆,因为水的气液两相处于不平衡}
\item[2]{可逆,因为是准静态过程}
\item[3]{可逆,因为是准静态过程}
\item[4]{不可逆,混合时会产生混合熵。}
  \eitem
  \ech
\end{frame}




\begin{frame}
\chtitle{热力学第二定律练习I}
\bch

\addfig{1.}{songfen.jpg}

\bitem
\item{论证摩擦生热过程是不可逆的}
\eitem

\ech
\end{frame}

\begin{frame}
\chtitle{热力学第二定律练习II}
\bch

\addfig{1.}{songfen.jpg}

\bitem
\item{有人想利用海洋不同深度处温度不同制造一种机器,把海水的内能转化为机械功,这是否违反热力学第二定律?}
\eitem

\ech
\end{frame}


\begin{frame}
\chtitle{热力学第二定律练习III}
\bch

\addfig{1.}{songfen.jpg}

\bitem
\item{给气筒里的气体加热,使它在保持内能不变的情况下膨胀推动活塞做功,这把热完全转化为了功,是否违反热力学第二定律?}
\eitem
\ech
\end{frame}


\begin{frame}
\chtitle{热力学第二定律练习IV}
\bch

\addfig{1.}{songfen.jpg}

\bitem
\item{论证绝热线和等温线不能有两个以上的交点。}
\eitem
\ech
\end{frame}




\section{Homework}

\begin{frame}
  \chtitle{第15周作业(序号接第14周)}
  \bch
  {\small 
  \bitem
\item[39]{教材习题4-9}
\item[40]{教材习题4-28}
\item[41]{教材习题4-20}
  \eitem
  }
  \ech
\end{frame}

\end{document}

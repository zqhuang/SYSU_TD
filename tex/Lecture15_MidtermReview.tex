\documentclass[CJK]{beamer}
\usepackage{CJKutf8}
\usepackage{beamerthemesplit}
\usetheme{Malmoe}
\useoutertheme[footline=authortitle]{miniframes}
\usepackage{amsmath}
\usepackage{amssymb}
\usepackage{graphicx}
\usepackage{eufrak}
\usepackage{color}
\usepackage{slashed}
\usepackage{simplewick}
\usepackage{tikz}
\graphicspath{{../figures/}}
\def\addfig#1#2{\begin{center}\includegraphics[width=#1 in]{#2}\end{center}}
\def\blacktext#1{{\color{black}#1}}
\def\bluetext#1{{\color{blue}#1}}
\def\redtext#1{{\color{red}#1}}
\def\darkbluetext#1{{\color[rgb]{0,0.2,0.6}#1}}
\def\skybluetext#1{{\color[rgb]{0.2,0.7,1.}#1}}
\def\cyantext#1{{\color[rgb]{0.,0.5,0.5}#1}}
\def\greentext#1{{\color[rgb]{0,0.7,0.1}#1}}
\def\darkgray{\color[rgb]{0.2,0.2,0.2}}
\def\lightgray{\color[rgb]{0.6,0.6,0.6}}
\def\gray{\color[rgb]{0.4,0.4,0.4}}
\def\blue{\color{blue}}
\def\red{\color{red}}
\def\green{\color{green}}
\def\darkblue{\color[rgb]{0,0.2,0.6}}
\def\skyblue{\color[rgb]{0.2,0.7,1.}}
\def\fdeg{{^\circ \mathrm{F}}}
\def\cdeg{^\circ \mathrm{C}}
\def\be{\begin{equation}}
\def\ee{\nonumber\end{equation}}
\def\bea{\begin{eqnarray}}
\def\eea{\nonumber\end{eqnarray}}
\def\ii{{\dot{\imath}}}
\def\bch{\begin{CJK}{UTF8}{gbsn}}
\def\ech{\end{CJK}}
\def\bitem{\begin{itemize}}
\def\eitem{\end{itemize}}
\def\bcenter{\begin{center}}
\def\ecenter{\end{center}}
\def\bex{\begin{minipage}{0.3\textwidth}\includegraphics[width=1in]{jugelizi.png}\end{minipage}\begin{minipage}{0.6\textwidth}}
\def\eex{\end{minipage}}
\def\chtitle#1{\frametitle{\bch#1\ech}}
\def\skipline{{\vskip0.1in}}
\def\skiplines{{\vskip0.2in}}
\def\lagr{{\mathcal{L}}}
\def\hamil{{\mathcal{H}}}
\def\vecv{{\mathbf{v}}}
\def\vecx{{\mathbf{x}}}
\def\vecy{{\mathbf{y}}}
\def\veck{{\mathbf{k}}}
\def\vecp{{\mathbf{p}}}
\def\vecn{{\mathbf{n}}}
\def\vecA{{\mathbf{A}}}
\def\vecP{{\mathbf{P}}}
\def\vecsigma{{\mathbf{\sigma}}}
\def\hatJn{{\hat{J_\vecn}}}
\def\hatJx{{\hat{J_x}}}
\def\hatJy{{\hat{J_y}}}
\def\hatJz{{\hat{J_z}}}
\def\hatj#1{\hat{J_{#1}}}
\def\hatphi{{\hat{\phi}}}
\def\hatq{{\hat{q}}}
\def\hatpi{{\hat{\pi}}}
\def\vel{\upsilon}
\def\Dint{{\mathcal{D}}}
\def\adag{{\hat{a}^\dagger}}
\def\bdag{{\hat{b}^\dagger}}
\def\cdag{{\hat{c}^\dagger}}
\def\ddag{{\hat{d}^\dagger}}
\def\hata{{\hat{a}}}
\def\hatb{{\hat{b}}}
\def\hatc{{\hat{c}}}
\def\hatd{{\hat{d}}}
\def\hatN{{\hat{N}}}
\def\hatH{{\hat{H}}}
\def\hatp{{\hat{p}}}
\def\Fup{{F^{\mu\nu}}}
\def\Fdown{{F_{\mu\nu}}}
\def\newl{\nonumber \\}
\def\SIkm{\,\mathrm{km}}
\def\SIyr{\,\mathrm{yr}}
\def\SIGyr{\,\mathrm{Gyr}}
\def\SIeV{\,\mathrm{eV}}
\def\SIkeV{\,\mathrm{keV}}
\def\SIMeV{\,\mathrm{MeV}}
\def\SIGeV{\,\mathrm{GeV}}
\def\SIcal{\,\mathrm{cal}}
\def\SIkcal{\,\mathrm{kcal}}
\def\SImol{\,\mathrm{mol}}
\def\SIm{\,\mathrm{m}}
\def\SIcm{\,\mathrm{cm}}
\def\SIfm{\,\mathrm{fm}}
\def\SImm{\,\mathrm{mm}}
\def\SInm{\,\mathrm{nm}}
\def\SImum{\,\mathrm{\mu m}}
\def\SIJ{\,\mathrm{J}}
\def\SIkJ{\,\mathrm{kJ}}
\def\SIs{\,\mathrm{s}}
\def\SIkg{\,\mathrm{kg}}
\def\SIg{\,\mathrm{g}}
\def\SIK{\,\mathrm{K}}
\def\SImmHg{\,\mathrm{mmHg}}
\def\SIPa{\,\mathrm{Pa}}
\def\vece{\mathrm{e}}
\def\bmat#1{\left(\begin{array}{#1}}
\def\emat{\end{array}\right)}
\def\bcase#1{\left\{\begin{array}{#1}}
\def\ecase{\end{array}\right.}
\def\calM{{\mathcal{M}}}
\def\calT{{\mathcal{T}}}
\def\calR{{\mathcal{R}}}
\def\barpsi{\bar{\psi}}
\def\baru{\bar{u}}
\def\barv{\bar{\upsilon}}
\def\bmini#1{\begin{minipage}{#1\textwidth}}
\def\emini{\end{minipage}}
\def\qeq{\stackrel{?}{=}}
\def\torder#1{\mathcal{T}\left(#1\right)}
\def\rorder#1{\mathcal{R}\left(#1\right)}
\def\contr#1#2{\contraction{}{#1}{}{#2}#1#2}
\def\trof#1{\mathrm{Tr}\left(#1\right)}
\def\trace{\mathrm{Tr}}
\def\comm#1{\ \ \ \left(\mathrm{used}\ #1\right)}
\def\tcomm#1{\ \ \ (\text{#1})}
\def\slp{\slashed{p}}
\def\slk{\slashed{k}}
\def\wulian{\includegraphics[width=0.18in]{emoji_wulian.jpg}}
\def\bye{\includegraphics[width=0.18in]{emoji_bye.jpg}}
\def\calp{{\mathfrak{p}}}
\def\veccalp{\mathbf{\mathfrak{p}}}
\def\atm{\,\mathrm{atm}}
\def\angstrom{\,\text{\AA}}
\def\Tthree{T_{\tiny \textcircled{3}}}
\def\pthree{p_{\tiny \textcircled{3}}}

\def\courseurl{http://zhiqihuang.top}

\def\tpage#1#2{
\begin{frame}
\bch
\begin{center}
\begin{large}
热学 \\
第#1讲 #2

\end{large}

\skiplines

黄志琦


\end{center}

\skiplines

{\small 
教材:《热学》第二版,赵凯华,罗蔚茵,高等教育出版社


课件下载
}
\courseurl 
\ech
\end{frame}
}

\def\bfr#1{
\begin{frame}
\chtitle{#1} 
\bch
}

\def\efr{
\ech 
\end{frame}
}

\title{Lesson 15 Sad Midterm}
  \author{}
  \date{}
\begin{document}
\tpage{15}{期中回顾}

\section{Review}

\begin{frame}
\chtitle{Midterm Review}
\bch
\bitem
\item{quiz分数榜和midterm分数相关度仅有0.38}
\item{选择题1,2题成功地侮辱了大家的智商,本次考试起步分10分。}
\item{选择题第4题是整个上半学期内容的基本出发点,大概有20\%的人做错。}
\item{填空题很惨淡。当我看到某张答卷上$(1.002)^4\approx 1.0024$的操作时,似乎明白了什么…}
\item{反映出很多和热学无关的基础训练不足的问题:数学需要报警,单位搞不清,没有量纲的概念,语言表述不清晰,手残…}
  
\eitem
\ech
\end{frame}


\begin{frame}
\chtitle{选择题第3题}
\bchL
热力学第零定律和热力学第一定律之间的关系是:
\bitem
\item[A]{是互相独立的实验定律}
\item[B]{第一定律以第零定律为前提}
\item[C]{第零定律以第一定律为前提}    
\eitem
\echL
\end{frame}


\begin{frame}
\chtitle{选择题第4题}
\bchL
设某气体的分子速率分布函数为$F(\upsilon)$。气体中任取一个分子,其速率$\upsilon$大于$\upsilon_0$的概率为:
\bitem
\item[A]{$\int_{\upsilon_0}^\infty F(\upsilon)d\upsilon$}
\item[B]{$\int_{\upsilon_0}^\infty 4\pi\upsilon^2 F(\upsilon)d\upsilon$}
\item[C]{$\int_{\upsilon_0}^\infty \frac{1}{4\pi\upsilon^2} F(\upsilon)d\upsilon$}    
\eitem
\echL
\end{frame}





\begin{frame}
\chtitle{选择题第5题}
\bchL
处于平衡态的单一成分理想气体,如果温度升高2\%,其分子平均速率
\bitem
\item[A]{约变大4\%}
\item[B]{约变大2\%}
\item[C]{约变大1\%}
  \eitem
\echL
\end{frame}

\begin{frame}
\chtitle{选择题第6题}
\bchL
一个绝热的刚性容器被中间隔板分成体积相等的左右两部分。左侧装有平衡态的理想气体,右侧是真空。把隔板突然抽掉,气体扩散到整个容器中并重新达到平衡,气体的温度和压强如何变化?
\bitem
\item[A]{温度降低,压强不变}
\item[B]{温度不变,压强变小}
\item[C]{温度降低,压强变小}
  \eitem
\echL
\end{frame}

\begin{frame}
\chtitle{填空题第3题}
\bchL
根据能均分定理,温度为$T$的气体分子内部的一个(简谐)振动自由度对分子平均能量的贡献为\uline{1}.
\echL
\end{frame}


\begin{frame}
\chtitle{填空题第4题}
\bchL
处于平衡态的单一成分二维理想气体,其分子平均速率和方均根速率之比等于\uline{1}.
\echL
\end{frame}


\begin{frame}
\chtitle{填空题第5题}
\bchL
在一个以匀角速度绕中心轴高速旋转的圆柱形容器内装有平衡态的理想气体,测得气体在中心轴处的压强为 $1.000\atm$,距离中心轴$1\SIm$处的压强为$1.002\atm$。那么气体在离中心轴$2\SIm$处的压强大约为\uline{1}.
\echL
\end{frame}

\begin{frame}
\chtitle{简答题第2题}
\bchL
容积为$100\SIL$的封闭容器内装有平均速率为$800\SIm/\SIs$的平衡态理想气体,其分子泻流速率为多少?(5分)在容器上开一个面积为$10^{-6}\SIm^2$的小孔,气体分子开始从小孔中逸出。经过$1\SIs$后,容器内分子数密度大约下降了百分之多少?
\echL
\end{frame}


\begin{frame}
\chtitle{简答题第3题}
\bchL
把大气近似为温度为$T$,分子质量为$m$的无限高的单一成分等温理想气体。重力加速度为$g$ (忽略重力加速度随高度的变化)。
\bitem
\item[(1)]{把重力势能的零点取在地面,计算气体分子的平均重力势能。}
\item[(2)]{随机任取两个气体分子,计算这两个分子的高度差大于$h$的概率。}  
\eitem
\echL
\end{frame}


\begin{frame}
\chtitle{上讲内容回顾}
\bch 
\bitem
\item{$pVT$系统(二自由度系统)的偏导数技巧}
\item{迄今学过的态函数:温度$T$,熵$S$,压强$p$,体积$V$,内能$U$,焓$H=U+pV$,自由能$F=U-TS$和自由焓$G=H-TS$}
\item{$U, H, F, G$的全微分以及由此导出的麦克斯韦关系}
\item{内能和状态方程的关系:
 $$\pfrac UVT = \pfrac p{\ln T}V - p $$}
\eitem
\ech
\end{frame}


\begin{frame}
\chtitle{快速答题}
\bch

\addfig{0.5}{think1.jpg}

$\pfrac{X}{Y}{X}$是合法的偏导数符号吗?为什么?

\ech
\end{frame}


\begin{frame}
\chtitle{快速答题}
\bch

\addfig{0.5}{think1.jpg}

写出下列函数的全微分:
\bitem
\item[1.]{ $$f(x,y);$$}
\item[2.]{ $$f(x,y, z);$$}  
\item[3.]{$$f(x, y, z(x,y)). $$}
\eitem


\ech
\end{frame}



\begin{frame}
\chtitle{思考题}
\bch

\addfig{1}{think3.jpg}

你能用隐函数的求导法则证明切换固定量的公式
      $$ \pfrac WXY = \pfrac WXZ + \pfrac WZX \pfrac ZXY$$
吗?
\ech
\end{frame}

\end{document}
